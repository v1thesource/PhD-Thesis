\chapter{Iodine defect equilibria in \zirconia}

\emph{The work in this chapter has been published in:} \\ A. Kenich \emph{et al.} J. Nucl. Mater. \textbf{511} (2018) 390-395. Ref \cite{kenichiodine2018}.

\label{ch:results2}

\section{Introduction}

Stress-corrosion cracking (SCC) in nuclear fuel pins is an issue related to early integrity of fuel assemblies in light water reactors (LWRs). SCC studies of the internal surface of zirconium-based fuel claddings have been conducted, which indicate that iodine is likely to be one of the main corrosive species involved in promoting crack growth \cite{rosenbaum1966interaction, bcoxpelletclad1990,fregonese2000failure,Sidky1998}. The exact mechanism for iodine SCC has not yet been determined due to difficulties observing the internal cladding surface in-situ, while experimental studies are not yet capable of reproducing the conditions under which such failures occur. We have therefore adopted a quantum-mechanical simulation approach to model the behaviour of iodine within the oxide layer of the cladding, the layer preceding the zirconium metal. 

Iodine is produced in the fuel pellet directly from fission and also from the decay of tellurium precursors. Both iodine and tellurium are relatively common fission products, with combined independent yields from thermal fission of U$_{235}$ above 5\% \cite{kennett1956mass, iodine129fissionyield, imanishi1976independent, iodinefissionyields, iodine132, amiel1975odd}. The majority of thermal fission events occur in the outer rim of the fuel pellet, and a fission product penetration depth of up to 8 $\mu$m in \zirconia\ \cite{degueldre2001behaviour} suggests a large degree of implantation within the oxide and the Zr metal into which the oxide grows, raising the concentration of I well above the equilibrium value. Iodine and many of its compounds (ZrI$_{4}$, CsI) are volatile and fuel pellets contain many cracks and spaces through which iodine may be rapidly transported to the cladding. When reactor power is increased during start-up, iodine is released in substantial quantities from the UO$_{2}$ pellet \cite{peehs1982experimental}. This is believed to cause crack propagation in the cladding when combined with stresses imposed on the cladding by the fuel pellet, a phenomenon known as pellet-cladding interaction (PCI). Upper limits on power ramping and holding times have therefore been established by fuel suppliers to mitigate potential PCI failures \cite{yagnik2005effect}. While these restrictions have reduced or prevented the incidence of PCI failures, they also impose costs on the operator due to longer ramping periods. This also restricts the ability of the nuclear reactor to load-follow grid demand. Cladding/fuel materials resistant to PCI failure are therefore of great interest in the nuclear power industry, promoting research into solutions such as cladding liners and doped fuel pellets \cite{nonon2005pci,yang2012effect}. 

%\item Power ramping: Increasing power, such as during reactor start-up, can lead to cladding failure.   %power must be ramped up gradually in order to avoid excessive temperature gradients in the fuel pins, but also to manage fission product concentrations. Due to the different half-lives of various fission products, a power ramp will cause a transient increase in the iodine concentration within a fuel pin


Iodine is an oxidising agent, which, under standard conditions, will oxidise Zr metal to produce ZrI$_{4}$. However, oxygen is also present in the internal fuel pin environment, both from the native \zirconia\ layer on the cladding, and the evolution of oxygen from the fuel pellet during burnup. Liberated oxygen will compete with iodine in the oxidation of the Zr metal, but whereas iodine promotes crack growth under stress, oxygen provides a more protective effect, self-limiting its diffusion into the metal \cite{farina2002stress, causey2005review}. Furthermore, oxygen is a more powerful oxidising agent than iodine, reacting together to produce I${_2}$O$_{5}$. For these reasons, the internal oxide layer of the cladding is often considered a barrier to the ingress of iodine into the Zr metal. 

Unlike oxygen and hydrogen, which readily diffuse into Zr metal to occupy interstitial sites, iodine atoms have been predicted in atomistic studies to have very high energy barriers to bulk interstitial diffusion \cite{rossi2015first,legris2005ab,carlot2002energetically}. This is due to the relatively large radius of the iodine atom, which imposes large local strains when penetrating the Zr lattice. This suggests that iodine will instead be transported towards crack tips via grain boundaries. Indeed, intergranular cracking has been observed in PCI failures, but only for a few hundred nm before a more rapid transgranular crack propagation \cite{fregonese2000failure}. Conversely, no atomic scale studies of iodine in \zirconia\ were found in the literature.  

\zirconia\ grown thermally on Zr metal exists mainly in either the monoclinic or tetragonal phase \cite{Howard1988,teufer1962crystal}. We can expect the internal \zirconia\ layer of the cladding to be mostly monoclinic in early life, with the stress-stabilised tetragonal phase appearing near the oxide/metal interface due to cohesive strains resulting from the lattice mismatch. With increasing burnup however, it is expected that more tetragonal and possibly even the cubic phase of \zirconia\ forms due to vacancy formation and residual stresses in the lattice from radiation damage \cite{sickafus1999radiation}. Amorphisation due to radiation damage has also been observed in the cubic phase from Cs$^{+}$ implantation \cite{amorphization2000wang}. In this study however, while defect energies for the cubic phase are reported, we focus our analysis on monoclinic and tetragonal \zirconia\ phases, partly due to difficulties predicting the behaviour of the pure high-temperature cubic phase using energies calculated from a static energy technique. 

The effectiveness of the oxide layer as a barrier to iodine is debated, with one study presuming that the oxide is bypassed entirely by iodine due to fracturing, leaving the Zr metal underneath exposed \cite{rossi2015first}. The outermost part of the oxide, which is porous, exhibits networks of interconnected grain boundary diffusion pathways towards the oxide/metal interface which are certainly wide enough (1-3 nm) to allow iodine transport \cite{ni2010porosity}. The oxygen-saturated Zr at the oxide/metal interface is not, however, taken into account, and it is expected that this will influence the corrosion mechanism due to iodine-oxygen competition: even the much smaller hydrogen atom has its rate of diffusion into the metal reduced by the presence of oxygen, as shown in both computational \cite{glazoff2014oxidation} and experimental hydrogen pick-up studies \cite{couet2014hydrogen}. This means that some barrier to iodine ingress must already exist near the oxide/metal interface. The varying levels of oxygen across the oxide layer itself also have an effect on defect behaviour, and will therefore influence the initiation mechanisms behind PCI failures. Thus here, we predict iodine incorporation energies and defect equilibria in \zirconia\ as a function of oxygen pressure through Brouwer diagrams, in order to predict the resulting iodine defect response.


\subsection{Pellet-cladding interaction}

Pellet-cladding interaction refers to the interaction between the fuel and the cladding at higher burnups where the gas gap has been closed by the swelling of the fuel pellets. PCI has both a mechanical and a chemical component, sometimes referred to specifically as pellet cladding mechanical interaction (PCMI) and pellet cladding chemical interaction (PCCI) respectively.


\section{Methodology}
\subsection{Computational details}

Calculations were performed using the CASTEP 8.0 density functional theory (DFT) based code \cite{Clark2005}. Ultra-soft pseudo-potentials with a cut-off energy of 600 eV were employed. The Perdew, Burke and Ernzerhof (PBE) \cite{Perdew1996} parameterisation of the generalised gradient approximation (GGA) was used to describe the exchange correlation functional. A Monkhorst-Pack sampling scheme \cite{Monkhorst1976} was used for Brillouin zone integration, with a minimum \emph{k}-point separation of 0.09 \r{A}\textsuperscript{-1}. The Pulay method for density mixing \cite{Pulay1980} was used to improve simulation convergence. 

The electronic energy convergence criterion was set to $1\times10^{-6}$ eV and the maximum force between atoms limited to $1\times10^{-2}$ eV \r{A}\textsuperscript{-1}. A gradient-descent geometry optimisation task was run on the cell until consecutive iterations differed in energy and atomic displacement by less than $1\times10^{-5}$ eV and $5\times10^{-4}$ \r{A} respectively. 

\subsection{Incorporation energies}

The inner oxide of the fuel cladding will be highly defective due to radiation damage, resulting in a high concentration of pre-existing intrinsic defect sites relative to the concentration of iodine. We therefore consider the energy of iodine incorporation on to these existing defect sites. The energies to incorporate iodine at interstitial and substitutional sites in \zirconia\ were calculated from the set of defective and perfect supercell DFT energies. Incorporation energies were established to place iodine into vacancy sites of different charge to generate defects from \ch{I_{O}^{x}} to \ch{I_{O}^{**}}, and \ch{I_{Zr}^{x}} to \ch{I_{Zr}^{''''}}. I was also incorporated onto the interstitial site. 

The incorporation energy equation uses $\frac{1}{2}$I$_{2}$ as the reference state of iodine:

\begin{equation}
\label{interstitial_incorp_equation}
E_{inc}(\ch{I_{$i$}^{x}}) = E_{DFT}(\ch{I_{$i$}^{x}}) - (E_{DFT}(ZrO_2) + \frac{1}{2}\mu_{I_{2}})  % - \frac{E_{I_2}}{2}
\end{equation}

where $E_{inc}(\ch{I_{$i$}^{x}})$ is the incorporation energy of a neutral iodine interstitial, $E_{DFT}(\ch{I_{$i$}^{x}})$ is the energy of a neutral iodine interstitial, $E_{DFT}(ZrO_2)$ is the energy of a non-defective \zirconia\ supercell and $\mu_{I_{2}}$ is the chemical potential of an I$_{2}$ molecule, taken from a single point DFT calculation of the I$_{2}$ dimer. Similarly, for a substitutional defect:

\begin{equation}
\label{o_sub_incorp_equation}
E_{inc}(\ch{I_{O}^{$n$}}) = E_{DFT}(\ch{I_{O}^{$n$}}) - (E_{DFT}(\ch{V_{O}^{$n$}}) + \frac{1}{2}\mu_{I_{2}})  % - \frac{E_{I_2}}{2}
\end{equation}

where $\ch{I_{O}^{$n$}}$ is an iodine substitutional defect at an oxygen site of charge $n$ and $\ch{V_{O}^{$n$}}$ is the corresponding oxygen vacancy.


\subsection{Defect Equilibrium Response to Oxygen Partial Pressure}

Brouwer diagrams, also known as Kr{\"o}ger-Vink diagrams, were produced using a method outlined by Murphy et al. \cite{Murphy2014} to determine defect concentrations as a function of oxygen partial pressure. We start from the statement that the chemical potential of \zirconia\ is equivalent to the sum of the chemical potentials $\mu$ of its constituent species, Zr and O:

\begin{equation}
{\mu}_{ZrO_2(s)} = {\mu}_{Zr}(p_{O_2}, T) + {\mu}_{O_2}(p_{O_2}, T)
\label{mewZrO2results2}
\end{equation}

where $T$ denotes temperature and $p_{O_2}$ denotes oxygen partial pressure. The chemical potential of \zirconia\ in the solid state is assumed to have negligible dependence on $T$ and $p_{O_2}$ relative to ${\mu}_{Zr}$ and ${\mu}_{O_2}$. Energies can be obtained for bulk \zirconia\ and Zr, but the ground state of oxygen is not correctly reproduced in DFT \cite{Batyrev2000,Lozovoi2001}. Instead, we use the approach of Finnis et al. \cite{Finnis2005} to infer the oxygen chemical potential from standard state values. We can use the experimental Gibbs free energy to produce an equation where $\mu_{O_2}$ is the only unknown:

\begin{equation}
\Delta{G^{\plimsoll}_{f, ZrO_2}} = \mu_{ZrO_2(s)} - (\mu_{Zr(s)} + \mu^{\plimsoll}_{O_2})
\end{equation}

where $\Delta{G^{\plimsoll}_{f, ZrO_2}}$ is the experimental Gibbs energy at standard temperature and pressure and $\mu^{\plimsoll}_{O_2}$ is the oxygen chemical potential under the same conditions. The values of $\mu_{ZrO_2(s)}$ and $\mu_{Zr(s)}$ are calculated from the DFT energies. Once $\mu^{\plimsoll}_{O_2}$ is calculated, we can generalise the chemical potential of oxygen for any value of $T$ and $p_{O_2}$ by appending an ideal gas relationship $\Delta{\mu(T)}$ and a Boltzmann distribution:

\begin{equation}
\mu_{O_2}(p_{O_2},T) = \mu^{\plimsoll}_{O_2} + \Delta{\mu(T)} + \frac{1}{2}{k_B}log(\frac{p_{O_2}}{p^{\plimsoll}_{O_2}})
\end{equation}

Using our generalised formula for $\mu_{O_2}$, we fix the temperature within the range of thermal phase-stabilisation (1500 K for tetragonal \zirconia) and calculate $\mu_{O_2}$ for many different values of $p_{O_2}$ between $10^{-35}$ and 10$^{0}$ atm, corresponding to oxygen deficient and oxygen rich environments, respectively ($p_{O_2}$ in air is approximately 0.2 atm). While the tetragonal phase will be stress-stabilised in practice, thermal-stabilisation in such models has been shown to qualitatively approximate the effect of stress-stabilisation, while allowing a wider range of dopant behaviours to be predicted \cite{Bell2016}. Equilibrium defect concentrations are then calculated at each $\mu_{O_2}$ and plotted against $p_{O_2}$ to produce a Brouwer diagram. Brouwer diagrams at extrinsic defect concentrations of $10^{-5}$ and $10^{-3}$ parts/fu (i.e. parts per \zirconia\ formula unit) were generated to examine low and high dopant concentrations, respectively. These two concentrations were examined because the amount of fission products present at a particular point in a fuel pellet depends on macroscopic parameters, including its position in the core and the time since the last shutdown, but also microscopic parameters such as the radial position in the pellet. These two concentrations were selected because $10^{-3}$ parts/fu is high enough to model an aggregation of iodine (such as at a crack tip), and $10^{-5}$ parts/fu was found to be the concentration below which iodine did not have a significant effect on defect equilibria. 

\section{Results}

\subsection{Incorporation energies}

\subsubsection*{Interstitial Sites}
Neutral iodine incorporation energies in interstitial sites for each phase are reported in Table \ref{i_incorp_interstitial}. The $2a$ and $2c$ sites in monoclinic \zirconia\ provide the least unfavourable iodine incorporation energy, followed by the $2b$ and $8e$ sites in tetragonal \zirconia , although in all cases energies are positive and large, indicating a large energy penalty against interstitial incorporation. The difference in incorporation energies between monoclinic and tetragonal \zirconia\ is approximately 1 eV, whereas the difference between tetragonal and cubic is 3.5 eV, indicating especially unfavourable conditions in cubic \zirconia . These differences are likely due to the larger interstitial sites in the lower-temperature phases, as monoclinic \zirconia\ exhibits the least and cubic \zirconia\ the most dense cell. 
%It is expected that these densities will be representative because the decreased density due to thermal expansion at 650 K will be countered by attractive dispersion forces. 

While the incorporation energies of iodine in the interstitial sites of \zirconia\ are large, for a fixed iodine concentration, they become relevant as the intrinsic defect populations become small, such as at low temperatures relative to the melting point. This is because interstitial sites are always available, whereas at low intrinsic defect concentrations, substitutional sites become saturated and accommodation at a lattice site first requires the creation of a vacancy defect, which has a formation energy penalty associated with it. 

When Brouwer diagrams are generated, iodine will also be considered as a charged species at an interstitial site. This includes I$^{+}$ and I$^{-}$, where I$^{+}$ is a smaller ion that is more easily accommodated at an interstitial site.

\subsubsection*{Oxygen Sites}

Table \ref{i_incorp_oxygen} reports incorporation energies of iodine at various oxygen sites. In each phase, the lowest incorporation energy was that for accommodation at a vacant oxygen site such that iodine is in the 1- oxidation state, resulting in the overall defect \ch{I_{O}^{*}}. This anionic behaviour is expected from a halogen atom in a highly reducing site as it promotes the filling of the \emph{p} shell. % and these atoms have large electron affinities since they require only one electron to achieve the relatively stable noble gas electron configuration. 

\subsubsection*{Zirconium Sites}

Incorporation energies of iodine on zirconium sites are reported in Table \ref{i_incorp_zirconium}. The  incorporation energy decreases as the charge of the defect decreases from -4 to 0 (i.e. nominally from I$^{0}$ to I$^{4+}$. This is due to the decrease in the size of the iodine species with increasing positive charge, fitting better into the small Zr$^{4+}$ cation site. This alone does not guarantee the emergence of uncharged iodine defects (I$^{4+}$) on zirconium sites when all energy terms are considered. In particular, there is also an energy penalty incurred in the change in charge of iodine. A Mulliken population analysis revealed a charge localised on the iodine of +2.31 at the \ch{I_{Zr}^{x}} defect, and a +0.86 charge on the \ch{I_{Zr}^{''''}} defect, with the remaining charge accommodated by other ions in the lattice. 

\begin{table}[htp]
\onehalfspacing
\centering
\caption{Incorporation energies of neutral iodine interstitials in non-defective supercells.}
\label{i_incorp_interstitial}
\begin{tabular}{ccc} % \ch{V_{O}^{x}}
\hline
\hspace{0.7 cm} \multirow{2}{*}{\textbf{Structure}} \hspace{0.7 cm} & \multirow{2}{*}{\textbf{Site}} & \multirow{2}{*}{\textbf{Incorporation of \ch{I_{i}^{x}} (eV)}} \\
                                     &                                &                                                                \\ \hline
\multirow{4}{*}{\textbf{Monoclinic}} & $2a$               & 8.55                                                       \\
                                     & $2b$               & 10.81                                                      \\
                                     & $2c$               & 8.79                                                       \\
                                     & $2d$               & 10.94                                                      \\ \hline
\multirow{2}{*}{\textbf{Tetragonal}} & $2b$             & 9.49                                                       \\
                                     & $8e$               & 9.53                                                       \\ \hline
\multirow{2}{*}{\textbf{Cubic}}      & $24d$                      & 13.02                                                      \\
                                     & $4b$             & 13.08                                                     
\end{tabular}
\end{table}


\begin{table}[htp] % Iodine O site incorporation
\onehalfspacing
\centering
\caption{Incorporation energies of iodine in oxygen sites of the monoclinic, tetragonal, and cubic \zirconia\ phases.}
\label{i_incorp_oxygen}
\begin{tabular}{cccc} % \ch{V_{O}^{x}}
\hline
\multirow{2}{*}{\textbf{Structure}} & \multicolumn{3}{c}{\textbf{Incorporation energy (eV)}} \\ \cline{2-4} 
                                    & \hspace{0.7 cm} \textbf{\ch{I_{O}^{**}}} \hspace{0.7 cm} & \textbf{\ch{I_{O}^{*}}} & \textbf{\ch{I_{O}^{x}}} \\ \hline
\textbf{Monoclinic (3 co-ord)}      & 4.54             & 2.90             & 3.67             \\
\textbf{Monoclinic (4 co-ord)}      & 5.63             & 3.77             & 4.87             \\
\textbf{Tetragonal}                 & 6.19             & 4.02             & 4.44             \\
\textbf{Cubic}                      & 8.37             & 5.74             & 6.66            
\end{tabular}
\end{table}

\begin{table}[htp] % Iodine Zr site incorporation
\onehalfspacing
\centering
\caption{Incorporation energies of iodine in zirconium sites of \zirconia.}
\label{i_incorp_zirconium}
\begin{tabular}{ccllll}
\hline
\hspace{0.7 cm} \multirow{2}{*}{\textbf{Structure}} \hspace{0.7 cm} & \multicolumn{5}{c}{\hspace{0.7 cm} \textbf{Incorporation energy (eV)}} \hspace{0.7 cm}                                                          \\ \cline{2-6} 
                                    & \multicolumn{1}{l}  {\textbf{\hspace{0.45 cm} \ch{I_{Zr}^{''''}}}} \hspace{0.45 cm} & \textbf{\ch{I_{Zr}^{'''}}} \hspace{0.45 cm} & \textbf{\ch{I_{Zr}^{''}}} \hspace{0.45 cm} & \textbf{\ch{I_{Zr}^{'}}} \hspace{0.45 cm} & \textbf{\ch{I_{Zr}^{x}}} \\ \hline
\textbf{Monoclinic}                 & 6.78                             &       3.65            &        0.89          &         -2.84         &     -5.08             \\
\textbf{Tetragonal}                 & 7.58                            &         3.64         &        1.69          &      -2.13            &     -4.57             \\
\textbf{Cubic}                      & 9.70                            &         6.81         &        3.01          &        0.38          &      -3.14           
\end{tabular}
\end{table}

%\subsection{Temperature dependence}

%To examine the temperature dependence of the defect equilibria, the concentration of dopant iodine was held constant while temperature was changed.


\subsection{Dopant concentration dependence}

To examine the dependence of the defect equilibria on iodine dopant concentration, the temperature was held constant while iodine concentration was changed.

\begin{landscape}
\begin{figure}[ht] % Mono conc sweep
\begin{center}
\begin{tikzpicture} % 10e-3 iodine conc in mono
	\begin{axis}
		[width=8.22cm, xlabel={\ch{log_{10}}($p_{O_{2}}$) (atm)}, ylabel={\ch{log_{10}}([D]) (per f.u.)}, ymin=-10, ymax=0, xmin=-35, xmax=0, legend style={draw={}, at={(0.40,0.97)}, anchor=north west, legend columns=3, nodes={scale=0.75, transform shape}}]
        \addplot[no marks, draw=blue!70!black] table [x=pO2, y=electrons,]{dat/1e5iconcmono650.dat};  \node at (-28.1,-7.5) {\ch{e^{'}}};  \addlegendentry{\ch{e^{'}}};
        \addplot[no marks, draw=red!85!black] table [x=pO2, y=holes,]{dat/1e5iconcmono650.dat}; \addlegendentry{\ch{h^{\textperiodcentered}}}; % \node at (-1,-4.5) {\ch{h^{\textperiodcentered}}};
%         \addplot[no marks, draw=black!70!green] table [x=pO2, y=VO{2},]{dat/1e5iconcmono650.dat}; \addlegendentry{\ch{V_{O}^{\textperiodcentered\textperiodcentered}}};
%         \addplot[no marks, draw=black!55!green] table [x=pO2, y=VO{1},]{dat/1e5iconcmono650.dat}; \addlegendentry{\ch{V_{O}^{\textperiodcentered}}};
%         \addplot[no marks, draw=black!30!green] table [x=pO2, y=VO{0},]{dat/1e5iconcmono650.dat}; \addlegendentry{\ch{V_{O}^{x}}};
%         \addplot[no marks, draw=yellow!85!blue] table [x=pO2, y=VM{-4},]{dat/1e5iconcmono650.dat}; \addlegendentry{\ch{V_{Zr}^{''''}}};
%         \addplot[no marks, draw=yellow!75!blue] table [x=pO2, y=VM{-3},]{dat/1e5iconcmono650.dat}; \addlegendentry{\ch{V_{Zr}^{'''}}};
%         \addplot[no marks, draw=yellow!65!blue] table [x=pO2, y=VM{-2},]{dat/1e5iconcmono650.dat}; \addlegendentry{\ch{V_{Zr}^{''}}};
%         \addplot[no marks, draw=yellow!55!blue] table [x=pO2, y=VM{-1},]{dat/1e5iconcmono650.dat}; \addlegendentry{\ch{V_{Zr}^{'}}};
%         \addplot[no marks, draw=yellow!45!blue] table [x=pO2, y=VM{0},]{dat/1e5iconcmono650.dat}; \addlegendentry{\ch{V_{Zr}^{x}}};
%         \addplot[no marks, draw=red!60!yellow] table [x=pO2, y=Oi{-2},]{dat/1e5iconcmono650.dat}; \addlegendentry{\ch{O_{i}^{''}}};
%         \addplot[no marks, draw=red!50!yellow] table [x=pO2, y=Oi{-1},]{dat/1e5iconcmono650.dat}; \addlegendentry{\ch{O_{i}^{'}}};
%         \addplot[no marks, draw=red!40!yellow] table [x=pO2, y=Oi{0},]{dat/1e5iconcmono650.dat}; \addlegendentry{\ch{O_{i}^{x}}};
%         \addplot[no marks, draw=green!80!pink] table [x=pO2, y=Mi{4},]{dat/1e5iconcmono650.dat}; \addlegendentry{\ch{Zr_{i}^{\textperiodcentered\textperiodcentered\textperiodcentered\textperiodcentered}}};
%         \addplot[no marks, draw=green!70!pink] table [x=pO2, y=Mi{3},]{dat/1e5iconcmono650.dat}; \addlegendentry{\ch{Zr_{i}^{\textperiodcentered\textperiodcentered\textperiodcentered}}};
%         \addplot[no marks, draw=green!60!pink] table [x=pO2, y=Mi{2},]{dat/1e5iconcmono650.dat}; \addlegendentry{\ch{Zr_{i}^{\textbf{\textperiodcentered\textperiodcentered}}}};
%         \addplot[no marks, draw=green!50!pink] table [x=pO2, y=Mi{1},]{dat/1e5iconcmono650.dat}; \addlegendentry{\ch{Zr_{i}^{\textperiodcentered}}};
%         \addplot[no marks, draw=green!40!pink] table [x=pO2, y=Mi{0},]{dat/1e5iconcmono650.dat}; \addlegendentry{\ch{Zr_{i}^{x}}};
%         \addplot[no marks, dashed, draw=red!70!black] table [x=pO2, y=Ii{0},]{dat/1e5iconcmono650.dat}; \addlegendentry{\ch{I_{i}^{x}}};
%         \addplot[no marks, dashed, draw=red!50!black] table [x=pO2, y=Ii{-1},]{dat/1e5iconcmono650.dat}; \addlegendentry{\ch{I_{i}^{'}}};
        \addplot[no marks, dashed, draw=purple!60!white] table [x=pO2, y=Ii{1},]{dat/1e5iconcmono650.dat}; \addlegendentry{\ch{I_{i}^{\textperiodcentered}}}; 
        \addplot[no marks, dashed, draw=blue!50!white] table [x=pO2, y=IsubO{1},]{dat/1e5iconcmono650.dat}; \addlegendentry{\ch{I_{O}^{\textperiodcentered}}}; \node at (-22,-4.5) {\ch{I_{O}^{\textperiodcentered}}};
        \addplot[no marks, dashed, draw=green!60!black] table [x=pO2, y=IsubO{2},]{dat/1e5iconcmono650.dat}; \addlegendentry{\ch{I_{O}^{\textperiodcentered\textperiodcentered}}}; 
        \addplot[no marks, dashed, draw=black] table [x=pO2, y=IsubO{3},]{dat/1e5iconcmono650.dat}; \addlegendentry{\ch{I_{O}^{\textperiodcentered\textperiodcentered\textperiodcentered}}};
        \addplot[no marks, dashed, draw=orange!80!black] table [x=pO2, y=IsubZr{-3},]{dat/1e5iconcmono650.dat};  \node at (-22,-6.2) {\ch{I_{Zr}^{'''}}}; \addlegendentry{\ch{I_{Zr}^{'''}}};
%         \addplot[no marks, dashed, draw=green!50!black] table [x=pO2, y=IsubZr{-4},]{dat/1e5iconcmono650.dat}; \addlegendentry{\ch{I_{Zr}^{''''}}};
%         \addplot[no marks, dashed, draw=green!30!black] table [x=pO2, y=IsubZr{-5},]{dat/1e5iconcmono650.dat}; \addlegendentry{\ch{I_{Zr}^{'''''}}};
%         \addplot[no marks] table [x=pO2, y=Stoich,]{dat/1e5iconcmono650.dat}; \addlegendentry{Stoich};
\node at (-33.7,-0.5) {\textbf{a)}};
			\end{axis}            
\end{tikzpicture} % 10e-5 iodine conc in mono
\begin{tikzpicture} % 10e-3 iodine conc in mono
	\begin{axis}
		[width=8.22cm, xlabel={\ch{log_{10}}($p_{O_{2}}$) (atm)}, yticklabels={}, ymin=-10, ymax=0, xmin=-35, xmax=0]
        \addplot[no marks, draw=blue!70!black] table [x=pO2, y=electrons,]{dat/1e3iconcmono650.dat}; \node at (-29,-7) {\ch{e^{'}}};
        \addplot[no marks, draw=red!85!black] table [x=pO2, y=holes,]{dat/1e3iconcmono650.dat}; 
%         \addplot[no marks, draw=black!70!green] table [x=pO2, y=VO{2},]{dat/1e3iconcmono650.dat}; 
%         \addplot[no marks, draw=black!55!green] table [x=pO2, y=VO{1},]{dat/1e3iconcmono650.dat}; 
%         \addplot[no marks, draw=black!30!green] table [x=pO2, y=VO{0},]{dat/1e3iconcmono650.dat}; 
%         \addplot[no marks, draw=yellow!85!blue] table [x=pO2, y=VM{-4},]{dat/1e3iconcmono650.dat}; 
%         \addplot[no marks, draw=yellow!75!blue] table [x=pO2, y=VM{-3},]{dat/1e3iconcmono650.dat}; 
%         \addplot[no marks, draw=yellow!65!blue] table [x=pO2, y=VM{-2},]{dat/1e3iconcmono650.dat}; 
%         \addplot[no marks, draw=yellow!55!blue] table [x=pO2, y=VM{-1},]{dat/1e3iconcmono650.dat}; 
%         \addplot[no marks, draw=yellow!45!blue] table [x=pO2, y=VM{0},]{dat/1e3iconcmono650.dat}; 
%         \addplot[no marks, draw=red!60!yellow] table [x=pO2, y=Oi{-2},]{dat/1e3iconcmono650.dat}; 
%         \addplot[no marks, draw=red!50!yellow] table [x=pO2, y=Oi{-1},]{dat/1e3iconcmono650.dat}; 
%         \addplot[no marks, draw=red!40!yellow] table [x=pO2, y=Oi{0},]{dat/1e3iconcmono650.dat}; 
%         \addplot[no marks, draw=green!80!pink] table [x=pO2, y=Mi{4},]{dat/1e3iconcmono650.dat}; 
%         \addplot[no marks, draw=green!70!pink] table [x=pO2, y=Mi{3},]{dat/1e3iconcmono650.dat}; 
%         \addplot[no marks, draw=green!60!pink] table [x=pO2, y=Mi{2},]{dat/1e3iconcmono650.dat}; 
%         \addplot[no marks, draw=green!50!pink] table [x=pO2, y=Mi{1},]{dat/1e3iconcmono650.dat}; 
%         \addplot[no marks, draw=green!40!pink] table [x=pO2, y=Mi{0},]{dat/1e3iconcmono650.dat}; 
%         \addplot[no marks, dashed, draw=red!70!black] table [x=pO2, y=Ii{0},]{dat/1e3iconcmono650.dat}; 
%         \addplot[no marks, dashed, draw=red!50!black] table [x=pO2, y=Ii{-1},]{dat/1e3iconcmono650.dat}; 
        \addplot[no marks, dashed, draw=purple!60!white] table [x=pO2, y=Ii{1},]{dat/1e3iconcmono650.dat}; 
        \addplot[no marks, dashed, draw=blue!50!white] table [x=pO2, y=IsubO{1},]{dat/1e3iconcmono650.dat}; \node at (-22,-2.5) {\ch{I_{O}^{\textperiodcentered}}};
        \addplot[no marks, dashed, draw=green!60!black] table [x=pO2, y=IsubO{2},]{dat/1e3iconcmono650.dat}; 
        \addplot[no marks, dashed, draw=black] table [x=pO2, y=IsubO{3},]{dat/1e3iconcmono650.dat}; 
        \addplot[no marks, dashed, draw=orange!80!black] table [x=pO2, y=IsubZr{-3},]{dat/1e3iconcmono650.dat}; \node at (-22,-4.4) {\ch{I_{Zr}^{'''}}}; 
%         \addplot[no marks, dashed, draw=green!50!black] table [x=pO2, y=IsubZr{-4},]{dat/1e3iconcmono650.dat}; 
%         \addplot[no marks, dashed, draw=green!30!black] table [x=pO2, y=IsubZr{-5},]{dat/1e3iconcmono650.dat}; 
%         \addplot[no marks] table [x=pO2, y=Stoich,]{dat/1e3iconcmono650.dat}; 
\node at (-33.7,-0.5) {\textbf{b)}};
			\end{axis}            
\end{tikzpicture}
		\caption{Monoclinic phase Brouwer diagrams of point defects at iodine concentrations of a) $10^{-5}$ and b) $10^{-3}$, at a temperature of 650 K.}
		\label{figure:tikzbrouwerconcmono}
	\end{center}
\end{figure} % 10e-3 iodine conc in mono
\end{landscape}


\begin{landscape}
\begin{figure}[ht] % 10e-5 iodine conc in tet
\begin{center}
\begin{tikzpicture}
	\begin{axis}
		[width=8.22cm, xlabel={\ch{log_{10}}($p_{O_{2}}$) (atm)}, ylabel={\ch{log_{10}}([D]) (per f.u.)}, ymin=-10, ymax=0, xmin=-35, xmax=0, legend style={{draw=}, at={(0.40,0.97)}, anchor=north west, legend columns=3, nodes={scale=0.75, transform shape}}]
        \addplot[no marks, draw=blue!70!black] table [x=pO2, y=electrons,]{dat/1e5iconctet1500.dat}; \addlegendentry{\ch{e^{'}}}; \node at (-26.0,-1.9) {\ch{e^{'}}};
        \addplot[no marks, draw=red!85!black] table [x=pO2, y=holes,]{dat/1e5iconctet1500.dat}; \addlegendentry{\ch{h^{\textperiodcentered}}}; \node at (-7,-3.6) {\ch{h^{\textperiodcentered}}};
        \addplot[no marks, draw=black!70!green] table [x=pO2, y=VO{2},]{dat/1e5iconctet1500.dat}; \addlegendentry{\ch{V_{O}^{\textperiodcentered\textperiodcentered}}}; \node at (-26.7,-3.3) {\ch{V_{O}^{\textperiodcentered\textperiodcentered}}};
%         \addplot[no marks, draw=black!55!green] table [x=pO2, y=VO{1},]{dat/1e5iconctet1500.dat}; \addlegendentry{\ch{V_{O}^{\textperiodcentered}}};
%         \addplot[no marks, draw=black!30!green] table [x=pO2, y=VO{0},]{dat/1e5iconctet1500.dat}; \addlegendentry{\ch{V_{O}^{x}}};
        \addplot[no marks, draw=yellow!85!blue] table [x=pO2, y=VM{-4},]{dat/1e5iconctet1500.dat}; \addlegendentry{\ch{V_{Zr}^{''''}}};
%         \addplot[no marks, draw=yellow!75!blue] table [x=pO2, y=VM{-3},]{dat/1e5iconctet1500.dat}; \addlegendentry{\ch{V_{Zr}^{'''}}};
%         \addplot[no marks, draw=yellow!65!blue] table [x=pO2, y=VM{-2},]{dat/1e5iconctet1500.dat}; \addlegendentry{\ch{V_{Zr}^{''}}};
%         \addplot[no marks, draw=yellow!55!blue] table [x=pO2, y=VM{-1},]{dat/1e5iconctet1500.dat}; \addlegendentry{\ch{V_{Zr}^{'}}};
%         \addplot[no marks, draw=yellow!45!blue] table [x=pO2, y=VM{0},]{dat/1e5iconctet1500.dat}; \addlegendentry{\ch{V_{Zr}^{x}}};
%         \addplot[no marks, draw=red!60!yellow] table [x=pO2, y=Oi{-2},]{dat/1e5iconctet1500.dat}; \addlegendentry{\ch{O_{i}^{''}}};
%         \addplot[no marks, draw=red!50!yellow] table [x=pO2, y=Oi{-1},]{dat/1e5iconctet1500.dat}; \addlegendentry{\ch{O_{i}^{'}}};
%         \addplot[no marks, draw=red!40!yellow] table [x=pO2, y=Oi{0},]{dat/1e5iconctet1500.dat}; \addlegendentry{\ch{O_{i}^{x}}};
%         \addplot[no marks, draw=green!80!pink] table [x=pO2, y=Mi{4},]{dat/1e5iconctet1500.dat}; \addlegendentry{\ch{Zr_{i}^{\textperiodcentered\textperiodcentered\textperiodcentered\textperiodcentered}}};
%         \addplot[no marks, draw=green!70!pink] table [x=pO2, y=Mi{3},]{dat/1e5iconctet1500.dat}; \addlegendentry{\ch{Zr_{i}^{\textperiodcentered\textperiodcentered\textperiodcentered}}};
%         \addplot[no marks, draw=green!60!pink] table [x=pO2, y=Mi{2},]{dat/1e5iconctet1500.dat}; \addlegendentry{\ch{Zr_{i}^{\textbf{\textperiodcentered\textperiodcentered}}}};
%         \addplot[no marks, draw=green!50!pink] table [x=pO2, y=Mi{1},]{dat/1e5iconctet1500.dat}; \addlegendentry{\ch{Zr_{i}^{\textperiodcentered}}};
%         \addplot[no marks, draw=green!40!pink] table [x=pO2, y=Mi{0},]{dat/1e5iconctet1500.dat}; \addlegendentry{\ch{Zr_{i}^{x}}};
%         \addplot[no marks, dashed, draw=red!70!black] table [x=pO2, y=Ii{0},]{dat/1e5iconctet1500.dat}; \addlegendentry{\ch{I_{i}^{x}}};
%         \addplot[no marks, dashed, draw=red!50!black] table [x=pO2, y=Ii{-1},]{dat/1e5iconctet1500.dat}; \addlegendentry{\ch{I_{i}^{'}}};
        \addplot[no marks, dashed, draw=purple!60!white] table [x=pO2, y=Ii{1},]{dat/1e5iconctet1500.dat}; \addlegendentry{\ch{I_{i}^{\textperiodcentered}}};
        \addplot[no marks, dashed, draw=blue!50!white] table [x=pO2, y=IsubO{1},]{dat/1e5iconctet1500.dat}; \addlegendentry{\ch{I_{O}^{\textperiodcentered}}};
        \addplot[no marks, dashed, draw=green!60!black] table [x=pO2, y=IsubO{2},]{dat/1e5iconctet1500.dat}; \addlegendentry{\ch{I_{O}^{\textperiodcentered\textperiodcentered}}};
        \addplot[no marks, dashed, draw=black] table [x=pO2, y=IsubO{3},]{dat/1e5iconctet1500.dat}; \addlegendentry{\ch{I_{O}^{\textperiodcentered\textperiodcentered\textperiodcentered}}};
        \addplot[no marks, dashed, draw=orange!80!black] table [x=pO2, y=IsubZr{-3},]{dat/1e5iconctet1500.dat}; \addlegendentry{\ch{I_{Zr}^{'''}}};
%         \addplot[no marks, dashed, draw=pink] table [x=pO2, y=IsubZr{-4},]{dat/1e5iconctet1500.dat}; \addlegendentry{\ch{I_{Zr}^{''''}}};
%         \addplot[no marks, dashed, draw=purple] table [x=pO2, y=IsubZr{-5},]{dat/1e5iconctet1500.dat}; \addlegendentry{\ch{I_{Zr}^{'''''}}};
%         \addplot[no marks] table [x=pO2, y=Stoich,]{dat/1e5iconctet1500.dat}; \addlegendentry{Stoich};
\node at (-33.7,-0.5) {\textbf{a)}};
			\end{axis}            
\end{tikzpicture} % 10e-5 iodine conc in tet
\begin{tikzpicture} % 10e-3 iodine conc in tet
	\begin{axis}
		[width=8.22cm, xlabel={\ch{log_{10}}($p_{O_{2}}$) (atm)}, yticklabels={}, ymin=-10, ymax=0, xmin=-35, xmax=0]
        \addplot[no marks, draw=blue!70!black] table [x=pO2, y=electrons,]{dat/1e3iconctet1500.dat}; \node at (-27,-1.7) {\ch{e^{'}}};
        \addplot[no marks, draw=red!85!black] table [x=pO2, y=holes,]{dat/1e3iconctet1500.dat}; \node at (-2.5,-2.1) {\ch{h^{\textperiodcentered}}};
        \addplot[no marks, draw=black!70!green] table [x=pO2, y=VO{2},]{dat/1e3iconctet1500.dat}; 
%         \addplot[no marks, draw=black!55!green] table [x=pO2, y=VO{1},]{dat/1e3iconctet1500.dat}; 
%         \addplot[no marks, draw=black!30!green] table [x=pO2, y=VO{0},]{dat/1e3iconctet1500.dat}; 
        \addplot[no marks, draw=yellow!85!blue] table [x=pO2, y=VM{-4},]{dat/1e3iconctet1500.dat}; 
%         \addplot[no marks, draw=yellow!75!blue] table [x=pO2, y=VM{-3},]{dat/1e3iconctet1500.dat}; 
%         \addplot[no marks, draw=yellow!65!blue] table [x=pO2, y=VM{-2},]{dat/1e3iconctet1500.dat}; 
%         \addplot[no marks, draw=yellow!55!blue] table [x=pO2, y=VM{-1},]{dat/1e3iconctet1500.dat}; 
%         \addplot[no marks, draw=yellow!45!blue] table [x=pO2, y=VM{0},]{dat/1e3iconctet1500.dat}; 
%         \addplot[no marks, draw=red!60!yellow] table [x=pO2, y=Oi{-2},]{dat/1e3iconctet1500.dat}; 
%         \addplot[no marks, draw=red!50!yellow] table [x=pO2, y=Oi{-1},]{dat/1e3iconctet1500.dat}; 
%         \addplot[no marks, draw=red!40!yellow] table [x=pO2, y=Oi{0},]{dat/1e3iconctet1500.dat}; 
%         \addplot[no marks, draw=green!80!pink] table [x=pO2, y=Mi{4},]{dat/1e3iconctet1500.dat}; 
%         \addplot[no marks, draw=green!70!pink] table [x=pO2, y=Mi{3},]{dat/1e3iconctet1500.dat}; 
%         \addplot[no marks, draw=green!60!pink] table [x=pO2, y=Mi{2},]{dat/1e3iconctet1500.dat}; 
%         \addplot[no marks, draw=green!50!pink] table [x=pO2, y=Mi{1},]{dat/1e3iconctet1500.dat}; 
%         \addplot[no marks, draw=green!40!pink] table [x=pO2, y=Mi{0},]{dat/1e3iconctet1500.dat}; 
%         \addplot[no marks, dashed, draw=red!70!black] table [x=pO2, y=Ii{0},]{dat/1e3iconctet1500.dat}; 
%         \addplot[no marks, dashed, draw=red!50!black] table [x=pO2, y=Ii{-1},]{dat/1e3iconctet1500.dat}; 
        \addplot[no marks, dashed, draw=purple!60!white] table [x=pO2, y=Ii{1},]{dat/1e3iconctet1500.dat}; 
        \addplot[no marks, dashed, draw=blue!50!white] table [x=pO2, y=IsubO{1},]{dat/1e3iconctet1500.dat}; \node at (-11,-2.6) {\ch{I_{O}^{\textperiodcentered}}};
        \addplot[no marks, dashed, draw=green!60!black] table [x=pO2, y=IsubO{2},]{dat/1e3iconctet1500.dat}; 
        \addplot[no marks, dashed, draw=black] table [x=pO2, y=IsubO{3},]{dat/1e3iconctet1500.dat}; 
        \addplot[no marks, dashed, draw=orange!80!black] table [x=pO2, y=IsubZr{-3},]{dat/1e3iconctet1500.dat}; 
%         \addplot[no marks, dashed, draw=pink] table [x=pO2, y=IsubZr{-4},]{dat/1e3iconctet1500.dat}; 
%         \addplot[no marks, dashed, draw=purple] table [x=pO2, y=IsubZr{-5},]{dat/1e3iconctet1500.dat}; 
%         \addplot[no marks] table [x=pO2, y=Stoich,]{dat/1e3iconctet1500.dat}; 
\node at (-33.7,-0.5) {\textbf{b)}};
			\end{axis}            
\end{tikzpicture} % 10e-3 iodine conc in tet
		\caption{Tetragonal phase Brouwer diagrams of point defects at iodine concentrations of a) $10^{-5}$ and b) $10^{-3}$, at a temperature of 1500 K.}
		\label{figure:tikzbrouwerconctet}
	\end{center}
\end{figure}
\end{landscape}

\subsection{Brouwer Diagrams} \label{Formation}

\subsubsection*{Monoclinic Phase}

Brouwer diagrams associated with the monoclinic phase, at 650 K, at which temperature this \zirconia\ phase is stable, are shown in Figure \ref{figure:tikzbrouwerconcmono}. At 650 K, this phase exhibits a relatively low concentration of intrinsic defects; concentrations of \ch{V_{O}^{**}} and \ch{V_{Zr}^{''''}} remained below $10^{-10}$ parts/fu across the majority of oxygen pressures at both iodine concentrations and do not appear in the diagrams. At lower iodine concentrations, the intrinsic electronic defects, \ch{e^{'}} and \ch{h^{*}}, were more significant, with \ch{h^{*}} defects being a major fraction of the total defect population near stoichiometry (i.e. at an oxygen pressure of approximately $10^{-7.5}$ atm). 

Between oxygen pressures of $10^{-35}$ and $10^{-10}$ atm, the dominant defects were \ch{I_{O}^{*}} charge-compensated by \ch{I_{Zr}^{'''}}. Above an oxygen pressure of $10^{-10}$ atm, a combination of \ch{I_{i}^{*}}, \ch{I_{Zr}^{'''}} and \ch{I_{O}^{***}} defects were dominant. This demonstrates that iodine will adopt a +1 oxidation state in order to facilitate iodine incorporation into the lattice. The effective ionic radius of I$^{-}$ is 2.20 \r{A} in VI-fold coordination, compared to 1.38 \r{A} for O$^{2-}$ in IV-fold coordination, as is the case in \zirconia\ \cite{Shannon1976}. Iodine with a higher positive charge state will have a smaller ionic radius, and thus impose less strain on the lattice (and therefore a smaller energy penalty) in each defect configuration, including substitution on a Zr site. At the highest oxygen pressures, the Brouwer diagrams show that oxidation of iodine, substituted at an oxygen site, to the +1 oxidation state (i.e. \ch{I_{O}^{***}}) becomes a necessary charge compensating defect. This is because the energy penalty to form hole defects in this broad band insulator is too great, as is the formation of other positive charge defects such as \ch{Zr_{i}^{****}}. This may translate to iodine out-competing oxygen for oxygen sites in monoclinic \zirconia , with higher oxygen pressures providing very little in terms of a barrier effect. 

% \begin{itemize}
% %\item Figure 1 shows monoclinic Brouwer diagrams generated at different assumed iodine concentrations.
% %\item The monoclinic Brouwer diagrams were generated at a temperature of 650 K. This is because the structure is stable at this temperature, and it is representative of the temperature which the \zirconia\ layer on the internal surface of the cladding would experience.
% %\item At an iodine concentration of 1e-5 and low oxygen pressures, the dominant defects were iodine -1 substitutional defects on the oxygen site, charged compensated by iodine +1 substitutional defects on the zirconium site. At higher oxygen pressures, the zirconium substitutional defects remained but were now charge compensated by iodine +1 substitutional defects on the oxygen site. At very high oxygen pressures, hole defects were preferred to oxygen substitutional defects.

% \item At an iodine concentration of 1e-3, intrinsic defects were found to be negligible compared to the extrinsic iodine defects. The same pattern was seen as with an iodine concentration of 1e-5, except hole defects were no longer significant at high oxygen pressures. Very low concentrations of iodine +1 interstitial defects began to appear at oxygen pressures greater than 1e-20.
% \end{itemize}
% The monoclinic Brouwer diagram (Figure \ref{figure:monoBrouwer}) predicts that at 635 K, few types of defects will be present and at very low (\textless 10 ppb \zirconia ) concentrations. This is typical of defect behaviour in a ceramics at temperatures far below their melting points \cite{kingery1997physical,ball2006computer}. Fully-charged zirconium vacancies, charge-compensated by holes, are the major defect type we expect to observe at $p_{O_{2}}$ \textgreater $10^{-15}$. Below this, only electronic defects compensated by electron hole defects are expected. We briefly see increased concentrations of uncharged oxygen interstitial defects at very high levels of $p_{O_{2}}$.

\subsubsection*{Tetragonal Phase}

Brouwer diagrams for the tetragonal phase are shown in Figure \ref{figure:tikzbrouwerconctet}. As these diagrams were generated at a temperature of 1500 K (at which the tetragonal phase becomes stable), intrinsic defect concentrations were significantly higher than in the monoclinic diagrams for all oxygen pressures (though trends remained the same). Intrinsic defects \ch{e^{'}}, \ch{h^{*}}, \ch{V_{O}^{**}} and \ch{V_{Zr}^{''''}} were dominant across most oxygen pressures at an iodine concentration of $10^{-5}$ parts/fu. Only around stoichiometry do extrinsic defect concentrations approach intrinsic values (which as mentioned earlier is why this concentration of iodine was chosen). Across all oxygen pressures, \ch{I_{O}^{*}} and \ch{I_{Zr}^{'''}} are the major iodine defects. Between $10^{-15}$ and $10^{-5}$ atm, Figure \ref{figure:tikzbrouwerconctet} illustrates that the major iodine defect swaps from being \ch{I_{O}^{*}} to \ch{I_{Zr}^{'''}}. 

When the iodine concentration was increased to $10^{-3}$ parts/fu, a significant change in defect equilibria was predicted. The oxygen pressure at stoichiometry increased from $10^{-10}$ to $10^{-6.5}$ atm (for monoclinic \zirconia, it remained at $10^{-7.5}$ atm regardless of iodine concentration). Nevertheless, \ch{I_{O}^{*}} and \ch{I_{Zr}^{'''}} remain the dominant defect pair between oxygen pressures of $10^{-15}$ and $10^{-5}$ atm (as they are at the lower iodine concentration). However, \ch{I_{O}^{*}} and \ch{I_{Zr}^{'''}} became higher concentration defects than both intrinsic \ch{V_{O}^{**}} and \ch{V_{Zr}^{''''}} defects. We also observe that Zr vacancies no longer serve as the main negative charge-compensation defect near stoichiometry, leaving \ch{I_{Zr}^{'''}} as the most energetically favourable negatively-charged defect.  

Unlike in the Brouwer diagrams for the monoclinic phase, for the tetragonal phase, the concentration of iodine substitutional defects on oxygen sites decreases more steeply at high oxygen pressures, peaking near stoichiometry. \ch{I_{O}^{***}} in particular, which was the dominant defect at high oxygen pressures in monoclinic \zirconia , becomes insignificant under the same conditions in the tetragonal phase, with iodine confined to Zr sites. This behaviour is indicative of a `barrier' effect against iodine at high oxygen partial pressures, with oxygen out-competing iodine for oxygen sites. Given that the inner oxide is likely to have a higher tetragonal phase fraction than the external oxide, due to the incorporation of fission products, this result could help to explain why there appears to be an oxygen effect on PCI-related SCC of zirconium alloys \cite{hofmann1984stress}. 

Another effect considered was the space charge of the system. Electrons have a higher rate of diffusion than oxygen vacancies in \zirconia , leading to a build-up of oxygen vacancies near the metal-oxide interface as corrosion progresses \cite{bojinov2010influence}. This results in an overall positive charge (since the dominant oxygen vacancy is \ch{V_{O}^{**}}) referred to as a space charge. When included in our Brouwer diagrams, this space charge had a negligible effect on the concentration or charge state of iodine up to a charge of $10^{-1}$ holes per f.u. \zirconia . This corresponds to a high concentration of oxygen vacancies relative to the equilibrium concentration, predicting that a significant deviation from equilibrium is not expected near the metal oxide interface as a result of a positive space charge.


\section{Summary}

Iodine exhibits lower incorporation energies when occupying defects in monoclinic \zirconia\ than in the tetragonal phase. However, as monoclinic is the low-temperature phase, intrinsic defect concentrations will also be low, thereby requiring additional energy input to produce vacancies when the concentration of iodine is much larger than that of the intrinsic defects. This leads to relatively large concentrations of iodine interstitial defects predicted in the monoclinic Brouwer diagrams, as interstitial sites are always available in the lattice. 

Defects involving iodine in the +1 oxidation state are present in significant concentrations, especially in monoclinic \zirconia , indicating that filling of the $p$ electronic sub-shell is not always energetically favourable compared to forming the smaller iodine ionic radius developed through oxidation. 

The competition between iodine and oxygen for anion sites in \zirconia\ is phase and oxygen pressure dependent. At high oxygen pressures in monoclinic \zirconia , iodine in the +1 oxidation state is predicted to occupy oxygen sites and remains the dominant defect. In tetragonal \zirconia\ at high oxygen pressures, however, the concentration of iodine defects on anion sites decreases steeply, indicating a preference for iodine accommodated at zirconium cation sites. This is indicative of a barrier effect in the tetragonal phase with oxygen out-competing iodine for anion sites.
