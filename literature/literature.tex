\chapter{Literature Review}

\label{literature_review}

\section{PCI}

When a fuel pin is first in-pile, there is a gap between the fuel pellet and the cladding (see § \ref{ss_fuelpin}). This gap slowly closes over time due to creep-down of the cladding due to coolant pressure, as well as thermal expansion and swelling of the fuel pellet (illustrated in Figure \ref{figure:pcmi}) due to radiation damage and the accumulation of fission products. At a high enough burnup, the fuel pellet makes contact with the cladding. This phenomenon is PCI and involves both mechanical and chemical interactions which contribute to observed fuel failures \cite{bcoxpelletclad1990}.

\begin{figure}[ht] % PCMI bambooing
\centering
\includegraphics[width=13cm]{images/pcmi.png}
\caption[Illustration of fuel swelling and clad deformation due to PCI. \textbf{a)} Fresh fuel before irradiation. \textbf{b)} Thermal expansion and swelling of fuel pellets and closing of fuel-pellet gap during operation. \textbf{c)} Fuel contact with cladding and subsequent `bambooing' of the cladding.]{Illustration of fuel swelling and clad deformation due to PCI. \textbf{a)} Fresh fuel before irradiation. \textbf{b)} Thermal expansion and swelling of fuel pellets and closing of fuel-pellet gap during operation. \textbf{c)} Fuel contact with cladding and subsequent `bambooing' of the cladding. Adapted from \cite{alam2011review}.}
\label{figure:pcmi}
\end{figure}

PCI-related failure of nuclear fuel pins has been known since 1963, when the first reported failure occurred in a highly rated fuel pin during reactor start-up \cite{lyons1963high}. Many studies have since been made regarding the topic \cite{alam2011review, bcoxpelletclad1990}. The exact cause of PCI failures has not yet been determined, however it is likely that both the mechanical and chemical effects of contact between the fuel pellet and the cladding are necessary for it to occur. It is also known that PCI failures are typically preceded by power transients, such as during reactor startup where several power ramps are performed over many hours \cite{bcoxpelletclad1990}.

\subsection{Fuel pellet relocation}

Irradiated fuel pellets will sometimes crack and break into fragments whilst in the cladding. These fragments are unconstrained and are able to move radially outwards (i.e. towards the pellet-cladding gap). This phenomenon is called `relocation', although this term is also used to refer to movement of the pellet fragments in the axial direction within the cladding \cite{sheppard1982data}. Relocation of fuel fragments means that firm contact between the fuel and cladding can occur earlier in the fuel pin's life (i.e. before fuel swelling closes the fuel-cladding gap).

The conceptual model of fuel pellet relocation is shown in Figure \ref{figure:relocation} as a plot of the `effective' gap (based on cladding temperature and fuel centreline temperature) against rod power. Initially, the fuel pellet has no fragmentation and a wide fuel-cladding gap (region I). The gap decreases slowly due to thermal expansion as rod power is increased until point A where the fuel pellet cracks and fragments. The fuel pellet fragments move radially outward and the gap decreases significantly (region II). Point B marks the onset of `soft' PCI, meaning that the fuel pellet fragments make some contact with the cladding, but are not yet fully constrained (e.g. radial or azimuthal motion is still possible). The fragments' lack of mechanical constraints prevent a stress from being imparted to the cladding, thus the mechanical interaction is considered `soft' (region III). At point C, thermal expansion due to rod power is large enough that some pellet fragments become mechanically constrained, marking the onset of `hard' PCI (region IV). As discussed in § \ref{ss_fuelpin}, chipping of fuel pellets can cause debris to occupy the pellet-cladding gap, thereby reducing the rod power at which onset of hard PCI occurs.

\begin{figure}[ht] % Relocation
\centering
\includegraphics[width=15cm]{images/relocation.png}
\caption[Concept of pellet-cladding gap model showing stages of fuel pellet fragment relocation with onset of soft and hard PCI. ]{Concept of pellet-cladding gap model showing stages of fuel pellet fragment relocation with onset of soft and hard PCI. Adapted from \cite{Oguma1983} with fuel pin cross-sections from \cite{walton1983fuel}.}
\label{figure:relocation}
\end{figure}

\subsection{Pellet-clad gap and bonding}

The gap between the fuel and the cladding allows fuel pellets to be inserted into the fuel pin easily during manufacture, but this clearance is also designed to accommodate some increase in fuel pellet volume. It is important to consider the thermal expansion of the fuel pellet, as the centreline temperature of a PWR fuel pellet during a power transient can vary between 1500 and 1800 $^{\circ}$C, depending on the burnup of the fuel, magnitude of the reactivity insertion and location within the core \cite{Bagger1994}. In addition, fuel pellets will swell due to radiation damage throughout their operational lifetime. Once the pellet-clad gap has closed entirely, any pellet expansion during a power transient will translate to a force exerted on the cladding, generating hoop stresses which open cracks on the inner cladding surface. This is the mechanical component of PCI.

In LWRs, the pellet-cladding gap closes completely at high burnups of approximately 50 GWd/tU \cite{Nogita1997, Suzuki2006, Rudling2008}. When the surface of the fuel pellet makes contact with the cladding, there is also a chemical interaction between the UO$_{2}$ (and fission products) and the internal surface oxide of the cladding. UO$_{2}$ has some solid solubility in ZrO$_{2}$, and a bonded reaction layer, which has a chemical composition (U, Zr)O$_{2}$ is observed. Due to the large U atom, cation substitution allows some of this mixed layer to adopt the crystal structure of cubic fluorite, the high temperature phase of ZrO$_{2}$. The uranium nuclei in both this bonding layer and the outer rim of the fuel pellet experience the highest number of fission events (due to proximity to the moderator), and therefore this region contains fission products that contribute to the chemical degradation of the fuel cladding, in particular iodine. This is the chemical component of PCI.

\subsection{Reactor power ramps}

It has been established that power ramping of a reactor is associated with PCI failures \cite{penn1977candu, MacDonald1979, Hardy1977198, Knaab1987}. This presents a problem for reactors when it comes to events such as start-up, load-following and any other power transients experienced by the fuel pins. Figure \ref{figure:reactor_startup} shows how reactor power varies over time during a typical PWR start-up procedure. A combination of low ramping rates and long holds at low power (to remain below ramping limits, to condition\footnote{Fuel is considered `conditioned' after it has operated at a specified power level for a certain period of time.} fuel \cite{billaux2005pellet}, and conduct coolant chemistry checks) require the entire procedure to be completed in a period of 90 hours, with several operator switch-overs in between. This is a costly procedure for the utility owner to perform, with millions of US\$ foregone in electricity sales for larger reactors.

\begin{figure}[ht]
\centering
\includegraphics[height=10cm]{images/reactor_startup.png}
\caption[Reactor startup procedure for a typical US PWR. Dashed line indicates \% withdrawal of control rods.]{Reactor startup procedure for a typical US PWR. Dashed line indicates \% withdrawal of control rods. Adapted from \cite{ramping}.}
% M. F. James, R. W. Mills and D. R. Weaver (1991) UKAEA Reports, AEA-TRS-1015, AEA-TRS-1018 
         %  and AEA-TRS-1019.
\label{figure:reactor_startup}
\end{figure}

Scheduled reactor shutdowns or extended reduced power operation (ERPO) events occur whenever refuelling or maintenance of the reactor is required (and in some cases there may be a grid demand to reduce power output).  Refuelling is typically performed every 1 to 2 years, while unscheduled maintenance may be required at any time. Emergency shutdowns may also occur and have their own challenges to consider (e.g. xenon poisoning, decay heat removal), though they are rarer. In each case, it is necessary to go through the lengthy power ramping procedure, and since these shutdowns cannot be avoided, being able to ramp up power faster would be a significant improvement. Ramp rates in reactors are restricted to between 3-4\% of full power per hour above a certain threshold level to avoid PCI failures \cite{ramping}. Additionally, fuel conditioning holds (operation at certain power levels for long periods) are performed to further reduce the incidence of PCI failures. 

The limits described above present a challenge not just when restarting reactors, but also for the implementation of load-following in reactors. PWRs have thermal feedback loops which provide some level of intrinsic load-following behaviour. For example, an increase in steam demand leads to increased boiling in the steam generators. The subsequent decrease in coolant temperature in the primary circuit leaving the steam generator causes a reactivity increase and therefore a power increase in the reactor. The reactor returns to critical (reactivity = 0) after some fluctuation, and the average temperature of the primary coolant remains unchanged.

\subsection{PCI Failures}

PCI failures are typically associated with cracks which span the thickness of the cladding (so-called \emph{pinhole} defects), and lead to fission product contamination of the primary loop coolant. A fuel cladding breach is detected when a sharp spike in radioactivity is registered by sensors in the primary loop (i.e. a signal that is distinguishable above the previous background). This means that the time to detection of a failure from the failure-inducing event (e.g. a power ramp) is known, but the time taken for the PCI failure to occur is unknown \cite{bcoxpelletclad1990}. 

Figure \ref{figure:fuelcladdingcrack} shows the typical cracking behaviour of fuel pellets and stress-corrosion cracking (SCC) of the Zr cladding from a PWR fuel pin subjected to two annual operating cycles. Fragmentation of the fuel pellet can be seen and the absence of a pellet-cladding gap caused by swelling of the pellet. The axial cross-section in Figure \ref{figure:fuelcladdingcrack}b shows the gap at the chamfers almost completely closed (axial expansion of the pellet). The pellet dishes are also visible as dark horizontal discs. Note that these samples have been cooled down and so the swelling seen does not include the thermal expansion effect. 

\begin{figure}[ht] % abc Fuel cladding crack
\centering
\includegraphics[width=\linewidth]{images/fuelcladdingcrack.png}
\caption[Cracking of PWR fuel pellets and cladding. \textbf{a)} Transversal macrography of a fuel rod irradiated for two annual PWR operating cycles. \textbf{b)} Axial macrography of a fuel rod irradiated for two annual PWR operating cycles. \textbf{c)} SCC cladding failure.]{Cracking of PWR fuel pellets and cladding. \textbf{a)} Transversal macrography of a fuel rod irradiated for two annual PWR operating cycles. \textbf{b)} Axial macrography of a fuel rod irradiated for two annual PWR operating cycles. \textbf{c)} SCC cladding crack. Adapted from \cite{brochard2001modelling}.}
\label{figure:fuelcladdingcrack}
\end{figure}

Figure \ref{figure:fuelcladdingcrack}c shows an SCC crack which has not progressed through the entire thickness of the cladding and exhibits intergranular morphology. The crack initiation site is also aligned with a radial crack in the fuel pellet. There is much evidence in post-irradiation examinations of fuel pins that I-SCC of the cladding will occur ahead of these radial pellet cracks and at the ends of the pellet \cite{Peehs2009, Haddad1983, haynes2015modelling, Wood1975, Nobrega1985, Syrett1981, Shimada1983}.

\subsubsection{Cladding Zr liners}

Claddings in BWRs are more prone to PCI failures, and therefore differ from claddings in PWRs by including an additional layer of Zr metal up to 0.1 mm thick, called a \emph{liner}, bonded to the inner surface of the cladding \cite{Kitano2006, Takagi1996}. The liner material is softer than the the Zr alloy used in the rest of the cladding for the purpose of suppressing cracks. The introduction of a liner in BWR fuel pins has been successful in reducing the incidence of PCI failures \cite{andersson2003fuel, klinger2003experience}, and this has prompted interest in introducing liners to PWR claddings \cite{groeschel2003failure}. Zr liners are still susceptible to SCC however, and their function is to alleviate the mechanical component of PCI rather than the chemical component. Figure \ref{figure:linercracks}a shows incipient cracks in a Zr liner after a ramp test in a BWR. The cracks are up to 10 $\mu$m in length, spanning the thickness of the oxide layer and stopping near the metal/oxide interface. Crack arrest in this case was attributed solely to the softness of metal, although this same study showed that the hardness of the metal measured in this region was almost double that of the rest of the liner (due to radiation hardening). It should be noted that the metal near the metal/oxide interface will contain up to 29 at.\% oxygen and this has implications for the corrosion process (see § \ref{oxygensolubility}).

\begin{figure}[ht] % Liner cracks
\centering
\includegraphics[width=14.5cm]{images/linercracks.png}
\caption[Optical micrographs of incipient cracks in BWR cladding liners at a burnup of 30 MWd/kg U following \textbf{a)} a ramp test with a peak LHR of 57 kW/m and \textbf{b)} subsequent irradiation for 422 hours with a peak LHR of 47 kW/m. Black arrow indicates oxidised crack.]{Optical micrographs of incipient cracks in BWR cladding liners at a burnup of 30 MWd/kg U following \textbf{a)} a ramp test with a peak LHR of 57 kW/m and \textbf{b)} subsequent irradiation for 422 hours with a peak LHR of 47 kW/m. Black arrow indicates oxidised crack. Adapted from \cite{Kitano2006}.}
\label{figure:linercracks}
\end{figure}

Figure \ref{figure:linercracks}b shows an incipient crack in a BWR cladding liner which has been oxidised in a long (422 hour) irradiation test. This shows that repassivation of the cladding can occur during operation, preventing further attack at a crack site by iodine and other fission products. Additionally, a bonding layer between the pellet and cladding liner several microns thick is observed. In this case, the crack in the liner does not follow a crack in the fuel pellet as expected (see Figure \ref{figure:fuelcladdingcrack}), but this particular sample did not fail so it is not clear if this type of crack would eventually lead to a PCI failure. 

\subsubsection{Failure thresholds}

The propensity for a fuel pin to fail due to PCI is a function of both burnup and ramp rate. Many ramp rate studies have therefore been conducted to establish failure thresholds \cite{Thomas1979, mogard1980studsvik, franklin1985performance, mogard1985international, djurle1984super, howe1991ramp, baba1983power, suzuki1994burnup, wesley1994mark, djurle1983studsvik, hollowell1982international}. Figure \ref{figure:BWRrampthreshold} shows one such ramp test on BWR fuel at a burnup of approximately 30 MWd/kg U. The ramps, measured as the change in the linear power rating (also known as linear heat rating or simply linear rating) with units kW/m, lead to cladding failures after a certain duration when above a threshold level. In the case of the BWR fuel in Figure \ref{figure:BWRrampthreshold}, this threshold linear rating is around 41 kW/m. As linear rating increases from this level, the time to failure falls significantly, leading to failure within several minutes. A time delay between cladding failure and subsequent detection of fission products in the coolant is also shown (in this case, up to ten times as long as the failure time). This time delay and its implications are discussed in more detail in Chapter \ref{ch:results3}. Of the fuel claddings which did not fail near the failure boundary, SCC damage was present on the internal surface. 

\begin{figure}[ht] % BWR ramp threshold failure
\centering
\includegraphics[width=14.5cm]{images/BWRrampthreshold.png}
\caption[TRANS-RAMP I PCI failure progression for BWR fuel rods at a burnup of 30 MWd/kg U.]{TRANS-RAMP I PCI failure progression for BWR fuel rods at a burnup of 30 MWd/kg U. Taken from \cite{Mogard1988}}
\label{figure:BWRrampthreshold}
\end{figure}

SCC damage has also been observed to occur with a threshold effect based on fuel burnup \cite{Wood1974}. Figure \ref{figure:SCCthreshold} shows the results of I-SCC tests on irradiated Zircaloy cladding C-rings. A failure threshold exists at a burnup of approximately 24 MWh/kg U (i.e. 1 MWd/kg U). All samples were loaded and held for 25 hours, and the failure fraction was plotted against burnup. Yield strengths of samples from the same irradiated claddings are also provided and clearly show a radiation hardening effect, increasing yield strength of the samples from 470 MPa at a burnup of 1 MWd/kg U, to 730 MPa at approximately 20 MWd/kg U. This is accompanied by a large decrease in both ductility and ability to resist crack propagation. C-ring samples have failure rates approaching 100\% at these higher burnups \cite{bcoxpelletclad1990}.

\begin{figure}[ht] % SCC initiation threshold
\centering
\includegraphics[width=14.5cm]{images/SCCthreshold.png}
\caption[I-SCC susceptibility of stress relieved Zircaloy fuel cladding at 573 K as a function of burnup and cladding yield strengths at 553 K. Numbers next to data points indicate number of rings from the sample used in yield data.]{I-SCC susceptibility of stress relieved Zircaloy fuel cladding at 573 K as a function of burnup and cladding yield strengths at 553 K. Numbers next to data points indicate number of rings from the sample used in yield data. Figure from \cite{bcoxpelletclad1990} with yield strength data from \cite{Bement1964}.}
\label{figure:SCCthreshold}
\end{figure}

\subsection{Fission products and SCC}

Although PCI failures were found to occur during power ramping, it was not yet known whether these failures were due to fission product induced SCC or tensile failure of the cladding due to radiation embrittlement. 

In 1971, a series of tests were carried out at Chalk River Nuclear Laboratories to determine if fission products were necessary for PCI failures to occur \cite{MacDonald1979}. The experiments involved taking highly irradiated zirconium fuel pins (fluence of $8 \times 10^{24}$ n/m$^{2}$ with 1 MeV neutrons) and then inserting fresh, unirradiated UO$_{2}$ fuel pellets into them. These fuel pins were then inserted back into a reactor and subjected to large power ramps with peak linear power ratings of up to 77 kW/m, as shown in Figure \ref{figure:fueltests}. These ramps (phase I and II) would typically cause failures in fuel pins with similar irradiation histories (see Figures \ref{figure:BWRrampthreshold} and \ref{figure:SCCthreshold}). In the initial ramp tests however, all the fuel pins survived the ramps intact. Six fuel pins were then irradiated in the reactor at low power to a burnup in excess of 50 MWh/kg U (i.e 2.1 MWd/kg U) in order to build up fission products in the fuel (phase III). 

In a subsequent ramp test (phase IV), two of the high burnup fuel pins failed in the reactor. This finding provided the strongest evidence to date that fission products are necessary for PCI failure of zirconium-based claddings. The fission product most likely to cause cracking, based on known SCC susceptibility of Zr metal, is iodine.

\begin{landscape} % Chalk river fresh fuel old cladding
\begin{figure}[ht]
\centering
\includegraphics[width=\linewidth]{images/fueltests.png}
\caption[Power ramping test data for the different phases of the Chalk River Nuclear Lab experiment.]{Power ramping test data for the different phases of the Chalk River Nuclear Lab experiment. Taken from \cite{MacDonald1979}.}
\label{figure:fueltests}
\end{figure}
\end{landscape}

As discussed earlier, iodine is one of the most prevalent fission products and it is known to corrode zirconium metal \cite{iodinezrmetal, Sidky1998, rosenbaum1966interaction, Fregonese1998, Lewis2011, anghel2010experimental}. The exact mechanism by which this occurs in fuel pins is not yet known, though a combination of radiolysis, I$_{2}$ diffusion and chemical attack (I-SCC) are considered to be most likely. 

The most commonly proposed mechanism is illustrated in Figure \ref{figure:vanarkel}. In the first step, iodine and caesium are produced through fission of the fuel and diffuse towards the outer surface of the fuel pellet (though not necessarily together \cite{Grimes1992}). A thin film of CsI is deposited on the outer surface of the fuel pellet and subsequently decomposes via radiolysis, liberating iodine in vapour form. The iodine vapour then diffuses towards a crack site in the cladding and reacts with Zr to produce ZrI$_{4}$. The ZrI$_{4}$ then breaks away from the metal due to the high surface energies, causing pitting and progressing the crack tip further into the metal. This model however, fails to consider the effect of the oxide on the internal layer of the cladding (as a barrier layer to ingress of corrosive species), and it does not take into account the presence of oxygen in the pellet-cladding gap and in the metal matrix (in solution) near the metal-oxide interface. Both these factors are important as the oxide provides a protective effect (otherwise I-SCC failures would be far more prevalent, regardless of ramp rate limitations and conditioning holds) and where the oxide has been ruptured, repassivation and competition between iodine and oxygen will occur. The effect of this internal oxide layer is one of the gaps in knowledge that is addressed in this thesis.

\begin{figure}[ht] % Van Arkel
\centering
\includegraphics[width=\linewidth]{images/vanarkel.png}
\caption[Schematic of a proposed I-SCC process for cladding crack penetration.]{Schematic of a proposed I-SCC process for cladding crack penetration. Taken from \cite{Lewis2011}.}
\label{figure:vanarkel}
\end{figure}

\subsection{Iodine availability}

The amount of iodine available in the fuel pin is dependent on many different factors (e.g. temperature, pressure, power history, axial location in fuel pin, diffusion rate through fuel), making it difficult to measure. Additionally, iodine nuclei (and fission products in general) will implant in either the fuel pellet or the cladding due to their large kinetic energies following a fission event. To get an idea of the quantity of iodine in a reactor, a calculation was performed to obtain a lower-bound for the mass of radioactive iodine present in a 1000 MWe reactor core at steady-state operation (see Appendix \ref{ss_i_conc}), yielding a value of 203.84 grams. The distribution of iodine, however, is not known. It is necessary to consider things such as decay of precursors, radiolysis and fission product implantation, described in more detail below.

\subsubsection{Iodine precursors}

Lewis \emph{et al}. \cite{Lewis2017} showed that the iodine inventory will increase sharply during both startup and shutdown, and attributed the iodine spike during shutdown solely to cracking of the fuel pellet (and subsequent release of trapped iodine). However, this ignores the contribution of iodine precursors (tellurium isotopes) to the iodine inventory. For example, \ch{I^{131}} concentration is shown to increase following shutdown. Production of \ch{I^{131}} is dominated by $\beta$- decay of \ch{Te^{131}} (half-life 25 minutes, see Figure \ref{table:decaydata_chap1}). Immediately following shutdown from steady state operation, the rate of \ch{I^{131}} production from \ch{Te^{131}} decay remains the same (because there is still \ch{Te^{131}} in the fuel), but `burning' of \ch{I^{131}} by neutron capture falls to zero. The thermal neutron absorption cross-section of \ch{I^{131}}, shown in Figure \ref{figure:te_i_xsection}, is significant, being of comparable magnitude to the thermal fission cross-section of \ch{U^{235}}. With the absorption reduced to zero, \ch{I^{131}} can only be removed by nuclear decay. Decay of \ch{I^{131}} (half-life 8.023 days) is much slower than \ch{Te^{131}}, and therefore the quantity of \ch{I^{131}} will spike immediately after shutdown. All isotopes of tellurium in the mass range 131 to 138 have high fission yields (often higher than iodine) and short half lives, contributing significantly to the iodine inventory in the fuel. This effect has largely been overlooked in the literature.

\begin{figure}[ht] % I, U cross-sections
\centering
\includegraphics[width=14cm]{images/te_i_xsection.png}
\caption[Neutron absorption and fission spectra of \ch{I^{131}} and \ch{U^{235}}, respectively.]{Neutron absorption and fission spectra of \ch{I^{131}} and \ch{U^{235}}, respectively.}
\label{figure:te_i_xsection}
\end{figure}

\subsubsection{Radiolysis}

Thermodynamic calculations performed by Konashi \emph{et al}. \cite{Konashi1983} estimate the equilibrium partial pressure of iodine in the fuel-cladding gap to be as low as 10$^{-17}$ atm when there is no radiolysis of CsI, and up to 10$^{-8}$ atm with the effect of CsI radiolysis included. For comparison, mandrel tests of irradiated Zr claddings at 350 $^{\circ}$C show that susceptibility to I-SCC is reduced when the iodine partial pressure is below 60 Pa (approximately 6$\times 10^{-4}$ atm) \cite{anghel2010experimental}. While these calculated values of the iodine pressure are too low to induce corrosion in Zr, they demonstrate that radiation will increase the partial pressure of iodine by several orders of magnitude. Increasing reactor power (e.g. during a ramp) will increase radiation flux and therefore dissociation of CsI, however, no figures are yet available which demonstrate a clear link between this contribution to the iodine pressure and PCI failures. 

\subsubsection{Implantation} \label{implantation}

Fission products immediately following a fission event have kinetic energies of up to 90 MeV. These are large, highly ionising particles which transfer their energy to the surrounding atoms within several microns of where they are produced, due to the large electronic and nuclear stopping power of crystalline solids such as UO$_{2}$ and ZrO$_{2}$. Computer simulations are often used to predict the distribution of ions in a material subjected to some ion fluence. Figure \ref{figure:srimtrim}a shows TRIM calculations of the amount of iodine that is implanted in ZrO$_{2}$ at different incident ion energy levels. Iodine at 6.5 MeV is predicted to penetrate ZrO$_{2}$ up to a thickness of 1.6 $\mu$m (the internal oxide thickness is typically between 5 and 10 $\mu$m). Less energetic ions will have distribution peaks nearer to the oxide surface. There is good agreement between TRIM predictions and experimental data (Figure \ref{figure:srimtrim}b) for the distribution of 50 keV iodine ions implanted in ZrO$_{2}$, with the distribution peak at 12 nm from the surface.

\begin{figure}[ht] % SRIM/TRIM values for iodine in ZrO2
\centering
\includegraphics[width=\linewidth]{images/srimtrim.png}
\caption[\textbf{a)} TRIM distributions for iodine implantation in ZrO$_{2}$ at 50 keV, 800 keV and 6.5 MeV. \textbf{b)} Rutherford backscattering spectrometry profile of 50 keV iodine implanted in ZrO$_{2}$ samples.]{\textbf{a)} TRIM distributions for iodine implantation in ZrO$_{2}$ at 50 keV, 800 keV and 6.5 MeV. \textbf{b)} Rutherford backscattering spectrometry profile of 50 keV iodine implanted in ZrO$_{2}$ samples. Taken from \cite{brossard1998use}.}
\label{figure:srimtrim}
\end{figure}

Implantation of fission products has also been studied experimentally in yttria-stabilised zirconia (YSZ) due to interest in inert matrix fuels and wasteforms for actinide wastes. At dopant concentrations between 0.4 and 1 at.\%, caesium was found to preferentially occupy cation sites (i.e. Cs substitutional defect formation), whereas iodine seemed to occupy random sites in the lattice \cite{Thome1999}. It was not clear, however, if iodine was occupying interstitial sites or forming defect clusters (or even compounds with Zr). These studies provided limited insight for pure ZrO$_{2}$ because stabilisation of the cubic phase with yttrium will significantly increase the concentration of oxygen vacancy defects, and these defects in addition to the presence of yttrium ions will influence the types of iodine and caesium defects which form in ZrO$_{2}$.

Further to the mechanisms described above, it is important to consider the role of oxygen in the I-SCC process. Fuel pins do not regularly fail during normal operation, despite iodine being produced continuously from fission of the fuel. As previously mentioned, this is because the inner surface of the cladding is not pure Zr metal, but rather a protective oxide which provides an effective barrier against corrosive species such as iodine.

\section{Oxidation of zirconium}

The oxidation of zirconium to produce \zirconia\ occurs during manufacture of the fuel cladding when the Zr metal is exposed to oxygen in air. \zirconia\ is a ceramic with material properties that make it desirable in many industrial applications, including solid-oxide fuel cells \cite{radford1979zirconia}, refractory linings \cite{whittemore1952fused}, and nuclear waste storage \cite{wang2012ceramics}. However, in the context of nuclear fuel cladding, the most important function of \zirconia\ is the barrier it provides against the ingress of corrosive species. 

\zirconia\ grown thermally on Zr metal exists mainly in either the monoclinic or tetragonal phase \cite{Howard1988,teufer1962crystal}. We can expect the internal \zirconia\ layer of the cladding to be mostly monoclinic in early life, with the stress-stabilised tetragonal phase appearing near the oxide/metal interface due to cohesive strains resulting from the lattice mismatch. With increasing burnup, it is expected that more tetragonal and possibly even the cubic phase of \zirconia\ forms due to anion vacancy formation and residual stresses in the lattice from radiation damage \cite{sickafus1999radiation}. Amorphisation due to radiation damage has also been observed in the cubic phase from Cs$^{+}$ implantation \cite{amorphization2000wang}. In this thesis however, while defect energies for the cubic phase are reported (see Figures \ref{isolated_defects}, \ref{table:bound_defects}, \ref{figure:cubicinter}), the focus is on monoclinic and tetragonal \zirconia\ phases, partly due to difficulties predicting the behaviour of the pure high-temperature cubic phase using energies calculated from a static energy technique. 

\subsection{Oxide growth mechanism}

An oxide layer will form on the surface of zirconium metal even at very low oxygen partial pressures \cite{causey2005review}. The oxidation process is mainly driven by the ingress of oxygen ions. Initially, the oxide is highly protective, growing slowly into the metal until it reaches a thickness of approximately 2-3 $\mu$m \cite{garzarolli1991oxide, dawson1968kinetics}, after which the oxide growth mechanism enters a `post-transition' stage where the oxidation kinetics follow a cubic-rate law  \cite{porte1960oxidation}. At low temperatures relative to the melting point or high pressures, and after reaching a critical thickness (called the transition point), parts of the initial oxide will fail and the oxidation rate will increase again. This process is illustrated in Figure \ref{figure:oxide_weight_gain}. 

\begin{figure}[ht]
\centering
\includegraphics[height=10.5cm]{images/zro2_oxide_weight_gain.png}
\caption[Diagrammatic representation of the cyclical oxidation behaviour of \zirconia .]{Diagrammatic representation of the cyclical oxidation behaviour of \zirconia . Taken from \cite{cox1963some}.}
\label{figure:oxide_weight_gain}
\end{figure}

\subsection{Oxygen solubility of zirconium} \label{oxygensolubility}

Considering the Zr-O binary phase diagram in Figure \ref{figure:binary_phase_diagram}, oxygen is soluble in zirconium up to 29 at.\% when below 400 $^{\circ}$C, commensurate with the operating temperature of a typical PWR (330 \textdegree C at the outer surface of the cladding). Solubility increases slightly up to 35 at.\% as temperature is increased to the liquidus line at 2065 $^{\circ}$C. This is important to note because, in the literature, studies often assume that there is pure Zr metal immediately beneath the \zirconia\ layer \cite{rossi2015first}. This assumption leads to an underestimation of the extent to which repassivation occurs when the oxide layer is breached, and disregards the effect of the thin ZrO interface that can precede the metal. The amount of oxygen required to grow more oxide near the interface will therefore be at least 37\% lower than expected when using this assumption.

The presence of oxygen in the Zr metal will also have an effect on thermodynamic calculations of extrinsic defect formation. Atoms such as Te and I will have to compete with O atoms (and potentially, self-interstitial Zr atoms) for interstitial sites in the metal. This decreases the diffusivity through the Zr matrix because of the lower availability of empty sites to facilitate diffusion. An energy input is required to remove O or Zr atoms occupying these sites, making diffusion paths involving these sites less preferable.


\begin{figure}[ht]
\centering
\includegraphics[width=14cm]{images/zro2_binary_phase.png}
\caption[Phase diagram of the Zr-O binary system.]{Phase diagram of the Zr-O binary system. Taken from \cite{Abriata1986}.}
\label{figure:binary_phase_diagram}
\end{figure}

\subsection{Outer oxide vs inner oxide} \label{section:outervsinner}

As mentioned previously, the cladding of an LWR fuel pin develops an oxide on both the inner and outer surfaces due to exposure to oxygen in air during manufacture. Both the outer and inner oxide layers provide protection against corrosion, though the corrosive environment is different. 

The outer oxide layer is in contact with the coolant, which is mostly light water with some additional dissolved species such as boron and hydrogen to control reactivity and pH. Figure \ref{figure:outer_oxide} shows a section of the cladding with the outer oxide visible. This layer is mostly monoclinic \zirconia\ with small (nano) grains of tetragonal \zirconia\ distributed randomly throughout. These grains of tetragonal \zirconia\ are autostabilised during growth of the oxide because of the large volume expansion associated with oxidation (Zr has a Pilling-Bedworth ratio of 1.57). Of course, transmission electron microscope (TEM) foils under examination are always stress-relieved, whereas the oxide in reactor conditions will be under 1-2 GPa of residual stress due to the growth of the oxide (see § \ref{section:tet_stress_stabilisation}).

\begin{figure}[ht]
\centering
\includegraphics[height=10.5cm]{images/outer_oxide.png}
\caption[Scanning transmission electron microscrope (STEM) image of the outer oxide layer formed in an autoclave under simulated PWR water conditions showing the prevalence of different \zirconia\ phases.]{Scanning transmission electron microscrope (STEM) image of the outer oxide layer formed in an autoclave under simulated PWR water conditions showing the prevalence of different \zirconia\ phases. Taken from \cite{Hu2016}.}
\label{figure:outer_oxide}
\end{figure}

The internal oxide layer is much more challenging to examine due to the need to prepare samples in hot cells. This layer is typically very brittle due to radiation damage and implantation of fission products. At a high enough burnup or LHR, the \zirconia\ layer makes contact with the UO$_{2}$ fuel, with which it will bond. Figure \ref{figure:inner_oxide} shows a section of the cladding with the inner oxide layer bonded to the pellet. The crystal structure of the \zirconia\ in this layer is debated. One study reported no monoclinic phase in high burnup fuel pins, with cubic phase \zirconia\ throughout most of the layer and an amorphous phase nearer the pellet side \cite{Nogita1997}, while other studies report mostly tetragonal phase in this layer \cite{ciszak2017etude, gibert1998influence}. 
%After removal from the reactor and the subsequent cooling period, the inner oxide

\begin{figure}[ht] % PCI bonding layer high burnup
\centering
\includegraphics[height=10.5cm]{images/pci_bondinglayer.png}
\caption[High resolution scanning electron microscrope (SEM) image of the bonding layer between a PWR UO$_{2}$ fuel pellet and Zr cladding in a fuel pin at an approximate burnup of 60 GWd/tU.]{High resolution SEM image of the bonding layer between a PWR UO$_{2}$ fuel pellet and Zr cladding in a fuel pin at an approximate burnup of 60 GWd/tU. Adapted from \cite{Lozano1998}.}
\label{figure:inner_oxide}
\end{figure}

Figure \ref{figure:bonding_layer_composition} shows the composition of the inner oxide of a high burnup fuel pin. The fission product (Ba, Mo, Nd) content is highest at the beginning of the \zirconia\ layer and decreases almost linearly with distance towards the Zr metal. This is due to fission product implantation rather than diffusion in the oxide as these elements have low volatility and low cation diffusion rates in UO$_{2}$ where they are produced \cite{S.G.PrussinD.R.OlanderP.Goubeault1984, Prussin1988}, although diffusion rates can be increased due to irradiation. % can you use the weight composition and fission yields to estimate what percentage of iodine implantation will be?

\begin{figure}[ht!] % Elemental composition of fuel pellet
\centering
\includegraphics[width=14cm]{images/bonding_layer_composition.png}
\caption[Elemental composition of the bonded UO$_{2}$/ZrO$_{2}$ layer from a PWR UO$_{2}$ fuel pellet with a burnup of 61 GWd/tU.]{Elemental composition of the bonded UO$_{2}$/ZrO$_{2}$ layer from a PWR UO$_{2}$ fuel pellet with a burnup of 61 GWd/tU. Taken from \cite{Lozano1998}.}
\label{figure:bonding_layer_composition}
\end{figure}


\subsection{Sources of oxygen}

The internal oxide layer is present before fuel claddings are pressurised with helium gas and sealed. This is from the normal oxidation of Zr in air, where the oxygen pressure is 0.21 atm. After capping of the fuel rods, the only significant source of oxygen is from the UO$_{2}$ fuel pellets.

Uranium oxides have a wide range of non-stoichiometric compositions, with U/O ratios ranging from 1.67 to 2.24 in solid UO$_{2 \pm x}$, as shown in Figure \ref{figure:U_O_phase_diagram}. The oxide form U$_{3}$O$_{8-y}$ also exists and is more kinetically and thermodynamically stable than UO$_{2}$, but has lower density, making it less suitable for use as a fuel form. The different stoichiometries have different equilibrium O$_{2}$ pressures at constant temperature, allowing some level of internal cladding environment control depending on whether more or less oxygen is desired. 

Oxygen and oxygen precursors may also be produced from fission of U$^{235}$, but this contribution is insignificant compared to changing the stoichiometry of the fuel pellet: that is, liberation of oxygen from UO$_{2 \pm x}$ due to fission (which is a function of fuel stoichiometry) is a more significant contributor to the oxygen pressure than direct production via fission.

\begin{figure}[ht!]
\centering
\includegraphics[height=12cm]{images/UO_phase_diagram.png}
\caption[Partial U-O temperature binary phase diagram between O/U ratios of 1.2 and 2.25.]{Partial U-O temperature binary phase diagram between O/U ratios of 1.2 and 2.25. Figure taken from \cite{katz2007chemistry}, with phase boundaries from \cite{rand1978thermodynamic, chevalier2002progress, gueneau2002thermodynamic}.}
\label{figure:U_O_phase_diagram}
\end{figure}

\section{Atomistic scale computer simulation}

Conducting experiments on active nuclear materials is a difficult undertaking. Handling irradiated materials is an expensive process, and materials such as uranium are highly controlled (though they are relatively benign before irradiation compared to many typical chemical laboratory materials and solvents). Furthermore, experiments which require samples to be irradiated must be left to cool-down (due to material activation) for up to a year before they can be analysed in a specialised lab \cite{efthymiopoulos2011hiradmat}. Thus, any errors in the experimental procedure or problems with samples are not revealed until months later when material analysis is performed. This makes it difficult to study many material phenomena, especially if they are time-dependent. In-situ reactor experiments are also problematic, requiring sensor equipment to be made tolerant to the high radiation environments as well as being acceptable to and consistent with the reactor operation safety case. The risks and costs mean that experimental work is mostly restricted to the largest labs and researchers with enough funding to undertake such work. For these reasons, computer simulation methods are a preferable alternative for nuclear research and can also serve to guide experimental work.

\subsection{Classical approach - molecular dynamics}

Molecular dynamics (MD) uses classical mechanics as the basis for calculations \cite{Andersen1980}. These types of simulations typically use pair potentials (though many-bodied potentials are also used). These are mathematical functions which effectively describe the energy of interaction between two particles. Pair potentials are created by fitting functions to several parameters from empirical data, such as equilibrium bond lengths, thermal properties or even values from quantum mechanical calculations. The simple form of pair potentials allows MD simulations to scale up to billions of atoms, corresponding to a length scale of approximately 0.1 $\mu$m. 

In the literature, many molecular dynamics studies of \zirconia\ exist \cite{Schelling2004, Khan1998, Lee2013, Fisher1998, Pietrucci2008, Li1995, Miller2013, Aidhy2015}. However, these studies typically focus on dopant-stabilised zirconias (i.e. cubic ones as empirical potentials do not capture monoclinic or tetragonal phases and their transitions accurately). The large system sizes possible in molecular dynamics simulations are often necessary for examining properties such as ion diffusion, thermal conductivity or melting points \cite{Davis2010}. Studying fission products in \zirconia\ however, requires potentials which can accurately model interactions of atoms such as Zr, O, and I in the solid phase. A good potential for such a system has not yet been published and so a quantum mechanical study of the \zirconia\ system is the focus of this thesis. The work herein may then be used in the future development of such potentials.

%These relatively large system sizes make MD a useful tool for studying a range of materials phenomena which are difficult to model at smaller scales, such as dislocations and long-range diffusion. 

%Add a figure here showing lennard jones. Also show the basic equations
%These are mainly pair potentials (although many-bodied potentials are also used) which are some combination of a short-range repulsion term (Pauli exclusion, nuclear repulsion if van der waals is taken into account) and a long range Coulombic attraction term

\subsection{Quantum mechanical approach - DFT}

Another method for modelling materials at the atomic scale is to use a quantum mechanical approach. In this thesis, the framework of density functional theory (DFT) is used throughout for quantum mechanical calculations (see § \ref{section:dft}). These techniques use a more fundamental approach than molecular dynamics, and are sometimes referred to as \emph{ab initio} methods (although several empirical approximations are often used in DFT). The CASTEP 8.0 software package was used for all DFT calculations \cite{Clark2005}.

System sizes are far more constrained when using quantum mechanical methods. The solution algorithms require CPU time proportional to $N^4$, where $N$ is the number of basis functions (functions used to describe electron orbitals). This steep scaling leads to large computational costs even for simple molecules with a few atoms before applying DFT methods \cite{Challacombe1997, Schwegler1996}. There are several ways to significantly reduce the computational complexity without sacrificing too much accuracy (e.g. the pseudopotential method, periodic boundaries, cell constraints and symmetry). This allows system sizes on the order of hundreds of atoms to be studied, corresponding to a length scale of approximately 1 nm. While this length scale is much smaller than what can be achieved using MD, the use of a more fundamental parameter (electron density) in calculations provides a stronger scientific basis when material properties are derived from DFT models. Additionally, DFT allows electronic properties such as electron orbital occupancy and band gaps to be studied.

In the literature, DFT studies of \zirconia\ are predominantly focused on the dopant-stabilised cubic phase because of its use in fuel cells and medical applications \cite{orera1990intrinsic,jiang2011first}, with few studies on the undoped system \cite{mackrodt1986theoretical,aarhammar2009energetics}. Pure oxide studies also tend to focus on only one of the three common phase, typically the monoclinic \cite{zheng2007first,foster2002modelling,foster2001structure} and tetragonal phases \cite{Gionco2013, Eichler2004, Zhang2014}. Two notable studies have looked at all three phases. The first focused on the electronic structure and optical properties of \zirconia\ \cite{French1994}, while the second examined the structural properties and band structure of \zirconia\ \cite{Kralik1998}. Both studies utilised the local density approximation (LDA) for the exchange-correlation functional. In this thesis, a generalised gradient approximation (GGA) is used for the exchange-correlation functional (see § \ref{section:kohnsham}). Although GGAs are considered an improvement upon the LDA, it is always useful to compare results with data from older studies. 

%The LDA has been improved upon by utilising generalised gradient approximations (GGAs), improving the accuracy of more recent models (see § \ref{section:kohnsham}). However, it is always useful to compare to data from older studies. 

Lattice dielectric properties in the three phases have been calculated using DFT \cite{Zhao2002a}, and these have been used in this thesis to predict energies and defect equilibria. Various data from these studies have been used either for comparison or to aid in new calculations which have then been published.

\subsection{Band gap}

Conductors and insulators are two common ways to describe materials. While this binary characterisation may work as an approximation for many materials, in reality there is more of a continuum between these two states (e.g. semiconductors), and at the heart of this lies the band gap.

Electron energy levels in a crystalline lattice are quantised, restricting the range of possible electron energies to discrete quantities. More specifically, electrons can only occupy unique quantum states, defined by parameters such as quantum spin and angular momentum. In a crystal, where there are large numbers of electrons and many possible configurations of them, we refer to energy \emph{bands} which are comprised of many quantised energies. There are sometimes gaps between energy bands in crystals (illustrated in Figure \ref{figure:band_gap}) corresponding to energy levels that cannot be occupied, meaning that if electrons were to be added at the lowest energy levels one by one, there would occasionally be relatively large jumps in energy as an electron is forced to enter a higher energy band. 

\begin{figure}[ht]
\centering
\includegraphics[width=\linewidth]{images/band_gap.png}
\caption[Illustration of the band gap in diamond as a function of interatomic spacing.]{Illustration of the band gap in diamond as a function of interatomic spacing. Taken from \cite{Chetvorno2017}.}
\label{figure:band_gap}
\end{figure}

Two energy bands, the \emph{valence band} and \emph{conduction band}, help determine a material's metallic or non-metallic character. The valence band contains energy levels occupied by the valence electrons (at absolute zero), while the conduction band contains energy levels which are high enough that electrons may freely move throughout the crystal. In metals, the valence and conduction bands have some amount of overlap, meaning that once the valence band is full, the highest occupied electron energy states are within the conduction band and so the material acts as a conductor. Materials like \zirconia , however, have large energy gaps between the valence and conduction bands, known as the band gap. These materials are called \emph{band insulators} (as opposed to \emph{Mott insulators} where there is no conventional band gap, but electron-electron interactions impede electron promotion to higher energies), because the band gap is an energy barrier preventing the valence electrons from moving freely around the crystal. 

In addition to the valence and conduction bands, a value for the electron chemical potential or Fermi level of the material is needed to determine how the energy bands are filled. If the Fermi level is exactly halfway between the valence band maximum (VBM) and the conduction band minimum (CBM), then the additional energy input required to promote an electron to the conduction band is half the value of the band gap. The Fermi level is strongly dependent on extrinsic defects and temperature. Extrinsic defects (or dopants) can be introduced to materials such as semiconductors in order to change the concentration of electronic defects (electrons and holes), while an increase in temperature will result in an increase in the Fermi level because of the larger quantity of thermal energy available.

It should be noted that band gaps reported in DFT studies using LDA/GGA methods are significantly lower than those obtained experimentally. This is a known problem in DFT, and an exchange-correlation functional which reproduces correct band gap energies in semiconductors and insulators (without overfitting to experimental data) is not yet known. The GW method, which uses a self-energy energy term in place of an exchange-correlation functional, allows more accurate\footnote{The GW approximation still has inaccuracies when modelling strongly correlated systems, but works well with $s$-$p$ systems.} estimates of the band gap than with DFT, but at a significantly higher computational expense. The band gap from DFT calculations may also be increased by appending an additional potential term, known as a +U parameter, to valence electron orbitals (see § \ref{subsection:plus_U}), or by using hybrid potentials which can incorporate the exact exchange energy.

%\section{Fission product empirical potential}
%
%In order to study the interaction of fission products with larger features in the cladding microstructure, such as dislocations and grain boundaries, it is necessary to develop empirical potentials for use in molecular dynamics simulations. Grain boundary transport is of particular interest, and this would require something on the order of 10$^{4}$ atoms to simulate to a reasonable degree of accuracy. This cannot be done using DFT currently due to the significant amount of computing resources required to run such a simulation. 
%
%The development of an iodine and xenon potential with \zirconia\ should be prioritised in order to run simulations to determine the migration of iodine within \zirconia, followed by the behaviour of xenon at the equilibrium iodine sites.
%
%\section{Grain boundary transport}
%
%Grain boundaries are interesting areas for studying species migration because diffusion towards the metal is expected to be more rapid through them than through bulk \zirconia .
%
%\section{Zr/ZrO/\zirconia\ interface study}
%
%The inner oxide is not a homogeneous structure, as described in § \ref{ch:crystallography}. Figure \ref{fig:zro_interface} clearly shows the existence of a ZrO phase up to 200 nm thick at the interface between \zirconia\ and Zr metal. The presence of ZrO and even oxygen-saturated Zr metal will have an effect on the thermodynamic equilibria of different fission products. An interface study can be conducted using DFT, to determine stresses at the interfaces of Zr and ZrO, and ZrO and \zirconia . Studying the aggregate effect of these interfaces on fission product behaviour may require larger molecular dynamics simulations, however. The crystal structure of the ZrO phase has been studied using both simulation and high-resolution electron microscopy, with two likely crystal structures being proposed \cite{Nicholls2015}. Further atomistic studies must be conducted to determine the stability of each crystal structure of ZrO when constrained by \zirconia\ and oxygen-saturated Zr metal interfaces.