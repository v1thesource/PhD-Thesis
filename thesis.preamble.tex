\usepackage{graphicx}
\usepackage{verbatim}
\usepackage{latexsym}
\usepackage{mathchars}
\usepackage{setspace}
\usepackage{acronym}
\usepackage{xr-hyper}
\usepackage{hyperref}
%\usepackage{nomencl}
\usepackage{animate}
\usepackage{pgfplots}

\pgfplotsset{compat=1.14}
\usepackage{tikz}
\usetikzlibrary{lindenmayersystems}
\pgfdeclarelindenmayersystem{A}{%
  \symbol{F}{\pgflsystemstep=0.6\pgflsystemstep\pgflsystemdrawforward}
  \rule{A->F[+A][-A]}
}

\hypersetup{allcolors=[rgb]{0.07,0.15,0.3}, colorlinks=true}
\usepackage{amsmath, amssymb, amsthm}
\usepackage{multirow}
\usepackage{chemformula}[2014/04/08]
\setchemformula{kroeger-vink=true}

\usepackage{gensymb}
\usepackage{cite}

\usepackage{stmaryrd}
\usepackage{proof}
\usepackage{mathpartir}
\usepackage{url}
\usepackage{multicol}
\usepackage{lscape} % Landscape view
\usepackage{float}





\setlength{\parskip}{\medskipamount}  % a little space before a \par
\setlength{\parindent}{0pt}	      % don't indent first lines of paragraphs
%UHEAD.STY  If this is included after \documentstyle{report}, it adds
% an underlined heading style to the LaTeX report style.
% \pagestyle{uheadings} will put underlined headings at the top
% of each page. The right page headings are the Chapter titles and
% the left page titles are supplied by \def\lefthead{text}.

% Ted Shapin, Dec. 17, 1986

\makeatletter
\def\chapapp2{Chapter}

\def\appendix{\par
 \setcounter{chapter}{0}
 \setcounter{section}{0}
 \def\chapapp2{Appendix}
 \def\@chapapp{Appendix}
 \def\thechapter{\Alph{chapter}}}

\def\ps@uheadings{\let\@mkboth\markboth
% modifications
\def\@oddhead{\protect\underline{\protect\makebox[\textwidth][l]
		{\sl\rightmark\hfill\rm\thepage}}}
\def\@oddfoot{}
\def\@evenfoot{}
\def\@evenhead{\protect\underline{\protect\makebox[\textwidth][l]
		{\rm\thepage\hfill\sl\leftmark}}}
% end of modifications
\def\chaptermark##1{\markboth {\ifnum \c@secnumdepth >\m@ne
 \chapapp2\ \thechapter. \ \fi ##1}{}}%
\def\sectionmark##1{\markright {\ifnum \c@secnumdepth >\z@
   \thesection. \ \fi ##1}}}
\makeatother
%%From: marcel@cs.caltech.edu (Marcel van der Goot)
%%Newsgroups: comp.text.tex
%%Subject: illegal modification of boxit.sty
%%Date: 28 Feb 92 01:10:02 GMT
%%Organization: California Institute of Technology (CS dept)
%%Nntp-Posting-Host: andromeda.cs.caltech.edu
%%
%%
%%Quite some time ago I posted a file boxit.sty; maybe it made it
%%to some archives, although I don't recall submitting it. It defines
%%	\begin{boxit}
%%	...
%%	\end{boxit}
%%to draw a box around `...', where the `...' can contain other
%%environments (e.g., a verbatim environment). Unfortunately, it had
%%a problem: it did not work if you used it in paragraph mode, i.e., it
%%only worked if there was an empty line in front of \begin{boxit}.
%%Luckily, that is easily corrected.
%%
%%HOWEVER, apparently someone noticed the problem, tried to correct it,
%%and then distributed this modified version. That would be fine with me,
%%except that:
%%1. There was no note in the file about this modification, it only has my
%%   name in it.
%%2. The modification is wrong: now it only works if there is *no* empty
%%   line in front of \begin{boxit}. In my opinion this bug is worse than
%%   the original one.
%%
%%In particular, the author of this modification tried to force an empty
%%line by inserting a `\\' in the definition of \Beginboxit. If you have
%%a version of boxit.sty with a `\\', please delete it. If you have my
%%old version of boxit.sty, please also delete it. Below is an improved
%%version.
%%
%%Thanks to Joe Armstrong for drawing my attention to the bug and to the
%%illegal version.
%%
%%                                          Marcel van der Goot
%% .---------------------------------------------------------------
%% | Blauw de viooltjes,                    marcel@cs.caltech.edu
%% |    Rood zijn de rozen;
%% | Een rijm kan gezet
%% |    Met plaksel en dozen.
%% |


% boxit.sty
% version: 27 Feb 1992
%
% Defines a boxit environment, which draws lines around its contents.
% Usage:
%   \begin{boxit}
%	... (text you want to be boxed, can contain other environments)
%   \end{boxit}
%
% The width of the box is the width of the contents.
% The boxit* environment behaves the same, except that the box will be
% at least as wide as a normal paragraph.
%
% The reason for writing it this way (rather than with the \boxit#1 macro
% from the TeXbook), is that now you can box verbatim text, as in
%   \begin{boxit}
%   \begin{verbatim}
%   this better come out in boxed verbatim mode ...
%   \end{verbatim}
%   \end{boxit}
%
%						Marcel van der Goot
%						marcel@cs.caltech.edu
%

\def\Beginboxit
   {\par
    \vbox\bgroup
	   \hrule
	   \hbox\bgroup
		  \vrule \kern1.2pt %
		  \vbox\bgroup\kern1.2pt
   }

\def\Endboxit{%
			      \kern1.2pt
		       \egroup
		  \kern1.2pt\vrule
		\egroup
	   \hrule
	 \egroup
   }	

\newenvironment{boxit}{\Beginboxit}{\Endboxit}
\newenvironment{boxit*}{\Beginboxit\hbox to\hsize{}}{\Endboxit}
\pagestyle{empty}

\setlength{\parskip}{2ex plus 0.5ex minus 0.2ex}
\setlength{\parindent}{0pt}

\makeatletter  %to avoid error messages generated by "\@". Makes Latex treat "@" like a letter

\linespread{1.5}
\def\submitdate#1{\gdef\@submitdate{#1}}

\def\maketitle{
  \begin{titlepage}{
    %\linespread{1.5}
    \Large Imperial College of Science, Technology and Medicine \\
    Department of Materials
    \rm
    \vskip 3in
    \Large \bf \@title \par
  }
  \vskip 0.3in
  \par
  {\Large \@author}
  \vskip 4in
  \par
  Submitted in part fulfilment of the requirements for the degree of 
  \linebreak
  Doctor of Philosophy and the Diploma of Imperial College
  \linebreak
  \linebreak
  \@submitdate
  \vfil
  \end{titlepage}
}

\def\titlepage{
  \newpage
  \centering
  \linespread{1}
  \normalsize
  \vbox to \vsize\bgroup\vbox to 9in\bgroup
}
\def\endtitlepage{
  \par
  \kern 0pt
  \egroup
  \vss
  \egroup
  \cleardoublepage
}

\def\abstract{
  \begin{center}{
    \large\bf Abstract}
  \end{center}
  \small
  %\def\baselinestretch{1.5}
  \linespread{1.5}
  \normalsize
}
\def\endabstract{
  \par
}

\newenvironment{acknowledgements}{
  \cleardoublepage
  \begin{center}{
    \large \bf Acknowledgements}
  \end{center}
  \small
  \linespread{1.5}
  \normalsize
}{\cleardoublepage}
\def\endacknowledgements{
  \par
}

\newenvironment{dedication}{
  \cleardoublepage
  \begin{center}{
    \large \bf Dedication}
  \end{center}
  \small
  \linespread{1.5}
  \normalsize
}{\cleardoublepage}
\def\enddedication{
  \par
}

\def\preface{
    \pagenumbering{roman}
    \pagestyle{plain}
    \setstretch{0.8}
}

\def\body{
    \cleardoublepage    
    \pagestyle{uheadings}
    \tableofcontents
    \pagestyle{plain}
    \cleardoublepage
    \pagestyle{plain}
    \vspace*{55px}
\textbf{\Huge List of abbreviations}

\onehalfspacing

\begin{acronym}[1234567890123456] % Give the longest label here so that the list is nicely aligned
\acro{DFT}{Density functional theory}
\acro{PCI}{Pellet cladding interaction}
\acro{LWR}{Light water reactor}
\acro{PWR}{Pressurised water reactor}
\acro{BWR}{Boiling water reactor}
\acro{GCR}{Gas cooled reactor}
\acro{SCC}{Stress corrosion cracking}
\acro{NDT}{Non destructive testing}
\acro{GGA}{Generalised gradient approximation}
\acro{PBE}{Perdew-Burke-Ernzerhof}
\acro{LDA}{Local density approximation}
\acro{VBM}{Valence band maximum}
\acro{CBM}{Conduction band minimum}
\acro{DOS}{Density of states}
\acro{PP}{Pseudopotential}
\acro{RPV}{Reactor pressure vessel}
\acro{YSZ}{Yttria-Stabilised Zirconia}
\acro{PSZ}{Partially-Stabilised Zirconia}
\acro{VASP}{Vienna Ab Initio Simulation Package}
\end{acronym}

\singlespacing

    \cleardoublepage
    \pagestyle{uheadings}
    \listoftables
    \pagestyle{plain}
    \cleardoublepage
    \pagestyle{uheadings}
    \listoffigures
    \pagestyle{plain}
    \cleardoublepage
    \pagestyle{uheadings}
    \pagenumbering{arabic}
    \singlespacing
}

\makeatother  %to avoid error messages generated by "\@". Makes Latex treat "@" like a letter

\newcommand{\ipc}{{\sf ipc}}

\newcommand{\zirconia}{ZrO\texorpdfstring{$_{2}$}{}}
\newcommand*{\plimsoll}{{\ensuremath{-\kern-4pt{\ominus}\kern-4pt-}}}

\newcommand{\Prob}{\bbbp}
\newcommand{\Real}{\bbbr}
\newcommand{\real}{\Real}
\newcommand{\Int}{\bbbz}
\newcommand{\Nat}{\bbbn}

\newcommand{\NN}{{\sf I\kern-0.14emN}}   % Natural numbers
\newcommand{\ZZ}{{\sf Z\kern-0.45emZ}}   % Integers
\newcommand{\QQQ}{{\sf C\kern-0.48emQ}}   % Rational numbers
\newcommand{\RR}{{\sf I\kern-0.14emR}}   % Real numbers
\newcommand{\KK}{{\cal K}}
\newcommand{\OO}{{\cal O}}
\newcommand{\AAA}{{\bf A}}
\newcommand{\HH}{{\bf H}}
\newcommand{\II}{{\bf I}}
\newcommand{\LL}{{\bf L}}
\newcommand{\PP}{{\bf P}}
\newcommand{\PPprime}{{\bf P'}}
\newcommand{\QQ}{{\bf Q}}
\newcommand{\UU}{{\bf U}}
\newcommand{\UUprime}{{\bf U'}}
\newcommand{\zzero}{{\bf 0}}
\newcommand{\ppi}{\mbox{\boldmath $\pi$}}
\newcommand{\aalph}{\mbox{\boldmath $\alpha$}}
\newcommand{\bb}{{\bf b}}
\newcommand{\ee}{{\bf e}}
\newcommand{\mmu}{\mbox{\boldmath $\mu$}}
\newcommand{\vv}{{\bf v}}
\newcommand{\xx}{{\bf x}}
\newcommand{\yy}{{\bf y}}
\newcommand{\zz}{{\bf z}}
\newcommand{\oomeg}{\mbox{\boldmath $\omega$}}
\newcommand{\res}{{\bf res}}
\newcommand{\cchi}{{\mbox{\raisebox{.4ex}{$\chi$}}}}
%\newcommand{\cchi}{{\cal X}}
%\newcommand{\cchi}{\mbox{\Large $\chi$}}

% Logical operators and symbols
\newcommand{\imply}{\Rightarrow}
\newcommand{\bimply}{\Leftrightarrow}
\newcommand{\union}{\cup}
\newcommand{\intersect}{\cap}
\newcommand{\boolor}{\vee}
\newcommand{\booland}{\wedge}
\newcommand{\boolimply}{\imply}
\newcommand{\boolbimply}{\bimply}
\newcommand{\boolnot}{\neg}
\newcommand{\boolsat}{\!\models}
\newcommand{\boolnsat}{\!\not\models}


\newcommand{\op}[1]{\mathrm{#1}}
\newcommand{\s}[1]{\ensuremath{\mathcal #1}}

% Properly styled differentiation and integration operators
\newcommand{\diff}[1]{\mathrm{\frac{d}{d\mathit{#1}}}}
\newcommand{\diffII}[1]{\mathrm{\frac{d^2}{d\mathit{#1}^2}}}
\newcommand{\intg}[4]{\int_{#3}^{#4} #1 \, \mathrm{d}#2}
\newcommand{\intgd}[4]{\int\!\!\!\!\int_{#4} #1 \, \mathrm{d}#2 \, \mathrm{d}#3}

% Large () brackets on different lines of an eqnarray environment
\newcommand{\Leftbrace}[1]{\left(\raisebox{0mm}[#1][#1]{}\right.}
\newcommand{\Rightbrace}[1]{\left.\raisebox{0mm}[#1][#1]{}\right)}

% Funky symobols for footnotes
\newcommand{\symbolfootnote}{\renewcommand{\thefootnote}{\fnsymbol{footnote}}}
% now add \symbolfootnote to the beginning of the document...

\newcommand{\normallinespacing}{\renewcommand{\baselinestretch}{1.5} \normalsize}
\newcommand{\mediumlinespacing}{\renewcommand{\baselinestretch}{1.2} \normalsize}
\newcommand{\narrowlinespacing}{\renewcommand{\baselinestretch}{1.0} \normalsize}
\newcommand{\bump}{\noalign{\vspace*{\doublerulesep}}}
\newcommand{\cell}{\multicolumn{1}{}{}}
\newcommand{\spann}{\mbox{span}}
\newcommand{\diagg}{\mbox{diag}}
\newcommand{\modd}{\mbox{mod}}
\newcommand{\minn}{\mbox{min}}
\newcommand{\andd}{\mbox{and}}
\newcommand{\forr}{\mbox{for}}
\newcommand{\EE}{\mbox{E}}

\newcommand{\deff}{\stackrel{\mathrm{def}}{=}}
\newcommand{\syncc}{~\stackrel{\textstyle \rhd\kern-0.57em\lhd}{\scriptstyle L}~}

\def\coop{\mbox{\large $\rhd\!\!\!\lhd$}}
\newcommand{\sync}[1]{\raisebox{-1.0ex}{$\;\stackrel{\coop}{\scriptscriptstyle
#1}\,$}}

\newtheorem{definition}{Definition}[chapter]
\newtheorem{theorem}{Theorem}[chapter]

\newcommand{\Figref}[1]{Figure~\ref{#1}}
\newcommand{\fig}[3]{
 \begin{figure}[!ht]
 \begin{center}
 \scalebox{#3}{\includegraphics{figs/#1.ps}}
 \vspace{-0.1in}
 \caption[ ]{\label{#1} #2}
 \end{center}
 \end{figure}
}

\newcommand{\figtwo}[8]{
 \begin{figure}
 \parbox[b]{#4 \textwidth}{
 \begin{center}
 \scalebox{#3}{\includegraphics{figs/#1.ps}}
 \vspace{-0.1in}
 \caption{\label{#1}#2}
 \end{center}
 }
 \hfill
 \parbox[b]{#8 \textwidth}{
 \begin{center}
 \scalebox{#7}{\includegraphics{figs/#5.ps}}
 \vspace{-0.1in}
 \caption{\label{#5}#6}
 \end{center}
 }
 \end{figure}
}
