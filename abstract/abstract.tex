\addcontentsline{toc}{chapter}{Abstract}

\begin{abstract}
Zirconium alloys are used as a cladding material in most nuclear reactors worldwide due to properties uniquely suited to the operating environment of a reactor. In this thesis, density functional theory (DFT) simulations were conducted to investigate the behaviour of fission product dopants in the inner cladding oxide, and to examine the role this layer plays in limiting corrosion in the context of pellet-cladding interaction (PCI). 

Simulations in undoped monoclinic, tetragonal and cubic \zirconia\ yielded structure properties in addition to intrinsic defect energies, volumes and defect equilibria. Fully-charged Schottky defects \{2\ch{V_{O}^{**}}:\ch{V_{Zr}^{''''}}\}$^{\times}$ had the smallest formation energies in each phase, followed by O Frenkels and then Zr Frenkels. Defective cubic \zirconia\ simulations are sensitive to finite-size effects, and would often break symmetry or collapse into the tetragonal phase when defect clusters were introduced. Free energy calculations predicted a transition from monoclinic to tetragonal as temperature was increased, but not from tetragonal to cubic. % as would be expected. 

Iodine defects adopt oxidation states of +1 (\ch{I_{O}^{***}}, \ch{I_{i}^{*}} and \ch{I_{Zr}^{'''}}) or -1 (\ch{I_{O}^{*}}) in \zirconia , with fewer defects in the 0 oxidation state (\ch{I_{O}^{**}}). At high oxygen partial pressures ($p_{O_{2}}$), iodine defects in tetragonal \zirconia\ fall significantly. Iodine defects in monoclinic \zirconia\ changed by small amounts as $p_{O_{2}}$ was increased. This demonstrated competition between iodine and oxygen in \zirconia , and that it is dependent on both $p_{O_{2}}$ and phase. High $p_{O_{2}}$ in the tetragonal phase provides the greatest barrier to iodine ingress.

During reactor power ramps, the quantity of fission products implanted in the oxide layer will increase. Decay rates of Te and I isotopes were found to be commensurate with time to failure in irradiation tests. Defect equilibria and volumes of Te, I, Xe and Cs were obtained in tetragonal \zirconia\ to investigate the effect of nuclear transmutation while dopant atoms are present. Defect evolution on the O site is predicted to be \ch{Te_{O}^{**}} $\rightarrow$ \ch{I_{O}^{*}} $\rightarrow$ \ch{Xe_{O}^{**}} $\rightarrow$ \ch{Cs_{O}^{**}}. On the Zr site, Brouwer diagrams predict \ch{Te_{Zr}^{'''}} $\rightarrow$ \ch{I_{Zr}^{'''}} $\rightarrow$ \ch{Xe_{Zr}^{''''}} $\rightarrow$ \ch{Cs_{Zr}^{'''}}. These defects have large defect volumes and will generate stresses which may promote crack formation.
\end{abstract}

%\begin{itemize}
%\item The third study was about tellurium, iodine, xenon and caesium in the tetragonal phase only.
%\item We propose a new initiation mechanism for PCI failures, whereby iodine diffuses deep into the \zirconia\ layer, past the monoclinic portion but short of the oxide-metal interface. \zirconia\ in this region of the oxide is predominantly tetragonal phase. The iodine nuclei then decay into xenon nuclei, which are larger and have less coherence with the \zirconia\ matrix. These xenon atoms impose a significant strain locally which will open cracks and initiate new ones. At a critical concentration of iodine, this effect bares enough fresh metal surface such that the corrosive effect of iodine outpaces the development of a passivating oxide layer, leading to failure of the clad.
%end{itemize}
