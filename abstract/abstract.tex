\addcontentsline{toc}{chapter}{Abstract}

\begin{abstract}

\begin{itemize}
\item The inner oxide of LWR fuel cladding was studied.
\item In particular, its role in PCI failures.
\item We used quantum mechanical simulation methods to predict defect energies in \zirconia .
\item We used these energies to determine defect equilibria in the different phases of \zirconia .
\item The first study was entirely on undoped \zirconia . We predicted intrinsic defect energies and equilibria, as well as defect volumes and relative structural stability of the monoclinic, tetragonal and cubic phases.
\item Along the way, we found that simulating the cubic phase using QM methods would produce an easily destabilised structure, prone to collapse when defects are introduced.
\item The second study was focused on iodine defects in the monoclinic and tetragonal phase.
\item We found that there is significant competition between iodine and oxygen for anion sites in the tetragonal phase. This is not the case for the monoclinic phase.
\item The third study was about tellurium, iodine, xenon and caesium in the tetragonal phase only.
\item We propose a new initiation mechanism for PCI failures, whereby iodine diffuses deep into the \zirconia\ layer, past the monoclinic portion but short of the oxide-metal interface. \zirconia\ in this region of the oxide is predominantly tetragonal phase. The iodine nuclei then decay into xenon nuclei, which are larger and have less coherence with the \zirconia\ matrix. These xenon atoms impose a significant strain locally which will open cracks and initiate new ones. At a critical concentration of iodine, this effect bares enough fresh metal surface such that the corrosive effect of iodine outpaces the development of a passivating oxide layer, leading to failure of the clad.
\end{itemize}
\end{abstract}