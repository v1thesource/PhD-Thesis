\chapter{Intrinsic defect study}

\label{ch:results1} % 3

\section{Introduction} % 3.1
\subsection{Applications} % 3.1.1

The intrinsic defect structure of \zirconia\ is important to understand because useful material properties may be exploited by doping with other ions. For example, \zirconia\ doped with enough yttrium cations will stabilise the cubic phase and increase the concentration of oxygen vacancies, which is the main charge carrier in a YSZ solid oxide fuel cell.

\subsection{Crystal structures} % 3.1.2

\zirconia\ grown at standard conditions on Zr metal exists mainly in either the monoclinic or tetragonal phase \cite{Howard1988,teufer1962crystal}. A high temperature cubic phase has also been observed in pure \zirconia , but it's similarity to the tetragonal phase often leads to them being mischaracterised in experimental studies. 

\subsection{Previous work} % 3.1.3

Previous works studying intrinsic defects in the \zirconia\ system have utilised quantum mechanical methods to determine defect formation energies in the monoclinic phase \cite{zheng2007first,foster2002modelling,foster2001structure} and defect equilibria in the tetragonal phase \cite{youssef2012intrinsic}. The cubic phase is mainly studied as a dopant-stabilised system \cite{orera1990intrinsic,jiang2011first}, with few undoped defect studies in the literature \cite{mackrodt1986theoretical,aarhammar2009energetics}. Building upon previous quantum mechanical studies, we provide a comprehensive account of intrinsic defect energies, defect volumes, and defect equilibria for all three common crystal structures of \zirconia , using state-of-the-art, accessible methods.

\section{Methodology}
\subsection{Simulation parameters}

Density functional theory (DFT) calculations were performed using CASTEP 8.0 \cite{Clark2005}. Ultra-soft pseudo-potentials were used throughout, employing a 600 eV cut-off energy. The Perdew, Burke and Ernzerhof (PBE) \cite{Perdew1996} parameterisation of the generalised gradient approximation (GGA) was used to describe the exchange correlation functional. A Monkhorst-Pack sampling scheme \cite{Monkhorst1976} was used for Brillouin zone integration, with a minimum \emph{k}-point separation of 0.09 \r{A}\textsuperscript{-1}. The Pulay method for density mixing \cite{Pulay1980} was used to improve convergence of simulations. 

The electrical energy convergence criterion was set to $1\times10^{-6} $ eV. The maximum force between atoms was limited to $1\times10^{-2}$ eV \r{A}\textsuperscript{-1}. A gradient-descent geometry optimisation task was run on the cell until consecutive iterations differed in energy and atomic displacement by less than $1\times10^{-5}$ eV and $5\times10^{-4}$ \r{A}, respectively. 


\subsection{Temperature dependence}

To determine the temperature dependence of the ground states for the pure crystal structures, a harmonic approximation method as described by Burr et al. was used \cite{burr2015crystal,jackson2016resolving}. A constant-volume phonon calculation was performed for each structure, from which the vibrational enthalpy $H_{vib}(T, V)$ and entropy $S_{vib}(T, V)$ contributions to the Helmholtz free energy were calculated up to a temperature of 2500 K. The complete Helmholtz free energy $F(T, V)$ was then obtained by including the internal energy $U(V)$ and configurational entropy $S_{conf}$ of the system:

\begin{equation} \label{helmholtz_equation}
F(T, V) = U(V) + H_{vib}(T, V) - TS_{vib}(T, V) - TS_{conf}
\end{equation}

\subsection{Defect formation energies}

\begin{itemize}
\item defective cell - perfect cell + electron energy + charge correction
\end{itemize}

\subsection{Brouwer diagrams}

Brouwer diagrams, also known as Kr{\"o}ger-Vink diagrams, were produced using a method outlined by Murphy et al. \cite{Murphy2014} to determine defect concentrations as a function of oxygen partial pressure. We start from the statement that the chemical potential of \zirconia\ is equivalent to the sum of the chemical potentials $\mu$ of its constituent species, Zr and O:

\begin{equation}
{\mu}_{ZrO_2(s)} = {\mu}_{Zr}(p_{O_2}, T) + {\mu}_{O_2}(p_{O_2}, T)
\label{mewZrO2results1}
\end{equation}

where $T$ denotes temperature and $p_{O_2}$ denotes oxygen partial pressure. The chemical potential of \zirconia\ in the solid state is assumed to have negligible dependence on $T$ and $p_{O_2}$ relative to ${\mu}_{Zr}$ and ${\mu}_{O_2}$. Energies can be obtained for bulk \zirconia\ and Zr, but the ground state of oxygen is not correctly reproduced in DFT \cite{Batyrev2000,Lozovoi2001}. Instead, we use the approach of Finnis et al. \cite{Finnis2005} to infer the oxygen chemical potential from standard state values. We can use the experimental Gibbs free energy to produce an equation where $\mu_{O_2}$ is the only unknown:

\begin{equation}
\Delta{G^{\plimsoll}_{f, ZrO_2}} = \mu_{ZrO_2(s)} - (\mu_{Zr(s)} + \mu^{\plimsoll}_{O_2})
\end{equation}

where $\Delta{G^{\plimsoll}_{f, ZrO_2}}$ is the experimental Gibbs energy at standard temperature and pressure and $\mu^{\plimsoll}_{O_2}$ is the oxygen chemical potential under the same conditions. The values of $\mu_{ZrO_2(s)}$ and $\mu_{Zr(s)}$ are calculated from the DFT energies. Once $\mu^{\plimsoll}_{O_2}$ is calculated, we can generalise the chemical potential of oxygen for any value of $T$ and $p_{O_2}$ by appending an ideal gas relationship $\Delta{\mu(T)}$ and a Boltzmann distribution:

\begin{equation}
\mu_{O_2}(p_{O_2},T) = \mu^{\plimsoll}_{O_2} + \Delta{\mu(T)} + \frac{1}{2}{k_B}log(\frac{p_{O_2}}{p^{\plimsoll}_{O_2}})
\end{equation}

Using our generalised formula for $\mu_{O_2}$, we fix the temperature within the range of thermal phase-stabilisation (1500 K for tetragonal \zirconia) and calculate $\mu_{O_2}$ for many different values of $p_{O_2}$ between $10^{-35}$ and 10$^{0}$ atm, corresponding to oxygen deficient and oxygen rich environments, respectively ($p_{O_2}$ in air is approximately 0.2 atm). While the tetragonal phase will be stress-stabilised in practice, thermal-stabilisation in such models has been shown to qualitatively approximate the effect of stress-stabilisation, while allowing a wider range of dopant behaviours to be predicted \cite{Bell2016}. Equilibrium defect concentrations are then calculated at each $\mu_{O_2}$ and plotted against $p_{O_2}$ to produce a Brouwer diagram. 

\section{Cubic phase collapse}

\begin{itemize}
\item When some oxygen Frenkel defects were introduced to the cubic phase supercell, relaxation under constant volume conditions caused a collapse into a pseudo-tetragonal structure.
\item This indicated that the cubic phase as modelled in DFT may not be fully stable.
\item Further investigation indicated that the structure of a supercell of c-\zirconia\ broke down even with constrained symmetry, a result corroborated by Burr et al. \cite{burr2017importance}. 
\end{itemize}

\section{Defect formation energies}

Defect volumes of isolated Frenkel defects can be seen in Table \ref{defect_volumes_clusters_isolated}.


\begin{table}[htp] % Isolated Frenkel volumes
\onehalfspacing
\centering
\caption{Isolated cluster defect volumes in the three \zirconia\ structures.}
\label{defect_volumes_clusters_isolated}
\begin{tabular}{cccc}
\hline
                      & \multicolumn{3}{c}{\textbf{Relaxation Volume (\r{A}\textsuperscript{3})}}  \\ \cline{2-4} 
\textbf{Defect}       & \textbf{Monoclinic} & \textbf{Tetragonal} & \textbf{Cubic} \\ \hline
\ch{V_{Zr}^{''''}} + \ch{Zr_{i}^{****}}          & 21.331	 & 25.4702 &	21.1309         \\
\ch{V_{Zr}^{'''}} + \ch{Zr_{i}^{***}}          & 19.7155 &	23.3463 &	19.9954      \\
\ch{V_{Zr}^{''}} + \ch{Zr_{i}^{**}}          & 18.1149 &	22.2525 &	19.68618           \\
\ch{V_{Zr}^{'}} + \ch{Zr_{i}^{*}}          & 19.78339 &	18.4096913 &	19.76396           \\
\ch{V_{Zr}^{x}} + \ch{Zr_{i}^{x}}          & 19.99485 &	18.1061 &	20.29223       \\
\ch{V_{O}^{**}} + \ch{O_{i}^{''}}           & 0.8839 &	2.4704 &	5.8217       \\
\ch{V_{O}^{*}} + \ch{O_{i}^{'}}           &  0.9486 &	5.032 &	4.1146        \\
\ch{V_{O}^{x}} + \ch{O_{i}^{x}}           &  0.9576 &	8.26065 &	7.83687          \\
\ch{V_{Zr}^{''''}} + 2\ch{V_{O}^{**}}       &  3.6979 &	-7.647 &	2.9448             \\
\ch{V_{Zr}^{''}} + 2\ch{V_{O}^{*}}       &  1.0707 &	-4.7866 &	1.5564         \\
\ch{V_{Zr}^{x}} + 2\ch{V_{O}^{x}}        & 0.64517 &	-0.8985 &	2.08973       \\ \hline
\end{tabular}
\end{table}

\begin{table}[htp] % Isolated formation energies
\onehalfspacing
\centering
\caption{Formation energies in eV of isolated \zirconia\ defects.}
\label{isolated_defects}
\begin{tabular}{cccll}
\hline
\multirow{2}{*}{\textbf{Defect}}                      & \multirow{2}{*}{\textbf{Equation}}                                        & \multicolumn{3}{c}{\textbf{Formation Energy (eV)}} \\ \cline{3-5}
	&	& \multicolumn{1}{l}{Monoclinic} & Tetragonal & Cubic \\ \hline
\multirow{5}{*}{\textbf{Zr Frenkel}} & \ch{Zr_{Zr}^{x}} $\rightarrow$ \ch{V_{Zr}^{''''}} + \ch{Zr_{i}^{****}}              & 5.428 & 5.639 & 5.610                             \\
                                     & \ch{Zr_{Zr}^{x}} $\rightarrow$ \ch{V_{Zr}^{'''}} + \ch{Zr_{i}^{***}}               & 8.695 & 8.939 & 8.476                            \\
                                     & \ch{Zr_{Zr}^{x}} $\rightarrow$ \ch{V_{Zr}^{''}} + \ch{Zr_{i}^{**}}                & 12.118 & 12.058 & 11.628                             \\
                                     & \ch{Zr_{Zr}^{x}} $\rightarrow$ \ch{V_{Zr}^{'}} + \ch{Zr_{i}^{*}}                & 16.021 &	15.696 &	13.319                             \\
                                     & \ch{Zr_{Zr}^{x}} $\rightarrow$ \ch{V_{Zr}^{x}} + \ch{Zr_{i}^{x}}                  & 20.563	& 20.094 &	18.170                            \\ \hline
\multirow{3}{*}{\textbf{O Frenkel}}  & \ch{O_{O}^{x}} $\rightarrow$ \ch{V_{O}^{**}} + \ch{O_{i}^{''}}                   & 4.457 &	4.000 & 	3.728                             \\
                                     & \ch{O_{O}^{x}} $\rightarrow$ \ch{V_{O}^{*}} + \ch{O_{i}^{'}}                   & 6.432	& 6.588 &	7.055                             \\
                                     & \ch{O_{O}^{x}} $\rightarrow$ \ch{V_{O}^{x}} + \ch{O_{i}^{x}}                     & 7.518 &	7.452 &	8.477                             \\ \hline
\multirow{3}{*}{\textbf{Schottky}}   & $\varnothing$ $\rightarrow$ \ch{V_{Zr}^{''''}} + 2\ch{V_{O}^{**}} & 5.120 &	3.778	& 1.752                             \\
                                     & $\varnothing$ $\rightarrow$ \ch{V_{Zr}^{''}} + 2\ch{V_{O}^{*}} & 11.353 &	10.832 &	9.624                             \\
                                     & $\varnothing$ $\rightarrow$ \ch{V_{Zr}^{x}} + 2\ch{V_{O}^{x}}   & 18.554 &	18.232 &	17.073  \\ \hline                          
\end{tabular}
\end{table}

\begin{table}[htp] % Bound formation energies
\onehalfspacing
\centering
\caption{Formation energies of bound defects in \zirconia.}
\label{table:bound_defects}
\begin{tabular}{cccc}
\hline
\multirow{2}{*}{\textbf{Defect}} & \multicolumn{3}{c}{\textbf{Formation Energy (eV)}} \\ \cline{2-4} 
 & \textbf{Monoclinic} & \textbf{Tetragonal} & \textbf{Cubic} \\ \hline
\textbf{O Frenkel} & 4.1212 & 4.0290 & 6.4397 \\
\textbf{Zr Frenkel} & 8.4232 & 7.8633 & 6.3274 \\
\textbf{NTV1} & 5.2272 & 3.5813 & 2.6961 \\
\textbf{NTV2} & 5.1405 & 4.2312 & 0.1798 \\
\textbf{NTV3} & 4.6620 & 3.3623 & 2.4089 \\ \hline
\end{tabular}
\end{table}

\section{Elastic constants and defect relaxation volumes}

Table \ref{stiffness_tensor} shows the calculated elastic constants for the monoclinic, tetragonal, and cubic phases of \zirconia . The cubic phase has the highest stiffness, likely due to the short Zr-O bond lengths in the energy-minimised structure (Figure \ref{figure:zrobonddistance}). It is expected that at high temperatures where the cubic phase is stable, the resulting increase in bond length would cause a reduction in stiffness. 

The monoclinic and tetragonal phases have similar stiffness in the principal axes, but vary significantly under shearing conditions. In particular, the tetragonal phase exhibits much smaller $C_{44}$ and $C_{55}$ components. This may be due to the strong directional anisotropy of the tetragonal phase due to the larger $c$ parameter. 


\begin{table}[htp] % Elastic constants
\onehalfspacing
\centering
\caption{Elastic constants for different phases of \zirconia\ from DFT calculations.}
\label{stiffness_tensor}
\begin{tabular}{cccc}
\hline
\multirow{2}{*}{\textbf{Elastic Component}} & \multicolumn{3}{c}{\textbf{Stiffness (GPa)}}               \\ \cline{2-4} 
                                            & \textbf{Monoclinic} & \textbf{Tetragonal} & \textbf{Cubic} \\ \hline
$C_{11}$                                         & 338.86        & 334.30               & 523.38    \\
$C_{12}$                                         & 151.80        & 207.30               & 92.93     \\
$C_{13}$                                         & 89.37         & 48.93               & 92.93     \\
$C_{22}$                                         & 348.37        & 334.20               & 523.39    \\
$C_{23}$                                         & 143.04        & 48.93               & 92.93    \\
$C_{33}$                                         & 262.17        & 250.50               & 523.38   \\
$C_{44}$                                         & 76.35         & 9.38                & 61.98    \\
$C_{55}$                                         & 71.65         & 9.38                & 61.98   \\
$C_{66}$                                         & 114.19        & 152.60               & 61.99     \\ \hline
\end{tabular}
\end{table}

\begin{table}[htp] % Isolated defect volumes
\onehalfspacing
\centering
\caption{Isolated defect volumes in the three \zirconia\ structures.}
\label{defect_volumes_raw}
\begin{tabular}{cccc}
\hline
                      & \multicolumn{3}{c}{\textbf{Volume (\r{A}\textsuperscript{3})}}  \\ \cline{2-4} 
\textbf{Defect}       & \textbf{Monoclinic} & \textbf{Tetragonal} & \textbf{Cubic} \\ \hline
\ch{V_{Zr}^{''''}}             & 55.9495             & 67.4062             & 48.468         \\
\ch{V_{Zr}^{'''}}             & 42.4752             &         51.0824     &     36.944      \\
\ch{V_{Zr}^{''}}            & 29.9013             &  34.2764            &     25.9264           \\
\ch{V_{Zr}^{'}}             & 17.1031             &  18.4276            &     15.0761           \\
\ch{V_{Zr}^{x}}              & 4.05915             &  4.70160            &    4.31893       \\
\ch{Zr_{i}^{****}}             & -34.6185            & -41.936             & -27.3371       \\
\ch{Zr_{i}^{***}}             &  -22.7597           &	-27.7361 		  &	-16.9486         \\
\ch{Zr_{i}^{**}}             &  -11.7864 	        &	-12.0239 		  &	-6.24022          \\
\ch{Zr_{i}^{*}}            &  2.68029			& -0.0179087 		  & 	4.68786             \\
\ch{Zr_{i}^{x}}              &  15.9357		 	& 13.4045	 		  & 15.9733         \\
\ch{V_{O}^{**}} {[}4coord{]} & -22.5154            & -37.5266            & -22.7616       \\
\ch{V_{O}^{*}} {[}4coord{]} &  -12.4114           &    -19.5315         &     -12.1850           \\
\ch{V_{O}^{x}} {[}4coord{]}  &  -0.686074          &  -2.80005           &      -1.1146          \\
\ch{V_{O}^{**}} {[}3coord{]} & -26.1258            &                     &                \\
\ch{V_{O}^{*}} {[}3coord{]} &  -14.4153           &                     &                \\
\ch{V_{O}^{x}} {[}3coord{]}  &   -1.70699          &                     &                \\
\ch{O_{i}^{''}}              & 27.0097             & 39.997              & 28.5833        \\
\ch{O_{i}^{'}}              &  15.3639            &    24.5635          &  16.2996              \\
\ch{O_{i}^{x}}               & 2.66459             &    11.0607          &   8.95147        \\ \hline
\end{tabular}
\end{table}

\section{Helmholtz energies}

The calculated Helmholtz energies plotted in Figure \ref{Figure:helmholtz} show the correct order of stability for the three phases of \zirconia\ at low temperatures (monoclinic $->$ tetragonal $->$ cubic) . However, as temperature is increased, only a transition from monoclinic to tetragonal is seen. The cubic phase Helmholtz energy does fall below the monoclinic curve, but never below the tetragonal curve, thus predicting no tetragonal to cubic phase transition. It must also be noted that the transition temperatures are not predicted accurately. The tetragonal phase is predicted to have a lower energy than monoclinic at approximately 400 K.

\begin{figure}[htp] % Helmholtz
\begin{center}
\begin{tikzpicture}
	\begin{axis}
		[width=11cm, xlabel={Temperature (K)}, ylabel={Helmholtz free energy (eV)}, ymin=-28, ymax=-10, xmin=0, xmax=3000, legend style={{draw=}, at={(0.95,0.95)}, anchor=north east, legend columns=1}]
		\addplot[no marks] table [x=temperature, y=monoclinic,]{dat/helmholtz.dat}; \addlegendentry{Monoclinic};
        \addplot[no marks, dashed] table [x=temperature, y=tetragonal, ]{dat/helmholtz.dat}; \addlegendentry{Tetragonal};
        \addplot[no marks, densely dotted, black] table [x=temperature, y=cubic,]{dat/helmholtz.dat}; \addlegendentry{Cubic};
			\end{axis}
		\end{tikzpicture}
		\caption{Helmholtz free energy as a function of temperature for the monoclinic, tetragonal, and cubic crystal structures of \zirconia .}
		\label{Figure:helmholtz}
	\end{center}
\end{figure}

\section{Defect equilibria}

The intrinsic defect equilibria are shown in Brouwer diagrams in Figures XXXmono, XXXtet and XXXcubic. The monoclinic phase exhibits the smallest overall concentration of intrinsic defects due to the low temperature (650 K) relative to the other tetragonal (1500 K) and cubic (XXX K) phases.

\section{Summary}

The main defects in \zirconia\ are oxygen vacancies and zirconium vacancies at low and high oxygen pressures respectively. 

The cubic phase cannot be modelled accurately, partly due to the instability of the phase. For this reason, it was decided that the cubic phase would not be considered when conducting extrinsic dopant simulation studies in \zirconia .
