\documentclass[a4paper,12pt,twoside]{report}
\usepackage[left=2cm,right=2cm,top=2cm,bottom=3cm]{geometry}
%\usepackage{showframe}

\documentclass[a4paper,12pt,twoside]{report}
\usepackage[left=2cm,right=2cm,top=2cm,bottom=3cm]{geometry}
%\usepackage{showframe}

\documentclass[a4paper,12pt,twoside]{report}
\usepackage[left=2cm,right=2cm,top=2cm,bottom=3cm]{geometry}
%\usepackage{showframe}

\documentclass[a4paper,12pt,twoside]{report}
\usepackage[left=2cm,right=2cm,top=2cm,bottom=3cm]{geometry}
%\usepackage{showframe}

\include{thesis.preamble}


\begin{document}

\title{\LARGE {\bf Atomistic Simulation of Fission Products in Zirconia Polymorphs}\\
 \vspace*{6mm}
}

\author{Alexandros Kenich}
\submitdate{April 2019}

\normallinespacing
\maketitle

\input{declaration}

\preface
\input{abstract/abstract}
\input{acknowledgements/acknowledgements}
%\input{dedication/dedication}
%\input{quotes/quotes}

\body

% body of thesis comes here
\doublespacing

\input{introduction/introduction}
\input{crystallography/crystallography}
\input{compmethodology/compmethodology}
\input{results1/results1}
\input{results2/results2}
\input{results3/results3}
\input{future/future}

\addcontentsline{toc}{chapter}{References}
\label{References}
\renewcommand\bibname{References}
\bibliographystyle{unsrt}
\bibliography{Mendeley}

\appendix
% Appendices come here
\addcontentsline{toc}{chapter}{Appendix}
\label{Appendix}

\chapter{ParaSweep}

ParaSweep is a generalised sensitivity analysis visualisation tool which was developed during this project. Initially, it was built to help visualise the effects of changing single parameters in Brouwer diagrams, such as temperature or concentration of defects. 


\chapter{CASTEP and HPC Scripts}

Throughout the course of this work, many useful scripts were created to help with preparing CASTEP jobs and analysing their outputs. These scripts have been made available online and for free at \href{https://github.com/v1thesource/CASTEP}{https://github.com/v1thesource/CASTEP}. The purpose of open-sourcing these scripts is to simplify the experience for new users of CASTEP and help them save a considerable amount of time.


\end{document}


\begin{document}

\title{\LARGE {\bf Atomistic Simulation of Fission Products in Zirconia Polymorphs}\\
 \vspace*{6mm}
}

\author{Alexandros Kenich}
\submitdate{April 2019}

\normallinespacing
\maketitle

\vspace*{140px}
\begin{center}
\textsc{\LARGE Declaration}
\end{center}
I declare that the work presented in this thesis is my own, and that all efforts from others are referenced. 

The copyright of this thesis rests with the author and is made available under a Creative Commons Attribution Non-Commercial No Derivatives licence. Researchers are free to copy, distribute or transmit the thesis on the condition that they attribute it, that they do not use it for commercial purposes and that they do not alter, transform or build upon it. For any reuse or redistribution, researchers must make clear to others the licence terms of this work. 

\begin{center}
\rule{125px}{0.2px}
\end{center}
\vfill
\pagebreak

\preface
\addcontentsline{toc}{chapter}{Abstract}

\begin{abstract}
Zirconium alloys are used as a cladding material in over 90\% of nuclear reactors worldwide due to properties which are uniquely suited to the operating environment of a reactor. In this thesis, density functional theory (DFT) simulations were conducted to investigate the behaviour of fission product dopants in the inner cladding oxide, and to examine the role this oxide layer plays in limiting corrosion in the context of pellet-cladding interaction (PCI). 

Simulations in undoped monoclinic, tetragonal and cubic \zirconia\ yielded non-defective structure properties in addition to intrinsic defect energies, volumes and defect equilibria. Fully-charged Schottky defects \{2\ch{V_{O}^{**}}:\ch{V_{Zr}^{''''}}\}$^{\times}$ were shown to have the smallest formation energies in each phase, followed by O Frenkel defects and then Zr Frenkel defects. Defective cubic \zirconia\ simulations are sensitive to finite-size effects, and would often break symmetry or collapse into the tetragonal phase when defect clusters were introduced. Free energy calculations predicted a transition from monoclinic to tetragonal as temperature is increased, but not from tetragonal to cubic. % as would be expected. 

Iodine adopts oxidation states of either +1 (\ch{I_{O}^{***}}, \ch{I_{i}^{*}} and \ch{I_{Zr}^{'''}}) or -1 (\ch{I_{O}^{*}}) when forming defects in \zirconia , with fewer defects in the 0 oxidation state (\ch{I_{O}^{**}}). At high oxygen partial pressures ($p_{O_{2}}$), iodine defects in tetragonal \zirconia\ fall significantly. In monoclinic \zirconia, iodine defects changed by only small amounts as $p_{O_{2}}$ was increased. This demonstrated competition between iodine and oxygen in \zirconia , and that it is dependent on both $p_{O_{2}}$ and phase. High $p_{O_{2}}$ in the tetragonal phase provides the greatest barrier to iodine ingress.

During reactor power ramps, the quantity of fission products implanted in the oxide layer will increase. Decay rates of major Te and I isotopes were found to be commensurate with time to failure in irradiation tests. Defect equilibria and volumes of Te, I, Xe and Cs were obtained in tetragonal \zirconia\ to investigate the effect of nuclear transmutation while dopant atoms are present. Defect evolution on the O site is predicted to be \ch{Te_{O}^{**}} $\rightarrow$ \ch{I_{O}^{*}} $\rightarrow$ \ch{Xe_{O}^{**}} $\rightarrow$ \ch{Cs_{O}^{**}}. On the Zr site, Brouwer diagrams predict \ch{Te_{Zr}^{'''}} $\rightarrow$ \ch{I_{Zr}^{'''}} $\rightarrow$ \ch{Xe_{Zr}^{''''}} $\rightarrow$ \ch{Cs_{Zr}^{'''}}. These defects have large defect volumes and will generate stresses which may promote crack formation.
\end{abstract}

%\begin{itemize}
%\item The third study was about tellurium, iodine, xenon and caesium in the tetragonal phase only.
%\item We propose a new initiation mechanism for PCI failures, whereby iodine diffuses deep into the \zirconia\ layer, past the monoclinic portion but short of the oxide-metal interface. \zirconia\ in this region of the oxide is predominantly tetragonal phase. The iodine nuclei then decay into xenon nuclei, which are larger and have less coherence with the \zirconia\ matrix. These xenon atoms impose a significant strain locally which will open cracks and initiate new ones. At a critical concentration of iodine, this effect bares enough fresh metal surface such that the corrosive effect of iodine outpaces the development of a passivating oxide layer, leading to failure of the clad.
%end{itemize}

\cleardoublepage

\addcontentsline{toc}{chapter}{Acknowledgements}

\begin{acknowledgements}

Firstly I would like to thank my supervisors Robin Grimes and Mark Wenman for giving me a chance (despite being a lowly mechanical engineer) and opening the door to pursue nuclear engineering at such a high level. I would also like to thank the EPSRC for funding my studentship through the ICO CDT, the Imperial College HPC team for their quick responses whenever something went wrong, Philipp Frankel and his research group at the University of Manchester for the fruitful discussions and insightful conferences over the years, the Department of Materials administration staff for all their help with my admin woes and the Centre for Nuclear Engineering for being my second home for half a decade.

In no particular order, I want to give a shoutout to the people who have left a strong impression on me and influenced my growth both as a scientist and as a person: Wael Al Jishi, Conor Galvin, Paul Fossati, Claudia Gasparrini, Navaratnarajah Kuganathan, Dhan-Sham Rana, Said El Chamaa, Filippo Vecchiato, Jana Smutna, Vlad Podgurschi, Matt Jackson, Lloyd Jones, Nipun Wickramasundara, Hussam Zaghal, Patrick Burr, William White, Mark Mawdsley, Richard Pearson, Sophie Morrison, Alan Charles, Jonathan Tate, Andy Wilson, John Brokx, Alexandru Paunoiu, Irina Dumitrescu, Julian Sutherland, Anca Semenescu and of course, Emma Warriss.

Finally, I give my everlasting gratitude and love to my parents, my wife Cristina and my daughter Livia-May.

\clearpage

\vfill
\begin{center}
\emph{In memory of Emma Warriss}
\end{center}
\vfill

\end{acknowledgements}
%\input{dedication/dedication}
%\input{quotes/quotes}

\body

% body of thesis comes here
\doublespacing

\chapter{Introduction} \label{introduction}

\section{Nuclear Power} % Complete

In the summer of 1956, the world's first commercial nuclear power plant was connected to the grid in the north of England. This marked a significant departure from previous forms of commercial energy production, which relied on relatively low energy density sources such as the combustion of coal, oil and gas. Before this, the closest anyone had come to utilising nuclear energy commercially was through geothermal power, where the thermal energy input is partly due to radiogenic heat from unstable isotopes in the Earth's mantle \cite{gando2011partial}. 
%which relied on the chemical reactions of coal oil and gas
%Combustion is a chemical process, where energy differences between reactants and products are exploited via electron exchange. Nuclear energy however, exploits the energy difference between nuclei. Both rely on the conversion of mass into energy, however, the amount of energy that can be extracted varies by several orders of magnitude

Combustion is a chemical process whereby energy differences between reactants and products are exploited via electron exchange. Nuclear energy exploits the energy difference between nuclei. Both rely on the conversion of mass into energy, however, the amount of energy that can be extracted from the nucleus is several orders of magnitude greater.
%Combustion is a chemical process, and its use in commercial energy production is fundamentally about exploiting the free energy difference when electrons are exchanged between some reactants to produce some products. Nuclear energy, however, is about the direct conversion of mass into energy. The difference between the two is staggering.

Consider methane, with an enthalpy of combustion of −887.2 kJ/mol \cite{thornton1917xv}. This is the equivalent of 9.14 eV per particle. By comparison, the total energy release from fission of one uranium-235 nucleus is at least $1.65 \times 10^{8}$ eV, as shown in Figure \ref{figure:fissionenergy}.

Combustion-based power as a technology has matured over hundreds of years, with modern optimisations only looking to offer fractional percent gains in efficiency. By comparison, nuclear power technology is far from mature, with large improvements yet to be realised. One such feature is load-following, an enormously useful feature for a power plant which is currently underutilised in nuclear reactors. Load-following, as currently practiced in some French and German nuclear power plants, is defined as operation where power output follows a variable load programme on a daily basis with several power changes (i.e. to follow the change in electricity demand over a 24 hour period). These power variations can be as large as 50\% of a reactor's rated power \cite{lokhov2011technical}. The biggest obstacle to load-following in nuclear reactors is the issue of pellet-cladding interaction (PCI), which is the basis of the work in this thesis.

\begin{figure}[ht]
\centering
\includegraphics[height=13cm]{images/fission_energy_total.png}
\caption[Energy from thermal fission of U$^{235}$ as a function of mass ratios of daughter nuclei. Total energy release includes contributions from gamma rays and subsequent radioactive decays.]{Energy from thermal fission of U$^{235}$ as a function of mass ratios of daughter nuclei. Total energy release includes contributions from gamma rays and subsequent radioactive decays. Taken from \cite{aras1965ranges}.}
\label{figure:fissionenergy}
\end{figure}

\subsection{Fission}

Commercial nuclear power plants extract energy through the process of fission, where a large nucleus is split into smaller nuclei. While it is also possible to extract energy from certain small nuclei by the process of fusing them into larger ones, no fusion reactor currently exists which achieves a net positive energy output. At a fundamental level, both fission and fusion rely upon mass-energy equivalence. The relationship between mass and energy is shown using Einstein's equation:
\begin{equation}
\label{emc2}
    E = mc^{2}
\end{equation}
where $E$ is the energy of the system, $m$ is the mass and $c$ is the speed of light in a vacuum. Using this equation we can analyse a typical fission reaction:
\begin{equation}
    \ch{U^{235}_{92}} + n^{1}_{0} \xrightarrow[]{absorption} \ch{U^{236}_{92}} \xrightarrow[]{fission} \ch{I^{132}_{53}} + \ch{Y^{101}_{39}} + 3n^{1}_{0}
\label{eqn:fission} 
\end{equation}
While the number of protons and neutrons are conserved throughout the reaction, a mass difference calculation will show that there is actually less mass in the products than the reactants by approximately 0.188 amu (3.127$\times 10^{-28}$ kg). This missing mass, known as the \emph{mass defect}, is converted to energy ($\sim$175 MeV). In this way, the total mass-energy of the system is conserved. Some of this energy is carried away as kinetic energy of the fission products (in Equation \ref{eqn:fission}, I and Y) and also the kinetic energy of the neutrons. The neutrons at this stage have energies \goodtilde{1} MeV and are known as \emph{fast} neutrons.

This change in mass arises due to the phenomenon of \emph{binding energy}. In order for two or more nucleons to be thermodynamically stable when bound together, the total free energy of the bound configuration must be less than the sum of constituent nucleon free energies. Much as with energy stored in a chemical bond, the binding energy represents the energy required to separate the nucleus into individual protons and neutrons. 

Larger nuclei will generally have a greater total binding energy value compared to smaller nuclei, but the mass defect per nucleon will not necessarily be the same in a larger nucleus. It is therefore useful to normalise the binding energy by the mass number. Different isotopes have different binding energies, and any nuclear reaction that increases the binding energy per nucleon will be exothermic, whether by fission or fusion. Figure \ref{figure:bindingenergy} shows a plot of binding energy per nucleon against mass number with the relevant isotopes from Equation \ref{eqn:fission}. 

%235.0439299 + 1.008664 (236.0525939) -> 131.907997 + 100.93031 + 3.025992 (235.864299)
\begin{figure}[ht]
\centering
\includegraphics[width=14cm]{images/Binding_energy_curve.png}
\caption[Plot of binding energy per nucleon against mass number. Arrows indicate the reaction shown in equation \ref{eqn:fission}.]{Plot of binding energy per nucleon against mass number. Arrows indicate the reaction shown in equation \ref{eqn:fission}. Adapted from \cite{Fastfission}.}
\label{figure:bindingenergy}
\end{figure}

\subsection{Reactor design} % Complete

Commercial nuclear reactors are large boilers in a Rankine cycle, designed to maximise heat transfer to a working fluid. All nuclear plants use steam turbines on the generation side, though the reactor coolant may be another fluid in a separate loop, such as carbon dioxide in gas-cooled reactors (GCRs), or even in a separate water loop such as in pressurised water reactors (PWRs). 

The most prevalent reactor type is the PWR, followed by the boiling water reactor (BWR). A schematic of a PWR power plant is shown in Figure \ref{figure:pwrschematic}. This design incorporates a primary coolant loop and heat exchanger to a secondary loop at a lower pressure. Steam is generated on the low pressure side of the heat exchanger which then drives a steam turbine. There are several other reactor types used around the world (enumerated in Table \ref{figure:world_reactors}). The work in this thesis is focused on zirconium-based claddings which are used worldwide in all commercial reactors except GCRs and sodium-cooled fast reactors. In total, zirconium fuel cladding is used in over 95\% of all nuclear reactor fuel pins, and so performance improvements in these cladding materials have an effect across the entire industry.

\begin{figure}[ht] % Schematic of a PWR
\centering
\includegraphics[width=\linewidth]{images/pwrschematic.png}
\caption[Schematic illustration of a PWR power plant.]{Schematic illustration of a PWR power plant. Taken from \cite{lokhov2011technical}.}
\label{figure:pwrschematic}
\end{figure}

The fission of uranium takes place inside a steel reactor pressure vessel (RPV) in PWRs and BWRs, which holds the fuel pins, control rods and other reactor internals. The working fluid in a nuclear reactor is typically under high pressure, with PWR RPV operating pressures between 150 and 160 bar, while BWRs operate at lower pressures of around 70 bar \cite{kok2016nuclear, Server2010, Durmayaz2001}. The pressure of the coolant acts on the fuel cladding, generating radial and hoop stresses which influence crack formation. 

The operating temperature of the coolant in a typical PWR is approximately 600 K. This is a low temperature relative to the melting points of Zr metal (2128 K) and ZrO$_{2}$ (2988 K). This temperature together with the high pressure is chosen in order to keep the coolant in the liquid phase for safety reasons, though this limits the thermodynamic efficiency of the plant. The highest temperature in a reactor will occur in the nuclear fuel, with PWR fuel pellets reaching centreline temperatures of up to 1673 K \cite{beyer1998review}.

\begin{table}[ht] % Reactors in the world
\centering
\caption[Type and number of different reactors operational worldwide at the end of 2017. Change from 2016 shown in parentheses.]{Type and number of different reactors operational worldwide at the end of 2017. Change from 2016 shown in parentheses. Taken from \cite{WNAreport2018}.}
\includegraphics[width=15cm]{images/WNA_report2018.png}
\label{figure:world_reactors}
\end{table}

For both PWRs and BWRs, the coolant is typically light water (as opposed to heavy water, D$_{2}$O). Water is used because it has many useful engineering properties. It has a high heat capacity (compared to the gaseous coolant in gas cooled reactors), has low activation in a free neutron environment and also serves as a good radiation shield. Furthermore, it is plentiful, cheap and easily purified.

In addition to its function as a coolant, water also acts as a neutron moderator, slowing down high-energy fast neutrons from fission events and nuclear decay processes. Moderation of neutrons is an important step in the nuclear reactor because slow thermal neutrons (i.e. neutrons at thermal equilibrium with the coolant) are significantly more likely to cause uranium nuclei to undergo fission than fast neutrons. Fast neutrons will typically escape from the fuel pin after they are generated, dispense most of their energy in the coolant via scattering with H and O nuclei, and then some will re-enter the fuel, where they may cause fission of a U$^{235}$ nucleus near the outer edge of the fuel pellet. Neutrons which do not end up fissioning fuel will either be absorbed parasitically by other nuclei (e.g. in the control rods or coolant), escape from the reactor entirely (neutron leakage), or decay into protons (free neutrons have a half-life of 10.61 minutes \cite{Christensen1972}).

\subsection{Fuel pellets and cladding} \label{ss_fuelpin}

Fuel assemblies in nuclear reactors are bundles of fuel pins (see Figure \ref{figure:fuelassembly}). In most commercial reactors, fuel pins are comprised of a zirconium-based cladding (tubes), which are filled with cylindrical UO$_{2}$ fuel pellets, each of which are approximately 1 cm$^{3}$ in volume (see Figure \ref{figure:fuelpellet}). Fuel pellets in PWRs and BWRs have dishes on the top and bottom faces of the cylinder as well as chamfered edges. The main function of the dishes is  to reduce the axial pellet-pellet stresses caused due to swelling of the pellet when irradiated \cite{marino2005crack}. Chamfers aid in the loading of fuel pellets into the cladding, as well as reducing the risk of chipping at the edges of the fuel pellet. This is important because chipping of the fuel pellet can lead to debris falling into the pellet-cladding gap where it can act as a stress raiser \cite{doerr2015nuclear}.

\begin{figure}[ht]
\centering
\includegraphics[width=11cm]{images/fuelassembly.png}
\caption[Schematic view of a PWR fuel assembly and a PWR fuel pin.]{Schematic view of a PWR fuel assembly and a PWR fuel pin. Adapted from \cite{Croff2003}.}
\label{figure:fuelassembly}
\end{figure} 

Once loaded with fuel pellets, the fuel pins are capped and filled with inert helium gas, pressurised to between 2 and 25 atm to improve heat transfer from the fuel pellets to the coolant as well as delaying inward creep deformation of the cladding due to the high coolant pressure \cite{King1980}. 

LWR fuel pellets are manufactured in a multi-stage process starting from enriched UF$_{6}$. The UF$_{6}$ must be converted into UO$_{2}$, which can be done using either a `dry' or `wet' process, referring to the use of liquid water in the process. The dry process, called the integrated dry route (IDR), is simpler and is described as follows:

\begin{itemize}
\item Enriched UF$_{6}$ (a solid at room temperature and pressure) is heated into vapour form using an autoclave.
\item UF$_{6}$ vapour is mixed with steam and fed into a rotary kiln.
\item Hydrogen gas is added to the mixture and the UF$_{6}$ is reduced to solid UO$_{2}$. The gaseous HF is recovered, leaving pure UO$_{2}$ crystals.
\end{itemize}

The UO$_{2}$ from this stage is then blended to homogenise the particle sizes and achieve a desired particle surface area. At this stage, additives may be introduced to the UO$_{2}$ (e.g. burnable poisons, lubricants, dopants to improve densification or to control microstructure). This powder is then fed into a pellet pressing die where it is pressed into a cylindrical shape, called a `green' pellet. Green pellets are then sintered in a furnace at temperatures of up to 2000 K in order to consolidate and increase the density of the pellets \cite{pramanik2010innovative}. These fired pellets are then machined to the appropriate dimensions, including chamfers and dishes, before final inspection and loading into a cladding tube.

\begin{figure}[ht]
\centering
\includegraphics[width=10cm]{images/fuelpellet.png}
\caption[UO$_{2}$ LWR fuel pellet showing dishes and chamfers.]{UO$_{2}$ LWR fuel pellet showing dishes and chamfers. Adapted from \cite{tulenko2013development}.}
\label{figure:fuelpellet}
\end{figure}

In the early stages of a fuel pin's life, there is a small gap between the fuel pellet and the cladding, known as the pellet-cladding gas gap. This gas gap slowly closes with increasing fuel burn-up due to swelling of the fuel pellets and inward creep deformation of the cladding due to the coolant pressure. The pellet-cladding system is shown using a schematic view of the cross section of a PWR fuel pin in Figure \ref{figure:gas_gap}. The cladding internal oxide layer covers the entire internal surface of the cladding and is the first barrier to corrosive species. 

\begin{figure}[ht]
\centering
\includegraphics[width=10cm]{images/gas_gap.png}
\caption{Schematic cross-section of a single PWR fuel pin with an expanded view of the pellet-gap-cladding system.}
\label{figure:gas_gap}
\end{figure}

\subsection{Effects of radiation on materials} 

While ionising radiation is always present in the environment as background radiation, the intensity of radiation in a nuclear reactor is so great that it causes significant engineering challenges because of how it affects change in reactor materials.

Radiation hardening (also known as radiation embrittlement) is a phenomenon which affects most materials subjected to ionising radiation. It is characterised by a loss of plasticity caused by radiation damage over time, leading to an increased risk of cracks and failure of components. While zirconium is a very useful nuclear material due to its neutron transparency, it is still susceptible to radiation damage \cite{Wisner1998}. Beyond certain levels of radiation damage, phase changes may also occur. %In \zirconia\ however,  

Amorphisation is another effect of radiation damage, which has been observed in the (Zr, U)O$_{2}$ bonding layer in fuel pins \cite{Nogita1997}. This is characterised by an overall loss of long-range order of atoms in a crystal. This typically occurs beyond a certain threshold of radiation damage depending on the material, called the critical amorphisation dose. Amorphisation causes a loss in long range crystallographic structure and a corresponding reduction in structural stability (amorphous materials have a higher Gibbs free energy than their crystalline counterparts), and causes swelling of the material \cite{Einfal2013}. In the literature, there is evidence of amorphisation in cubic stabilised \zirconia\ when bombarded with Cs$^{+}$ ions up to a fluence of $1 \times 10^{21}$ ions m$^{-2}$ \cite{amorphization2000wang}. However, no amorphisation is seen at an Xe$^{2+}$ fluence of $2 \times 10^{21}$ ions m$^{-2}$, or an I$^{+}$ fluence of $5 \times 10^{19}$ ions m$^{-2}$ \cite{sickafus1999radiation}.

One material phenomenon exclusive to nuclear reactor environments is neutron activation. The high free neutron environment leads to neutron capture in various nuclei within the reactor, including those of the fuel assemblies, coolant and RPV. There are many possible (n, x) reactions that may occur in materials experiencing a neutron flux, but of particular concern is transmutation of nuclei following a nuclear capture event. When a stable nucleus captures a neutron and becomes unstable, the nucleus may then emit particles to reduce its free energy, altering its atomic number in the process. This new element will have different chemical properties compared to the parent nucleus by virtue of a different electronic structure. This will change the elemental composition of a material, typically in an unfavourable way with dopants that negatively affect some desired material property. The extremely large number of nuclei relative to neutron flux means that this effect is small, though over time this becomes more significant due to the accumulation of these dopant elements.

In high-radiation environments, it is also possible for some molecules to be split by gamma photons above certain energies through the process of radiolysis. Corrosive fission products such as iodine will be present inside the fuel pin, but may exist in the form of (for example) CsI, which is not highly corrosive. Radiolysis however, decomposes CsI into Cs and I$_{2}$ vapour which can diffuse towards the cladding and promote cracking \cite{Konashi1983}.

\subsection{Fission products, their distribution and decay chains} \label{fissionyieldsection}

Nuclei which can undergo fission will produce daughter nuclei (fission products) with specific characteristics. At first, these nuclei will almost always be neutron-rich, as compared to their stable isotopes. This is the result of the higher neutron to proton (N/Z) ratios of larger nuclei. Figure \ref{figure:NZcurve} shows how the nuclei of abundant isotopes start with N/Z ratios of around 1 for small elements (e.g. \ch{He_{2}^{4}}, \ch{C_{6}^{12}}, \ch{O_{8}^{16}}), whereas larger elements will have isotopes with N/Z ratios approaching 1.6 (e.g. \ch{Pb_{82}^{208}}, \ch{Th_{90}^{232}}, \ch{U_{92}^{238}}).

\begin{figure}[ht]
\centering
\includegraphics[height=13cm]{images/Isotopes_and_half-life.png}
\caption[Plot of neutron number against proton number for nuclei with half-lives greater than 10${^{-8}}$ s.]{Plot of neutron number against proton number for nuclei with half-lives greater than 10${^{-8}}$ s. Taken from \cite{BenRG}.}
\label{figure:NZcurve}
\end{figure} 

Fissionable nuclei are typically very large and when fission occurs, the daughter nuclei will inherit a high N/Z ratio. These neutron-rich nuclei will generally decay by $\beta-$ particle emission to reduce their N/Z ratio and increase stability. One decay event is usually not enough to achieve a stable nucleus, so several decays as part of a decay chain are expected. An example is given for Te in Equations \ref{eqn:te_decay} and \ref{eqn:i_decay}.
\begin{gather}
\ch{Te^{134}_{52}} \xrightarrow[]{\beta-} \ch{I^{134}_{53}}+ e^{-} + \overline{\nu}_{e}
\label{eqn:te_decay} \\
\ch{I^{134}_{53}} \xrightarrow[]{\beta-} \ch{Xe^{134}_{54}} + e^{-} + \overline{\nu}_{e}
\label{eqn:i_decay}
\end{gather}
Another characteristic feature of fission products is that their masses are bi-modally distributed. Figure \ref{figure:fissionyield} shows calculated fission yields as a function of mass number, indicating a 40:60 rather than 50:50 mass distribution among daughter nuclei when heavy nuclei are fissioned. This is attributed to smaller nuclei (which have high binding energies per nucleon) separating from the nucleus first at the moment when fission occurs. This results in the majority of fission products centring around atomic masses of 95 and 135, producing a disproportionate number of nuclei such as Sr, Y and Zr on the low side and Te, I and Cs on the high side of the distribution. The distribution of fission products varies slightly depending on which nucleus is fissioned (e.g. U$^{233}$, U$^{235}$, Pu$^{239}$) and the energy of the fissioning neutron. Over time, transmutation of U$^{238}$ to Pu$^{239}$ and consumption of U$^{235}$ means that an increasing proportion of energy output will be due to fission of Pu$^{239}$, but fuel assemblies are typically retired before significant build-up of Pu$^{239}$ inventory and so the net yields of fission products will only change slightly over the life of the fuel.
% Iodine 0.1142684237, Tellurium 0.180582

\begin{figure}[ht!]
\centering
\includegraphics[height=12cm]{images/fissionyield.jpg}
\caption[Plot of the percentage yield of nuclei with a given mass following a fission event. Range of masses corresponding to isotopes of iodine shown in purple and isotopes of yttrium shown in green, based on Equation \ref{eqn:fission}.]{Plot of the percentage yield of nuclei with a given mass following a fission event. Range of masses corresponding to isotopes of iodine shown in purple and isotopes of yttrium shown in green, based on Equation \ref{eqn:fission}. Adapted from \cite{England1992}.}
% M. F. James, R. W. Mills and D. R. Weaver (1991) UKAEA Reports, AEA-TRS-1015, AEA-TRS-1018 
         %  and AEA-TRS-1019.
\label{figure:fissionyield}
\end{figure}

Iodine is an important fission product because it is known to corrode Zr metal. It is part of the Te $\rightarrow$ Cs decay chain, with most isotopes exhibiting half-lives ranging from a few seconds to several days. Select data on Te and I fission yields are presented in Table \ref{table:decaydata_chap1}. In total, the independent yield of I isotopes from U$^{235}$ fission is 10.4\%, while the independent yield of Te isotopes (I precursors) is 17.6\%, with \ch{Te_{52}^{134}} having the highest independent yield of all possible fission products (6.3\%). These particular elements will also usually be paired with Zr and Y fission products, the latter of which is a common phase stabiliser dopant in \zirconia .

\begin{table}[ht!] % Yields and half lives
\onehalfspacing
\caption{Independent fission product yields and half-lives for the major iodine isotopes and precursors in a thermal neutron reactor. Yields taken from the Joint Evaluated Fission and Fusion File (JEFF 3.3). All isotopes undergo single $\beta$- decay. Metastable states are included.}  \label{table:decaydata_chap1}
\begin{center}
\begin{tabular}{c c c}
\hline
Isotope & Independent Yield (\%) & Half-life \\
\hline
\ch{Te^{131}} & 0.126
 & 25.0 m \cite{auble1976nuclear} \\
\ch{I^{131}} & 0.00296
 & 8.02 d \cite{I131halflife} \\
\ch{Te^{132}} & 1.50 & 3.18 d \cite{Te132} \\
\ch{I^{132}} & 0.0284  & 2.30 h \cite{Te132} \\
\ch{Te^{133}} & 3.94 & 12.5 m \cite{khazov2011nuclear} \\
\ch{I^{133}} & 0.198  & 20.9 h \cite{I133} \\
\ch{Te^{134}} & 6.30 & 41.8 m \cite{Sonzogni2004} \\
\ch{I^{134}} & 0.767 & 52.5 m \cite{Sonzogni2004} \\
\ch{Te^{135}} & 3.48 & 19.0 s \cite{tellurium135halflife} \\
\ch{I^{135}} & 2.46 & 6.58 h \cite{tellurium135halflife} \\ 
\ch{Te^{136}} & 1.67 & 17.6  s \cite{Mccutchan2018} \\
\ch{I^{136}} & 3.04 & 83.4 s \cite{Mccutchan2018} \\
\ch{Te^{137}} & 0.484 & 2.49 s \cite{browne2007nuclear} \\
\ch{I^{137}} & 2.70 & 24.5 s \cite{browne2007nuclear} \\ 
\ch{Te^{138}} & 0.113 & 1.40 s \cite{chen2017nuclear} \\ 
\ch{I^{138}} & 1.24 & 6.26 s \cite{chen2017nuclear} \\ 
\hline
\end{tabular}
\end{center}
\end{table}

\subsection{Formation of plutonium and minor actinides}

Uranium fuel in LWRs is enriched to contain up to 5\% \ch{U^{235}}, which is a fissile isotope. The rest of the uranium is comprised of the more abundant \ch{U^{238}} which is a \emph{fertile} isotope. Fertile isotopes are not\footnote{\ch{U^{238}} has a thermal neutron fission cross section which is 7 orders of magnitude smaller than that of \ch{U^{235}}, so thermal fission of \ch{U^{238}} is insignificant.} fissioned directly, but can be converted into fissile isotopes through nuclear reactions. In the case of \ch{U^{238}}, the fissile isotope \ch{Pu^{239}} is produced through the following reactions:
\begin{gather}
\ch{U^{238}_{92}} + n^{1}_{0} \xrightarrow[]{absorption} \ch{U^{239}_{92}} \\
\ch{U^{239}_{92}} \xrightarrow[]{\beta-} \ch{Np^{239}_{93}} + e^{-} + \overline{\nu}_{e} \\
\ch{Np^{239}_{93}} \xrightarrow[]{\beta-} \ch{Pu^{239}_{94}} + e^{-} + \overline{\nu}_{e}
\end{gather}
UO$_{2}$ fuel pellets do not contain any plutonium before irradiation, however, production of \ch{Pu^{239}} and heavier isotopes during reactor operation is so significant that they can account for over 50\% of the fissile isotope inventory in a LWR \cite{OECD1989}. A fraction of the \ch{Pu^{239}} nuclei will capture additional neutrons, producing heavier isotopes of plutonium which may be fissile (e.g. \ch{Pu^{241}}). The production of fissile nuclei from fertile nuclei is known as \emph{breeding}.

Plutonium nuclei which are not burned through fission can either be extracted from spent fuel by reprocessing, or will undergo nuclear decay (specifically $\alpha$ or $\beta$- decay) and produce elements such as neptunium, americium and curium which are known as \emph{minor actinides}\footnote{In the nuclear power industry, uranium and plutonium are known as the major actinides.}. These minor actinides are produced from neutron irradiation of uranium (both \ch{U^{238}} and \ch{U^{235}}) and nuclear decay. The radiotoxicity of spent nuclear fuel is dominated by plutonium and the minor actinides, with global production rates of isotopes like \ch{Np^{237}}, \ch{Am^{241}} and \ch{Cm^{244}} measured in several metric tons per year \cite{Ewing2004}. 

In a nuclear fuel pin, the cladding serves as a physical barrier preventing the release of fission products and actinide wastes. The first barrier is the UO$_{2}$ matrix itself, however, volatile species can migrate to the pellet-cladding gap. This gap gradually closes during operation and the fuel pellet makes contact with the cladding, leading to mechanical and chemical interactions. This is discussed in detail in Chapter \ref{literature_review}.
\chapter{Crystallography and Point Defects}

\label{ch:crystallography}

\section{\zirconia\ phases and stabilisation}

\zirconia\ is unusual in exhibiting three commonly reported polytypes in its binary phase diagram (Figure \ref{figure:binary_phase_diagram}). Each will now be described and contrasted.

\begin{figure}[htp]
\centering
\includegraphics[height=9cm]{images/zro2_binary_phase.png}
\caption{Binary phase diagram of the Zr and O$_{2}$ system. Taken from \cite{Abriata1986TheSystem}.}
\label{figure:binary_phase_diagram}
\end{figure}

\subsection{Monoclinic}

A unit cell of monoclinic \zirconia\ is illustrated in Figure \ref{figure:coordination}. The dashed line (approximately 3.7\r{A} in length) shows the Zr-O bond which is broken when transitioning to monoclinic from the tetragonal phase.

\begin{figure}[htp] % Mono coordination figure
\centering
\includegraphics[height=6.2cm]{images/coordination.png}
\caption{A monoclinic zirconia unit cell indicating the two different oxygen bond coordinations. Small spheres represent oxygen ions while large spheres represent zirconium ions. Taken from \cite{Xia2010}.
\label{figure:coordination}}
\end{figure}

\begin{figure}[htp] % Mono Zr centre
\centering
\includegraphics[height=6cm]{images/zr_centre_mono.png}
\caption{Zirconium centre in monoclinic \zirconia\ showing nearest oxygen atoms and their respective bond co-ordinations. Zirconium atoms are shown in green and oxygen atoms in red.}
\label{figure:monoschottky}
\end{figure}


\begin{table}[htp]
\centering
\onehalfspacing
\caption{\zirconia\ crystal structures and their stable temperatures at 1 atm \cite{Howard1988}.}
\label{table:phases}
\begin{tabular}{ccc}
\hline
{Crystal Structure} & {Space Group}    & {Temperature Range (K)} \\ \hline
\multicolumn{1}{c}{Monoclinic} & \multicolumn{1}{c}{$P2_1/c$} & \multicolumn{1}{c}{$T$ \textless\ 1440}     \\
\multicolumn{1}{c}{Tetragonal} & \multicolumn{1}{c}{$P4_2/nmc$} & \multicolumn{1}{c}{1440 \textless\ $T$ \textless\ 2640}        \\
\multicolumn{1}{c}{Cubic} & \multicolumn{1}{c}{$Fm\overline{3}m$}     & \multicolumn{1}{c}{2640 \textless\ $T$ \textless\ 2950}      \\ \hline
\end{tabular}
\end{table}

\subsection{Tetragonal}

\subsection{Cubic}

\subsection{Other phases}

\subsubsection{Cotunnite}

Two orthorhombic phases of \zirconia\ have also been observed at high pressures.

\begin{figure}[htp]
  \centering
      \includegraphics[height=9cm]{images/cotunnite_structure.png}
  \caption{Illustration of the OII cotunnite crystal structure of \zirconia . Zirconium and oxygen ions are shaded dark and light respectively. Taken from \cite{Haines1997CharacterizationHafnia}.}
  \label{fig:cotunnite_structure}
\end{figure}

\subsubsection*{Volume expansion}

The phase transitions in \zirconia\ are accompanied by a change in volume, where the monoclinic phase is the least dense and the cubic phase is the most dense (see Figure \ref{figure:zrobonddistance}). This is especially significant in the case of the martensitic t-\zirconia\ to m-\zirconia\ transition, where the volume increases by around 9\% \cite{Gupta1977}. This has substantial implications for the creation and opening of cracks as \zirconia\ is a ceramic material with low toughness. This is especially relevant in a reactor scenario where temperature cycling (shutdown/startup or load-following behaviour) may lead to fatigue if the phase transition threshold is passed.

Another consequence of this large volume expansion is that a significant hysteresis effect is observed in the monoclinic/tetragonal phase transition, as shown in Figure \ref{fig:phasediagram}. 
%as the resulting coherency strain is likely to result in reduced mobility of fission products that have been embedded in the bulk crystal. 

\begin{figure}[htp]
  \centering
      \includegraphics[height=10cm]{images/zirconiaphasediagram.png}
  \caption{Pressure-temperature phase diagram for \zirconia . Dash-dotted lines represent more recent data. Diamonds mark transition points during an increase in pressure/temperature, while open circles are used for a decrease in pressure/temperature. Solid circles represent transition points for a fresh, single crystal sample. Taken from \cite{gando2011partial}. \label{fig:phasediagram}}
\end{figure}

\begin{figure}
\begin{center}
\begin{tikzpicture}
	\begin{axis}
		[width=12cm, xlabel={Nearest neighbour Zr-O bond distance (\r{A})}, ylabel={Relative occurrence}, ymin=0, ymax=140, xmin=2.0, xmax=2.50, legend style={{draw=}, at={(0.95,0.95)}, anchor=north east, legend columns=1}]
		\addplot[no marks] table [x=zr_o_dist, y=monoclinic,]{dat/zr_o_bond_distances.dat}; \addlegendentry{Monoclinic};
        \addplot[no marks, dashed] table [x=zr_o_dist, y=tetragonal, ]{dat/zr_o_bond_distances.dat}; \addlegendentry{Tetragonal};
        \addplot[no marks, densely dotted] table [x=zr_o_dist, y=cubic,]{dat/zr_o_bond_distances.dat}; \addlegendentry{Cubic};
			\end{axis}
		\end{tikzpicture}
		\caption{Density plot of the nearest neighbour Zr-O bond distances in \zirconia\ for each crystal structure. Specific volumes from DFT simulations are 11.99 \r{A}$^{3}$ion$^{-1}$, 11.51 \r{A}$^{3}$ion$^{-1}$, and 11.13 \r{A}$^{3}$ion$^{-1}$ for monoclinic, tetragonal, and cubic phases respectively.}
		\label{figure:zrobonddistance}
	\end{center}
\end{figure}


\subsection{Pressure stabilisation (isochoric + autostabilisation)}

The tetragonal and cubic phases of \zirconia\ are stabilised at high pressure. Since the oxide has a larger volume than the underlying metal (pilling-bedworth ratio of 1.5X), the growth of the oxide will itself impose stresses which may stabilise the tetragonal phase.

\subsection{Dopant stabilisation (lower valence cations)}

Some dopants will also stabilise the tetragonal and cubic phases of \zirconia. The most technologically significant of which is yttrium, which at concentrations of 15\% (atomic), fully stabilises the cubic phase. Zirconia stabilised this way is known as yttria-stabilised zirconia (YSZ). The way this works is by trivalent yttrium promoting the inclusion of charge compensating oxygen vacancy defects (see equation XXX). This works in a similar way with several other cation dopants such as trivalent scandium and divalent magnesium.

\section{Point Defects}

\subsection{Kr\"{o}ger-Vink notation}

Kr\"{o}ger-Vink notation \cite{kroger1956relations} is used throughout this thesis to describe defects. It is widely used in physical chemistry and is a useful shorthand for describing chemical reactions where conservation of mass, charge and lattice sites is required. The notation syntax is of the form \ch{x^{y}_{z}}, where x is the substituted atom or missing atom (i.e. a vacancy V), y is the charge of the defect (relative to the lattice species that originally occupied the site) and z is the site the defect occupies. Positive and negative charges are indicated with dots (\ch{^{*}}) and dashes (\ch{^{'}}) respectively, otherwise a cross (\ch{^{x}}) is used to denote a neutral defect. The site may be either a lattice site (such as Zr or O in \zirconia ) or an interstitial site ($i$). Table \ref{table:krogervink} shows examples of several different types of defects and their respective Kr\"{o}ger-Vink notation.

\begin{table}[htp] % Kroger-Vink notation table
\onehalfspacing
\centering
\caption{Examples of Kr\"{o}ger-Vink notation for several defects in \zirconia .}
\label{table:krogervink}
\begin{tabular}{cc}
\hline
Defect & Kr\"{o}ger-Vink Notation \\ \hline
Anion vacancy & \ch{V_{O}^{**}} \\
Cation vacancy & \ch{V_{Zr}^{''''}} \\
Anion interstitial & \ch{O_{i}^{''}} \\
Cation interstitial & \ch{Zr_{i}^{****}} \\
Iodine (I$^{-}$ anion) on oxygen site & \ch{I_{O}^{*}} \\ \hline
\end{tabular}
\end{table}


\chapter{Computational Methodology}

\label{ch:compmethodology}

\section{Density functional theory} \label{section:dft}

Quantum mechanics is currently the most complete modern theory which describes the behaviour of matter at the length scale of atoms. It can be used to predict things such as the energy levels of atoms, the interactions of light with matter, and the thermodynamic stability of systems of atoms. Ideally, the mathematical formalisms of quantum mechanics would be used to predict the properties and behaviour of all possible types of molecules and materials. In reality, this is very difficult to achieve, requiring several approximations and abstractions in order to produce methods which sacrifice some degree of physical accuracy in order to be computationally tractable. Currently, the most successful approach to predict the behaviour of most solids is provided by density functional theory.
% The mathematical formalisms of quantum mechanics must themselves be discretised to be used in computational simulation methods.

\subsection{The Schr\"{o}dinger equation}

The time-independent Schr\"{o}dinger equation is used to find the total energy of a system:
\begin{equation}
E\Psi(\textbf{r}) = \hat{H}\Psi(\textbf{r})
\label{equation:schrodinger}
\end{equation}

where $E$ is the total energy of the system, $\Psi$ is the wave function associated with the electrons, and $\hat{H}$ is the energy Hamiltonian operator. $\hat{H}$ includes the kinetic energy contributions ($\hat{T}$) and potential energy contributions ($\hat{V}$), shown in atomic units in equations \ref{equation:kineticcontribution} and \ref{equation:potentialcontribution} respectively:
\begin{gather}
\hat{H} = \hat{T} + \hat{V} \label{equation:hamiltonian}\\
\hat{T} = -\sum_i{\frac{1}{2}}\nabla^2_{r_i} - \sum_i{\frac{1}{2M_i}}\nabla^2_{R_i} \label{equation:kineticcontribution} \\
\hat{V} = \sum_{i,j=i+1}{\frac{1}{2|r_i - r_j|}} + \sum_{i,j=i+1}{\frac{Z_i Z_j}{2|R_i - R_j|}} - \sum_{i,j}{\frac{Z_i}{2|R_i - r_j|}} \label{equation:potentialcontribution}
\end{gather}

where $r_{i}$ is the position of electron $i$, $R_{i}$ is the position of nucleus $i$ and $M_{i}$ is the mass of nucleus $i$. Thus, the second term on the right of equation \ref{equation:kineticcontribution} relates to the kinetic energy of any associated nuclei, and the first term to electrons. 

If $\Psi(\textbf{r})$ is the wave function, the electron density at position \textbf{r} ($\rho(\textbf{r})$) is given by:
\begin{equation}
\rho(\textbf{r}) = \Psi(\textbf{r})^2
\end{equation}

\subsection{Kohn-Sham Method} \label{section:kohnsham}

DFT was developed by Kohn and Sham in 1964 \cite{Kohn1965} as an ab initio method for predicting $\rho(\textbf{r})$ associated with an ensemble of atoms. The Kohn-Sham Hamiltonian (Equation \ref{equation:kohnsham}) is used in the Schr\"odinger equation:
\begin{equation}
\hat{H}(\rho(\textbf{r})) = E_{KE}(\rho(\textbf{r})) + E_{P}(\rho(\textbf{r})) + E_{XC}(\rho(\textbf{r}))
\label{equation:kohnsham}
\end{equation}

where $E_{KE}$ and $E_{P}$ are the kinetic and potential energy functionals (functions of functions), $E_{XC}$ is the exchange correlation functional, and \textbf{r} is the position vector. The main approximation is to consider that the electrons only interact with nuclei and the average field generated by all other electrons, and not other electrons explicitly, thus allowing all the terms to be evaluated using the electron density rather than position. An exchange correlation term is then used to include the non-classical electron-electron interactions, namely electron exchange and correlation. Additionally, the exchange correlation term includes the difference in kinetic energy due to the use of non-interacting electrons. While Kohn and Sham proposed an exchange-correlation functional in the Hamiltonian, a general form of the functional has not yet been found. Several forms have been considered, each with strengths and weaknesses when applied to different systems. One basic form of the functional which is frequently used is the LDA \cite{Kohn1965}:
\begin{equation}
E_{LDA}(\rho(\textbf{r})) = \int\rho(\textbf{r})e_{uniform}(\rho(\textbf{r}))dr
\label{equation:LDA}
\end{equation}

where $e_{uniform}$ is the normalised exchange-correlation energy of a uniform electron gas (an idealised system). This exchange-correlation functional generates accurate results in materials such as metals where the electron density is relatively uniform, while systems with more rapidly changing electron densities (e.g. highly ionic materials) require more complex functionals. A natural extension of the LDA is to also take into account the gradient of the electron density, thus allowing a smoother functional fit when electron density is highly variable as a function of position. Such functionals are collectively referred to as GGAs \cite{Langreth1980, Langreth1983, Becke1988, perdew2008restoring}. One GGA which has enjoyed widespread use for many different types of systems is the Perdew-Burke-Ernzerhof (PBE) GGA \cite{Perdew1996}. The accuracy of this functional when modelling solid phase systems is well-established, and its frequent use in DFT studies provides ample reference material for comparing results. After conducting several convergence tests (see § \ref{section:convergence}), the PBE GGA was chosen as the exchange-correlation functional to be used for all calculations in this thesis.

\subsubsection{Born-Oppenheimer approximation}

The Born-Oppenheimer approximation is a two-step process for evaluating atomic forces which greatly reduces the computational costs of atomistic simulations. It exploits the large difference in mass between nuclei and electrons in order to separate their interactions. This allows us to decompose the total wave function into a product of an electronic wave function and a nuclear wave function via a separation of variables approach. The first step involves ignoring the kinetic energy contribution of nuclei by assuming they are stationary, thus the nuclear kinetic energy term in Equation \ref{equation:kineticcontribution} can be removed. The stationary nuclei assumption also simplifies the nuclear-nuclear Coulombic repulsion term in Equation \ref{equation:potentialcontribution} because $|R_i - R_j|$ becomes a constant throughout the calculation. An electronic Schr\"{o}dinger equation is then solved where electronic positions are variables and nuclear positions are fixed parameters. This solution contains information of the shape of the electronic orbitals. The next step is to take the electronic distribution and calculate the resultant forces on the nuclei. The nuclear positions are then modified to minimise these forces, followed by feeding these nuclear positions back into the electronic Schr\"{o}dinger equation to obtain the new electronic distribution. This process is repeated until the required convergence criterion (such as energy change per iteration and forces on nuclei) are satisfied.

\subsection{Pseudopotentials}

The electron-electron interaction component of the potential energy presents a problem when it comes to scaling experimental models. The number of terms in this interaction grows quadratically with the number of electrons in the system, quickly becoming computationally intractable for even small systems. However, it is known that in chemical reactions, the majority of chemical behaviour is determined by relatively few valence electrons, while the more numerous core electrons have a far smaller effect. 

Consider the zirconium atom with 40 electrons, of which 4 (4$d^2$5$s^2$) are typically involved in bonding and chemical reactions. By considering only these valence electrons for Coulombic-term calculations, we reduce the system size by 90\%, which provides a more than tenfold reduction in computational requirements.

Although the core electrons do not participate in chemical reactions, they still influence the properties of the atom, such as the atomic radius. Instead of modelling the core electrons explicitly, we can approximate their aggregate effect with a potential energy function. This is what we aim to achieve by using the pseudopotential method. An example indicative pseudopotential is shown in figure \ref{figure:pseudopotential}.

\begin{figure}[ht] % Pseudopotential Image
\begin{center}
\includegraphics[height=10cm]{images/pseudopotential.png}
\end{center}
\caption[Sketch of an all-electron potential V$_{AE}$ and a pseudopotential V$_{PS}$ with their corresponding wave functions. r$_{cut}$ indicates the radius beyond which both the potentials and their wave functions are the same.]{Sketch of an all-electron potential V$_{AE}$ and a pseudopotential V$_{PS}$ with their corresponding wave functions. r$_{cut}$ indicates the radius beyond which both the potentials and their wave functions are the same. Adapted from \cite{Payne1992}.}
\label{figure:pseudopotential}
\end{figure}

Figures \ref{figure:o_pp} and \ref{figure:zr_pp} show the actual pseudopotentials used throughout this work for oxygen and zirconium respectively. The potentials are shown broken down by the electronic sub-shells occupied by the valence electrons. The pseudopotentials are shown in order of increasing sub-shell energies, thus the 5s electron orbitals are filled before the 4d orbitals in zirconium. Two lines for the all-electron wavefunction are shown, corresponding to the different electron angular momenta.

\begin{figure}[ht] % Oxygen pseudopotentials
\begin{center}
\begin{tikzpicture}
	\begin{groupplot}[group style={group size=1 by 2}, width=14cm, height=7cm]
	\nextgroupplot[
    axis y line*=middle, axis x line*=bottom, y label style={at={(-0.03,-0.1)}}, ylabel=Wavefunction $\Psi$, axis y line shift=0.5cm,
    xtick=\empty, ymin=-1.1, ymax=1.1, ytick={-1, -0.5, 0, 0.5, 1}, xmax=2.5, 
    x axis line style={white}]
         \addplot[no marks, dashed, draw=red!80!white] table [x=distance, y=O_AE_s1,]{dat/o_pp.dat}; 
		\addplot[no marks, draw=red!80!white] table [x=distance, y=O_pp_s1,]{dat/o_pp.dat}; 
		\addplot[no marks, dashed, draw=blue!80!white] table [x=distance, y=O_AE_s2,]{dat/o_pp.dat}; 
		\addplot[no marks, draw=blue!80!white] table [x=distance, y=O_pp_s2,]{dat/o_pp.dat};
		\node at (0.1, 0.99) {\textbf{O (2s)}};
		\draw[black] % horizontal line
				(-0.11, 0)
				-- % = line-to
				++ % = calculate a vector sum
				(axis direction cs:2.61, 0);
		\draw[black, dotted] % Vertical line
				(1.299,1)
				-- % = line-to
				++ % = calculate a vector sum
				(axis direction cs:0,-2);
				
    \nextgroupplot[
    axis y line*=middle, axis x line*=bottom, axis x line shift=0, x axis line style={white}, axis y line shift=0.5cm,
    ymin=-1.1, ymax=1.1, ytick={-1, -0.5, 0, 0.5, 1}, xtick={0, 0.5, 1, 1.5, 2, 2.5}, xmax=2.5, xlabel=Distance from nucleus (Bohr)]
         \addplot[no marks, dashed, draw=red!80!white] table [x=distance, y=O_AE_p1,]{dat/o_pp.dat}; 
		\addplot[no marks, draw=red!80!white] table [x=distance, y=O_pp_p1,]{dat/o_pp.dat}; 
		\addplot[no marks, dashed, draw=blue!80!white] table [x=distance, y=O_AE_p2,]{dat/o_pp.dat}; 
		\addplot[no marks, draw=blue!80!white] table [x=distance, y=O_pp_p2,]{dat/o_pp.dat}; 
		\node at (0.1, 0.99) {\textbf{O (2p)}};
		\draw[black] % horizontal line
				(-0.11, 0)
				-- % = line-to
				++ % = calculate a vector sum
				(axis direction cs:2.61, 0);
		\draw[black, dotted] % vertical line
				(1.299,1)
				-- % = line-to
				++ % = calculate a vector sum
				(axis direction cs:0,-2);
	\end{groupplot}
		\end{tikzpicture}
		\caption{Plots of the valence $s$ and $p$ orbital potentials for oxygen with two projectors per angular momentum. Dashed lines indicate the all-electron potentials while solid lines indicate the corresponding pseudopotential. Dotted vertical line marks the radius beyond which the potentials match.}
		\label{figure:o_pp}
	\end{center}
\end{figure}

\clearpage

\begin{figure}[h!] % Zirconium pseudopotentials
\begin{center}
\begin{tikzpicture}
	\begin{groupplot}[group style={group size=1 by 3}, width=14cm, height=7cm]
	\nextgroupplot[ % 4p
    axis y line*=middle, axis x line*=bottom, axis y line shift=0.5cm,
    xtick=\empty, ymin=-1.2, ymax=1.2, ytick={-1, -0.5, 0, 0.5, 1}, xmax=2.5, 
    x axis line style={white}]
         \addplot[no marks, dashed, draw=red!80!white] table [x=distance, y=Zr_AE_p1,]{dat/zr_pp.dat}; 
		\addplot[no marks, draw=red!80!white] table [x=distance, y=Zr_pp_p1,]{dat/zr_pp.dat}; 
		\addplot[no marks, dashed, draw=blue!80!white] table [x=distance, y=Zr_AE_p2,]{dat/zr_pp.dat}; 
		\addplot[no marks, draw=blue!80!white] table [x=distance, y=Zr_pp_p2,]{dat/zr_pp.dat};
		\node at (0.1, 0.99) {\textbf{Zr (4p)}};
		\draw[black] % horizontal line
				(-0.11, 0)
				-- % = line-to
				++ % = calculate a vector sum
				(axis direction cs:2.61, 0);
		\draw[black, dotted] % Vertical line
				(2.105,1)
				-- % = line-to
				++ % = calculate a vector sum
				(axis direction cs:0,-2);
				
	\nextgroupplot[ % 5s
    axis y line*=middle, axis x line*=bottom, axis y line shift=0.5cm, ylabel=Wavefunction $\Psi$,
    xtick=\empty, ymin=-1.2, ymax=1.2, ytick={-1, -0.5, 0, 0.5, 1}, xmax=2.5, 
    x axis line style={white}]
         \addplot[no marks, dashed, draw=red!80!white] table [x=distance, y=Zr_AE_s1,]{dat/zr_pp.dat}; 
		\addplot[no marks, draw=red!80!white] table [x=distance, y=Zr_pp_s1,]{dat/zr_pp.dat}; 
		\addplot[no marks, dashed, draw=blue!80!white] table [x=distance, y=Zr_AE_s2,]{dat/zr_pp.dat}; 
		\addplot[no marks, draw=blue!80!white] table [x=distance, y=Zr_pp_s2,]{dat/zr_pp.dat};
		\node at (0.1, 0.99) {\textbf{Zr (5s)}};
		\draw[black] % horizontal line
				(-0.11, 0)
				-- % = line-to
				++ % = calculate a vector sum
				(axis direction cs:2.61, 0);
		\draw[black, dotted] % Vertical line
				(2.105,1)
				-- % = line-to
				++ % = calculate a vector sum
				(axis direction cs:0,-2);
				
    \nextgroupplot[ % 4d
    axis y line*=middle, axis x line*=bottom, axis x line shift=0, x axis line style={white}, axis y line shift=0.5cm,
    ymin=-1.2, ymax=1.2, ytick={-1, -0.5, 0, 0.5, 1}, xtick={0, 0.5, 1, 1.5, 2, 2.5}, xmax=2.5, xlabel=Distance from nucleus (Bohr)]
         \addplot[no marks, dashed, draw=red!80!white] table [x=distance, y=Zr_AE_d1,]{dat/zr_pp.dat}; 
		\addplot[no marks, draw=red!80!white] table [x=distance, y=Zr_pp_d1,]{dat/zr_pp.dat}; 
		\addplot[no marks, dashed, draw=blue!80!white] table [x=distance, y=Zr_AE_d2,]{dat/zr_pp.dat}; 
		\addplot[no marks, draw=blue!80!white] table [x=distance, y=Zr_pp_d2,]{dat/zr_pp.dat}; 
		\node at (0.1, 0.99) {\textbf{Zr (4d)}};
		\draw[black] % horizontal line
				(-0.11, 0)
				-- % = line-to
				++ % = calculate a vector sum
				(axis direction cs:2.61, 0);
		\draw[black, dotted] % vertical line
				(2.105,1)
				-- % = line-to
				++ % = calculate a vector sum
				(axis direction cs:0,-2);
	\end{groupplot}
		\end{tikzpicture}
		\caption{Plots of the valence $s$, $p$ and $d$ orbital potentials for zirconium with two projectors per angular momentum. Dashed lines indicate the all-electron potentials while solid lines indicate the corresponding pseudopotential. Dotted vertical line marks the radius beyond which the potentials match.}
		\label{figure:zr_pp}
	\end{center}
\end{figure}

%\begin{figure}[ht] % Oxygen pseudopotential
%\begin{center}
%\includegraphics[height=7.3cm]{images/oxygen_otf_pp.png}
%\end{center}
%\caption{Plots of the valence $s$ and $p$ orbital potentials for oxygen with two projectors per angular momentum. Dashed lines indicate the all-electron potentials while solid lines indicate the corresponding pseudopotential. Dotted vertical line marks the radius beyond which the potentials match.}
%\label{figure:oxygen_pseudopotential}
%\end{figure}

%\begin{figure}[ht] % Zirconium pseudopotential
%\begin{center}
%\includegraphics[height=12cm]{images/zirconium_otf_pp.png}
%\end{center}
%\caption{Plots of the valence $s$, $p$ and $d$ orbital potentials for zirconium with two projectors per angular momentum. Dashed lines show the all-electron potentials while solid lines indicate the corresponding pseudopotential. Dotted vertical line marks the radius beyond which the potentials match.}
%\label{figure:zirconium_pseudopotential}
%\end{figure}


\section{Periodic boundaries}

\subsection{Bloch's theorem}

The repeating nature of a crystal structure, defined by the lattice vectors plus a basis set of atoms that are repeated, is well-suited for computer models. It allows us to define periodicity in three dimensions for a given unit cell. An example of this periodicity is illustrated in Figure \ref{figure:periodicboundary} in two dimensions. A model based on this periodicity is justified as follows:

\begin{itemize}
\item Nuclei are arranged in a periodically repeating pattern, thus their potentials acting on electrons are also periodic.
\item If the potential is periodic, it follows that the electron density is also periodic.
\item The electron density is equivalent to the square of the wave function magnitude, thus the magnitude of the wave function is also periodic.
\end{itemize}

\begin{figure}[ht] % Periodic boundary image
\begin{center}
\includegraphics[width=\linewidth]{images/PeriodicBoundaryThesis.png}
\end{center}
\caption{Two dimensional illustration of periodic boundary around a primitive cell.}
\label{figure:periodicboundary}
\end{figure}

Knowing that the magnitude of the wave function is periodic greatly simplifies the calculation process; only one `period' of the function needs to be evaluated. However, the phase of the wave function can take any of an infinite number of values and still satisfy the periodicity condition. At this point, we consider Bloch's theorem which states that the possible wave functions are all quasi-periodic, and thus the wave function can be expressed as:  % Patrick's Fig 2.3 is really useful for describing this
\begin{equation}
\label{equation:bloch}
\psi_k(\textbf{r}) = e^{i\textbf{k}.\textbf{r}}u_k(\textbf{r})
\end{equation}

Where $\psi_k(\textbf{r})$ is the wave function evaluated at position \textbf{r}, $e^{i\textbf{k}.\textbf{r}}$ is an arbitrary phase factor, and $u_k(\textbf{r})$ is a periodic function with the same periodicity as the wave function. Solutions to this equation exist for any value of \textbf{k} and so the general solution can be expressed as an integral over the first Brillouin zone, the primitive lattice cell in reciprocal space. Instead of evaluating the integral over the range of \textbf{k} (a computationally costly task as it is done for many wave functions), a sum of values at discrete points, known as \textbf{k}-points, is used. This approximation is valid because the wave function varies slowly over \textbf{k}, thus allowing the integral to be approximated with several appropriately spaced \textbf{k}-points. In general, a finer \textbf{k}-point grid results in increased accuracy, but at an increased computational cost \cite{Hasnip2010}. For all DFT calculations in this thesis, a Monkhorst-Pack sampling scheme \cite{Monkhorst1976} was used for Brillouin zone integration, with a minimum \textbf{k}-point separation of 0.09 \r{A}$^{-1}$.

\subsection{Plane-waves}

The electron density of a system is described in the context of a basis set. A basis set is a collection of functions (known as basis functions) which can be combined to produce some relevant output, typically the mathematical description for the shape of an electron orbital. For example, any sound wave can be generated from a combination of sine functions (basis functions). 

The purpose of a basis set in DFT calculations is to describe the varying amplitude of the electron density in space. Any complete basis set (e.g. plane-wave, correlation-consistent, split-valence) may be used to represent the behaviour of electron orbitals, but a plane-wave method was chosen due to its greater suitability for periodic systems (plane-waves are intrinsically periodic). Since the electron densities are represented by a finite sum of plane-waves with different energies, a truncation error will be incurred. Plane-waves of higher energies provide a smaller contribution to the overall density, so only plane-wave up to a chosen cut-off energy value are considered in order to reduce computational requirements. An appropriate plane-wave cut-off energy must therefore be determined through a convergence test. 

%Figure \ref{Figure:cutoffconvergence} shows the first convergence study where the total energy of simulations with various values of $E_{cutoff}$ were compared to a highly converged value, and then plotted on a log scale to see how precision is improved at larger values.


\section{Computational details}
\subsection{Cell dimensions and initialisation}

A supercell method is used for the study of various defects. The first step is to create a unit cell of \zirconia\ in each of the three crystal structures. Each unit cell is then fully relaxed through a geometry optimisation process (see § \ref{geometry_optimisation_method}). The resulting cell is used to construct supercells through tessellation in three dimensions, before being fully relaxed again. In this way, we generate systems with up to ten times as many atoms as the unit cell (supercell details can be found in Table \ref{table:supercells}). This is necessary because introducing defects into a small unit cell will result in the defect interacting with itself across the periodic boundary. A supercell increases the distance between the defect and its periodic image, using the bulk material as an interaction buffer. 

When constructing a supercell, it is important to consider making the supercell equally large in all directions, such that any directional bias in defect-defect interaction is minimised. Larger supercells carry an increased computational cost when running calculations, limiting the sizes we can achieve. For example, a constant-volume defect calculation with 300 atom supercells will take upwards of 500 hours to complete (on the Imperial College HPC using four 32-core nodes), whereas the equivalent 100 atom supercell will take just 72 hours (fully relaxed calculations are even more computationally expensive).


\begin{table}[ht] % Supercell details
\doublespacing
\centering
\caption{Composition of the supercells in terms of the number of individual unit cells stacked in each direction.} % Unit cells were stacked in such a way as to produce the most cubic supercell in order to minimise directional defect-defect interactions.}
\vspace*{2mm}
\label{table:supercells}
\begin{tabular}{cccccccc}
\hline
\multirow{2}{*}{{\bf \begin{tabular}[c]{@{}c@{}}Crystal \\ Structure\end{tabular}}} & \multicolumn{3}{c}{{\bf No. unit cells}} & \multicolumn{3}{c}{{\bf Supercell size (\AA)}} & \multirow{2}{*}{{\bf \begin{tabular}[c]{@{}c@{}}No.\\ atoms\end{tabular}}} \\ \cline{2-7}
 & \hspace{0.25 cm} a \hspace{0.2 cm} & b & c & a \hspace{0.0 cm} & b & c \hspace{0.35 cm} &  \\ \hline
\begin{tabular}[c]{@{}c@{}}Monoclinic\\ ($P2_1/c$)\end{tabular} & 2 & 2 & 2 & 10.37 & 10.47 & 10.75 & 96 \\ \hline
\begin{tabular}[c]{@{}c@{}}Tetragonal\\ ($P4_2/nmc$)\end{tabular} & 3 & 3 & 2 & 10.85 & 10.85 & 10.56 & 108 \\ \hline
\begin{tabular}[c]{@{}c@{}}Cubic\\ ($Fm\overline{3}m$)\end{tabular} & 2 & 2 & 2 & 10.22 & 10.22 & 10.22 & 96 \\ \hline
\end{tabular}
\end{table}

\subsection{Geometry optimisation} \label{geometry_optimisation_method}

The geometry optimisation task in CASTEP follows a simple steepest-descent algorithm which attempts to satisfy certain convergence criteria, depending on the constraints applied to the system. This is an iterative process which takes an initial system state, modifies ion positions slightly and then calculates the difference in properties between the states to check for convergence. 

The variational principle in quantum mechanics tells us that the lowest system energy calculated is always an upper bound for the ground state energy, thus providing a way to check if modifications to the system are actually optimising the geometry. The exception is when the system converges upon a local minimum, which may not be an experimentally observed state. This can be avoided to some extent by having good initial ion placement from which to optimise.

\subsection{Convergence criteria for geometry optimisation} \label{convergence_criteria}

Four convergence criteria are used for the geometry optimisation tasks throughout this work, one of which is only used when performing constant-pressure calculations, such as when a supercell is being fully relaxed. These criteria are evaluated with respect to the previous iteration during the geometry optimisation task:

\begin{itemize}
\item \emph{Change in energy per ion}: The largest change in the energy per ion between iterations must be below $10^{-5}$ eV. Below this value, the total energy improvement towards the ground state for a 100 atom supercell is less than 0.001 eV, and is therefore considered converged.
\item \emph{Maximum force on an ion}: The maximum force requirement on any single ion in an iteration must be below $10^{-2}$ eV/\r{A}. This is required to make sure that the ion position will not change significantly in the following iteration, possibly bringing another convergence criterion above its threshold.
\item \emph{Maximum change in ion position}: This must be below $5 \times 10^{-4}$ \r{A} between iterations to be considered converged. This criterion specifies the maximum `rattle' of the ion that is tolerated once the minimum energy is reached (i.e. displacements above this value may still be important for achieving a correct atomic configuration). 
\item \emph{Maximum stress (constant-pressure only)}: During unconstrained relaxation, the maximum change in stress between iterations should be below 50 MPa. This is necessary to avoid large deviations which may distort the symmetry of the supercell, resulting in anomalous energy values.
\end{itemize}

Using these convergence criteria, non-defective supercells of \zirconia\ were relaxed under constant pressure. The resulting structure was used as the starting point to which defects were introduced, and subsequently relaxed again, this time under constant volume conditions to simulate low defect concentrations \cite{Murphy2014, Bell2015}. Finally, all DFT calculations on doped and defective structures in this thesis employed the Pulay method for density mixing \cite{Pulay1980} to take into consideration changes in electronic behaviour of the system caused by the defect and to speed up convergence.

\subsection{Charged cell correction} \label{charged_cell_correction}

When calculating the energy of a defect with an overall non-zero charge, this charge introduces a systematic error in the energy value which is a function of the charge magnitude. This is typically the case in high band-gap materials such as \zirconia\ where electron mobility is far lower than in metals or semiconductors, allowing defects such as \ch{V_{O}^{**}} and \ch{V_{Zr}^{''''}} to be thermodynamically stable in the lattice. 

The source of the error from charged defects is self-interaction across the periodic boundary, made necessary by the finite cell size. A common solution is to append a Makov-Payne correction term when calculating formation energies of defects \cite{Makov1995, Makov1996}. This works well in many cases, but does not take into consideration the anisotropy in the material's dielectric properties, as is the case in tetragonal \zirconia\ due to the non-unity lattice $c/a$ ratio. These effects are better captured when using a screened Madelung correction \cite{Murphy2013}. This method provides a more complete description of the dielectric properties by utilising a dielectric tensor rather than a single value of the dielectric constant (or relative permittivity). Dielectric tensors for the different phases of \zirconia\ were taken from the literature \cite{Zhao2002a, Zhao2002}. The screened Madelung correction is therefore used in preference to a Makov-Payne correction throughout this thesis.

\subsection{Helmholtz free energy} \label{helmholtz_method}

In order to examine the relationship between temperature and energy for the different \zirconia\ phases, phonon calculations were performed in CASTEP using a method outlined by Burr \emph{et al.} \cite{burr2015crystal,jackson2016resolving}. This entails using the harmonic approximation to determine the shape of the potential well that an atom sits in. The potential well is approximated by a spherically symmetric harmonic well, centred at an atom's equilibrium position in the lattice. At a temperature of 0 K, an atom will occupy the lowest region of its potential well, known as the ground state (though they will still have energy, known as zero-point energy). As temperature increases, the atom will sometimes occupy higher energy states in the potential well due to increased thermal vibrations, moving from its equilibrium position. The total energy ($A(T, V)$) of this system, known as the Helmholtz free energy, is calculated using the internal energy ($U(V)$), vibrational enthalpy ($H_{v}(T, V)$), vibrational entropy ($S_{v}(T, V)$) and configurational entropy ($S_{conf}$):
\begin{equation} \label{vibrational}
A(T, V) = U(V) + H_{v}(T, V) - TS_{v}(T, V) - TS_{conf} 
\end{equation}
where $T$ is temperature and $V$ is volume. The configurational entropy is calculated using Boltzmann statistics:
\begin{equation}
S_{conf} = k_{B}\ch{ln}(\Omega)
\end{equation}
where $k_{B}$ is the Boltzmann constant and $\Omega$ denotes the number of possible configurations (i.e. valid permutations of energy level occupancy). The vibrational terms in Equation \ref{vibrational} are obtained by performing a constant-volume phonon calculation in CASTEP and then integrating over the resulting phonon density of states (DOS). This is done over a range of temperatures for each crystal structure of \zirconia .

\subsection{Incorporation energies}

The inner oxide of the fuel cladding will be highly defective due to radiation damage, resulting in a high concentration of pre-existing intrinsic defect sites relative to the concentration of fission products. We therefore consider the energy of fission product incorporation on to these existing defect sites. The energies to incorporate atoms at interstitial and substitutional sites in \zirconia\ were calculated from the set of defective and perfect supercell DFT energies. For iodine, incorporation energies were established to place atoms into vacancy sites of different charge to generate defects from \ch{I_{O}^{x}} to \ch{I_{O}^{**}}, and \ch{I_{Zr}^{x}} to \ch{I_{Zr}^{''''}}. I was also incorporated onto the interstitial sites.

The incorporation energy equation for iodine uses $\frac{1}{2}$I$_{2}$ as the reference state of iodine, while Te, Xe and Cs use the DFT energy calculated as a single atom in a large cell:
\begin{equation}
\label{interstitial_incorp_equation}
E_{inc}(\ch{I_{$i$}^{x}}) = E_{DFT}(\ch{I_{$i$}^{x}}) - (E_{DFT}(ZrO_2) + \frac{1}{2}\mu_{I_{2}})  % - \frac{E_{I_2}}{2}
\end{equation}
where $E_{inc}(\ch{I_{$i$}^{x}})$ is the incorporation energy of a neutral iodine interstitial, $E_{DFT}(\ch{I_{$i$}^{x}})$ is the energy of a neutral iodine interstitial, $E_{DFT}(ZrO_2)$ is the energy of a non-defective \zirconia\ supercell and $\mu_{I_{2}}$ is the chemical potential of an I$_{2}$ molecule, taken from a single point DFT calculation of the I$_{2}$ dimer. For incorporation of a charged interstitial (e.g. $\ch{I_{i}^{*}}$), the energy required to add or remove an electron is included in the calculation:
\begin{equation}
\label{interstitial_incorp_equation_charged}
E_{inc}(\ch{I_{$i$}^{n}}) = E_{DFT}(\ch{I_{$i$}^{n}}) - (E_{DFT}(ZrO_2) + \frac{1}{2}\mu_{I_{2}} + n(E_{VBM} + \mu_{e}))
\end{equation}
Similarly, for a substitutional defect:
\begin{equation}
\label{o_sub_incorp_equation}
E_{inc}(\ch{I_{O}^{$n$}}) = E_{DFT}(\ch{I_{O}^{$n$}}) - (E_{DFT}(\ch{V_{O}^{$n$}}) + \frac{1}{2}\mu_{I_{2}})  % - \frac{E_{I_2}}{2}
\end{equation}
where $\ch{I_{O}^{$n$}}$ is an iodine substitutional defect at an oxygen site of charge $n$ and $\ch{V_{O}^{$n$}}$ is the corresponding oxygen vacancy.

\subsection{Stiffness matrix generation}

The elastic stiffness matrices for the pure monoclinic, tetragonal and cubic phases of \zirconia\ were calculated using CASTEP's \emph{elastic constants} task. The calculation of elastic constants is a multi-step process involving up to 36 individual DFT calculations. Several scripts have been made available to simplify this process (see Appendix \ref{castep_scripts}).

The first step is to generate multiple different \texttt{.cell} files, each with either a small deviation in the lattice parameter or an additional shear on the cell. This requires starting with a unit cell that has already been completely relaxed via the \emph{geometry optimisation} task. In total, 36 \texttt{.cell} files are generated, each corresponding to a single element of the eventual stiffness matrix. The next step is to run a single point DFT calculation on each \texttt{.cell} file with the \emph{calculate stress} parameter enabled to output the resulting stress matrix. The final step is to use Hooke's Law to calculate the elastic stiffness constants using the known stress and strain state. 

%\subsection{Strain method for defect volumes}
%
%The volumes of the defective supercells were kept constant because constant pressure calculations have been shown to sometimes break the symmetry of the supercell \cite{samanta2010thermodynamic}, leading to unreliable energy values. This is partly due to the assumed arrangement of the defects that may not be commensurate with the cell symmetry. This approach to calculating defect volumes then relies on calculating the elastic constants of the non-defective supercell, followed by extracting the resultant stress tensor from a defect simulation. The strain tensor of the defective cell can then be calculated using Hooke's law, giving the relaxation volume. 

\subsection{Defect relaxation volumes} \label{isobaricmethod}

Defect relaxation volumes of point defects were calculated using an isobaric method, requiring two calculations to be performed under constant-pressure using the geometry optimisation task in CASTEP. The defect relaxation volume ($\Delta$V) is defined as: 
\begin{equation}
\Delta V = V_{def} - V_{perf}
\end{equation}
where $V_{def}$ is the relaxation volume of the defective supercell and $V_{perf}$ is the relaxation volume of the non-defective (perfect) supercell. In this thesis, mentions of `volume' will refer to relaxation volumes unless stated otherwise.

After completing an energy calculation, CASTEP provides the volume of the resulting cell, defined as the space enclosed by the repeating unit of atoms within the calculated lattice parameters. By subtracting the volume of a non-defective cell from the volume of a defective cell, we obtain a value for the total defect volume. 

It is important to consider that if there is a non-zero charge on the system, this will affect the calculated volume. Two systems with the same type, amount and arrangement of atoms, but different overall charges, will have different energies (due to the number of electrons). Different electronic orbital occupancies will affect the inter-atomic forces and therefore the shape of the cell. In order to compensate for this effect, a `corrected' relaxation volume, as described by Goyal \emph{et al.} \cite{goyal2017conundrum} is calculated when the defect has a non-zero charge:
\begin{equation}
\Delta V = V_{def}^{q} - V_{perf}^{q}
\end{equation}
where $q$ is the defect charge. This formulation uses the volume of a non-defective supercell with equal charge magnitude as the reference structure. This method has been shown to yield more reasonable defect volumes than when using neutral non-defective supercells as the reference structure. Defect volumes without this correction applied are provided in Appendix \ref{uncorrected_volumes} for comparison.

\section{Defect energies and equilibria} 

\subsection{Defect formation energies}

Defect formation energies are calculated using equation \ref{equation:formation_energy}:
\begin{equation} \label{equation:formation_energy}
    E_{f} = E_{def} - (E_{perf} \pm \sum_{i} n_i\mu_i + q(E_{VBM} + \mu_{e})) + E_{corr}
\end{equation}
where $E_{f}$ is the formation energy, $E_{def}$ is the energy of the defective supercell, $E_{perf}$ is the energy of a non-defective supercell, $q$ is the defect charge, $E_{VBM}$ is the valence band maximum, $\mu_{e}$ is the Fermi level relative to the VBM and $E_{corr}$ is a charged-cell correction term (see § \ref{charged_cell_correction}). Since $\mu_{e}$ is not a fixed value, plots of formation energy against $\mu_{e}$ are produced to examine the behaviour of defects across the entire range of the band gap. These are reported in Figures \ref{figure:monovacancies}, \ref{figure:tetvacancies} and \ref{figure:cubicvacancies}.

\subsection{Defect equilibria} \label{brouwer_method} % 

Typically in materials, several types of defects will exist simultaneously. These defects will be present at an equilibrium concentration based on their thermodynamic stability. Predicting the defect equilibria is possible with statistical mechanics and some approximations. For example, it is expected that a crystal lattice will usually be overall charge-neutral (exceptions can be made under certain conditions, see § \ref{space_charge}), otherwise we would see a build-up of charge with a large Coulomb energy penalty which would be thermodynamically unsustainable.

Brouwer diagrams, also known as Kr{\"o}ger-Vink diagrams, were produced using a method outlined by Murphy \emph{et al}. \cite{Murphy2014, Murphy2014a} through which it is possible to determine defect concentrations as a function of oxygen partial pressure. We start from the statement that the chemical potential of \zirconia\ is equivalent to the sum of chemical potentials $\mu$ of its constituent species, Zr and O:
\begin{equation}
{\mu}_{ZrO_2(s)} = {\mu}_{Zr}(p_{O_2}, T) + {\mu}_{O_{2}}(p_{O_{2}}, T)
\label{mewZrO2compmethodology}
\end{equation}
where $T$ denotes temperature and $p_{O_2}$ denotes oxygen partial pressure. The chemical potential of \zirconia\ in the solid state is assumed to have negligible dependence on $T$ and $p_{O_2}$ relative to ${\mu}_{Zr}$ and ${\mu}_{O_2}$. Energies can be obtained for bulk \zirconia\ and Zr, but the ground state of oxygen is not correctly reproduced in DFT \cite{Batyrev2000,Lozovoi2001}. Instead, we use the approach of Finnis \emph{et al}. \cite{Finnis2005} to infer the oxygen chemical potential from standard state values. We can use the experimental Gibbs free energy to produce an equation where $\mu_{O_2}$ is the only unknown:
\begin{equation}
\Delta{G^{\plimsoll}_{f, ZrO_2}} = \mu_{ZrO_2(s)} - (\mu_{Zr(s)} + \mu^{\plimsoll}_{O_2})
\end{equation}
where $\Delta{G^{\plimsoll}_{f, ZrO_2}}$ is the experimental Gibbs energy at standard temperature and pressure and $\mu^{\plimsoll}_{O_2}$ is the oxygen chemical potential under the same conditions. Only monoclinic \zirconia\ is stable under standard conditions, with $\Delta{G^{\plimsoll}_{f, ZrO_2}}$ = -1042.746 kJ/mol (10.807 eV) \cite{brown2005chemical}. Values of the Gibbs free energy of formation for the tetragonal (10.697 eV) and cubic (10.595 eV) phases were obtained by adding the energy difference between the phases from DFT calculations. The values of $\mu_{ZrO_2(s)}$ and $\mu_{Zr(s)}$ are calculated using DFT. Once $\mu^{\plimsoll}_{O_2}$ is calculated, we can generalise the chemical potential of oxygen for any value of $T$ and $p_{O_2}$ by appending an ideal gas relationship $\Delta{\mu(T)}$ and a Boltzmann distribution:
\begin{gather}
\mu_{O_2}(p_{O_2},T) = \mu^{\plimsoll}_{O_2} + \Delta{\mu(T)} + \frac{1}{2}{k_B}log(\frac{p_{O_2}}{p^{\plimsoll}_{O_2}}) \\
\Delta \mu(T) = -\frac{1}{2}(S^{\plimsoll}_{O_{2}}- C^{\plimsoll}_{p})(T-T^{\plimsoll}) + C^{\plimsoll}_{p}T\textup{log}\left ( \frac{T}{T^{\plimsoll}} \right )
\end{gather} 
where $S^{\plimsoll}_{O_{2}}$ is the molecular entropy at standard temperature and pressure (T$^{\plimsoll}$ = 273.15 K, P$^{\plimsoll}$ = $10^{5}$ Pa), and $C^{\plimsoll}_{p}$ is the constant pressure heat capacity of oxygen. These quantities have values of $S^{\plimsoll}_{O_{2}}$ = 0.0021 eV/K and $C^{\plimsoll}_{p}$ = 0.000302 eV/K \cite{weast1984crc}. 

Using our generalised formula for $\mu_{O_2}$, we fix the temperature within the range of thermal phase-stabilisation (e.g. 1500 K for tetragonal \zirconia) and calculate $\mu_{O_2}$ for many different values of $p_{O_2}$ between $10^{-35}$ and 10$^{0}$ atm, corresponding to oxygen deficient and oxygen rich environments, respectively ($p_{O_2}$ in air is approximately 0.2 atm). While the tetragonal phase will be stress-stabilised in practice, thermal-stabilisation in such models has been shown to qualitatively approximate the effect of stress-stabilisation, while allowing a wider range of dopant behaviours to be predicted \cite{Bell2016}. 

Once a value of $\mu_{O_2}$ is calculated, defect concentrations can then be calculated using Boltzmann statistics. These concentrations were calculated using the method outlined by Kasamatsu \emph{et al}. \cite{Kasamatsu2012} whereby the effect of defects competing for the same lattice site is taken into account. The next step is to calculate the concentration of electron and hole defects. This is done by using the charge-neutrality condition to determine the Fermi level (electrochemical potential) in the system:
\begin{equation}
\sum_{i}q_{i}c_{i} - N_{c}\textrm{exp}{(-\frac{E_{g}-\mu_{e}}{k_{B}T})} + N_{v}\textrm{exp}{(-\frac{\mu_{e}}{k_{B}T})} = 0
\label{charge_neutrality}
\end{equation}

Where $c_{i}$ is the concentration of defect $i$, $q_{i}$ is its respective charge, $N_{c}$ and $N_{v}$ are the integrated density of states for the conduction and valence bands, $E_{g}$ is the band gap and $\mu_{e}$ is the Fermi level. 

\subsubsection{Temperature and pressure stabilisation}

As discussed in Chapter \ref{ch:crystallography}, the tetragonal and cubic phases are stabilised at elevated temperatures and pressures. However, DFT calculations of supercells under stress require significantly greater computational resources to yield sufficiently converged energy results, and so all DFT calculations in this thesis are performed on relaxed supercells. To account for this lack of stress stabilisation, Brouwer diagrams are generated at higher temperatures (where the tetragonal and cubic phases are thermally stabilised) rather than at 650 K, which is the expected temperature at the internal surface of the cladding. This approach to compensating for stress stabilisation follow that of similar studies published by other groups \cite{youssef2012intrinsic, Youssef2014, Otgonbaatar2014}.

\subsection{Effect of space charge} \label{space_charge}

Electrons have a higher rate of diffusion than oxygen vacancies in \zirconia , leading to a build-up of oxygen vacancies near the metal-oxide interface as corrosion progresses \cite{bojinov2010influence}. In the case of \zirconia , this effect will be pronounced because the layer is thin. This results in an overall positive charge (since the dominant oxygen vacancy is \ch{V_{O}^{**}}) referred to as a space charge. This effect can be taken into account when generating Brouwer diagrams by assuming an overall charge in the crystal structure instead of charge-neutrality:
\begin{equation}
\sum_{i}q_{i}c_{i} - N_{c}\textrm{exp}{(-\frac{E_{g}-\mu_{e}}{k_{B}T})} + N_{v}\textrm{exp}{(-\frac{\mu_{e}}{k_{B}T})} = q_{s}c_{s}
\label{charge_non_neutrality}
\end{equation}
where $q_{s}$ is the charge of a unit of the artificial space charge defect and $c_{s}$ is the concentration.

Figure \ref{figure:spacechargeexample} shows an example of the defect equilibria in tetragonal \zirconia\ with an overall positive space charge. In order for such a condition to be satisfied, higher concentrations of positively charged oxygen vacancy and hole defects are predicted to be present, while zirconium vacancy defects fall significantly. When extrinsic defects are also present in the lattice in significant concentrations, the space charge condition may influence which defect types are dominant at different oxygen pressures, as different oxidation states may be necessary to satisfy the charge condition.

%Another effect considered was the space charge of the system. Electrons have a higher rate of diffusion than oxygen vacancies in ZrO2, leading to a build-up of oxygen vacancies near the metal-oxide interface as corrosion progresses [44]. This results in an overall positive charge (since the dominant oxygen vacancy is referred to as a space charge. When included in our Brouwer diagrams, this space charge had a negligible effect on the concentration or charge state of iodine up to a charge of  holes per f.u. ZrO2. This corresponds to a high concentration of oxygen vacancies relative to the equilibrium concentration, predicting that a significant deviation from the equilibrium is not expected near the metal oxide interface as a result of a positive space charge.

\begin{figure}[htp] % Tet intrinsic no space charge
\begin{center}
\begin{tikzpicture}
	\begin{groupplot}[group style={group size=1 by 2}, width=14cm, height=11cm]
	\nextgroupplot[
		 ylabel={\ch{log_{10}}([D]) (per f.u.)}, ymin=-10, ymax=0, xmin=-35, xmax=0, legend style={{draw=}, at={(0.40,0.97)}, anchor=north west, legend columns=2, nodes={scale=1, transform shape}}]
        \addplot[no marks, draw=blue!70!black] table [x=pO2, y=electrons,]{dat/intrinsic_tet.dat}; \addlegendentry{\ch{e^{'}}}; \node at (-26.0,-2) {\ch{e^{'}}};
        \addplot[no marks, draw=red!85!black] table [x=pO2, y=holes,]{dat/intrinsic_tet.dat}; \addlegendentry{\ch{h^{\textperiodcentered}}}; \node at (-6.8,-3.6) {\ch{h^{\textperiodcentered}}};
        \addplot[no marks, draw=black!70!green] table [x=pO2, y=VO{2},]{dat/intrinsic_tet.dat}; \addlegendentry{\ch{V_{O}^{\textperiodcentered\textperiodcentered}}}; \node at (-28,-3) {\ch{V_{O}^{\textperiodcentered\textperiodcentered}}};
%         \addplot[no marks, draw=black!55!green] table [x=pO2, y=VO{1},]{dat/intrinsic_tet.dat}; \addlegendentry{\ch{V_{O}^{*}}};
%         \addplot[no marks, draw=black!30!green] table [x=pO2, y=VO{0},]{dat/intrinsic_tet.dat}; \addlegendentry{\ch{V_{O}^{x}}};
        \addplot[no marks, draw=yellow!85!blue] table [x=pO2, y=VM{-4},]{dat/intrinsic_tet.dat}; \addlegendentry{\ch{V_{Zr}^{''''}}}; \node at (-2.9,-4.6) {\ch{V_{Zr}^{''''}}};
%         \addplot[no marks, draw=yellow!75!blue] table [x=pO2, y=VM{-3},]{dat/intrinsic_tet.dat}; \addlegendentry{\ch{V_{Zr}^{'''}}};
%         \addplot[no marks, draw=yellow!65!blue] table [x=pO2, y=VM{-2},]{dat/intrinsic_tet.dat}; \addlegendentry{\ch{V_{Zr}^{''}}};
%         \addplot[no marks, draw=yellow!55!blue] table [x=pO2, y=VM{-1},]{dat/intrinsic_tet.dat}; \addlegendentry{\ch{V_{Zr}^{'}}};
%         \addplot[no marks, draw=yellow!45!blue] table [x=pO2, y=VM{0},]{dat/intrinsic_tet.dat}; \addlegendentry{\ch{V_{Zr}^{x}}};
%         \addplot[no marks, draw=red!60!yellow] table [x=pO2, y=Oi{-2},]{dat/intrinsic_tet.dat}; \addlegendentry{\ch{O_{i}^{''}}};
%         \addplot[no marks, draw=red!50!yellow] table [x=pO2, y=Oi{-1},]{dat/intrinsic_tet.dat}; \addlegendentry{\ch{O_{i}^{'}}};
%         \addplot[no marks, draw=red!40!yellow] table [x=pO2, y=Oi{0},]{dat/intrinsic_tet.dat}; \addlegendentry{\ch{O_{i}^{x}}};
%         \addplot[no marks, draw=green!80!pink] table [x=pO2, y=Mi{4},]{dat/intrinsic_tet.dat}; \addlegendentry{\ch{Zr_{i}^{****}}};
%         \addplot[no marks, draw=green!70!pink] table [x=pO2, y=Mi{3},]{dat/intrinsic_tet.dat}; \addlegendentry{\ch{Zr_{i}^{***}}};
%         \addplot[no marks, draw=green!60!pink] table [x=pO2, y=Mi{2},]{dat/intrinsic_tet.dat}; \addlegendentry{\ch{Zr_{i}^{\textbf{**}}}};
%         \addplot[no marks, draw=green!50!pink] table [x=pO2, y=Mi{1},]{dat/intrinsic_tet.dat}; \addlegendentry{\ch{Zr_{i}^{*}}};
%         \addplot[no marks, draw=green!40!pink] table [x=pO2, y=Mi{0},]{dat/intrinsic_tet.dat}; \addlegendentry{\ch{Zr_{i}^{x}}};
%         \addplot[no marks] table [x=pO2, y=Stoich,]{dat/intrinsic_tet.dat}; \addlegendentry{Stoich};
\node at (-33.7,-0.5) {\textbf{a)}}; 
			%\end{axis}     
%\end{tikzpicture}
%\begin{tikzpicture} % 1e-1
	\nextgroupplot[
		 xlabel={\ch{log_{10}}($p_{O_{2}}$) (atm)}, ylabel={\ch{log_{10}}([D]) (per f.u.)}, ymin=-10, ymax=0, xmin=-35, xmax=0, legend style={{draw=}, at={(0.40,0.97)}, anchor=north west, legend columns=4, nodes={scale=1, transform shape}}]
        \addplot[no marks, draw=blue!70!black] table [x=pO2, y=electrons,]{dat/intrinsic_spacecharge01.dat}; \node at (-26.5,-2.5) {\ch{e^{'}}};
        \addplot[no marks, draw=red!85!black] table [x=pO2, y=holes,]{dat/intrinsic_spacecharge01.dat}; \node at (-13,-3) {\ch{h^{\textperiodcentered}}};
        \addplot[no marks, draw=black!70!green] table [x=pO2, y=VO{2},]{dat/intrinsic_spacecharge01.dat}; \node at (-28,-0.8) {\ch{V_{O}^{\textperiodcentered\textperiodcentered}}};
%         \addplot[no marks, draw=black!55!green] table [x=pO2, y=VO{1},]{dat/intrinsic_spacecharge01.dat}; \addlegendentry{\ch{V_{O}^{*}}};
%         \addplot[no marks, draw=black!30!green] table [x=pO2, y=VO{0},]{dat/intrinsic_spacecharge01.dat}; \addlegendentry{\ch{V_{O}^{x}}};
        %\addplot[no marks, draw=yellow!85!blue] table [x=pO2, y=VM{-4},]{dat/intrinsic_spacecharge01.dat}; \node at (-3,-3) {\ch{V_{Zr}^{''''}}};
%         \addplot[no marks, draw=yellow!75!blue] table [x=pO2, y=VM{-3},]{dat/intrinsic_spacecharge01.dat}; \addlegendentry{\ch{V_{Zr}^{'''}}};
%         \addplot[no marks, draw=yellow!65!blue] table [x=pO2, y=VM{-2},]{dat/intrinsic_spacecharge01.dat}; \addlegendentry{\ch{V_{Zr}^{''}}};
%         \addplot[no marks, draw=yellow!55!blue] table [x=pO2, y=VM{-1},]{dat/intrinsic_spacecharge01.dat}; \addlegendentry{\ch{V_{Zr}^{'}}};
%         \addplot[no marks, draw=yellow!45!blue] table [x=pO2, y=VM{0},]{dat/intrinsic_spacecharge01.dat}; \addlegendentry{\ch{V_{Zr}^{x}}};
%         \addplot[no marks, draw=red!60!yellow] table [x=pO2, y=Oi{-2},]{dat/intrinsic_spacecharge01.dat}; \addlegendentry{\ch{O_{i}^{''}}};
%         \addplot[no marks, draw=red!50!yellow] table [x=pO2, y=Oi{-1},]{dat/intrinsic_spacecharge01.dat}; \addlegendentry{\ch{O_{i}^{'}}};
%         \addplot[no marks, draw=red!40!yellow] table [x=pO2, y=Oi{0},]{dat/intrinsic_spacecharge01.dat}; \addlegendentry{\ch{O_{i}^{x}}};
%         \addplot[no marks, draw=green!80!pink] table [x=pO2, y=Mi{4},]{dat/intrinsic_spacecharge01.dat}; \addlegendentry{\ch{Zr_{i}^{****}}};
%         \addplot[no marks, draw=green!70!pink] table [x=pO2, y=Mi{3},]{dat/intrinsic_spacecharge01.dat}; \addlegendentry{\ch{Zr_{i}^{***}}};
%         \addplot[no marks, draw=green!60!pink] table [x=pO2, y=Mi{2},]{dat/intrinsic_spacecharge01.dat}; \addlegendentry{\ch{Zr_{i}^{\textbf{**}}}};
%         \addplot[no marks, draw=green!50!pink] table [x=pO2, y=Mi{1},]{dat/intrinsic_spacecharge01.dat}; \addlegendentry{\ch{Zr_{i}^{*}}};
%         \addplot[no marks, draw=green!40!pink] table [x=pO2, y=Mi{0},]{dat/intrinsic_spacecharge01.dat}; \addlegendentry{\ch{Zr_{i}^{x}}};
%         \addplot[no marks] table [x=pO2, y=Stoich,]{dat/intrinsic_spacecharge01.dat}; \addlegendentry{Stoich};
\node at (-33.7,-0.5) {\textbf{b)}};
	\end{groupplot}
			%\end{axis}              
\end{tikzpicture}
		\caption{Tetragonal phase Brouwer diagrams of intrinsic point defects at a temperature of 1500 K \textbf{a)} without a space charge and \textbf{b)} with a space charge of $10^{-1}$ e$^{-1}$/fu.}
		\label{figure:spacechargeexample}
	\end{center}
\end{figure}

\section{Convergence testing} \label{section:convergence}

\subsection{Plane-wave cut-off energy}

In order to determine an appropriate value for the plane-wave cut-off energy, a convergence test was performed to determine the relative error in predicted energy compared to a highly converged value. This convergence test was conducted by running multiple geometry optimisation procedures under fully relaxed conditions on a unit cell of \zirconia\ for each phase. A small \textbf{k}-point spacing of 0.01 \r{A}$^{-1}$ was used for each task (highly converged), while increasing the plane-wave cut-off energy from 300 eV to 750 eV in 50 eV increments. The energy of each run was recorded and compared to the energy of a highly converged value taken when a cut-off energy of 900 eV was used. This provides a value for the truncation error at different cut-off energies. Figure \ref{Figure:cutoffconvergence} shows a log plot of the energy error for each phase of \zirconia\ as the cut-off energy is increased. 

The error is shown to be independent of phase, with all lines lying on a single path. This might be expected because the atoms in each phase are the same, and therefore the electrons involved in the calculations remain unchanged, however, interatomic distances are different in the different phases and thus so are the electron densities, so this equivalence in convergence does not necessarily have to follow. A cut-off energy of 600 eV was found to produce an error below 0.01 eV, and was subsequently used for future calculations as it provides a good compromise between computational cost and accuracy.

\begin{figure}[ht] % Plane-wave cut-off convergence
	\begin{center}
		\begin{tikzpicture}
			\begin{axis}
				[width=\linewidth*0.7, xlabel={E\textsubscript{cutoff} (eV)}, ylabel={log$_{10}$(error) / formula unit}, ymin=-3.5, legend style={{draw=}, at={(0.95,0.95)}, anchor=north east,}]
				\addplot[no marks] table [x=cutoffenergy, y=logerrormono,]{dat/convergence.dat}; \addlegendentry{Monoclinic};
			    \addplot[no marks, dashed] table [x=cutoffenergy, y=logerrortet,]{dat/convergence.dat}; \addlegendentry{Tetragonal};
			    \addplot[no marks, densely dotted] table [x=cutoffenergy, y=logerrorcubic,]{dat/convergence.dat}; \addlegendentry{Cubic};
                \draw[red,-stealth]
				(600,-1.96)
				-- % = line-to
				++ % = calculate a vector sum
				(axis direction cs:0,-1.46);
                \addplot [only marks,mark=*]
coordinates { (600,-1.95) };
			\end{axis}
		\end{tikzpicture}
		\caption{Plot of the log error of DFT energy against plane-wave cut-off energy for a perfect cell of each crystal structure. The error is calculated with respect to a highly converged value, calculated at a plane-wave cut-off energy of 900 eV. The red arrow indicates the cut-off energy beyond which the error is below 0.01 eV.}
		\label{Figure:cutoffconvergence}
	\end{center}
\end{figure}

\subsection{\textbf{k}-point convergence}

Too fine a grid in reciprocal space (i.e. a large number of \textbf{k}-points) results in prohibitively computationally expensive simulations, whereas too coarse a grid may have a large truncation error when energies are calculated. To find the optimum spacing of \textbf{k}-points, a convergence study was performed across a range of \textbf{k}-point spacings, with the output energies compared to a highly converged simulation to obtain a value for the error. 

Figure \ref{Figure:kpoint_convergence} shows the energy error for each phase of \zirconia\ as a function of the \textbf{k}-point spacing (given in reciprocal space as \r{A}$^{-1}$). The highly converged energy value was calculated with a \textbf{k}-point spacing of 0.01 \r{A}$^{-1}$ for error calculations. The plot shows a stepwise change in the error value as the grid spacing is reduced. This is because calculations demand an integer number of \textbf{k}-points, and larger spacings do not provide sufficient resolution to effectively fit an integer number of \textbf{k}-points into the reciprocal grid, so that the program snaps to the nearest appropriate grid number. An optimum \textbf{k}-point spacing was chosen at 0.09 \r{A}$^{-1}$, which was the largest spacing that kept the error below 0.01 eV for all phases, highlighted in the plot by the red arrow.

\begin{figure}[ht]
\begin{center}
\begin{tikzpicture}
	\begin{axis}
		[width=\linewidth*0.7, xlabel={\textbf{k}-point spacing (\r{A}\textsuperscript{-1})}, ylabel={log[error]}, ymin=-7, ymax=1, xmin=0, xmax=0.22, legend style={{draw=}, at={(0.05,0.95)}, anchor=north west, legend columns=1}, xticklabel
style={/pgf/number format/.cd,fixed,precision=5}]
		\addplot[no marks] table [x=kpoint_spacing, y=monoclinic,]{dat/kpoint_convergence.dat}; \addlegendentry{Monoclinic};
        \addplot[no marks, dashed] table [x=kpoint_spacing, y=tetragonal, ]{dat/kpoint_convergence.dat}; \addlegendentry{Tetragonal};
        \addplot[no marks, densely dotted] table [x=kpoint_spacing, y=cubic,]{dat/kpoint_convergence.dat}; \addlegendentry{Cubic};
        \draw[red,-stealth]
				(0.09,-2.35)
				-- % = line-to
				++ % = calculate a vector sum
				(axis direction cs:0,-4.6);
                \addplot [only marks,mark=*]
coordinates { (0.09,-2.35) };
			\end{axis}
		\end{tikzpicture}
		\caption{Log of the error in the total energy of the system as a function of \textbf{k}-point spacing. The error is calculated relative to a highly converged energy value at a \textbf{k}-point spacing of 0.01 \r{A}\textsuperscript{-1}. The red arrow indicates the \textbf{k}-point spacing which yields an error below 0.01 eV for all structures.}
		\label{Figure:kpoint_convergence}
	\end{center}
\end{figure}

\subsection{Exchange-correlation functionals}

There are a range of possible exchange-correlation functionals available in CASTEP, spanning both empirical and non-empirical types. Empirical exchange-correlation functionals are typically optimised to capture specific properties or systems particularly well, but perform less well for generalised systems. Non-empirical exchange-correlation functionals, while still not perfect, are preferred for modelling the widest range of properties. In a sense, non-empirical functions benefit from not being `over-fit' to experimental data. They are also more prevalent in the literature, thereby providing a rich corpus of work for comparison studies.

While the PBE-GGA exchange-correlation functional in this work had already been selected, it was helpful to conduct an energy convergence study of the systems across the different functionals available in CASTEP in order to determine how other functionals compared. Only 6 of the 14 functionals available in CASTEP were able to yield a converged energy calculation within a reasonable amount of time, as shown in Figure \ref{Figure:xc_test}. This is because several hybrid functionals partially incorporate the exact exchange using the Hartree-Fock method \cite{hartree1928wave}, significantly increasing the computational cost of an energy calculation. 

The calculated energies indicate that each functional correctly predicts the order of phase stability in \zirconia , though the magnitude of the energy difference between phases varied. These differences are small, approximately 0.1 eV/f.u., but their effects are compounded when defects are introduced into the cell. The total energies were more varied, with several eV differences between functionals, however, lower total energies across different exchange-correlation functionals do not necessarily suggest that a better minima has been found. For example, the PW91 functional resulted in even lower energies than PBE, despite PW91 preceding PBE and both having been developed by Perdew \emph{et al}. \cite{perdew1991unified, perdew1992atoms}. It is the energy difference between systems calculated with the same exchange-correlation functional which is important. 

To better gauge the performance of each functional, further studies across a much larger range of parameters, and even materials, would need to be conducted, however this is beyond the scope of this thesis.

\begin{figure}[ht!] % XC functional study
  \begin{center}
    \begin{tikzpicture}
      \begin{axis}
        [ybar, ymin=2350, ymax=2362, width=\linewidth*0.7, xtick=data, xlabel={XC Functional}, ylabel={Unit cell energy (-eV/f.u.)}, xticklabels from table={dat/xc_test.dat}{functional}, area legend, legend style={at={(0.04,0.96)},
anchor=north west, legend columns=1}] %axis x line=middle, ymin=5.1, ymax=5.35, xmin=0, xmax=12, legend style={{draw=}, at={(0.18,0.95)}, anchor=north east, legend columns=1}
        \addplot[style={black, fill=red!30!white, mark=none}] table [x expr=\coordindex, y=mono_energy]{dat/xc_test.dat};
        \addplot[style={black, fill=blue!30!white, mark=none}] table [x expr=\coordindex, y=tet_energy]{dat/xc_test.dat};
        \addplot[style={black, fill=green!30!white, mark=none}] table [x expr=\coordindex, y=cubic_energy]{dat/xc_test.dat};
        \legend{Monoclinic, Tetragonal, Cubic}
      \end{axis}
    \end{tikzpicture}
    \caption{Calculated energy of a unit cell of monoclinic, tetragonal and cubic \zirconia\ when using different exchange-correlation functionals.}
    \label{Figure:xc_test}
  \end{center}
\end{figure}

\subsection{On-the-fly pseudopotentials}

Ultra soft pseudopotentials are generated in CASTEP automatically (known as on-the-fly or OTF pseudopotentials) when none are specified for a particular element. Energies must be calculated and compared with the same set of pseudopotentials in order to keep simulations self-consistent. A single point calculation was performed on a unit cell of \zirconia\ and the resulting OTF pseudopotentials (one for oxygen and one for zirconium) were saved and used for all subsequent calculations. 

It is important to determine the variance in energy values of different pseudopotentials generated OTF in order to avoid systematic error. To assess error, 9 different pairs of OTF pseudopotentials were generated\footnote{OTF pseudopotential generation in CASTEP uses a random number seed which is generated whenever a calculation is run without specifying a pseudopotential. By default, these are ultra-soft pseudopotentials.} and used to calculate the total energy of a monoclinic \zirconia\ supercell. The difference in energy was then calculated with respect to the pseudopotential pair that resulted in the lowest energy. These deviations in total energy are shown in Figure \ref{Figure:otf_pp_test}. Across all calculations, the largest difference in total energy was 0.0012 eV, while the average difference was 0.0006 eV. Since here the only concern is with choosing other parameters to achieve a precision of 0.01 eV, and the largest deviation calculated is an order of magnitude below that, it is not necessary to take any special measures to correct any systematic error from randomly generated OTF pseudopotentials.

%\begin{figure}[ht] % +U cubic
%\begin{center}
%\begin{tikzpicture}
%	\begin{axis}
%		[width=11cm, xlabel={+U on Zr \emph{d} orbitals (eV)}, ylabel={Lattice parameter (\r{A})}, ymin=5.1, ymax=5.35, xmin=0, xmax=12, legend style={{draw=}, at={(0.18,0.95)}, anchor=north east, legend columns=1}]
%		\addplot[no marks] table [x=plusU, y=a,]{dat/plus_u_cubic.dat}; \addlegendentry{$a$};
%        %\addplot[no marks, dashed] table [x=plusU, y=b, ]{dat/plus_u_cubic.dat}; \addlegendentry{b};
%        %\addplot[no marks, densely dotted, black] table [x=plusU, y=c,]{dat/plus_u_cubic.dat}; \addlegendentry{c};
%			\end{axis}
%		\end{tikzpicture}
%		\caption{Individual lattice parameters as a function of +U term in cubic \zirconia .}
%		\label{Figure:plusucubic}
%	\end{center}
%\end{figure}

\begin{figure}[ht] % OTF PP study
  \begin{center}
    \begin{tikzpicture}
      \begin{axis}
        [ybar, width=\linewidth*0.7, xlabel={OTF pseudopotential pair}, ylabel={$\Delta$E with respect to lowest energy (meV)}, ] %axis x line=middle, ymin=5.1, ymax=5.35, xmin=0, xmax=12, legend style={{draw=}, at={(0.18,0.95)}, anchor=north east, legend columns=1}
        \addplot table [x=pp_pair, y=energy_diff_wrt_first,]{dat/otf_pp_test.dat};
      \end{axis}
    \end{tikzpicture}
    \caption{Energy deviation in meV of supercells with candidate OTF pseudopotential pairs. Energy deviations are shown with respect to the pseudopotential pair that resulted in the lowest total energy calculated.}
    \label{Figure:otf_pp_test}
  \end{center}
\end{figure}

\subsection{Chemical potential of oxygen}

The chemical potential of oxygen is required when performing any defect formation energy calculation where an atom of oxygen is added or removed (see Equation \ref{equation:formation_energy}). Calculating the chemical potential of oxygen requires special consideration of the electronic structure of O$_{2}$. The ground state of the O$_{2}$ molecule is known as triplet oxygen ($^{3}\Sigma^{-}_{g}$), an allotrope which exhibits a resultant spin magnetic moment (oxygen is paramagnetic). This is in contrast to singlet oxygen ($^{1}\Delta_{g}$) with a spin magnetic moment of zero. 

Two calculations, one for triplet and another for singlet oxygen, were performed using CASTEP. Large cells of 15 \r{A} x 15 \r{A} x 15 \r{A} were used to run geometry optimisation tasks on two oxygen atoms initially separated by 1.3 \r{A}. For the triplet oxygen calculation, a net electronic spin of +2 on the $p$ electrons was enforced, while the singlet oxygen calculation specified a net spin of 0.

The calculated bond lengths of triplet and singlet oxygen were 1.225 \r{A} and 1.227 \r{A} respectively. These bond lengths are within 2\% of the experimental value of 1.207 \cite{Lide2016}, with the triplet state prediction being slightly closer to this value. The calculated energies from DFT for triplet and singlet oxygen were -871.92 eV and -870.70 eV respectively. This gave an energy difference of 1.22 eV between the two forms of diatomic oxygen. While triplet oxygen was correctly predicted as the lower energy allotrope, the energy difference reported in the literature from microwave spectroscopy measurements is 0.9773 eV \cite{Atkins2006}, an almost 25\% difference compared to the DFT value. This large difference is attributed to the exchange-correlation functional and the inability to correctly model electron correlation effects in some cases. In this thesis, the DFT calculated energy of triplet oxygen was used only for formation energy against Fermi level plots, while defect equilibria calculations utilised a different method to calculate this value (see § \ref{brouwer_method}).

\subsection{Chemical potential of iodine}

To determine the chemical potential of iodine, an energy minimisation of the iodine dimer was performed. Unlike oxygen, iodine dimers do not exhibit a non-zero spin magnetic moment, thus avoiding a source of error in energy calculations with the PBE exchange-correlation functional. Similar to the \zirconia\ unit cell calculations, the lattice parameter after relaxation (bond length in this case) is compared to experimental data to assess the quality of the simulation parameters.

Figure \ref{figure:iodine_dimer} illustrates the energy minimisation of two iodine atoms in a cell of size 15 \r{A} x 15 \r{A} x 15 \r{A}, initially separated by 3.0 \r{A}. The geometry optimisation task finds an energy minima when the iodine atoms are bonded, at a separation of 2.69 \r{A}. This agrees well with the experimental value of 2.6745 \r{A} \cite{ukaji1966effect}.

\begin{figure}[ht] % Iodine dimer geometry optimisation
\centering
\includegraphics[width=14cm]{images/iodine_molecule.png}
\caption{Energy minimisation of two iodine atoms from an initial separation of 3.0 \r{A}.}
\label{figure:iodine_dimer}
\end{figure}

\subsection{+U study}
\label{subsection:plus_U}

In some DFT studies, an additional potential energy term (Hubbard U parameter or +U) is sometimes included to better capture the Coulomb interaction of localised electrons. An LDA or GGA functional alone will typically not describe this interaction correctly, especially for localised $d$ and $f$ electrons\footnote{Multiple occupation of $d$ and $f$ orbitals incurs an energy penalty which is not accurately modelled by the exchange-correlation functional.}. Of particular concern is the calculated value of the band gap from DFT simulations, as this value may deviate by up to 30\% from experimental values. Remedying this shortcoming with an appropriate +U parameter could therefore be valuable in obtaining accurate energies. 

In the literature, one GGA+U study of Fe-doped tetragonal \zirconia\ has shown that the inclusion of a +U term on Zr $d$ orbital electrons between 0 and 3.3 eV changes the electronic properties (in particular the electronic density of states) of the system significantly, and that the best agreement with experimental data occurs when U = 0 eV \cite{Sangalli2013}. Other GGA+U studies on bulk tetragonal \zirconia\ found that a +U term of 4 eV led to calculated lattice parameters which were in good agreement (within 0.05 \r{A}) with experimental values, but that the calculated band gap was still underestimated by 1.28 eV \cite{RuizPuigdollers2016, Chen2015, Puigdollers2015}. One LDA+U study in \zirconia\ used a +U parameter of 1 Ry (13.6 eV), and reproduced the correct order of stability of the monoclinic, tetragonal and cubic phase. This shows how the value of the +U parameter can vary significantly depending on the other approximations being used in the DFT calculations, and therefore an appropriate +U value for this thesis would have to be found independently. A +U study of the zirconium atom, with an electronic configuration of [Kr]$4d^{2}5s^{2}$, was performed to determine the response to and therefore the viability of this additional potential term for the $d$ electrons.

Figure \ref{Figure:plusubandgap} shows the effect on the calculated band gap when introducing a +U term. While the +U term does increase the band gap, the effect is not significant in bringing the prediction in-line with experimental values. Even with +U terms of 10 eV, the calculated band gap falls short of the experimental band gap by at least 1.5 eV. Moreover, with +U terms greater than 4 eV, we begin to see erratic behaviour in the development of both the band gap, and also in the predicted crystal structure. 

For the tetragonal phase, the calculated band gap (4.2 eV) does not agree with that calculated by Puigdollers \emph{et al.} \cite{RuizPuigdollers2016} (4.5 eV) when a +U term of 4 eV is used. The PBE GGA exchange-correlation functional and a plane-wave basis set was used in both studies, however, their study utilised the VASP 5.3 DFT software package while CASTEP 8.0 was used in this thesis. Different software packages may use different minimisation methods which could contribute to differing values, but determining the cause of this anomaly would require a separate study of the two codes at a low level which is beyond the scope of this thesis.

%Cubic is fine, it just keeps expanding with +U as we expect. The tetragonal phase expands in the short a&b directions but contracts in the long c direction (i.e. becomes more 'cubic') up until 6 eV, after which it grows in the same manner as cubic.

%The monoclinic phase is harder to explain. There is a cross-over in the length of the a and b parameter at around 4 eV, and then the beta angle (the ~99 deg between a and c) snaps into 90 deg at 11 eV, and when I look at the output structure at this energy, the coordination number of Zr is 6, down from 7. That's why the lattice parameters don't fall into a=b=c, because it doesn't become cubic.

%As you can see from the band gap plot below, just +U by itself is not enough to reproduce the experimental band gap, even for monoclinic. We're off by about 1.5 eV in each case.

\begin{figure}[ht] % +U band gaps
\begin{center}
\begin{tikzpicture}
	\begin{axis}
		[width=\linewidth*0.7, xlabel={+U on Zr \emph{d} orbitals (eV)}, ylabel={Band gap (eV)}, ymin=3.2, ymax=5, xmin=0, xmax=12, legend style={{draw=}, at={(0.35,0.95)}, anchor=north east, legend columns=1}]
		\addplot[no marks] table [x=plusU, y=bandgap,]{dat/plus_u_mono.dat}; \addlegendentry{Monoclinic};
        \addplot[no marks, dashed] table [x=plusU, y=bandgap, ]{dat/plus_u_tet.dat}; \addlegendentry{Tetragonal};
        \addplot[no marks, densely dotted, black] table [x=plusU, y=bandgap,]{dat/plus_u_cubic.dat}; \addlegendentry{Cubic};
			\end{axis}
		\end{tikzpicture}
		\caption{Calculated band gaps for different +U values in monoclinic, tetragonal and cubic \zirconia .}
		\label{Figure:plusubandgap}
	\end{center}
\end{figure}

\subsubsection{Monoclinic}

In monoclinic \zirconia , the use of a +U term causes the lattice parameters to change disproportionately to each other, as seen in Figure \ref{Figure:plusumono}. All lattice parameters increased with larger +U terms, however, expansion in the $a$ direction proceeded faster than in the $b$ direction, resulting in the $a$ lattice parameter becoming larger at a +U of 4 eV. +U terms larger than 10.5 eV caused the lattice parameters to snap suddenly onto new values. A  investigation of the atomic positions revealed that the monoclinic crystal structure had collapsed into an orthorhombic structure (i.e. the cell experienced a shear strain which resulted in a $\beta$ of 90\textdegree), with the co-ordination number of zirconium ions falling to 6 from 7.

\begin{figure}[ht] % +U mono
\begin{center}
\begin{tikzpicture}
	\begin{axis}
		[width=\linewidth*0.7, xlabel={+U on Zr \emph{d} orbitals (eV)}, ylabel={Lattice parameter (\r{A})}, ymin=4.9, ymax=6.3, xmin=0, xmax=12, legend style={{draw=}, at={(0.18,0.95)}, anchor=north east, legend columns=1}]
		\addplot[no marks] table [x=plusU, y=a,]{dat/plus_u_mono.dat}; \addlegendentry{$a$};
        \addplot[no marks, dashed] table [x=plusU, y=b, ]{dat/plus_u_mono.dat}; \addlegendentry{$b$};
        \addplot[no marks, densely dotted, black] table [x=plusU, y=c,]{dat/plus_u_mono.dat}; \addlegendentry{$c$};
			\end{axis}
		\end{tikzpicture}
		\caption{Individual lattice parameters as a function of +U term in monoclinic \zirconia .}
		\label{Figure:plusumono}
	\end{center}
\end{figure}

\subsubsection{Tetragonal}

In tetragonal \zirconia , increasing the +U term (Figure \ref{Figure:plusutet}) has a strong anisotropic effect on the lattice parameters. Unusually, the $c$ parameter falls (up to an +U energy of 6 eV) while the $a$ parameter increases. Typically it would be expected that both parameters would increase, perhaps at different rates, because the +U term increases the total energy (by increasing the Coulombic contribution) in the system. This increase in energy leads to higher stresses and therefore larger interatomic spacings (cells are permitted to relax in these calculations).

Systems will always tend towards the lowest energy configuration. Therefore the reduction of the $c$ parameter suggests that it is already in a high energy configuration in the $c$ direction (overextended) and can reduce its energy by being compressed in that direction (i.e. becoming more cubic). This is consistent with the interpretation that lower temperature phases of \zirconia\ are distortions of the cubic fluorite phase caused by a small cation radius. 

Above a +U parameter of 6 eV however, the $c$ parameter suddenly begins to increase. Upon further inspection of the resulting cell, it was found that it had transitioned completely to cubic fluorite from tetragonal. This can also be confirmed by observing that the relationship between the parameters becomes $2a^2 = c^2$ (i.e. the $c$ parameter is the same length as the unit cell's [110] diagonal, see Figure \ref{figure:tetvscubic}).

\begin{figure}[ht] % +U tet
\begin{center}
\begin{tikzpicture}
	\begin{axis}
		[width=\linewidth*0.7, xlabel={+U on Zr \emph{d} orbitals (eV)}, ylabel={$a$ parameter (\r{A})}, ymin=3.6, ymax=3.8, xmin=0, xmax=12, legend style={{draw=}, at={(0.18,0.95)}, anchor=north east, legend columns=1}, tick pos=left]
		\addplot[no marks] table [x=plusU, y=a,]{dat/plus_u_tet.dat}; \addlegendentry{$a$};
        %\addplot[no marks, dashed] table [x=plusU, y=b, ]{dat/plus_u_tet.dat}; \addlegendentry{b};
        \addplot[no marks, dashed, black] table [x=plusU, y=c,]{dat/plus_u_tet.dat}; \addlegendentry{$c$};
			\end{axis}
            \begin{axis}[width=\linewidth*0.7,
     xmin = 0, xmax = 12,
     ymin = 5.16, ymax = 5.32,
     hide x axis,
     hide y axis, tick pos=right]
     \addplot[no marks, dashed, black] table [x=plusU, y=c,]{dat/plus_u_tet.dat};
   			\end{axis}
            \pgfplotsset{every axis y label/.append style={rotate=180}}
   \begin{axis}[width=\linewidth*0.7,
         xmin=0, xmax=12,
         ymin=5.16, ymax=5.32,
         hide x axis,
         axis y line*=right,
         ylabel={$c$ parameter (\r{A})}
     ]
   \end{axis}
		\end{tikzpicture}
		\caption{Individual lattice parameters as a function of +U term in tetragonal \zirconia .}
		\label{Figure:plusutet}
	\end{center}
\end{figure}

\subsubsection{Cubic}

The effect of a +U term on a lattice parameter of cubic \zirconia\ is shown in Figure \ref{Figure:plusucubic}. Notably, the symmetry of the cell remains intact even up to a +U term of 12 eV, unlike in the monoclinic and tetragonal phases. The lattice parameter also increases superlinearly as the +U term is increased. This is the typical response that is expected when a +U term is introduced. 

\begin{figure}[ht] % +U cubic
\begin{center}
\begin{tikzpicture}
	\begin{axis}
		[width=\linewidth*0.7, xlabel={+U on Zr \emph{d} orbitals (eV)}, ylabel={Lattice parameter (\r{A})}, ymin=5.1, ymax=5.35, xmin=0, xmax=12, legend style={{draw=}, at={(0.18,0.95)}, anchor=north east, legend columns=1}]
		\addplot[no marks] table [x=plusU, y=a,]{dat/plus_u_cubic.dat}; \addlegendentry{$a$};
        %\addplot[no marks, dashed] table [x=plusU, y=b, ]{dat/plus_u_cubic.dat}; \addlegendentry{b};
        %\addplot[no marks, densely dotted, black] table [x=plusU, y=c,]{dat/plus_u_cubic.dat}; \addlegendentry{c};
			\end{axis}
		\end{tikzpicture}
		\caption{Lattice parameter as a function of +U term in cubic \zirconia .}
		\label{Figure:plusucubic}
	\end{center}
\end{figure}

There was however one instance of unexpected behaviour. Knowing that the tetragonal phase collapses to cubic at +U terms greater than 6 eV, we would expect both to exhibit the same band gap at this +U value. Looking at the band gap results from Figure \ref{Figure:plusubandgap}, we see that the band gaps of tetragonal and cubic \zirconia\ continue to be different above 6 eV, despite the crystal structures being the same in this region. One difference between the cells is that the cubic unit cell has 12 atoms while the tetragonal unit cell has 6 atoms. While this suggests a size effect, the band gap difference did not appear when comparing the band gaps of unit cells and supercells in the absence of a +U term. 

After considering the impact of a +U term in DFT calculations, the decision was made not to include the term. While it would provide a small improvement in the calculated band gap for the cubic phase, the effect on cell symmetry of the other phases would present a confounding variable, especially when placing defects into the structure. It is therefore more useful to run calculations without a +U term to maintain consistency of results between different phases in this thesis.
\chapter{Structure properties and intrinsic defects} \label{ch:results1}  

\label{ch:defects}

\section{Introduction}  

It is important to fully understand the behaviour of intrinsic defects in \zirconia\ before performing studies with dopant ions. In this chapter, intrinsic defects in monoclinic, tetragonal and cubic \zirconia\ are compared and contrasted, including values for formation energy, defect volumes and defect equilibria. Elastic constants, electronic density of states, band gaps and free energies of the non-defective structures are also reported. % because useful material properties may be exploited to improve performance, e.g. by doping with other ions to stabilise one crystal structure. For example, \zirconia\ doped with enough yttrium cations will stabilise the cubic phase and increase the concentration of oxygen vacancies. This would then affect the behaviour of dopant atoms also present in the lattice.

\subsection{Previous work} 

Previous works studying intrinsic defects in the \zirconia\ system have utilised quantum mechanical methods to determine defect formation energies in the monoclinic phase \cite{zheng2007first,foster2002modelling,foster2001structure} and defect equilibria in the tetragonal phase \cite{youssef2012intrinsic}. The cubic phase is mainly studied as a dopant-stabilised system \cite{orera1990intrinsic,jiang2011first}, with few undoped defect studies in the literature \cite{mackrodt1986theoretical,aarhammar2009energetics}. Building upon previous quantum mechanical studies, a comprehensive account of intrinsic defect energies, defect volumes, and defect equilibria for all three common crystal structures of \zirconia\ is provided, using state-of-the-art, accessible methods.

\section{Methodology}
\subsection{Simulation parameters}

As discussed in Chapter \ref{ch:compmethodology}, DFT calculations were performed using CASTEP 8.0 \cite{Clark2005}. Ultra-soft pseudo-potentials were used throughout, employing a 600 eV cut-off energy. The Perdew, Burke and Ernzerhof (PBE) \cite{Perdew1996} parameterisation of the generalised gradient approximation (GGA) was used to describe the exchange correlation functional. A Monkhorst-Pack sampling scheme \cite{Monkhorst1976} was used for Brillouin zone integration, with a minimum \emph{k}-point separation of 0.09 \r{A}\textsuperscript{-1}. The Pulay method for density mixing \cite{Pulay1980} was used to improve convergence of simulations. 

The electrical energy convergence criterion was set to $1\times10^{-6} $ eV. The maximum force between atoms was limited to $1\times10^{-2}$ eV \r{A}\textsuperscript{-1}. A gradient-descent geometry optimisation task was run on the cell until consecutive iterations differed in energy and atomic displacement by less than $1\times10^{-5}$ eV and $5\times10^{-4}$ \r{A}, respectively. 


\subsection{Helmholtz free energy}

To determine the temperature dependence of the ground states for the pure crystal structures, a harmonic approximation method as described by Burr et al. was used \cite{burr2015crystal,jackson2016resolving}. A constant-volume phonon calculation was performed for each structure, from which the vibrational enthalpy $H_{vib}(T, V)$ and entropy $S_{vib}(T, V)$ contributions to the Helmholtz free energy were calculated up to a temperature of 2500 K. The complete Helmholtz free energy $F(T, V)$ was then obtained by including the internal energy $U(V)$ and configurational entropy $S_{conf}$ of the system:
\begin{equation} \label{helmholtz_equation}
F(T, V) = U(V) + H_{vib}(T, V) - TS_{vib}(T, V) - TS_{conf}
\end{equation}

\subsection{Brouwer diagrams}

Using the method outlined in § \ref{brouwer_method}, Brouwer diagrams of the intrinsic defect equilibria for monoclinic, tetragonal and cubic \zirconia\ were generated. The monoclinic and tetragonal Brouwer diagrams were generated at temperatures of 650 K and 1500 K respectively, corresponding to temperatures at which these phases are thermally stabilised. Defect concentrations are reported in parts/fu (i.e. parts per formula unit \zirconia). The cubic Brouwer diagram was generated at 2000K despite being thermally stabilised at temperatures greater than 2400 K. This was because such high temperatures resulted in very large intrinsic defect concentrations such that defect behaviour could not be meaningfully examined (the system was completely defective). As discussed in § \ref{dis_form_energy_intrinsic} and § \ref{brouwer_discussion_intrinsic}, issues with modelling cubic \zirconia\ in DFT mean that calculated energy values may become unreliable, but are presented in this thesis for the purpose of completeness.

%\section{Cubic phase collapse}
%
%\begin{itemize}
%\item When some oxygen Frenkel defects were introduced to the cubic phase supercell, relaxation under constant volume conditions caused a collapse into a pseudo-tetragonal structure.
%\item This indicated that the cubic phase as modelled in DFT may not be fully stable.
%\item Further investigation indicated that the structure of a supercell of c-\zirconia\ broke down even with constrained symmetry, a result corroborated by Burr {et al}. \cite{burr2017importance}. 
%\end{itemize}

\subsection{Unit cells}

Having selected and optimised the parameters and functionals, unit cells of \zirconia\ in each phase were fully relaxed at constant pressure and the resulting structures were compared in detail to experimental data. Table \ref{lattice_params} shows the calculated lattice parameters and energy differences between the three \zirconia\ phases. 

The first thing to note is that the correct order of \zirconia\ phases is predicted in the total energy calculations, with monoclinic being the lowest energy phase and cubic being the highest. In addition, the energy difference between phases is small ($<$ 0.1 eV/fu). This is a good indication that the choice of exchange correlation functional can reproduce the energy landscape of the system accurately. This is especially important for when defects are introduced because they may promote stabilisation of one phase over another, and an inaccurate model will not capture this behaviour. In all cases the predicted cell volumes are consistently within approximately 2\% of experimental values. 

\begin{table}[ht] % Unit cell parameters
\onehalfspacing
\centering
\caption[Calculated unit cell parameters for the different crystal structures of \zirconia . Experimental data for pure monoclinic, yttria-stabilised tetragonal and magnesia-stabilised cubic phases at 295 K are shown in parentheses. Energy difference between structures is shown with respect to the cubic phase.]{Calculated unit cell parameters for the different crystal structures of \zirconia . Experimental data for monoclinic, tetragonal and cubic phases at 295 K are shown in parentheses \cite{Howard1988}. Energy difference between structures is shown with respect to the cubic phase.}
\label{lattice_params}
\resizebox{\textwidth}{!}{%
\begin{tabular}{ccccccc}
\hline Phase    & a (\AA) & b (\AA) & c (\AA) & $\beta$ ($\degree$) & Volume (\AA\textsuperscript{3}/fu) & $\Delta$E (eV/fu) \\ \hline
m-\zirconia   &    5.18 (5.15)          &    5.24 (5.21)         &    5.37 (5.32)         & 99.63 (99.23)             &       35.96 (35.22)                 &    -0.215              \\
t-\zirconia &    3.62 (3.61)         &              &    5.28  (5.18)        & 90             &   34.60 (33.75)                      &     -0.105             \\
c-\zirconia        &   5.11 (5.09)           &              &              & 90             &     33.36 (32.97)                   &      N/A     \\ \hline      
\end{tabular}}
\end{table}

\subsection{Electronic density of states} 

The electronic density of states for monoclinic, tetragonal and cubic \zirconia\ are generated in a two-step process. First, the non-defective structures are fully relaxed using the geometry optimisation task in CASTEP. This task will also calculate electronic eigenvalues for all k-points and save them to a \texttt{.bands} file. Second, the electronic band data is parsed from the \texttt{.bands} file using the OptaDOS code \cite{Nicholls2012, Morris2014} and the density of states is output to a text file. Further details on using OptaDOS to view the electronic density of states are given in Appendix \ref{castep_scripts}.

The electronic density of states for the three \zirconia\ phase are given in Figure \ref{figure:densityofstates}. In this figure, the valence and conduction bands can clearly be seen at 2-8 eV and 10-15 eV respectively. Most importantly, the energy values of the valence band maximum (VBM) and the conduction band minimum (CBM) for each phase can be obtained from this figure. These values are used to calculate the band gap in the different phases, shown in Figure \ref{table:bandgap} alongside experimental values. The VBM value is also used in the calculation of defect formation energies when electrons are added or removed from a system.

\begin{figure}[ht]
\begin{center}
\begin{tikzpicture}
	\begin{axis}
		[width=\linewidth*0.7, xlabel={Energy (eV)}, ylabel={Electronic density of states}, ymin=0, ymax=12, xmin=0, xmax=16, legend style={{draw=}, at={(0.05,0.95)}, anchor=north west, legend columns=1}]
       \addplot[no marks] table [x=mono_x, y=mono_y,]{dat/eDOS.dat}; \addlegendentry{Monoclinic};
       \addplot[no marks, dashed] table [x=tet_x, y=tet_y,]{dat/eDOS.dat}; \addlegendentry{Tetragonal};
       \addplot[no marks, densely dotted] table [x=cubic_x, y=cubic_y,]{dat/eDOS.dat}; \addlegendentry{Cubic};
			\end{axis}
		\end{tikzpicture}
		\caption{Electronic density of states for the different crystal structures of \zirconia\ showing the band gap predicted by DFT.}
		\label{figure:densityofstates}
	\end{center}
\end{figure}

The electronic density of states show that the VBM and CBM energies increase from monoclinic to tetragonal to cubic \zirconia . This means that at the same Fermi level, the total electronic energy will be smallest in the monoclinic phase and largest in the cubic phase. This corresponds to the correct order of thermal stability that is seen in experiments. Other features that can be seen are the band gaps of the different phases between 7 and 12 eV. These band gaps are significantly underestimated for each phase (see Table \ref{table:bandgap}), as is typical when using a GGA exchange-correlation functional.

\begin{table}[ht] % Band Gap
\onehalfspacing
\centering
\caption[Experimentally determined band gaps alongside values calculated from DFT simulations for each crystal structure of zirconia.]{Experimentally determined band gaps alongside values calculated from DFT simulations for each crystal structure of zirconia. Experimental values taken from \cite{French1994}.}
\begin{tabular}{ccc}
{\bf }                                       & \multicolumn{2}{c}{{\bf Band gap (eV)}}      \\ \hline
\multicolumn{1}{c}{{\bf Crystal Structure}} & \multicolumn{1}{c}{{\bf Expt.}} & {\bf DFT} \\ \hline
\multicolumn{1}{c}{Monoclinic}              & \multicolumn{1}{c}{5.83}        & 3.45      \\
\multicolumn{1}{c}{Tetragonal}              & \multicolumn{1}{c}{5.78}        & 4.00      \\
\multicolumn{1}{c}{Cubic}                   & \multicolumn{1}{c}{6.10}         &   3.55 \\ \hline
\label{table:bandgap}
\end{tabular}
\end{table}


\section{Frenkel and Schottky defects}

\subsection{Incorporation and defect formation energies}

\subsubsection*{Isolated Frenkel defects}

Zr and O Frenkel pair defect formation energies were determined via point defect DFT calculations for the three structures. The formation energies of the isolated Frenkel defect pairs were defined as: % Interstitial iodine defects were simulated in the neutral charge state at different interstitial sites in each phase. The incorporation energy of these defects, assuming a perfect lattice, was calculated using Equation \ref{equation_incorporation}:
\begin{equation}
\label{equation_frenkel}
E_{Frenkel} = E_{DFT}(V^{q}_{X}) + E_{DFT}(X^{-q}_{i}) - 2E_{DFT}(ZrO_2)% - \frac{E_{I_2}}{2}
\end{equation}

where $X$ is either Zr or O, $E_{DFT}(V^{q}_{X})$ is the energy of a supercell of \zirconia\ containing a single vacancy of charge $q$, $E_{DFT}(X^{-q}_{i})$ is the energy of a supercell of \zirconia\ containing a single interstitial with opposing charge $-q$, and $E_{DFT}(ZrO_2)$ is the energy of the non-defective supercell. Charges ranged from the fully charged case (+2 for oxygen vacancies, -4 for zirconium vacancies) to neutral. The interstitial sites, shown in Table \ref{table:interstitials}, were chosen based on standard vacant Wyckoff positions in each crystal structure \cite{theo1996international}.  In the case of oxygen vacancies in monoclinic \zirconia , a defect energy was obtained for both the (III) and (IV) co-ordinated oxygen sites, with the lowest energy value being used in the calculation of the Frenkel defect energy.

\begin{table}[ht] % Wyckoff positions of interstitials
\onehalfspacing
\centering
\caption{Wyckoff positions of interstitial sites used for each crystal structure.}
\label{table:interstitials}
\begin{tabular}{lcc}
\hline
\hspace{0.7 cm} {\bf Crystal Structure} \hspace{0.7 cm}                              & \hspace{0.7 cm} {\bf Interstitial Sites} \hspace{0.7 cm}                                               \\ \hline
\multicolumn{1}{c}{\textbf{Monoclinic}}              & $2a$, $2b$, $2c$, $2d$ \\
\multicolumn{1}{c}{\textbf{Tetragonal}}            & $2b$, $8e$                                   \\
\multicolumn{1}{c}{\textbf{Cubic}}       & $24d$, $4b$                                          \\ \hline
\end{tabular}
\end{table}

The isolated defect formation energies reported in Table \ref{isolated_defects} indicate that fully-charged Schottky defects have the lowest formation energy per atom (most energetically favourable) in all phases, followed by oxygen Frenkel defects and then zirconium Frenkel defects. A trend is seen where the high-temperature phases result in lower formation energies for both Schottky and oxygen Frenkel defects, whereas zirconium Frenkel defects have similar formation energies in all three phases. 

It has been suggested that the relatively small cation size leads to defect structures where oxygen vacancies are favoured over interstitial defects \cite{dwivedi1990computer}. As the zirconium ion is too small to maintain a strong 8-fold bond coordination with its neighbouring oxygen ions, the introduction of oxygen vacancies (which have the added effect of reducing cell volume) will have a stabilising effect.

\begin{table}[ht] % Isolated formation energies
\onehalfspacing
\centering
\caption{Formation energies in eV of isolated \zirconia\ defects.}
\label{isolated_defects}
\begin{tabular}{cccll}
\hline
\multirow{2}{*}{\textbf{Defect}}                      & \multirow{2}{*}{\textbf{Equation}}                                        & \multicolumn{3}{c}{\textbf{Formation Energy (eV)}} \\ \cline{3-5}
	&	& \multicolumn{1}{l}{Monoclinic} & Tetragonal & Cubic \\ \hline
\multirow{5}{*}{\textbf{Zr Frenkel}} & \ch{Zr_{Zr}^{x}} $\rightarrow$ \ch{V_{Zr}^{''''}} + \ch{Zr_{i}^{****}}              & 5.428 & 5.639 & 5.610                             \\
                                     & \ch{Zr_{Zr}^{x}} $\rightarrow$ \ch{V_{Zr}^{'''}} + \ch{Zr_{i}^{***}}               & 8.695 & 8.939 & 8.476                            \\
                                     & \ch{Zr_{Zr}^{x}} $\rightarrow$ \ch{V_{Zr}^{''}} + \ch{Zr_{i}^{**}}                & 12.118 & 12.058 & 11.628                             \\
                                     & \ch{Zr_{Zr}^{x}} $\rightarrow$ \ch{V_{Zr}^{'}} + \ch{Zr_{i}^{*}}                & 16.021 &	15.696 &	13.319                             \\
                                     & \ch{Zr_{Zr}^{x}} $\rightarrow$ \ch{V_{Zr}^{x}} + \ch{Zr_{i}^{x}}                  & 20.563	& 20.094 &	18.170                            \\ \hline
\multirow{3}{*}{\textbf{O Frenkel}}  & \ch{O_{O}^{x}} $\rightarrow$ \ch{V_{O}^{**}} + \ch{O_{i}^{''}}                   & 4.457 &	4.000 & 	3.728                             \\
                                     & \ch{O_{O}^{x}} $\rightarrow$ \ch{V_{O}^{*}} + \ch{O_{i}^{'}}                   & 6.432	& 6.588 &	7.055                             \\
                                     & \ch{O_{O}^{x}} $\rightarrow$ \ch{V_{O}^{x}} + \ch{O_{i}^{x}}                     & 7.518 &	7.452 &	8.477                             \\ \hline
\multirow{3}{*}{\textbf{Schottky}}   & $\varnothing$ $\rightarrow$ \ch{V_{Zr}^{''''}} + 2\ch{V_{O}^{**}} & 5.120 &	3.778	& 1.752                             \\
                                     & $\varnothing$ $\rightarrow$ \ch{V_{Zr}^{''}} + 2\ch{V_{O}^{*}} & 11.353 &	10.832 &	9.624                             \\
                                     & $\varnothing$ $\rightarrow$ \ch{V_{Zr}^{x}} + 2\ch{V_{O}^{x}}   & 18.554 &	18.232 &	17.073  \\ \hline                          
\end{tabular}
\end{table}

\subsubsection*{Bound Frenkel Defects}

Bound Zr and O Frenkel defect formation energies were calculated from DFT energies of supercells where a single ion was moved from its lattice site to an interstitial site. The formation energies of the bound Frenkel defect pairs were defined as:
% Interstitial iodine defects were simulated in the neutral charge state at different interstitial sites in each phase. The incorporation energy of these defects, assuming a perfect lattice, was calculated using Equation \ref{equation_incorporation}:

\begin{equation}
\label{equation_frenkel_bound}
E_{BoundFrenkel} = E_{DFT}(BoundFrenkel) - E_{DFT}(ZrO_2)% - \frac{E_{I_2}}{2}
\end{equation}

where $E_{DFT}(BoundFrenkel)$ is the energy of a supercell of \zirconia\ containing both a vacancy and interstitial defect of the same ion. The two defects were placed as far apart in the supercell as possible (7-8 \r{A}) to avoid recombination. The interstitial defect is assumed to fully compensate the charge of the vacancy defect, resulting in no overall charge on the supercell. The number and type of ions in the defective and non-defective supercell are the same, requiring no further steps to calculate the formation energy. The formation energies calculated for these defects in each crystal structure are presented in Table \ref{table:bound_defects}.

\begin{table}[ht] % Bound formation energies
%\setlength{\tabcolsep}{10pt} % Default value: 6pt
\onehalfspacing
\centering
\caption{Formation energies of bound defects in \zirconia.}
\label{table:bound_defects}
\begin{tabular}{cccc}
\hline
\multirow{2}{*}{\textbf{Defect}} & \multicolumn{3}{c}{\textbf{Formation Energy (eV)}} \\ \cline{2-4} 
 & \textbf{Monoclinic} & \textbf{Tetragonal} & \textbf{Cubic} \\ \hline
\textbf{O Frenkel} & 4.1212 & 4.0290 & 6.4397 \\
\textbf{Zr Frenkel} & 8.4232 & 7.8633 & 6.3274 \\
\textbf{NTV1} & 5.2272 & 3.5813 & 2.6961 \\
\textbf{NTV2} & 5.1405 & 4.2312 & 0.1798 \\
\textbf{NTV3} & 4.6620 & 3.3623 & 2.4089 \\ \hline
\end{tabular}
\end{table}

%Charges ranged from the fully charged case (+2 for oxygen, -4 for zirconium) to neutral. The interstitial sites, shown in Table \ref{table:interstitials}, were chosen based on standard vacant Wyckoff positions in each crystal structure \cite{theo1996international}.  In the case of oxygen vacancies in monoclinic \zirconia , a defect energy was obtained for both the (III) and (IV) co-ordinated oxygen sites. The lowest energies were used in the calculation of the Frenkel defect energy.

\subsubsection*{Isolated Schottky Defects}

Three Schottky energies were calculated for each structure, corresponding to fully charged, partially charged, and uncharged point defect energies. The Schottky formation energy was defined as:

\begin{equation}
\label{equation_schottky}
E_{Schottky} = E_{DFT}(V^{-2q}_{Zr}) + 2E_{DFT}(V^{q}_{O}) -\frac{3(n-1)}{n}E_{DFT}(ZrO_2)% - \frac{E_{I_2}}{2}
\end{equation}

where $n$ denotes the number of atoms in the supercell, $V^{q}_{O}$ denotes an oxygen vacancy with charge $q$, where $q$ varies from 2 to 0. This form maintains both the mass and charge balance of the Schottky defect description for \zirconia :

\begin{equation}
\label{generic_schottky}
Zr^{x}_{Zr} + 2O^{x}_{O} = V^{-2q}_{Zr} + 2V^{q}_{O} + ZrO_{2}
\end{equation}

This implies a rearrangement rather than complete removal of ions from the system. As with the Frenkel defects, the lowest energy vacancy energies were used to calculate Schottky formation energies. While there are multiple configurations of Schottky defects, such nuance cannot be accurately represented through a sum of individual vacancy defect energies. The values presented for Schottky defect formation energies should therefore be considered the lower bound for defect formation. 


\subsubsection*{Bound Schottky Defects}


\begin{figure}[ht] % Tet Zr centre
\centering
\includegraphics[width=8cm]{images/zr_centre_tet.png}
\caption{Zirconium centre  cell showing nearest oxygen atoms in tetragonal \zirconia. Schottky trios indicated by oxygen enumeration with Zr, O and a second oxygen in either the 1\textsuperscript{st}, 2\textsuperscript{nd} or 3\textsuperscript{rd} nearest neighbour with respect to the initial oxygen. Zirconium atoms are shown in green and oxygen atoms in red.}
\label{figure:tetschottky}
\end{figure}

\begin{figure}[ht] % Cubic Zr centre
\centering
\includegraphics[width=8cm]{images/sd_cubic_zro2.png}
\caption{Zirconium centre cell showing nearest oxygen atoms in cubic \zirconia. Schottky trios indicated by oxygen enumeration with Zr, O and a second oxygen in either the 1\textsuperscript{st}, 2\textsuperscript{nd} or 3\textsuperscript{rd} nearest neighbour with respect to the initial oxygen.. Zirconium atoms are shown in green and oxygen atoms in red.}
\label{figure:cubicschottky}
\end{figure}

Bound Schottky defects were modelled in a supercell of \zirconia\ by removing one Zr and two O atoms, in one of several possible nearest neighbour configurations as shown in Figures \ref{figure:monoschottky}, \ref{figure:tetschottky} and \ref{figure:cubicschottky}. Charge neutrality is maintained by the removal of a stoichiometric unit, therefore these defects were defined as neutral tri-vacancies (NTVs). The NTV formation energy was defined as:
\begin{equation}
\label{equation_NTV}
E_{NTV} = E_{DFT}(NTV) - \frac{n-3}{n}E_{DFT}(ZrO_2)% - \frac{E_{I_2}}{2}
\end{equation}

Where $E_{DFT}(NTV)$ is the energy of a supercell containing the NTV defect. As the defective supercell contains three fewer ions than the non-defective cell, the energy of the non-defective cell was adjusted by a proportional factor in our calculation. This form maintains both mass and charge balance of the Schottky defect description for \zirconia\ described in Equation \ref{generic_schottky}.




\section{Defect formation energies} \label{dis_form_energy_intrinsic}

\begin{figure}[ht] % Mono vacancies Fermi level
\begin{center}
\begin{tikzpicture}
	\begin{axis}
		[width=11cm, xlabel={Fermi level $\mu_{e}$ (eV)}, ylabel={Formation energy (eV) per \zirconia\ }, ymin=-10, ymax=18, xmin=0, xmax=6, legend style={{draw=}, at={(0.95,0.95)}, anchor=north east, legend columns=1}]
		\addplot[no marks, blue] table [x=ZRmono1, y=ZRmono2,]{dat/vacancies.dat}; \addlegendentry{Zr};
        \addplot[no marks, red, dashed] table [x=O3mono1, y=O3mono2,]{dat/vacancies.dat}; \addlegendentry{O (III)};
        \addplot[no marks, red] table [x=O4mono1, y=O4mono2,]{dat/vacancies.dat}; \addlegendentry{O (IV)};
			\end{axis}
		\end{tikzpicture}
		\caption{Monoclinic phase formation energies of intrinsic vacancy defects as a function of Fermi level. Gradient indicates defect charge. Oxygen coordination number shown in legend.}
		\label{figure:monovacancies}
	\end{center}
\end{figure}


\begin{figure}[ht] % Tet vacancies Fermi level
\begin{center}
\begin{tikzpicture}
	\begin{axis}
		[width=11cm, xlabel={Fermi level $\mu_{e}$ (eV)}, ylabel={Formation energy (eV) per \zirconia\ }, ymin=-10, ymax=18, xmin=0, xmax=6, legend style={{draw=}, at={(0.95,0.95)}, anchor=north east, legend columns=1}]
		\addplot[no marks, blue] table [x=ZRtet1, y=ZRtet2,]{dat/vacancies.dat}; \addlegendentry{Zr};
        \addplot[no marks, red] table [x=Otet1, y=Otet2,]{dat/vacancies.dat}; \addlegendentry{O};
			\end{axis}
		\end{tikzpicture}
		\caption{Tetragonal phase formation energies of intrinsic vacancy defects as a function of Fermi level. Gradient indicates defect charge.}
		\label{figure:tetvacancies}
	\end{center}
\end{figure}

\begin{figure}[ht]
\begin{center}
\begin{tikzpicture}
	\begin{axis}
		[width=11cm, xlabel={Fermi level $\mu_{e}$ (eV)}, ylabel={Formation energy (eV) per \zirconia\ }, ymin=-10, ymax=18, xmin=0, xmax=6, legend style={{draw=}, at={(0.95,0.95)}, anchor=north east, legend columns=1}]
		\addplot[no marks, blue] table [x=ZRcubic1, y=ZRcubic2,]{dat/vacancies.dat}; \addlegendentry{Zr};
        \addplot[no marks, red] table [x=Ocubic1, y=Ocubic2,]{dat/vacancies.dat}; \addlegendentry{O};
			\end{axis}
		\end{tikzpicture}
		\caption{Cubic phase formation energies of intrinsic vacancy defects as a function of Fermi level. Gradient indicates defect charge.}
		\label{figure:cubicvacancies}
	\end{center}
\end{figure}


\subsection{Defect Volumes}

Tables \ref{defect_volumes_raw} and \ref{defect_volumes_clusters_isolated} show the calculated point defect and cluster defect volumes respectively. The Frenkel and Schottky defect volumes are calculated from the sum of the relevant point defects that would result in an overall neutral charge, with clusters of fully-charged point defects being the expected defect structures in a real material.

\begin{table}[ht!] % Isolated defect volumes
\onehalfspacing
\centering
\caption{Isolated defect volumes in the three \zirconia\ structures.}
\label{defect_volumes_raw}
\begin{tabular}{cccc}
\hline
                      & \multicolumn{3}{c}{\textbf{Defect volume relative to non-defective cell (\r{A}\textsuperscript{3})}}  \\ \cline{2-4} 
\textbf{Defect}       & \textbf{Monoclinic} & \hspace{1cm} \textbf{Tetragonal} & \textbf{Cubic} \\ \hline
\ch{V_{Zr}^{''''}}             & 55.95             & 67.41             & 48.47         \\
\ch{V_{Zr}^{'''}}             & 42.48             &         51.08     &     36.94      \\
\ch{V_{Zr}^{''}}            & 29.90             &  34.28            &     25.93           \\
\ch{V_{Zr}^{'}}             & 17.10             &  18.43            &     15.08           \\
\ch{V_{Zr}^{x}}              & 4.06             &  4.70            &    4.32       \\
\ch{Zr_{i}^{****}}             & -34.62            & -41.94            & -27.34       \\
\ch{Zr_{i}^{***}}             &  -22.76           &	-27.74 		  &	-16.95         \\
\ch{Zr_{i}^{**}}             &  -11.79 	        &	-12.02 		  &	-6.24          \\
\ch{Zr_{i}^{*}}            &  2.68			& -0.02 		  & 	4.69             \\
\ch{Zr_{i}^{x}}              &  15.94		 	& 13.40	 		  & 15.97         \\
\ch{V_{O}^{**}} {[}4coord{]} & -22.52            & -37.53            & -22.76       \\
\ch{V_{O}^{*}} {[}4coord{]} &  -12.41           &    -19.53         &     -12.19           \\
\ch{V_{O}^{x}} {[}4coord{]}  &  -0.69          &  -2.80           &      -1.11          \\
\ch{V_{O}^{**}} {[}3coord{]} & -26.13            &                     &                \\
\ch{V_{O}^{*}} {[}3coord{]} &  -14.42           &                     &                \\
\ch{V_{O}^{x}} {[}3coord{]}  &   -1.71          &                     &                \\
\ch{O_{i}^{''}}              & 27.01             & 40.00              & 28.58        \\
\ch{O_{i}^{'}}              &  15.36            &    24.56         &  16.30              \\
\ch{O_{i}^{x}}               & 2.66             &    11.06          &   8.95        \\ \hline
\end{tabular}
\end{table}

\begin{table}[ht] % Isolated Frenkel volumes
\onehalfspacing
\centering
\caption{Isolated cluster defect volumes in the three \zirconia\ structures.}
\label{defect_volumes_clusters_isolated}
\begin{tabular}{cccc}
\hline
\multirow{2}{*}{\textbf{Defect}}   & \multicolumn{3}{c}{\textbf{Defect volume (\r{A}\textsuperscript{3})}}  \\ \cline{2-4} 
 & \textbf{Monoclinic} & \textbf{Tetragonal} & \textbf{Cubic} \\ \hline
\ch{V_{Zr}^{''''}} + \ch{Zr_{i}^{****}}          & 21.331	 & 25.4702 &	21.1309         \\
\ch{V_{Zr}^{'''}} + \ch{Zr_{i}^{***}}          & 19.7155 &	23.3463 &	19.9954      \\
\ch{V_{Zr}^{''}} + \ch{Zr_{i}^{**}}          & 18.1149 &	22.2525 &	19.68618           \\
\ch{V_{Zr}^{'}} + \ch{Zr_{i}^{*}}          & 19.78339 &	18.4096913 &	19.76396           \\
\ch{V_{Zr}^{x}} + \ch{Zr_{i}^{x}}          & 19.99485 &	18.1061 &	20.29223       \\
\ch{V_{O}^{**}} + \ch{O_{i}^{''}}           & 0.8839 &	2.4704 &	5.8217       \\
\ch{V_{O}^{*}} + \ch{O_{i}^{'}}           &  0.9486 &	5.032 &	4.1146        \\
\ch{V_{O}^{x}} + \ch{O_{i}^{x}}           &  0.9576 &	8.26065 &	7.83687          \\
\ch{V_{Zr}^{''''}} + 2\ch{V_{O}^{**}}       &  3.6979 &	-7.647 &	2.9448             \\
\ch{V_{Zr}^{''}} + 2\ch{V_{O}^{*}}       &  1.0707 &	-4.7866 &	1.5564         \\
\ch{V_{Zr}^{x}} + 2\ch{V_{O}^{x}}        & 0.64517 &	-0.8985 &	2.08973       \\ \hline
\end{tabular}
\end{table}

The oxygen Frenkel defect has the smallest defect volume in the monoclinic phase, followed by the tetragonal phase. This can be explained by the competition between phase density and matrix stiffness. As the monoclinic phase has the highest specific volume (see Table \ref{lattice_params}), we can argue that the monoclinic phase can best absorb the lattice strains imposed by the defect, despite having a lower stiffness than the cubic phase.

The zirconium Frenkel defect is significantly larger than the oxygen Frenkel, mostly due to the large positive strain contribution from the zirconium vacancy. This can explain the larger defect formation energy of zirconium Frenkel defects.

%\subsubsection*{Isolated Defects}
%
%The isolated defect formation energies reported in Table \ref{isolated_defects} indicate that fully-charged Schottky defects have the lowest formation energy per atom (most energetically favourable) in all phases, followed by oxygen Frenkel defects. A trend is seen where the high-temperature phases result in lower formation energies for both Schottky and oxygen Frenkel defects, whereas zirconium Frenkel defects have similar formation energies in all three phases. It has been suggested that the relatively small cation size leads to defect structures where oxygen vacancies are favoured over interstitial defects \cite{dwivedi1990computer}. As the zirconium ion is too small to maintain a strong 8-fold bond coordination with its neighbouring oxygen ions, the introduction of oxygen vacancies (which have the added effect of reducing cell volume) will have a stabilising effect.


\subsubsection*{Bound Defects}
The bound defect formation energies shown in Table \ref{table:bound_defects} show that NTV defects, on a per defect atom basis, are the most energetically favourable defects, followed by oxygen and zirconium Frenkel defects respectively. The NTV3 exhibited the smallest formation energy in all three crystal structures, with a single exception of the NTV2 in the cubic phase where a much smaller formation energy was observed due to collapse\footnote{Upon inspecting the output cell, it was found that all the oxygen atoms shifted positions along the [001] direction, becoming more like the tetragonal \zirconia\ structure. The cell size was constrained so the lattice parameters could not be changed, so this was not a true tetragonal cell.} of the supercell during geometry optimisation.

\section{Elastic constants and defect relaxation volumes}

Table \ref{stiffness_tensor} shows the calculated elastic constants for the monoclinic, tetragonal, and cubic phases of \zirconia . The cubic phase has the highest stiffness, likely due to the short Zr-O bond lengths in the energy-minimised structure (Figure \ref{figure:zrobonddistance}). It is expected that at high temperatures where the cubic phase is stable, the resulting increase in bond length would cause a reduction in stiffness. 

\begin{table}[ht] % Elastic constants
\onehalfspacing
\centering
\caption{Elastic constants for different phases of \zirconia\ from DFT calculations.}
\label{stiffness_tensor}
\begin{tabular}{cccc}
\hline
\multirow{2}{*}{\textbf{Elastic Component}} & \multicolumn{3}{c}{\textbf{Stiffness (GPa)}}               \\ \cline{2-4} 
                                            & \textbf{Monoclinic} & \textbf{Tetragonal} & \textbf{Cubic} \\ \hline
$C_{11}$                                         & 338.86        & 334.30               & 523.38    \\
$C_{12}$                                         & 151.80        & 207.30               & 92.93     \\
$C_{13}$                                         & 89.37         & 48.93               & 92.93     \\
$C_{22}$                                         & 348.37        & 334.20               & 523.39    \\
$C_{23}$                                         & 143.04        & 48.93               & 92.93    \\
$C_{33}$                                         & 262.17        & 250.50               & 523.38   \\
$C_{44}$                                         & 76.35         & 9.38                & 61.98    \\
$C_{55}$                                         & 71.65         & 9.38                & 61.98   \\
$C_{66}$                                         & 114.19        & 152.60               & 61.99     \\ \hline
\end{tabular}
\end{table}


The monoclinic and tetragonal phases have similar stiffness along the principal axes, but vary significantly under shearing conditions. In particular, the tetragonal phase exhibits much smaller $C_{44}$ and $C_{55}$ components. This may be attributed to the strong directional anisotropy of the tetragonal phase due to the larger $c$ parameter. 





\section{Helmholtz energies}

%The calculated Helmholtz energies plotted in Figure \ref{Figure:helmholtz} show the correct order of stability for the three phases of \zirconia\ at low temperatures (monoclinic $\rightarrow$ tetragonal $\rightarrow$ cubic) . However, as temperature is increased, only a transition from monoclinic to tetragonal is seen. The cubic phase Helmholtz energy does fall below the monoclinic curve, but never below the tetragonal curve, thus predicting no tetragonal to cubic phase transition. 

%It must also be noted that the transition temperatures are not predicted accurately. The tetragonal phase is predicted to have a lower energy than monoclinic at approximately 400 K, while experiments indicate that the transition temperature is above 1400 K. This difference is too large to attribute to a kinetic barrier.

The Helmholtz free energy results (Figure \ref{Figure:helmholtz}) show the correct order of crystal structure stability at low temperature. A transition from the monoclinic to tetragonal crystal structure is seen at 390K, but no further transition is seen from tetragonal to cubic. The low monoclinic-tetragonal transition temperature may be due to both the kinetic barrier \cite{bansal1972martensitic,bansal1974martensitic}, and the inability of the constant volume harmonic model to take into account the effects of thermal expansion. The lack of an observed transition to the cubic phase may indicate an inability to accurately simulate the high-temperature phase using DFT techniques. 

\begin{figure}[ht] % Helmholtz
\begin{center}
\begin{tikzpicture}
	\begin{axis}
		[width=\linewidth*0.7, xlabel={Temperature (K)}, ylabel={Helmholtz free energy (eV)}, ymin=-28, ymax=-10, xmin=0, xmax=3000, legend style={{draw=}, at={(0.95,0.95)}, anchor=north east, legend columns=1}]
		\addplot[no marks] table [x=temperature, y=monoclinic,]{dat/helmholtz.dat}; \addlegendentry{Monoclinic};
        \addplot[no marks, dashed] table [x=temperature, y=tetragonal, ]{dat/helmholtz.dat}; \addlegendentry{Tetragonal};
        \addplot[no marks, densely dotted, black] table [x=temperature, y=cubic,]{dat/helmholtz.dat}; \addlegendentry{Cubic};
			\end{axis}
		\end{tikzpicture}
		\caption{Helmholtz free energy as a function of temperature for the monoclinic, tetragonal, and cubic crystal structures of \zirconia .}
		\label{Figure:helmholtz}
	\end{center}
\end{figure}

\section{Defect equilibria} \label{brouwer_discussion_intrinsic}

\subsubsection*{Monoclinic}

The monoclinic Brouwer diagram (Figure \ref{figure:mono_intrinsic_brouwer}) predicts that at 635 K, few types of defects will be present and at very low (\textless 10 ppb \zirconia ) concentrations. This is typical of defect behaviour in a ceramics at temperatures far below their melting points \cite{kingery1997physical,ball2006computer}. Fully-charged zirconium vacancies, charge-compensated by holes, are the major defect type we expect to observe at $p_{O_{2}}$ \textgreater $10^{-15}$ atm. Below this, only electronic defects compensated by electron hole defects are expected. 

%We briefly see increased concentrations of uncharged oxygen interstitial defects at very high levels of $p_{O_{2}}$.
%The intrinsic defect equilibria are shown in Brouwer diagrams in Figures \ref{figure:mono_intrinsic_brouwer}, \ref{figure:tet_intrinsic_brouwer} and \ref{figure:cubic_intrinsic_brouwer}. The monoclinic phase exhibits the smallest overall concentration of intrinsic defects due to the low temperature (650 K) relative to the other tetragonal (1500 K) and cubic (XXX K) phases.

\begin{figure}[ht] % Mono intrinsic Brouwer
\begin{center}
\begin{tikzpicture}
	\begin{axis}
		[width=\linewidth*0.7, xlabel={\ch{log_{10}}($p_{O_{2}}$) (atm)}, ylabel={\ch{log_{10}}([D]) (per f.u.)}, ymin=-18, ymax=0, xmin=-35, xmax=0, legend style={{draw=}, at={(0.40,0.94)}, anchor=north west, legend columns=4, nodes={scale=1, transform shape}}]
        \addplot[no marks, draw=blue!70!black] table [x=pO2, y=electrons,]{dat/intrinsic_mono.dat}; \addlegendentry{\ch{e^{'}}}; \node at (-4.8,-13) {\ch{e^{'}}};
        \addplot[no marks, draw=red!85!black] table [x=pO2, y=holes,]{dat/intrinsic_mono.dat}; \addlegendentry{\ch{h^{\textperiodcentered}}}; \node at (-4.5,-8) {\ch{h^{\textperiodcentered}}};
        \addplot[no marks, draw=black!70!green] table [x=pO2, y=VO{2},]{dat/intrinsic_mono.dat}; \addlegendentry{\ch{V_{O}^{\textperiodcentered\textperiodcentered}}}; \node at (-33,-16.5) {\ch{V_{O}^{\textperiodcentered\textperiodcentered}}};
%         \addplot[no marks, draw=black!55!green] table [x=pO2, y=VO{1},]{dat/intrinsic_mono.dat}; \addlegendentry{\ch{V_{O}^{*}}};
%         \addplot[no marks, draw=black!30!green] table [x=pO2, y=VO{0},]{dat/intrinsic_mono.dat}; \addlegendentry{\ch{V_{O}^{x}}};
        \addplot[no marks, draw=yellow!85!blue] table [x=pO2, y=VM{-4},]{dat/intrinsic_mono.dat}; \addlegendentry{\ch{V_{Zr}^{''''}}}; \node at (-5,-10.5) {\ch{V_{Zr}^{''''}}};
%         \addplot[no marks, draw=yellow!75!blue] table [x=pO2, y=VM{-3},]{dat/intrinsic_mono.dat}; \addlegendentry{\ch{V_{Zr}^{'''}}};
%         \addplot[no marks, draw=yellow!65!blue] table [x=pO2, y=VM{-2},]{dat/intrinsic_mono.dat}; \addlegendentry{\ch{V_{Zr}^{''}}};
%         \addplot[no marks, draw=yellow!55!blue] table [x=pO2, y=VM{-1},]{dat/intrinsic_mono.dat}; \addlegendentry{\ch{V_{Zr}^{'}}};
%         \addplot[no marks, draw=yellow!45!blue] table [x=pO2, y=VM{0},]{dat/intrinsic_mono.dat}; \addlegendentry{\ch{V_{Zr}^{x}}};
%         \addplot[no marks, draw=red!60!yellow] table [x=pO2, y=Oi{-2},]{dat/intrinsic_mono.dat}; \addlegendentry{\ch{O_{i}^{''}}};
%         \addplot[no marks, draw=red!50!yellow] table [x=pO2, y=Oi{-1},]{dat/intrinsic_mono.dat}; \addlegendentry{\ch{O_{i}^{'}}};
%         \addplot[no marks, draw=red!40!yellow] table [x=pO2, y=Oi{0},]{dat/intrinsic_mono.dat}; \addlegendentry{\ch{O_{i}^{x}}};
%         \addplot[no marks, draw=green!80!pink] table [x=pO2, y=Mi{4},]{dat/intrinsic_mono.dat}; \addlegendentry{\ch{Zr_{i}^{****}}};
%         \addplot[no marks, draw=green!70!pink] table [x=pO2, y=Mi{3},]{dat/intrinsic_mono.dat}; \addlegendentry{\ch{Zr_{i}^{***}}};
%         \addplot[no marks, draw=green!60!pink] table [x=pO2, y=Mi{2},]{dat/intrinsic_mono.dat}; \addlegendentry{\ch{Zr_{i}^{\textbf{**}}}};
%         \addplot[no marks, draw=green!50!pink] table [x=pO2, y=Mi{1},]{dat/intrinsic_mono.dat}; \addlegendentry{\ch{Zr_{i}^{*}}};
%         \addplot[no marks, draw=green!40!pink] table [x=pO2, y=Mi{0},]{dat/intrinsic_mono.dat}; \addlegendentry{\ch{Zr_{i}^{x}}};
%         \addplot[no marks] table [x=pO2, y=Stoich,]{dat/intrinsic_mono.dat}; \addlegendentry{Stoich};
%\node at (-33.7,-0.5) {\textbf{a)}};
			\end{axis}            
\end{tikzpicture}
		\caption{Monoclinic phase Brouwer diagram of intrinsic defects at 650 K.}
		\label{figure:mono_intrinsic_brouwer}
	\end{center}
\end{figure}


\subsubsection*{Tetragonal}

Figure \ref{figure:tet_intrinsic_brouwer} shows a much greater concentration of defects across a wide range of $p_{O_{2}}$, mainly owing to an elevated temperature of 1500 K where the tetragonal crystal structure is fully stabilised. At low levels of $p_{O_{2}}$, electronic defects are again the dominant defect, but are now charge-compensated by the formation of fully-charged oxygen vacancies. A clear neutrality condition is seen at a $p_{O_{2}}$ of $10^{-11}$ atm where $[\ch{V_{O}^{**}}] = 2[\ch{V_{Zr}^{''''}}]$, with higher levels of $p_{O_{2}}$ being dominated by fully-charged zirconium vacancies charge-compensated by the formation of electron hole defects.

\begin{figure}[ht] % Tet intrinsic Brouwer
\begin{center}
\begin{tikzpicture}
	\begin{axis}
		[width=\linewidth*0.7, xlabel={\ch{log_{10}}($p_{O_{2}}$) (atm)}, ylabel={\ch{log_{10}}([D]) (per f.u.)}, ymin=-10, ymax=0, xmin=-35, xmax=0, legend style={{draw=}, at={(0.40,0.97)}, anchor=north west, legend columns=4, nodes={scale=1, transform shape}}]
        \addplot[no marks, draw=blue!70!black] table [x=pO2, y=electrons,]{dat/intrinsic_tet.dat}; \addlegendentry{\ch{e^{'}}}; \node at (-26.0,-2) {\ch{e^{'}}};
        \addplot[no marks, draw=red!85!black] table [x=pO2, y=holes,]{dat/intrinsic_tet.dat}; \addlegendentry{\ch{h^{\textperiodcentered}}}; \node at (-7,-3.6) {\ch{h^{\textperiodcentered}}};
        \addplot[no marks, draw=black!70!green] table [x=pO2, y=VO{2},]{dat/intrinsic_tet.dat}; \addlegendentry{\ch{V_{O}^{\textperiodcentered\textperiodcentered}}}; \node at (-28,-3) {\ch{V_{O}^{\textperiodcentered\textperiodcentered}}};
%         \addplot[no marks, draw=black!55!green] table [x=pO2, y=VO{1},]{dat/intrinsic_tet.dat}; \addlegendentry{\ch{V_{O}^{*}}};
%         \addplot[no marks, draw=black!30!green] table [x=pO2, y=VO{0},]{dat/intrinsic_tet.dat}; \addlegendentry{\ch{V_{O}^{x}}};
        \addplot[no marks, draw=yellow!85!blue] table [x=pO2, y=VM{-4},]{dat/intrinsic_tet.dat}; \addlegendentry{\ch{V_{Zr}^{''''}}}; \node at (-3,-4.5) {\ch{V_{Zr}^{''''}}};
%         \addplot[no marks, draw=yellow!75!blue] table [x=pO2, y=VM{-3},]{dat/intrinsic_tet.dat}; \addlegendentry{\ch{V_{Zr}^{'''}}};
%         \addplot[no marks, draw=yellow!65!blue] table [x=pO2, y=VM{-2},]{dat/intrinsic_tet.dat}; \addlegendentry{\ch{V_{Zr}^{''}}};
%         \addplot[no marks, draw=yellow!55!blue] table [x=pO2, y=VM{-1},]{dat/intrinsic_tet.dat}; \addlegendentry{\ch{V_{Zr}^{'}}};
%         \addplot[no marks, draw=yellow!45!blue] table [x=pO2, y=VM{0},]{dat/intrinsic_tet.dat}; \addlegendentry{\ch{V_{Zr}^{x}}};
%         \addplot[no marks, draw=red!60!yellow] table [x=pO2, y=Oi{-2},]{dat/intrinsic_tet.dat}; \addlegendentry{\ch{O_{i}^{''}}};
%         \addplot[no marks, draw=red!50!yellow] table [x=pO2, y=Oi{-1},]{dat/intrinsic_tet.dat}; \addlegendentry{\ch{O_{i}^{'}}};
%         \addplot[no marks, draw=red!40!yellow] table [x=pO2, y=Oi{0},]{dat/intrinsic_tet.dat}; \addlegendentry{\ch{O_{i}^{x}}};
%         \addplot[no marks, draw=green!80!pink] table [x=pO2, y=Mi{4},]{dat/intrinsic_tet.dat}; \addlegendentry{\ch{Zr_{i}^{****}}};
%         \addplot[no marks, draw=green!70!pink] table [x=pO2, y=Mi{3},]{dat/intrinsic_tet.dat}; \addlegendentry{\ch{Zr_{i}^{***}}};
%         \addplot[no marks, draw=green!60!pink] table [x=pO2, y=Mi{2},]{dat/intrinsic_tet.dat}; \addlegendentry{\ch{Zr_{i}^{\textbf{**}}}};
%         \addplot[no marks, draw=green!50!pink] table [x=pO2, y=Mi{1},]{dat/intrinsic_tet.dat}; \addlegendentry{\ch{Zr_{i}^{*}}};
%         \addplot[no marks, draw=green!40!pink] table [x=pO2, y=Mi{0},]{dat/intrinsic_tet.dat}; \addlegendentry{\ch{Zr_{i}^{x}}};
%         \addplot[no marks] table [x=pO2, y=Stoich,]{dat/intrinsic_tet.dat}; \addlegendentry{Stoich};
%\node at (-33.7,-0.5) {\textbf{a)}};
			\end{axis}            
\end{tikzpicture}
		\caption{Tetragonal phase Brouwer diagrams of intrinsic defects at 1500 K.}
		\label{figure:tet_intrinsic_brouwer}
	\end{center}
\end{figure}

\begin{figure}[ht] % cubic intrinsic Brouwer
\begin{center}
\begin{tikzpicture}
	\begin{axis}
		[width=\linewidth*0.7, xlabel={\ch{log_{10}}($p_{O_{2}}$) (atm)}, ylabel={\ch{log_{10}}([D]) (per f.u.)}, ymin=-10, ymax=0, xmin=-35, xmax=0, legend style={{draw=}, at={(0.60,0.97)}, anchor=north west, legend columns=3, nodes={scale=1, transform shape}}]
        \addplot[no marks, draw=blue!70!black] table [x=pO2, y=electrons,]{dat/intrinsic_cubic.dat}; \addlegendentry{\ch{e^{'}}}; \node at (-17.0,-1) {\ch{e^{'}}};
        \addplot[no marks, draw=red!85!black] table [x=pO2, y=holes,]{dat/intrinsic_cubic.dat}; \addlegendentry{\ch{h^{\textperiodcentered}}}; \node at (-8,-7.5) {\ch{h^{\textperiodcentered}}};
        \addplot[no marks, draw=black!70!green] table [x=pO2, y=VO{2},]{dat/intrinsic_cubic.dat}; \addlegendentry{\ch{V_{O}^{\textperiodcentered\textperiodcentered}}}; \node at (-20,-1.7) {\ch{V_{O}^{\textperiodcentered\textperiodcentered}}};
%         \addplot[no marks, draw=black!55!green] table [x=pO2, y=VO{1},]{dat/intrinsic_cubic.dat}; \addlegendentry{\ch{V_{O}^{*}}};
%         \addplot[no marks, draw=black!30!green] table [x=pO2, y=VO{0},]{dat/intrinsic_cubic.dat}; \addlegendentry{\ch{V_{O}^{x}}};
        \addplot[no marks, draw=yellow!85!blue] table [x=pO2, y=VM{-4},]{dat/intrinsic_cubic.dat}; \addlegendentry{\ch{V_{Zr}^{''''}}}; \node at (-20,-6) {\ch{V_{Zr}^{''''}}};
%         \addplot[no marks, draw=yellow!75!blue] table [x=pO2, y=VM{-3},]{dat/intrinsic_cubic.dat}; \addlegendentry{\ch{V_{Zr}^{'''}}};
%         \addplot[no marks, draw=yellow!65!blue] table [x=pO2, y=VM{-2},]{dat/intrinsic_cubic.dat}; \addlegendentry{\ch{V_{Zr}^{''}}};
%         \addplot[no marks, draw=yellow!55!blue] table [x=pO2, y=VM{-1},]{dat/intrinsic_cubic.dat}; \addlegendentry{\ch{V_{Zr}^{'}}};
%         \addplot[no marks, draw=yellow!45!blue] table [x=pO2, y=VM{0},]{dat/intrinsic_cubic.dat}; \addlegendentry{\ch{V_{Zr}^{x}}};
%         \addplot[no marks, draw=red!60!yellow] table [x=pO2, y=Oi{-2},]{dat/intrinsic_cubic.dat}; \addlegendentry{\ch{O_{i}^{''}}};
%         \addplot[no marks, draw=red!50!yellow] table [x=pO2, y=Oi{-1},]{dat/intrinsic_cubic.dat}; \addlegendentry{\ch{O_{i}^{'}}};
%         \addplot[no marks, draw=red!40!yellow] table [x=pO2, y=Oi{0},]{dat/intrinsic_cubic.dat}; \addlegendentry{\ch{O_{i}^{x}}};
%         \addplot[no marks, draw=green!80!pink] table [x=pO2, y=Mi{4},]{dat/intrinsic_cubic.dat}; \addlegendentry{\ch{Zr_{i}^{****}}};
%         \addplot[no marks, draw=green!70!pink] table [x=pO2, y=Mi{3},]{dat/intrinsic_cubic.dat}; \addlegendentry{\ch{Zr_{i}^{***}}};
%         \addplot[no marks, draw=green!60!pink] table [x=pO2, y=Mi{2},]{dat/intrinsic_cubic.dat}; \addlegendentry{\ch{Zr_{i}^{\textbf{**}}}};
%         \addplot[no marks, draw=green!50!pink] table [x=pO2, y=Mi{1},]{dat/intrinsic_cubic.dat}; \addlegendentry{\ch{Zr_{i}^{*}}};
%         \addplot[no marks, draw=green!40!pink] table [x=pO2, y=Mi{0},]{dat/intrinsic_cubic.dat}; \addlegendentry{\ch{Zr_{i}^{x}}};
%         \addplot[no marks, dashed, draw=red!70!black] table [x=pO2, y=Ii{0},]{dat/intrinsic_cubic.dat}; \addlegendentry{\ch{I_{i}^{x}}};
%         \addplot[no marks] table [x=pO2, y=Stoich,]{dat/intrinsic_cubic.dat}; \addlegendentry{Stoich};
%\node at (-33.7,-0.5) {\textbf{a)}};
			\end{axis}            
\end{tikzpicture} 
		\caption{Cubic phase Brouwer diagrams of intrinsic defects at 2000 K.}
		\label{figure:cubic_intrinsic_brouwer}
	\end{center}
\end{figure}

\section{Summary}

The main defects in \zirconia\ are oxygen vacancies and zirconium vacancies at low and high oxygen pressures respectively. 

The cubic phase cannot be modelled accurately due to instabilities outlined by Burr \emph{et al}. \cite{burr2017importance}. For this reason, it was decided that the cubic phase would not be considered when conducting extrinsic dopant simulation studies in \zirconia .

\chapter{Iodine defect energies and equilibria in \zirconia}

\emph{The work in this chapter has been published in:} \\ A. Kenich \emph{et al.} J. Nucl. Mater. \textbf{511} (2018) 390-395. Ref \cite{kenichiodine2018}.

\label{ch:results2}

\section{Introduction}

As discussed in Chapter \ref{introduction}, stress-corrosion cracking (SCC) in nuclear fuel pins is an issue related to early integrity of fuel assemblies in light water reactors (LWRs). SCC studies of the internal surface of zirconium-based fuel claddings have been conducted, which indicate that iodine is likely to be one of the main corrosive species involved in promoting crack growth \cite{rosenbaum1966interaction, bcoxpelletclad1990,fregonese2000failure,Sidky1998}. The exact mechanism for iodine SCC has not yet been determined due to difficulties observing the internal cladding surface in-situ, while experimental studies are not yet capable of reproducing the conditions under which such failures occur. The quantum-mechanical simulation approach is therefore particularly useful to model the behaviour of iodine within the oxide layer of the cladding, the layer preceding the zirconium metal. 

Iodine is produced in the fuel pellet directly from fission (see Chapter I for details) and also from the decay of tellurium precursors. As shown in Figure \ref{table:decaydata_chap1}, both iodine and tellurium are relatively common fission products, with combined independent yields from thermal fission of U$_{235}$ above 26\% \cite{kennett1956mass, iodine129fissionyield, imanishi1976independent, iodinefissionyields, iodine132, amiel1975odd, iaeafissionyield}. The majority of thermal fission events occur in the outer rim of the fuel pellet, and a fission product penetration depth of up to 8 $\mu$m in \zirconia\ \cite{degueldre2001behaviour} suggests a large degree of implantation within the oxide and the Zr metal into which the oxide grows, raising the concentration of I well above the equilibrium value. Iodine and many of its relevant compounds (as ZrI$_{4}$, CsI) are volatile and fuel pellets contain many cracks and spaces through which iodine may be rapidly transported to the cladding. When reactor power is increased during start-up, iodine is released in substantial quantities from the UO$_{2}$ pellet \cite{peehs1982experimental}. This is believed to cause crack propagation in the cladding when combined with stresses imposed on the cladding by the fuel pellet, and this contributes to pellet-cladding interaction (PCI), a phenomenon discussed in Chapter \ref{introduction}. Upper limits on power ramping and holding times have therefore been established by fuel suppliers to mitigate potential PCI failures \cite{yagnik2005effect}. While these restrictions have reduced or prevented the incidence of PCI failures, they also impose costs on the operator due to longer ramping periods. This also restricts the ability of the nuclear reactor to load-follow grid demand. Cladding/fuel materials resistant to PCI failure are therefore of great interest in the nuclear power industry, promoting research into solutions such as cladding liners and doped fuel pellets \cite{nonon2005pci,yang2012effect}. 

%\item Power ramping: Increasing power, such as during reactor start-up, can lead to cladding failure.   %power must be ramped up gradually in order to avoid excessive temperature gradients in the fuel pins, but also to manage fission product concentrations. Due to the different half-lives of various fission products, a power ramp will cause a transient increase in the iodine concentration within a fuel pin


Iodine is an oxidising agent, which, under standard conditions, will oxidise Zr metal to produce ZrI$_{4}$. However, oxygen is also present in the internal fuel pin environment, both from the native \zirconia\ layer on the cladding, and the evolution of oxygen from the fuel pellet during burnup. Liberated oxygen will compete with iodine in the oxidation of the Zr metal, but whereas iodine promotes crack growth under stress, oxygen provides a more protective effect, self-limiting its diffusion into the metal \cite{farina2002stress, causey2005review}. Furthermore, oxygen is a more powerful oxidising agent than iodine, reacting together to produce I${_2}$O$_{5}$. For these reasons, the internal oxide layer of the cladding is often considered a barrier to the ingress of iodine into the Zr metal. 

Unlike oxygen and hydrogen, which readily diffuse into Zr metal to occupy interstitial sites, iodine atoms have been predicted in atomistic studies to have very high energy barriers to bulk interstitial diffusion \cite{rossi2015first,legris2005ab,carlot2002energetically}. This is due to the relatively large radius of the iodine atom, which imposes large local strains when penetrating the Zr lattice. This suggests that iodine will instead be transported towards crack tips via grain boundaries. Indeed, intergranular cracking has been observed in PCI failures, but only for a few hundred nm before a more rapid transgranular crack propagation \cite{fregonese2000failure, une1984threshold, wood1983effects, lunde1981stress, vilpponen1981fuel}. Conversely, no atomic scale studies of iodine in \zirconia\ were found in the literature.  

As discussed in § \ref{section:outervsinner}, there is an oxide layer on the internal surface of the cladding consisting of monoclinic and tetragonal oxide grains. The effectiveness of the oxide layer as a barrier to iodine is debated, with one study presuming that the oxide is bypassed entirely by iodine due to fracturing, leaving the Zr metal underneath exposed \cite{rossi2015first}. The outermost part of the oxide, which is porous, exhibits networks of interconnected grain boundary diffusion pathways towards the oxide/metal interface which are certainly wide enough (1-3 nm) to allow iodine transport \cite{ni2010porosity}. The oxygen-saturated Zr at the oxide/metal interface is not, however, taken into account, and it is expected that this will influence the corrosion mechanism due to iodine-oxygen competition: even the much smaller hydrogen atom has its rate of diffusion into the metal reduced by the presence of oxygen, as shown in both computational \cite{glazoff2014oxidation} and experimental hydrogen pick-up studies \cite{couet2014hydrogen}. This means that some barrier to iodine ingress must already exist near the oxide/metal interface. The varying levels of oxygen across the oxide layer itself also have an effect on defect behaviour, and will therefore influence the initiation mechanisms behind PCI failures. Thus here, we predict iodine incorporation energies and defect equilibria in \zirconia\ as a function of oxygen pressure through Brouwer diagrams, in order to predict the resulting iodine defect response.


%\subsection{Pellet-cladding interaction}
%
%Pellet-cladding interaction refers to the interaction between the fuel and the cladding at higher burnups where the gas gap has been closed by the swelling of the fuel pellets. PCI has both a mechanical and a chemical component, sometimes referred to specifically as pellet cladding mechanical interaction (PCMI) and pellet cladding chemical interaction (PCCI) respectively.
%

\section{Methodology}
\subsection{Computational details}

As discussed in Chapter \ref{ch:compmethodology}, calculations were performed using CASTEP 8.0 \cite{Clark2005}. Ultra-soft pseudo-potentials with a cut-off energy of 600 eV were employed. The Perdew, Burke and Ernzerhof (PBE) \cite{Perdew1996} parameterisation of the generalised gradient approximation (GGA) was used to describe the exchange correlation functional. A Monkhorst-Pack sampling scheme \cite{Monkhorst1976} was used for Brillouin zone integration, with a minimum \emph{k}-point separation of 0.09 \r{A}\textsuperscript{-1}. The Pulay method for density mixing \cite{Pulay1980} was used to improve simulation convergence. 

The electronic energy convergence criterion was set to $1\times10^{-6}$ eV and the maximum force between atoms limited to $1\times10^{-2}$ eV \r{A}\textsuperscript{-1}, which are values demonstrated as appropriate in § \ref{convergence_criteria}. A gradient-descent geometry optimisation task was run on the cell until consecutive iterations differed in energy and atomic displacement by less than $1\times10^{-5}$ eV and $5\times10^{-4}$ \r{A} respectively, again demonstrated in § \ref{convergence_criteria}. 


\subsection{Defect Equilibrium Response to Oxygen Partial Pressure}

Brouwer diagrams were generated using the method previously outlined in § \ref{brouwer_method}. Defect concentrations for the monoclinic and tetragonal phases were calculated at 650 K and 1500 K respectively to reflect the temperatures at which these structures are stable. Brouwer diagrams at extrinsic defect concentrations of $10^{-5}$ and $10^{-3}$ parts/fu (i.e. parts per \zirconia\ formula unit) were generated to examine low and high dopant concentrations, respectively. These two concentrations were examined because the amount of fission products present at a particular point in a fuel pellet depends on macroscopic parameters, including its position in the core and the time since the last shutdown, but also microscopic parameters such as the radial position in the pellet. These two concentrations were selected because $10^{-3}$ parts/fu is high enough to model an aggregation of iodine (such as at a crack tip), and $10^{-5}$ parts/fu was found to be the concentration below which iodine did not have a significant effect on defect equilibria. 


%Brouwer diagrams, also known as Kr{\"o}ger-Vink diagrams, were produced using a method outlined by Murphy et al. \cite{Murphy2014} to determine defect concentrations as a function of oxygen partial pressure. We start from the statement that the chemical potential of \zirconia\ is equivalent to the sum of the chemical potentials $\mu$ of its constituent species, Zr and O:
%
%\begin{equation}
%{\mu}_{ZrO_2(s)} = {\mu}_{Zr}(p_{O_2}, T) + {\mu}_{O_2}(p_{O_2}, T)
%\label{mewZrO2results2}
%\end{equation}
%
%where $T$ denotes temperature and $p_{O_2}$ denotes oxygen partial pressure. The chemical potential of \zirconia\ in the solid state is assumed to have negligible dependence on $T$ and $p_{O_2}$ relative to ${\mu}_{Zr}$ and ${\mu}_{O_2}$. Energies can be obtained for bulk \zirconia\ and Zr, but the ground state of oxygen is not correctly reproduced in DFT \cite{Batyrev2000,Lozovoi2001}. Instead, we use the approach of Finnis et al. \cite{Finnis2005} to infer the oxygen chemical potential from standard state values. We can use the experimental Gibbs free energy to produce an equation where $\mu_{O_2}$ is the only unknown:
%
%\begin{equation}
%\Delta{G^{\plimsoll}_{f, ZrO_2}} = \mu_{ZrO_2(s)} - (\mu_{Zr(s)} + \mu^{\plimsoll}_{O_2})
%\end{equation}
%
%where $\Delta{G^{\plimsoll}_{f, ZrO_2}}$ is the experimental Gibbs energy at standard temperature and pressure and $\mu^{\plimsoll}_{O_2}$ is the oxygen chemical potential under the same conditions. The values of $\mu_{ZrO_2(s)}$ and $\mu_{Zr(s)}$ are calculated from the DFT energies. Once $\mu^{\plimsoll}_{O_2}$ is calculated, we can generalise the chemical potential of oxygen for any value of $T$ and $p_{O_2}$ by appending an ideal gas relationship $\Delta{\mu(T)}$ and a Boltzmann distribution:
%
%\begin{equation}
%\mu_{O_2}(p_{O_2},T) = \mu^{\plimsoll}_{O_2} + \Delta{\mu(T)} + \frac{1}{2}{k_B}log(\frac{p_{O_2}}{p^{\plimsoll}_{O_2}})
%\end{equation}
%
%Using our generalised formula for $\mu_{O_2}$, we fix the temperature within the range of thermal phase-stabilisation (1500 K for tetragonal \zirconia) and calculate $\mu_{O_2}$ for many different values of $p_{O_2}$ between $10^{-35}$ and 10$^{0}$ atm, corresponding to oxygen deficient and oxygen rich environments, respectively ($p_{O_2}$ in air is approximately 0.2 atm). While the tetragonal phase will be stress-stabilised in practice, thermal-stabilisation in such models has been shown to qualitatively approximate the effect of stress-stabilisation, while allowing a wider range of dopant behaviours to be predicted \cite{Bell2016}. 

\section{Results}

\subsection{Incorporation energies}

\subsubsection*{Interstitial Sites}
Neutral iodine incorporation energies at interstitial sites for each phase are reported in Table \ref{i_incorp_interstitial}. The $2a$ and $2c$ sites in monoclinic \zirconia\ provide the least unfavourable iodine incorporation energy, followed by the $2b$ and $8e$ sites in tetragonal \zirconia , although in all cases energies are positive and large, indicating a large energy penalty against interstitial incorporation. The difference in incorporation energies between monoclinic and tetragonal \zirconia\ is approximately 1 eV, whereas the difference between tetragonal and cubic is 3.5 eV, indicating especially unfavourable conditions in cubic \zirconia . These differences are likely due to the larger interstitial sites in the lower-temperature phases, as monoclinic \zirconia\ exhibits the least and cubic \zirconia\ the most dense cell, (see Chapter \ref{ch:crystallography}). 
%It is expected that these densities will be representative because the decreased density due to thermal expansion at 650 K will be countered by attractive dispersion forces. 

\begin{table}[ht]
\onehalfspacing
\centering
\caption{Incorporation energies of iodine interstitials in non-defective supercells.}
\label{i_incorp_interstitial}
\begin{tabular}{ccccc}
\hline
\multirow{2}{*}{\textbf{Structure}} & \multirow{2}{*}{\textbf{Site}} & \multicolumn{3}{c}{\textbf{Incorporation Energy (eV)}} \\ \cline{3-5} 
 &  &  \textbf{\ch{I^{x}_{i}}} & \textbf{\ch{I^{'}_{i}}} & \textbf{\ch{I^{*}_{i}}} \\ \hline % \hspace{0.7 cm}
\multirow{4}{*}{\textbf{Monoclinic}} & 2a & 8.55 & 12.10 & 6.55 \\
 & 2b & 10.81 & 16.40 & 5.63 \\
 & 2c & 8.79 & 12.20 & 4.62 \\
 & 2d & 10.94 & 13.66 & 6.92 \\ \hline
\multirow{2}{*}{\textbf{Tetragonal}} & 2b & 9.49 & 13.96 & 5.99 \\
 & 8e & 9.53 & 12.73 & 5.10 \\ \hline
\multirow{2}{*}{\textbf{Cubic}} & 24d & 13.02 & 18.24 & 7.62 \\
 & 4b & 13.08 & 16.46 & 9.82 \\ \hline
\end{tabular}%
\end{table}

While the incorporation energies of iodine in the interstitial sites of \zirconia\ are large, for a fixed iodine concentration, they become relevant as the intrinsic defect populations become small, such as at low temperatures relative to the melting point. This is because interstitial sites are always available, whereas at low intrinsic defect concentrations, substitutional sites become saturated and accommodation at a lattice site first requires the creation of a vacancy defect, which has a formation energy penalty associated with it. 

When Brouwer diagrams are generated, iodine will also be considered as a charged species at an interstitial site (see § \ref{results2_brouwer}). This includes I$^{+}$ and I$^{-}$, where I$^{+}$ is a smaller ion that is more easily accommodated at an interstitial site. Energy values for I$^{+}$ and I$^{-}$ are also reported in Table \ref{i_incorp_interstitial}, however, values for different charge states cannot be compared because an electron has been added or removed from the I atom to form the specific charge state. An interstitial site will always be uncharged (resultant charge is distributed onto nearby ions instead), meaning that there is no pre-existing charged interstitial site to incorporate an atom onto. It is therefore only appropriate to consider the \emph{formation} energy (i.e. including the addition or removal of an electron) of a charged interstitial defect. 

\subsubsection*{Oxygen Sites}

Table \ref{i_incorp_oxygen} reports incorporation energies of iodine at various oxygen sites. In each phase, the lowest incorporation energy was that for accommodation at a vacant oxygen site such that iodine is in the 1- oxidation state, resulting in the overall defect \ch{I_{O}^{*}}. This anionic behaviour is expected from a halogen atom in a highly reducing site as it promotes the filling of the \emph{p} shell. I is most readily accommodated in the monoclinic phase for all I charge states.

\begin{table}[ht] % Iodine O site incorporation
\onehalfspacing
\centering
\caption{Incorporation energies of iodine in oxygen sites of the monoclinic, tetragonal, and cubic \zirconia\ phases.}
\label{i_incorp_oxygen}
\begin{tabular}{cccc} % \ch{V_{O}^{x}}
\hline
\multirow{2}{*}{\textbf{Structure}} & \multicolumn{3}{c}{\textbf{Incorporation energy (eV)}} \\ \cline{2-4} 
                                    & \hspace{0.7 cm} \textbf{\ch{I_{O}^{**}}} \hspace{0.7 cm} & \textbf{\ch{I_{O}^{*}}} & \textbf{\ch{I_{O}^{x}}} \\ \hline
\textbf{Monoclinic (3 co-ord)}      & 4.54             & 2.90             & 3.67             \\
\textbf{Monoclinic (4 co-ord)}      & 5.63             & 3.77             & 4.87             \\
\textbf{Tetragonal}                 & 6.19             & 4.02             & 4.44             \\
\textbf{Cubic}                      & 8.37             & 5.74             & 6.66      \\     \hline  
\end{tabular}
\end{table}


 % and these atoms have large electron affinities since they require only one electron to achieve the relatively stable noble gas electron configuration. 

\subsubsection*{Zirconium Sites}

Incorporation energies of iodine on zirconium sites are reported in Table \ref{i_incorp_zirconium}. The  incorporation energy decreases as the charge of the defect decreases from -4 to 0 (i.e. nominally from I$^{0}$ to I$^{4+}$). This is due to the decrease in the size of the iodine species with increasing positive charge, fitting better into the small vacant Zr$^{4+}$ cation site. This alone does not guarantee the emergence of uncharged iodine defects (I$^{4+}$) on zirconium sites when all energy terms are considered. In particular, there is also an energy penalty incurred in the change in charge of iodine. A Mulliken population analysis revealed a charge localised on the iodine of +2.31 at the \ch{I_{Zr}^{x}} defect, and a +0.86 charge on the \ch{I_{Zr}^{''''}} defect, with the remaining charge accommodated by other ions in the lattice. Again, I is most readily accommodated in the monoclinic phase for all charge states.

\begin{table}[ht] % Iodine Zr site incorporation
\onehalfspacing
\centering
\caption{Incorporation energies of iodine in zirconium sites of \zirconia.}
\label{i_incorp_zirconium}
\begin{tabular}{ccllll}
\hline
\hspace{0.7 cm} \multirow{2}{*}{\textbf{Structure}} \hspace{0.7 cm} & \multicolumn{5}{c}{\hspace{0.7 cm} \textbf{Incorporation energy (eV)}} \hspace{0.7 cm}                                                          \\ \cline{2-6} 
                                    & \multicolumn{1}{l}  {\textbf{\hspace{0.45 cm} \ch{I_{Zr}^{''''}}}} \hspace{0.45 cm} & \textbf{\ch{I_{Zr}^{'''}}} \hspace{0.45 cm} & \textbf{\ch{I_{Zr}^{''}}} \hspace{0.45 cm} & \textbf{\ch{I_{Zr}^{'}}} \hspace{0.45 cm} & \textbf{\ch{I_{Zr}^{x}}} \\ \hline
\textbf{Monoclinic}                 & 6.78                             &       3.65            &        0.89          &         -2.84         &     -5.08             \\
\textbf{Tetragonal}                 & 7.58                            &         3.64         &        1.69          &      -2.13            &     -4.57             \\
\textbf{Cubic}                      & 9.70                            &         6.81         &        3.01          &        0.38          &      -3.14     \\      \hline
\end{tabular}
\end{table}

%\subsection{Temperature dependence}

%To examine the temperature dependence of the defect equilibria, the concentration of dopant iodine was held constant while temperature was changed.


%\subsection{Dopant concentration dependence}
%
%To examine the dependence of the defect equilibria on iodine dopant concentration, the temperature was held constant while iodine concentration was changed.

%\begin{figure}[ht] % Tet intrinsic no space charge
%\begin{center}
%\begin{tikzpicture}
%	\begin{groupplot}[group style={group size=1 by 2}, width=13cm, height=10.2cm]
%	\nextgroupplot[
%		 ylabel={\ch{log_{10}}([D]) (per f.u.)}, ymin=-10, ymax=0, xmin=-35, xmax=0, legend style={{draw=}, at={(0.40,0.97)}, anchor=north west, legend columns=2, nodes={scale=1, transform shape}}]
%
%	\nextgroupplot[
%		 xlabel={\ch{log_{10}}($p_{O_{2}}$) (atm)}, ylabel={\ch{log_{10}}([D]) (per f.u.)}, ymin=-10, ymax=0, xmin=-35, xmax=0, legend style={{draw=}, at={(0.40,0.97)}, anchor=north west, legend columns=4, nodes={scale=1, transform shape}}]
%      
%	\end{groupplot}           
%\end{tikzpicture}
%		\caption{Tetragonal phase Brouwer diagrams of intrinsic point defects at a temperature of 1500 K \textbf{a)} without a space charge and \textbf{b)} with a space charge of $10^{-1}$ e$^{-1}$ per f.u.}
%		\label{figure:spacechargeexample}
%	\end{center}
%\end{figure}

\subsection{Dopant interstitial defects}

The formation energies of iodine interstitial defects are useful not only to determine whether interstitial defects will form, but also which interstitial sites in particular they will occupy. As shown in Table \ref{table:interstitials}, ZrO$_{2}$ has four interstitial sites in the monoclinic phase, and two interstitial sites in the tetragonal and cubic phase (based on crystallographic data of the three phases). For each phase, the formation energies of each iodine interstitial defect as a function of Fermi level are calculated and provided below.

\subsubsection{Monoclinic interstitial defects}

Figure \ref{figure:monointer} shows how the formation energy of an iodine interstitial defect in monoclinic ZrO$_{2}$ varies based on site and Fermi level. In this phase, the 2$b$ Wyckoff position is the site of lowest formation energy across the entire range of Fermi levels that span the band gap, with a maximum of 8.1 eV at a Fermi level of 2.8 eV.  The next most favourable site is at 2$a$, with formation energies at least 0.4 eV greater at similar Fermi levels. Iodine at the 2$c$ and 2$d$ sites exhibits formation energies between 1 and 4 eV larger than at the 2$b$ site, making these sites significantly unfavourable. This shows that there are indeed four unique sites which iodine atoms can occupy in the monoclinic phase (as evidenced by the different formation energy evolutions at each site), and that of these sites, the 2$c$ is the most energetically favourable for iodine.

\begin{figure}[ht] % Mono iodine interstitials E vs Fermi
\begin{center}
\begin{tikzpicture}
	\begin{axis}
		[width=11.5cm, xlabel={Fermi level $\mu_{e}$ (eV)}, ylabel={Formation energy (eV) per \zirconia\ }, ymin=4, ymax=11, xmin=0, xmax=6, legend style={{draw=}, at={(0.5,0.05)}, anchor=south, legend columns=2}]
		\addplot[no marks, red] table [x=2a1, y=2a2,]{dat/monointer.dat}; \addlegendentry{$2a$};
        \addplot[no marks, red, dashed] table [x=2b1, y=2b2, ]{dat/monointer.dat}; \addlegendentry{$2b$};
        \addplot[no marks, blue] table [x=2c1, y=2c2,]{dat/monointer.dat}; \addlegendentry{$2c$};
        \addplot[no marks, blue, dashed] table [x=2d1, y=2d2,]{dat/monointer.dat}; \addlegendentry{$2d$};
			\end{axis}
		\end{tikzpicture}
		\caption{Iodine interstitial formation energies in monoclinic \zirconia\ as a function of Fermi level. Gradient indicates defect charge.}
		\label{figure:monointer}
	\end{center}
\end{figure}

It should also be noted that iodine will occupy these interstitial sites as either \ch{I^{*}_{i}} or \ch{I^{'}_{i}} (the slope of the curve indicates defect charge). \ch{I^{x}_{i}} appears for a small range of Fermi levels in the 2$c$ site, but the formation energy is so large compared to the other sites that this will not be present in the crystal. The preference for a charged state of +1 or -1 may be due to a combination of electron availability and ionic radius. \ch{I^{*}_{i}}, being positively charged, has a smaller ionic radius than \ch{I^{x}_{i}}. The smaller size of this defect comes with a smaller energy penalty when occupying an interstitial site, and at low Fermi levels the iodine is more susceptible to oxidation. At high Fermi levels, \ch{I^{'}_{i}} forms because electrons are more readily available and iodine has a high electron affinity, which compensates energetically for the increased ionic radius.

\subsubsection{Tetragonal interstitial defects}

\begin{figure}[ht] % Tet iodine interstitials E vs Fermi
\begin{center}
\begin{tikzpicture}
	\begin{axis}
		[width=11cm, xlabel={Fermi level $\mu_{e}$ (eV)}, ylabel={Formation energy (eV) per \zirconia\ }, ymin=6, ymax=11, xmin=0, xmax=6, legend style={{draw=}, at={(0.5,0.05)}, anchor=south, legend columns=1}]
		\addplot[no marks, red] table [x=2atet1, y=2atet2,]{dat/tetcubicinter.dat}; \addlegendentry{$2a$};
        \addplot[no marks, blue] table [x=8etet1, y=8etet2, ]{dat/tetcubicinter.dat}; \addlegendentry{$8e$};
			\end{axis}
		\end{tikzpicture}
		\caption{Iodine interstitial formation energies in tetragonal \zirconia\ as a function of Fermi level. Gradient indicates defect charge.}
		\label{figure:tetinter}
	\end{center}
\end{figure}

\subsubsection{Cubic interstitial defects}

\begin{figure}[ht] % Cubic iodine interstitials E vs Fermi
\begin{center}
\begin{tikzpicture}
	\begin{axis}
		[width=11cm, xlabel={Fermi level $\mu_{e}$ (eV)}, ylabel={Formation energy (eV) per \zirconia\ }, ymin=8, ymax=13, xmin=0, xmax=6, legend style={{draw=}, at={(0.5,0.05)}, anchor=south, legend columns=1}]
		\addplot[no marks, red] table [x=24cubic1, y=24cubic2,]{dat/tetcubicinter.dat}; \addlegendentry{$24d$};
        \addplot[no marks, blue] table [x=4bmcubic1, y=4bmcubic2, ]{dat/tetcubicinter.dat}; \addlegendentry{$4b$};
			\end{axis}
		\end{tikzpicture}
		\caption{Iodine interstitial formation energies in cubic \zirconia\ as a function of Fermi level. Gradient indicates defect charge.}
		\label{figure:cubicinter}
	\end{center}
\end{figure}

\subsection{Brouwer Diagrams}  \label{results2_brouwer}

\subsubsection*{Monoclinic Phase}

Brouwer diagrams associated with the monoclinic phase, at 650 K, at which temperature this \zirconia\ phase is stable, are shown in Figure \ref{figure:tikzbrouwerconcmono}. At 650 K, this phase exhibits a relatively low concentration of intrinsic defects; concentrations of \ch{V_{O}^{**}} and \ch{V_{Zr}^{''''}} remained below $10^{-10}$ parts/fu across the majority of oxygen pressures at both iodine concentrations and do not appear in the diagrams. At lower iodine concentrations, the intrinsic electronic defects, \ch{e^{'}} and \ch{h^{*}}, were more significant, with \ch{h^{*}} defects being a major fraction of the total defect population near stoichiometry (i.e. at an oxygen pressure of approximately $10^{-7.5}$ atm). 

\begin{figure}[ht!] % Mono conc sweep
\begin{center}
\begin{tikzpicture} % 10e-3 iodine conc in mono
	\begin{groupplot}[group style={group size=1 by 2}, width=13cm, height=10.2cm]
	\nextgroupplot[
		 ylabel={\ch{log_{10}}([D]) (per f.u.)}, ymin=-10, ymax=0, xmin=-35, xmax=0, legend style={{draw=}, at={(0.40,0.97)}, anchor=north west, legend columns=2, nodes={scale=1, transform shape}}]
        \addplot[no marks, draw=blue!70!black] table [x=pO2, y=electrons,]{dat/1e5iconcmono650.dat};  \node at (-28.1,-7.5) {\ch{e^{'}}};  \addlegendentry{\ch{e^{'}}};
        \addplot[no marks, draw=red!85!black] table [x=pO2, y=holes,]{dat/1e5iconcmono650.dat}; \addlegendentry{\ch{h^{\textperiodcentered}}}; % \node at (-1,-4.5) {\ch{h^{\textperiodcentered}}};
%         \addplot[no marks, draw=black!70!green] table [x=pO2, y=VO{2},]{dat/1e5iconcmono650.dat}; \addlegendentry{\ch{V_{O}^{\textperiodcentered\textperiodcentered}}};
%         \addplot[no marks, draw=black!55!green] table [x=pO2, y=VO{1},]{dat/1e5iconcmono650.dat}; \addlegendentry{\ch{V_{O}^{\textperiodcentered}}};
%         \addplot[no marks, draw=black!30!green] table [x=pO2, y=VO{0},]{dat/1e5iconcmono650.dat}; \addlegendentry{\ch{V_{O}^{x}}};
%         \addplot[no marks, draw=yellow!85!blue] table [x=pO2, y=VM{-4},]{dat/1e5iconcmono650.dat}; \addlegendentry{\ch{V_{Zr}^{''''}}};
%         \addplot[no marks, draw=yellow!75!blue] table [x=pO2, y=VM{-3},]{dat/1e5iconcmono650.dat}; \addlegendentry{\ch{V_{Zr}^{'''}}};
%         \addplot[no marks, draw=yellow!65!blue] table [x=pO2, y=VM{-2},]{dat/1e5iconcmono650.dat}; \addlegendentry{\ch{V_{Zr}^{''}}};
%         \addplot[no marks, draw=yellow!55!blue] table [x=pO2, y=VM{-1},]{dat/1e5iconcmono650.dat}; \addlegendentry{\ch{V_{Zr}^{'}}};
%         \addplot[no marks, draw=yellow!45!blue] table [x=pO2, y=VM{0},]{dat/1e5iconcmono650.dat}; \addlegendentry{\ch{V_{Zr}^{x}}};
%         \addplot[no marks, draw=red!60!yellow] table [x=pO2, y=Oi{-2},]{dat/1e5iconcmono650.dat}; \addlegendentry{\ch{O_{i}^{''}}};
%         \addplot[no marks, draw=red!50!yellow] table [x=pO2, y=Oi{-1},]{dat/1e5iconcmono650.dat}; \addlegendentry{\ch{O_{i}^{'}}};
%         \addplot[no marks, draw=red!40!yellow] table [x=pO2, y=Oi{0},]{dat/1e5iconcmono650.dat}; \addlegendentry{\ch{O_{i}^{x}}};
%         \addplot[no marks, draw=green!80!pink] table [x=pO2, y=Mi{4},]{dat/1e5iconcmono650.dat}; \addlegendentry{\ch{Zr_{i}^{\textperiodcentered\textperiodcentered\textperiodcentered\textperiodcentered}}};
%         \addplot[no marks, draw=green!70!pink] table [x=pO2, y=Mi{3},]{dat/1e5iconcmono650.dat}; \addlegendentry{\ch{Zr_{i}^{\textperiodcentered\textperiodcentered\textperiodcentered}}};
%         \addplot[no marks, draw=green!60!pink] table [x=pO2, y=Mi{2},]{dat/1e5iconcmono650.dat}; \addlegendentry{\ch{Zr_{i}^{\textbf{\textperiodcentered\textperiodcentered}}}};
%         \addplot[no marks, draw=green!50!pink] table [x=pO2, y=Mi{1},]{dat/1e5iconcmono650.dat}; \addlegendentry{\ch{Zr_{i}^{\textperiodcentered}}};
%         \addplot[no marks, draw=green!40!pink] table [x=pO2, y=Mi{0},]{dat/1e5iconcmono650.dat}; \addlegendentry{\ch{Zr_{i}^{x}}};
%         \addplot[no marks, dashed, draw=red!70!black] table [x=pO2, y=Ii{0},]{dat/1e5iconcmono650.dat}; \addlegendentry{\ch{I_{i}^{x}}};
%         \addplot[no marks, dashed, draw=red!50!black] table [x=pO2, y=Ii{-1},]{dat/1e5iconcmono650.dat}; \addlegendentry{\ch{I_{i}^{'}}};
        \addplot[no marks, dashed, draw=purple!60!white] table [x=pO2, y=Ii{1},]{dat/1e5iconcmono650.dat}; \addlegendentry{\ch{I_{i}^{\textperiodcentered}}}; 
        \addplot[no marks, dashed, draw=blue!50!white] table [x=pO2, y=IsubO{1},]{dat/1e5iconcmono650.dat}; \addlegendentry{\ch{I_{O}^{\textperiodcentered}}}; \node at (-22,-4.5) {\ch{I_{O}^{\textperiodcentered}}};
        \addplot[no marks, dashed, draw=green!60!black] table [x=pO2, y=IsubO{2},]{dat/1e5iconcmono650.dat}; \addlegendentry{\ch{I_{O}^{\textperiodcentered\textperiodcentered}}}; 
        \addplot[no marks, dashed, draw=black] table [x=pO2, y=IsubO{3},]{dat/1e5iconcmono650.dat}; \addlegendentry{\ch{I_{O}^{\textperiodcentered\textperiodcentered\textperiodcentered}}};
        \addplot[no marks, dashed, draw=orange!80!black] table [x=pO2, y=IsubZr{-3},]{dat/1e5iconcmono650.dat};  \node at (-22,-6.2) {\ch{I_{Zr}^{'''}}}; \addlegendentry{\ch{I_{Zr}^{'''}}};
%         \addplot[no marks, dashed, draw=green!50!black] table [x=pO2, y=IsubZr{-4},]{dat/1e5iconcmono650.dat}; \addlegendentry{\ch{I_{Zr}^{''''}}};
%         \addplot[no marks, dashed, draw=green!30!black] table [x=pO2, y=IsubZr{-5},]{dat/1e5iconcmono650.dat}; \addlegendentry{\ch{I_{Zr}^{'''''}}};
%         \addplot[no marks] table [x=pO2, y=Stoich,]{dat/1e5iconcmono650.dat}; \addlegendentry{Stoich};
\node at (-33.7,-0.5) {\textbf{a)}};
			\nextgroupplot[
		 xlabel={\ch{log_{10}}($p_{O_{2}}$) (atm)}, ylabel={\ch{log_{10}}([D]) (per f.u.)}, ymin=-10, ymax=0, xmin=-35, xmax=0, legend style={{draw=}, at={(0.40,0.97)}, anchor=north west, legend columns=4, nodes={scale=1, transform shape}}]
        \addplot[no marks, draw=blue!70!black] table [x=pO2, y=electrons,]{dat/1e3iconcmono650.dat}; \node at (-29,-7) {\ch{e^{'}}};
        \addplot[no marks, draw=red!85!black] table [x=pO2, y=holes,]{dat/1e3iconcmono650.dat}; 
%         \addplot[no marks, draw=black!70!green] table [x=pO2, y=VO{2},]{dat/1e3iconcmono650.dat}; 
%         \addplot[no marks, draw=black!55!green] table [x=pO2, y=VO{1},]{dat/1e3iconcmono650.dat}; 
%         \addplot[no marks, draw=black!30!green] table [x=pO2, y=VO{0},]{dat/1e3iconcmono650.dat}; 
%         \addplot[no marks, draw=yellow!85!blue] table [x=pO2, y=VM{-4},]{dat/1e3iconcmono650.dat}; 
%         \addplot[no marks, draw=yellow!75!blue] table [x=pO2, y=VM{-3},]{dat/1e3iconcmono650.dat}; 
%         \addplot[no marks, draw=yellow!65!blue] table [x=pO2, y=VM{-2},]{dat/1e3iconcmono650.dat}; 
%         \addplot[no marks, draw=yellow!55!blue] table [x=pO2, y=VM{-1},]{dat/1e3iconcmono650.dat}; 
%         \addplot[no marks, draw=yellow!45!blue] table [x=pO2, y=VM{0},]{dat/1e3iconcmono650.dat}; 
%         \addplot[no marks, draw=red!60!yellow] table [x=pO2, y=Oi{-2},]{dat/1e3iconcmono650.dat}; 
%         \addplot[no marks, draw=red!50!yellow] table [x=pO2, y=Oi{-1},]{dat/1e3iconcmono650.dat}; 
%         \addplot[no marks, draw=red!40!yellow] table [x=pO2, y=Oi{0},]{dat/1e3iconcmono650.dat}; 
%         \addplot[no marks, draw=green!80!pink] table [x=pO2, y=Mi{4},]{dat/1e3iconcmono650.dat}; 
%         \addplot[no marks, draw=green!70!pink] table [x=pO2, y=Mi{3},]{dat/1e3iconcmono650.dat}; 
%         \addplot[no marks, draw=green!60!pink] table [x=pO2, y=Mi{2},]{dat/1e3iconcmono650.dat}; 
%         \addplot[no marks, draw=green!50!pink] table [x=pO2, y=Mi{1},]{dat/1e3iconcmono650.dat}; 
%         \addplot[no marks, draw=green!40!pink] table [x=pO2, y=Mi{0},]{dat/1e3iconcmono650.dat}; 
%         \addplot[no marks, dashed, draw=red!70!black] table [x=pO2, y=Ii{0},]{dat/1e3iconcmono650.dat}; 
%         \addplot[no marks, dashed, draw=red!50!black] table [x=pO2, y=Ii{-1},]{dat/1e3iconcmono650.dat}; 
        \addplot[no marks, dashed, draw=purple!60!white] table [x=pO2, y=Ii{1},]{dat/1e3iconcmono650.dat}; 
        \addplot[no marks, dashed, draw=blue!50!white] table [x=pO2, y=IsubO{1},]{dat/1e3iconcmono650.dat}; \node at (-22,-2.5) {\ch{I_{O}^{\textperiodcentered}}};
        \addplot[no marks, dashed, draw=green!60!black] table [x=pO2, y=IsubO{2},]{dat/1e3iconcmono650.dat}; 
        \addplot[no marks, dashed, draw=black] table [x=pO2, y=IsubO{3},]{dat/1e3iconcmono650.dat}; 
        \addplot[no marks, dashed, draw=orange!80!black] table [x=pO2, y=IsubZr{-3},]{dat/1e3iconcmono650.dat}; \node at (-22,-4.4) {\ch{I_{Zr}^{'''}}}; 
%         \addplot[no marks, dashed, draw=green!50!black] table [x=pO2, y=IsubZr{-4},]{dat/1e3iconcmono650.dat}; 
%         \addplot[no marks, dashed, draw=green!30!black] table [x=pO2, y=IsubZr{-5},]{dat/1e3iconcmono650.dat}; 
%         \addplot[no marks] table [x=pO2, y=Stoich,]{dat/1e3iconcmono650.dat}; 
\node at (-33.7,-0.5) {\textbf{b)}};
			\end{groupplot}          
\end{tikzpicture}
		\caption{Monoclinic phase Brouwer diagrams of point defects at iodine concentrations of a) $10^{-5}$ and b) $10^{-3}$, at a temperature of 650 K.}
		\label{figure:tikzbrouwerconcmono}
	\end{center}
\end{figure} % 10e-3 iodine conc in mono

Between oxygen pressures of $10^{-35}$ and $10^{-10}$ atm, the dominant defects were \ch{I_{O}^{*}} charge-compensated by \ch{I_{Zr}^{'''}}. Above an oxygen pressure of $10^{-10}$ atm, a combination of \ch{I_{i}^{*}}, \ch{I_{Zr}^{'''}} and \ch{I_{O}^{***}} defects were dominant. This demonstrates that iodine will adopt a +1 oxidation state in order to facilitate iodine incorporation into the lattice. The effective ionic radius of I$^{-}$ is 2.20 \r{A} in VI-fold coordination, compared to 1.38 \r{A} for O$^{2-}$ in IV-fold coordination, as is the case in \zirconia\ \cite{Shannon1976}. Iodine with a higher positive charge state will have a smaller ionic radius, and thus impose less strain on the lattice (and therefore a smaller energy penalty) in each defect configuration, including substitution on a Zr site. At the highest oxygen pressures, the Brouwer diagrams show that oxidation of iodine, substituted at an oxygen site, to the +1 oxidation state (i.e. \ch{I_{O}^{***}}) becomes a necessary charge compensating defect. This is because the energy penalty to form hole defects in this broad band insulator is too great, as is the formation of other positive charge defects such as \ch{Zr_{i}^{****}}. This may translate to iodine out-competing oxygen for oxygen sites in monoclinic \zirconia , with higher oxygen pressures providing very little in terms of a barrier effect. 

% \begin{itemize}
% %\item Figure 1 shows monoclinic Brouwer diagrams generated at different assumed iodine concentrations.
% %\item The monoclinic Brouwer diagrams were generated at a temperature of 650 K. This is because the structure is stable at this temperature, and it is representative of the temperature which the \zirconia\ layer on the internal surface of the cladding would experience.
% %\item At an iodine concentration of 1e-5 and low oxygen pressures, the dominant defects were iodine -1 substitutional defects on the oxygen site, charged compensated by iodine +1 substitutional defects on the zirconium site. At higher oxygen pressures, the zirconium substitutional defects remained but were now charge compensated by iodine +1 substitutional defects on the oxygen site. At very high oxygen pressures, hole defects were preferred to oxygen substitutional defects.

% \item At an iodine concentration of 1e-3, intrinsic defects were found to be negligible compared to the extrinsic iodine defects. The same pattern was seen as with an iodine concentration of 1e-5, except hole defects were no longer significant at high oxygen pressures. Very low concentrations of iodine +1 interstitial defects began to appear at oxygen pressures greater than 1e-20.
% \end{itemize}
% The monoclinic Brouwer diagram (Figure \ref{figure:monoBrouwer}) predicts that at 635 K, few types of defects will be present and at very low (\textless 10 ppb \zirconia ) concentrations. This is typical of defect behaviour in a ceramics at temperatures far below their melting points \cite{kingery1997physical,ball2006computer}. Fully-charged zirconium vacancies, charge-compensated by holes, are the major defect type we expect to observe at $p_{O_{2}}$ \textgreater $10^{-15}$. Below this, only electronic defects compensated by electron hole defects are expected. We briefly see increased concentrations of uncharged oxygen interstitial defects at very high levels of $p_{O_{2}}$.

\subsubsection*{Tetragonal Phase}

Brouwer diagrams for the tetragonal phase are shown in Figure \ref{figure:tikzbrouwerconctet}. As these diagrams were generated at a temperature of 1500 K (at which the tetragonal phase becomes stable), intrinsic defect concentrations were significantly higher than in the monoclinic diagrams for all oxygen pressures (though trends remained the same). Intrinsic defects \ch{e^{'}}, \ch{h^{*}}, \ch{V_{O}^{**}} and \ch{V_{Zr}^{''''}} were dominant across most oxygen pressures at an iodine concentration of $10^{-5}$ parts/fu. Only around stoichiometry do extrinsic defect concentrations approach intrinsic values (which as mentioned earlier is why this concentration of iodine was chosen). Across all oxygen pressures, \ch{I_{O}^{*}} and \ch{I_{Zr}^{'''}} are the major iodine defects. Between $10^{-15}$ and $10^{-5}$ atm, Figure \ref{figure:tikzbrouwerconctet} illustrates that the major iodine defect swaps from being \ch{I_{O}^{*}} to \ch{I_{Zr}^{'''}}. 

\begin{figure}[ht!] % 10e-5 iodine conc in tet
\begin{center}
\begin{tikzpicture}
	\begin{groupplot}[group style={group size=1 by 2}, width=13cm, height=10.2cm]
	\nextgroupplot[
		 ylabel={\ch{log_{10}}([D]) (per f.u.)}, ymin=-10, ymax=0, xmin=-35, xmax=0, legend style={{draw=}, at={(0.40,0.97)}, anchor=north west, legend columns=2, nodes={scale=1, transform shape}}]
        \addplot[no marks, draw=blue!70!black] table [x=pO2, y=electrons,]{dat/1e5iconctet1500.dat}; \addlegendentry{\ch{e^{'}}}; \node at (-26.0,-1.9) {\ch{e^{'}}};
        \addplot[no marks, draw=red!85!black] table [x=pO2, y=holes,]{dat/1e5iconctet1500.dat}; \addlegendentry{\ch{h^{\textperiodcentered}}}; \node at (-7,-3.6) {\ch{h^{\textperiodcentered}}};
        \addplot[no marks, draw=black!70!green] table [x=pO2, y=VO{2},]{dat/1e5iconctet1500.dat}; \addlegendentry{\ch{V_{O}^{\textperiodcentered\textperiodcentered}}}; \node at (-26.7,-3.3) {\ch{V_{O}^{\textperiodcentered\textperiodcentered}}};
%         \addplot[no marks, draw=black!55!green] table [x=pO2, y=VO{1},]{dat/1e5iconctet1500.dat}; \addlegendentry{\ch{V_{O}^{\textperiodcentered}}};
%         \addplot[no marks, draw=black!30!green] table [x=pO2, y=VO{0},]{dat/1e5iconctet1500.dat}; \addlegendentry{\ch{V_{O}^{x}}};
        \addplot[no marks, draw=yellow!85!blue] table [x=pO2, y=VM{-4},]{dat/1e5iconctet1500.dat}; \addlegendentry{\ch{V_{Zr}^{''''}}};
%         \addplot[no marks, draw=yellow!75!blue] table [x=pO2, y=VM{-3},]{dat/1e5iconctet1500.dat}; \addlegendentry{\ch{V_{Zr}^{'''}}};
%         \addplot[no marks, draw=yellow!65!blue] table [x=pO2, y=VM{-2},]{dat/1e5iconctet1500.dat}; \addlegendentry{\ch{V_{Zr}^{''}}};
%         \addplot[no marks, draw=yellow!55!blue] table [x=pO2, y=VM{-1},]{dat/1e5iconctet1500.dat}; \addlegendentry{\ch{V_{Zr}^{'}}};
%         \addplot[no marks, draw=yellow!45!blue] table [x=pO2, y=VM{0},]{dat/1e5iconctet1500.dat}; \addlegendentry{\ch{V_{Zr}^{x}}};
%         \addplot[no marks, draw=red!60!yellow] table [x=pO2, y=Oi{-2},]{dat/1e5iconctet1500.dat}; \addlegendentry{\ch{O_{i}^{''}}};
%         \addplot[no marks, draw=red!50!yellow] table [x=pO2, y=Oi{-1},]{dat/1e5iconctet1500.dat}; \addlegendentry{\ch{O_{i}^{'}}};
%         \addplot[no marks, draw=red!40!yellow] table [x=pO2, y=Oi{0},]{dat/1e5iconctet1500.dat}; \addlegendentry{\ch{O_{i}^{x}}};
%         \addplot[no marks, draw=green!80!pink] table [x=pO2, y=Mi{4},]{dat/1e5iconctet1500.dat}; \addlegendentry{\ch{Zr_{i}^{\textperiodcentered\textperiodcentered\textperiodcentered\textperiodcentered}}};
%         \addplot[no marks, draw=green!70!pink] table [x=pO2, y=Mi{3},]{dat/1e5iconctet1500.dat}; \addlegendentry{\ch{Zr_{i}^{\textperiodcentered\textperiodcentered\textperiodcentered}}};
%         \addplot[no marks, draw=green!60!pink] table [x=pO2, y=Mi{2},]{dat/1e5iconctet1500.dat}; \addlegendentry{\ch{Zr_{i}^{\textbf{\textperiodcentered\textperiodcentered}}}};
%         \addplot[no marks, draw=green!50!pink] table [x=pO2, y=Mi{1},]{dat/1e5iconctet1500.dat}; \addlegendentry{\ch{Zr_{i}^{\textperiodcentered}}};
%         \addplot[no marks, draw=green!40!pink] table [x=pO2, y=Mi{0},]{dat/1e5iconctet1500.dat}; \addlegendentry{\ch{Zr_{i}^{x}}};
%         \addplot[no marks, dashed, draw=red!70!black] table [x=pO2, y=Ii{0},]{dat/1e5iconctet1500.dat}; \addlegendentry{\ch{I_{i}^{x}}};
%         \addplot[no marks, dashed, draw=red!50!black] table [x=pO2, y=Ii{-1},]{dat/1e5iconctet1500.dat}; \addlegendentry{\ch{I_{i}^{'}}};
        \addplot[no marks, dashed, draw=purple!60!white] table [x=pO2, y=Ii{1},]{dat/1e5iconctet1500.dat}; \addlegendentry{\ch{I_{i}^{\textperiodcentered}}};
        \addplot[no marks, dashed, draw=blue!50!white] table [x=pO2, y=IsubO{1},]{dat/1e5iconctet1500.dat}; \addlegendentry{\ch{I_{O}^{\textperiodcentered}}};
        \addplot[no marks, dashed, draw=green!60!black] table [x=pO2, y=IsubO{2},]{dat/1e5iconctet1500.dat}; \addlegendentry{\ch{I_{O}^{\textperiodcentered\textperiodcentered}}};
        \addplot[no marks, dashed, draw=black] table [x=pO2, y=IsubO{3},]{dat/1e5iconctet1500.dat}; \addlegendentry{\ch{I_{O}^{\textperiodcentered\textperiodcentered\textperiodcentered}}};
        \addplot[no marks, dashed, draw=orange!80!black] table [x=pO2, y=IsubZr{-3},]{dat/1e5iconctet1500.dat}; \addlegendentry{\ch{I_{Zr}^{'''}}};
%         \addplot[no marks, dashed, draw=pink] table [x=pO2, y=IsubZr{-4},]{dat/1e5iconctet1500.dat}; \addlegendentry{\ch{I_{Zr}^{''''}}};
%         \addplot[no marks, dashed, draw=purple] table [x=pO2, y=IsubZr{-5},]{dat/1e5iconctet1500.dat}; \addlegendentry{\ch{I_{Zr}^{'''''}}};
%         \addplot[no marks] table [x=pO2, y=Stoich,]{dat/1e5iconctet1500.dat}; \addlegendentry{Stoich};
\node at (-33.7,-0.5) {\textbf{a)}};
			\nextgroupplot[
		 xlabel={\ch{log_{10}}($p_{O_{2}}$) (atm)}, ylabel={\ch{log_{10}}([D]) (per f.u.)}, ymin=-10, ymax=0, xmin=-35, xmax=0, legend style={{draw=}, at={(0.40,0.97)}, anchor=north west, legend columns=4, nodes={scale=1, transform shape}}]
        \addplot[no marks, draw=blue!70!black] table [x=pO2, y=electrons,]{dat/1e3iconctet1500.dat}; \node at (-27,-1.7) {\ch{e^{'}}};
        \addplot[no marks, draw=red!85!black] table [x=pO2, y=holes,]{dat/1e3iconctet1500.dat}; \node at (-2.5,-2.1) {\ch{h^{\textperiodcentered}}};
        \addplot[no marks, draw=black!70!green] table [x=pO2, y=VO{2},]{dat/1e3iconctet1500.dat}; 
%         \addplot[no marks, draw=black!55!green] table [x=pO2, y=VO{1},]{dat/1e3iconctet1500.dat}; 
%         \addplot[no marks, draw=black!30!green] table [x=pO2, y=VO{0},]{dat/1e3iconctet1500.dat}; 
        \addplot[no marks, draw=yellow!85!blue] table [x=pO2, y=VM{-4},]{dat/1e3iconctet1500.dat}; 
%         \addplot[no marks, draw=yellow!75!blue] table [x=pO2, y=VM{-3},]{dat/1e3iconctet1500.dat}; 
%         \addplot[no marks, draw=yellow!65!blue] table [x=pO2, y=VM{-2},]{dat/1e3iconctet1500.dat}; 
%         \addplot[no marks, draw=yellow!55!blue] table [x=pO2, y=VM{-1},]{dat/1e3iconctet1500.dat}; 
%         \addplot[no marks, draw=yellow!45!blue] table [x=pO2, y=VM{0},]{dat/1e3iconctet1500.dat}; 
%         \addplot[no marks, draw=red!60!yellow] table [x=pO2, y=Oi{-2},]{dat/1e3iconctet1500.dat}; 
%         \addplot[no marks, draw=red!50!yellow] table [x=pO2, y=Oi{-1},]{dat/1e3iconctet1500.dat}; 
%         \addplot[no marks, draw=red!40!yellow] table [x=pO2, y=Oi{0},]{dat/1e3iconctet1500.dat}; 
%         \addplot[no marks, draw=green!80!pink] table [x=pO2, y=Mi{4},]{dat/1e3iconctet1500.dat}; 
%         \addplot[no marks, draw=green!70!pink] table [x=pO2, y=Mi{3},]{dat/1e3iconctet1500.dat}; 
%         \addplot[no marks, draw=green!60!pink] table [x=pO2, y=Mi{2},]{dat/1e3iconctet1500.dat}; 
%         \addplot[no marks, draw=green!50!pink] table [x=pO2, y=Mi{1},]{dat/1e3iconctet1500.dat}; 
%         \addplot[no marks, draw=green!40!pink] table [x=pO2, y=Mi{0},]{dat/1e3iconctet1500.dat}; 
%         \addplot[no marks, dashed, draw=red!70!black] table [x=pO2, y=Ii{0},]{dat/1e3iconctet1500.dat}; 
%         \addplot[no marks, dashed, draw=red!50!black] table [x=pO2, y=Ii{-1},]{dat/1e3iconctet1500.dat}; 
        \addplot[no marks, dashed, draw=purple!60!white] table [x=pO2, y=Ii{1},]{dat/1e3iconctet1500.dat}; 
        \addplot[no marks, dashed, draw=blue!50!white] table [x=pO2, y=IsubO{1},]{dat/1e3iconctet1500.dat}; \node at (-11,-2.6) {\ch{I_{O}^{\textperiodcentered}}};
        \addplot[no marks, dashed, draw=green!60!black] table [x=pO2, y=IsubO{2},]{dat/1e3iconctet1500.dat}; 
        \addplot[no marks, dashed, draw=black] table [x=pO2, y=IsubO{3},]{dat/1e3iconctet1500.dat}; 
        \addplot[no marks, dashed, draw=orange!80!black] table [x=pO2, y=IsubZr{-3},]{dat/1e3iconctet1500.dat}; 
%         \addplot[no marks, dashed, draw=pink] table [x=pO2, y=IsubZr{-4},]{dat/1e3iconctet1500.dat}; 
%         \addplot[no marks, dashed, draw=purple] table [x=pO2, y=IsubZr{-5},]{dat/1e3iconctet1500.dat}; 
%         \addplot[no marks] table [x=pO2, y=Stoich,]{dat/1e3iconctet1500.dat}; 
\node at (-33.7,-0.5) {\textbf{b)}};
			\end{groupplot}        
\end{tikzpicture} % 10e-3 iodine conc in tet
		\caption{Tetragonal phase Brouwer diagrams of point defects at iodine concentrations of a) $10^{-5}$ and b) $10^{-3}$, at a temperature of 1500 K.}
		\label{figure:tikzbrouwerconctet}
	\end{center}
\end{figure}

When the iodine concentration was increased to $10^{-3}$ parts/fu, a significant change in defect equilibria was predicted. The oxygen pressure at stoichiometry increased from $10^{-10}$ to $10^{-6.5}$ atm (for monoclinic \zirconia, it remained at $10^{-7.5}$ atm regardless of iodine concentration). Nevertheless, \ch{I_{O}^{*}} and \ch{I_{Zr}^{'''}} remain the dominant defect pair between oxygen pressures of $10^{-15}$ and $10^{-5}$ atm (as they are at the lower iodine concentration). However, \ch{I_{O}^{*}} and \ch{I_{Zr}^{'''}} became higher concentration defects than both intrinsic \ch{V_{O}^{**}} and \ch{V_{Zr}^{''''}} defects. We also observe that Zr vacancies no longer serve as the main negative charge-compensation defect near stoichiometry, leaving \ch{I_{Zr}^{'''}} as the most energetically favourable negatively-charged defect.  

Unlike in the Brouwer diagrams for the monoclinic phase, for the tetragonal phase, the concentration of iodine substitutional defects on oxygen sites decreases more steeply at high oxygen pressures, peaking near stoichiometry. \ch{I_{O}^{***}} in particular, which was the dominant defect at high oxygen pressures in monoclinic \zirconia , becomes insignificant under the same conditions in the tetragonal phase, with iodine confined to Zr sites. This behaviour is indicative of a `barrier' effect against iodine at high oxygen partial pressures, with oxygen out-competing iodine for oxygen sites. Given that the inner oxide is likely to have a higher tetragonal phase fraction than the external oxide, due to the incorporation of fission products, this result could help to explain why there appears to be an oxygen effect on PCI-related SCC of zirconium alloys \cite{hofmann1984stress}. 

Another effect considered was the space charge of the system. Electrons have a higher rate of diffusion than oxygen vacancies in \zirconia , leading to a build-up of oxygen vacancies near the metal-oxide interface as corrosion progresses \cite{bojinov2010influence}. This results in an overall positive charge (since the dominant oxygen vacancy is \ch{V_{O}^{**}}) referred to as a space charge. When included in our Brouwer diagrams, this space charge had a negligible effect on the concentration or charge state of iodine up to a charge of $10^{-1}$ holes per f.u. \zirconia . This corresponds to a high concentration of oxygen vacancies relative to the equilibrium concentration, predicting that a significant deviation from equilibrium is not expected near the metal oxide interface as a result of a positive space charge.

\section{Conclusions}

Iodine exhibits lower incorporation energies when occupying defects in monoclinic \zirconia\ than in the tetragonal phase. However, as monoclinic is the low-temperature phase, intrinsic defect concentrations will also be low, thereby requiring additional energy input to produce vacancies when the concentration of iodine is much larger than that of the intrinsic defects. This leads to relatively large concentrations of iodine interstitial defects predicted in the monoclinic Brouwer diagrams, as interstitial sites are always available in the lattice. 

Defects involving iodine in the +1 oxidation state are present in significant concentrations, especially in monoclinic \zirconia , indicating that filling of the $p$ electronic sub-shell is not always energetically favourable compared to forming the smaller iodine ionic radius developed through oxidation. 

The competition between iodine and oxygen for anion sites in \zirconia\ is phase and oxygen pressure dependent. At high oxygen pressures in monoclinic \zirconia , iodine in the +1 oxidation state is predicted to occupy oxygen sites and remains the dominant defect. In tetragonal \zirconia\ at high oxygen pressures, however, the concentration of iodine defects on anion sites decreases steeply, indicating a preference for iodine accommodated at zirconium cation sites. This is indicative of a barrier effect in the tetragonal phase with oxygen out-competing iodine for anion sites.

\chapter{Radioparagenesis of fission products in tetragonal \zirconia}

\label{ch:results3}

\section{Introduction}
\subsection{Radioparagenesis}

The nuclei of fission products immediately after a fission event are typically neutron-rich and unstable. In the case of iodine, the stable isotope is I-127, yet isotopes up to I-143 are produced during fission. This is the true for the fission of all large nuclei, including U-233 (thorium cycle), U-235 (conventional) and Pu-239 (breeder/MOX)

Stress-corrosion cracking (SCC) in nuclear fuel pins is an issue related to early loss of structural integrity of fuel assemblies in light water reactors (LWRs). In particular, the phenomenon of pellet-cladding interaction (PCI) in combination with SCC can lead to failures where the cladding is breached, exposing fuel to the coolant \cite{bcoxpelletclad1990}.     

This study follows previous work on defect equilibria in \zirconia\ to determine the oxide layer's effectiveness as a barrier to iodine \cite{kenichiodine2018}. It was found that the tetragonal phase of \zirconia\ is a greater barrier to iodine ingress than monoclinic \zirconia\ as the partial pressure of oxygen is increased. It is also known that tetragonal \zirconia\ will always be present on the inner surface of the cladding in significant quantities because it is self-stabilised by the stresses imposed as the oxide grows into the zirconium metal, in addition to compressive residual stresses induced by radiation damage. The iodine defect study, however, only informs us about one part of the SCC process. For a more holistic understanding, the life cycle of the iodine must be taken into account as well.     

%SCC studies of the internal surface of zirconium-based fuel claddings have been conducted, which indicate that iodine is likely to be one of the main corrosive species involved in promoting crack growth \cite{rosenbaum1966interaction, Cox1990Pellet-cladReview,Fregonese1998AmountIodine,Sidky1998IodineReview}. The exact mechanism for iodine SCC has not yet been determined due to difficulties observing the internal cladding surface in-situ, while experimental studies are not yet capable of reproducing the conditions under which such failures occur. This study focuses on the oxide on the internal surface of the fuel cladding, following from a previous study on iodine in the oxide layer. \\

Nuclear fuel claddings have unique materials challenges associated with them owing to the highly active environment and creation of unstable isotopes. Corrosive species in the pin such as iodine can be produced directly as a result of fission of uranium fuel. While it is known that iodine plays a role in SCC, one must also consider that these iodine nuclei are unstable. Fission of uranium will produce iodine precursors, mainly unstable isotopes of tellurium. Both iodine and tellurium are relatively common fission products, with combined independent yields from thermal fission of U$_{235}$ above 5\% \cite{kennett1956mass, iodine129fissionyield, imanishi1976independent, iodinefissionyields, iodine132, amiel1975odd}. 

Nuclei produced during fission are typically neutron-rich, resulting in decay modes such as $\beta-$ or neutron emission. In the case of tellurium, the vast majority of unstable isotopes will decay into iodine, which then decays into xenon with varying half-lives depending on the isotope. The decay chain continues with xenon nuclei decaying into caesium, many isotopes of which have half lives measured in years. At this point, fuel is typically retired long before a significant quantity of caesium decays into barium. For this reason we only consider the elements tellurium through caesium in this study. It should also be noted that the majority of thermal fission events occur in the outer rim of the fuel pellet, and a fission product penetration depth of up to 8 $\mu$m in \zirconia\ \cite{degueldre2001behaviour} suggests a large degree of fission product implantation within the oxide. With each nuclear decay comes a change in the chemical and therefore physical behaviour of the atom with its immediate environment. For example, an iodine dopant in \zirconia\ may decay into xenon which will then have a significantly different thermodynamic equilibrium site from the one it inherited.   

Determining the effect of each of these elements in the oxide layer may provide information about the initiation of SCC in fuel cladding. We have therefore adopted a quantum-mechanical calculation approach to model the behaviour of the decay chain elements tellurium through caesium within tetragonal phase zirconia. 

\begin{itemize}
\item We propose that crack initiation on the internal surface of the cladding may be in part due to radioparagenesis of fission products
\item One mechanism is neutron-rich iodine making its way through the monoclinic \zirconia\ before being stopped by the highly passivating tetragonal \zirconia\ closer to the metal interface.
\item The iodine nucleus then decays by beta- particle emission, converting from an iodine to a xenon nucleus.
\item This xenon ion quickly fills its valence shell to the noble gas configuration.
\item The uncharged xenon atom then imposes a large strain on the surrounding \zirconia\ due to the volume mismatch.
\item This strain weakens the monoclinic \zirconia , and promotes crack initiation (new surface relieves the strain imposed by the xenon).
\item The tetragonal \zirconia , now less constrained by the monoclinic layer, expands and becomes less inhibiting to iodine and oxygen ingress.
\item If the iodine partial pressure is high enough relative to the oxygen pressure, the \zirconia\ layer will fail to impede iodine corrosive attack on the zirconium metal.
\end{itemize}

Xenon in a reactor will also eventually decay by beta- emission into caesium, a much more chemically reactive element.

\subsection{Site preference of fission products}

\begin{itemize}

\item \textbf{Tellurium} is a group 6 element like oxygen, but it displays some metallic behaviour.
\item Because of its electronic structure, it may be expected to display preference for the oxygen site in \zirconia .
\item It's metallic properties and low electronegativity, however, suggest that it may be able to fill a cation site instead, but this would require the creation of oxygen vacancies since it has a lower valence than zirconium.
\item \textbf{Iodine} was shown in Chapter 4 to adopt either oxygen and zirconium sites under the right conditions
\item \textbf{Xenon} is a noble gas, but is still able to form compounds with very strong oxidising agents (e.g. XeF4). It's large size (comparison here) may make it unfavourable in both cation and anion sites, thus imposing a large lattice strain.
\item \textbf{Caesium} is a group 1 metal. Its second ionisation energy is very large (removing an electron from a full $p$ sub-shell), likely making it very unfavourable on a zirconium site, only made worse by its size.
\end{itemize}

\subsection{Fission product penetration}

\begin{itemize}
\item Fission products can penetrate up to 10 microns into the cladding, with most deposition occurring at 5 microns (REF)
\item This means we can expect some existing fission products in the cladding before crack-assisted diffusion becomes relevant
\item Therefore some defects will already exist, and the Brouwer diagrams lets us predict what the thermodynamically stable (most likely) ones will be.
\end{itemize}

\section{Methodology}
\subsection{Simulation parameters}

\begin{itemize}
\item energy per atom convergence
\item displacement per atom convergence
\item plane-wave cutoff
\item k-point spacing
\item PBE GGA exchange correlation functional
\end{itemize}

\subsection{Brouwer diagram generation}

\begin{itemize}
\item Defect concentration against oxygen partial pressure 
\item Find Fermi level that leads to charge neutrality
\end{itemize}

\subsection{Defect Volumes}

\begin{itemize}
\item compare constant pressure relaxation of defective to perfect supercell
\end{itemize}

\section{Defect equilibria}
\subsection{Tellurium}

\begin{landscape}
\begin{figure}[htp] % Tellurium
\begin{center}
\begin{tikzpicture}
	\begin{axis}
		[width=11.22cm, xlabel={\ch{log_{10}}($p_{O_{2}}$) (atm)}, ylabel={\ch{log_{10}}([D]) (per f.u.)}, ymin=-10, ymax=0, xmin=-35, xmax=0, legend style={{draw=}, at={(0.30,1.47)}, anchor=north west, legend columns=3, nodes={scale=0.75, transform shape}}]
        \addplot[no marks, draw=blue!70!black] table [x=pO2, y=electrons,]{dat/te_tet_10-5.dat}; \addlegendentry{\ch{e^{'}}}; %\node at (-26.0,-1.9) {\ch{e^{'}}};
        \addplot[no marks, draw=red!85!black] table [x=pO2, y=holes,]{dat/te_tet_10-5.dat}; \addlegendentry{\ch{h^{\textperiodcentered}}}; %\node at (-7,-3.6) {\ch{h^{\textperiodcentered}}};
        \addplot[no marks, draw=black!70!green] table [x=pO2, y=VO{2},]{dat/te_tet_10-5.dat}; \addlegendentry{\ch{V_{O}^{\textperiodcentered\textperiodcentered}}}; %\node at (-26.7,-3.3) {\ch{V_{O}^{\textperiodcentered\textperiodcentered}}};
         \addplot[no marks, draw=black!55!green] table [x=pO2, y=VO{1},]{dat/te_tet_10-5.dat}; \addlegendentry{\ch{V_{O}^{\textperiodcentered}}};
         \addplot[no marks, draw=black!30!green] table [x=pO2, y=VO{0},]{dat/te_tet_10-5.dat}; \addlegendentry{\ch{V_{O}^{x}}};
        \addplot[no marks, draw=yellow!85!blue] table [x=pO2, y=VM{-4},]{dat/te_tet_10-5.dat}; \addlegendentry{\ch{V_{Zr}^{''''}}};
         \addplot[no marks, draw=yellow!75!blue] table [x=pO2, y=VM{-3},]{dat/te_tet_10-5.dat}; \addlegendentry{\ch{V_{Zr}^{'''}}};
         \addplot[no marks, draw=yellow!65!blue] table [x=pO2, y=VM{-2},]{dat/te_tet_10-5.dat}; \addlegendentry{\ch{V_{Zr}^{''}}};
         \addplot[no marks, draw=yellow!55!blue] table [x=pO2, y=VM{-1},]{dat/te_tet_10-5.dat}; \addlegendentry{\ch{V_{Zr}^{'}}};
         \addplot[no marks, draw=yellow!45!blue] table [x=pO2, y=VM{0},]{dat/te_tet_10-5.dat}; \addlegendentry{\ch{V_{Zr}^{x}}};
         \addplot[no marks, draw=red!60!yellow] table [x=pO2, y=Oi{-2},]{dat/te_tet_10-5.dat}; \addlegendentry{\ch{O_{i}^{''}}};
         \addplot[no marks, draw=red!50!yellow] table [x=pO2, y=Oi{-1},]{dat/te_tet_10-5.dat}; \addlegendentry{\ch{O_{i}^{'}}};
         \addplot[no marks, draw=red!40!yellow] table [x=pO2, y=Oi{0},]{dat/te_tet_10-5.dat}; \addlegendentry{\ch{O_{i}^{x}}};
         \addplot[no marks, draw=green!80!pink] table [x=pO2, y=Mi{4},]{dat/te_tet_10-5.dat}; \addlegendentry{\ch{Zr_{i}^{\textperiodcentered\textperiodcentered\textperiodcentered\textperiodcentered}}};
         \addplot[no marks, draw=green!70!pink] table [x=pO2, y=Mi{3},]{dat/te_tet_10-5.dat}; \addlegendentry{\ch{Zr_{i}^{\textperiodcentered\textperiodcentered\textperiodcentered}}};
         \addplot[no marks, draw=green!60!pink] table [x=pO2, y=Mi{2},]{dat/te_tet_10-5.dat}; \addlegendentry{\ch{Zr_{i}^{\textbf{\textperiodcentered\textperiodcentered}}}};
        \addplot[no marks, draw=green!50!pink] table [x=pO2, y=Mi{1},]{dat/te_tet_10-5.dat}; \addlegendentry{\ch{Zr_{i}^{\textperiodcentered}}};
         \addplot[no marks, draw=green!40!pink] table [x=pO2, y=Mi{0},]{dat/te_tet_10-5.dat}; \addlegendentry{\ch{Zr_{i}^{x}}};
         \addplot[no marks, dashed, draw=red!70!black] table [x=pO2, y=Tei{0},]{dat/te_tet_10-5.dat}; \addlegendentry{\ch{Te_{i}^{x}}};
         \addplot[no marks, dashed, draw=red!50!black] table [x=pO2, y=Tei{-1},]{dat/te_tet_10-5.dat}; \addlegendentry{\ch{Te_{i}^{'}}};
        \addplot[no marks, dashed, draw=purple] table [x=pO2, y=Tei{1},]{dat/te_tet_10-5.dat}; \addlegendentry{\ch{Te_{i}^{\textperiodcentered}}};
        \addplot[no marks, dashed, draw=blue!50!white] table [x=pO2, y=TesubO{1},]{dat/te_tet_10-5.dat}; \addlegendentry{\ch{Te_{O}^{\textperiodcentered}}};
        \addplot[no marks, dashed, draw=orange] table [x=pO2, y=TesubO{2},]{dat/te_tet_10-5.dat}; \addlegendentry{\ch{Te_{O}^{\textperiodcentered\textperiodcentered}}};
        \addplot[no marks, dashed, draw=black] table [x=pO2, y=TesubO{3},]{dat/te_tet_10-5.dat}; \addlegendentry{\ch{Te_{O}^{\textperiodcentered\textperiodcentered\textperiodcentered}}};
        \addplot[no marks, dashed, draw=green] table [x=pO2, y=TesubZr{-3},]{dat/te_tet_10-5.dat}; \addlegendentry{\ch{Te_{Zr}^{'''}}};
         \addplot[no marks, dashed, draw=blue] table [x=pO2, y=TesubZr{-4},]{dat/te_tet_10-5.dat}; \addlegendentry{\ch{Te_{Zr}^{''''}}};
         \addplot[no marks, dashed, draw=red] table [x=pO2, y=TesubZr{-5},]{dat/te_tet_10-5.dat}; \addlegendentry{\ch{Te_{Zr}^{'''''}}};
%         \addplot[no marks] table [x=pO2, y=Stoich,]{Te_tet.dat}; \addlegendentry{Stoich};
%\node at (-33.7,-0.5) {\textbf{a)}};
			\end{axis}            
\end{tikzpicture}
\begin{tikzpicture} % TELLURIUM 2
	\begin{axis} % change width to 8.22cm for portrait
		[width=11.22cm, xlabel={\ch{log_{10}}($p_{O_{2}}$) (atm)}, yticklabels={}, ymin=-10, ymax=0, xmin=-35, xmax=0]
        \addplot[no marks, draw=blue!70!black] table [x=pO2, y=electrons,]{dat/te_tet_10-3.dat}; %\node at (-27,-1.7) {\ch{e^{'}}};
        \addplot[no marks, draw=red!85!black] table [x=pO2, y=holes,]{dat/te_tet_10-3.dat}; %\node at (-2.5,-2.1) {\ch{h^{\textperiodcentered}}};
        \addplot[no marks, draw=black!70!green] table [x=pO2, y=VO{2},]{dat/te_tet_10-3.dat}; 
         \addplot[no marks, draw=black!55!green] table [x=pO2, y=VO{1},]{dat/te_tet_10-3.dat}; 
         \addplot[no marks, draw=black!30!green] table [x=pO2, y=VO{0},]{dat/te_tet_10-3.dat}; 
        \addplot[no marks, draw=yellow!85!blue] table [x=pO2, y=VM{-4},]{dat/te_tet_10-3.dat}; 
%         \addplot[no marks, draw=yellow!75!blue] table [x=pO2, y=VM{-3},]{dat/te_tet_10-3.dat}; 
%         \addplot[no marks, draw=yellow!65!blue] table [x=pO2, y=VM{-2},]{dat/te_tet_10-3.dat}; 
%         \addplot[no marks, draw=yellow!55!blue] table [x=pO2, y=VM{-1},]{dat/te_tet_10-3.dat}; 
%         \addplot[no marks, draw=yellow!45!blue] table [x=pO2, y=VM{0},]{dat/te_tet_10-3.dat}; 
%         \addplot[no marks, draw=red!60!yellow] table [x=pO2, y=Oi{-2},]{dat/te_tet_10-3.dat}; 
%         \addplot[no marks, draw=red!50!yellow] table [x=pO2, y=Oi{-1},]{dat/te_tet_10-3.dat}; 
%         \addplot[no marks, draw=red!40!yellow] table [x=pO2, y=Oi{0},]{dat/te_tet_10-3.dat}; 
%         \addplot[no marks, draw=green!80!pink] table [x=pO2, y=Mi{4},]{dat/te_tet_10-3.dat}; 
%         \addplot[no marks, draw=green!70!pink] table [x=pO2, y=Mi{3},]{dat/te_tet_10-3.dat}; 
%         \addplot[no marks, draw=green!60!pink] table [x=pO2, y=Mi{2},]{dat/te_tet_10-3.dat}; 
%         \addplot[no marks, draw=green!50!pink] table [x=pO2, y=Mi{1},]{dat/te_tet_10-3.dat}; 
%         \addplot[no marks, draw=green!40!pink] table [x=pO2, y=Mi{0},]{dat/te_tet_10-3.dat}; 
        \addplot[no marks, dashed, draw=red!70!black] table [x=pO2, y=Tei{0},]{dat/te_tet_10-3.dat}; 
        \addplot[no marks, dashed, draw=red!50!black] table [x=pO2, y=Tei{-1},]{dat/te_tet_10-3.dat}; 
        \addplot[no marks, dashed, draw=purple] table [x=pO2, y=Tei{1},]{dat/te_tet_10-3.dat}; 
        \addplot[no marks, dashed, draw=blue!50!white] table [x=pO2, y=TesubO{1},]{dat/te_tet_10-3.dat}; %\node at (-11,-2.6) {\ch{I_{O}^{\textperiodcentered}}};
        \addplot[no marks, dashed, draw=orange] table [x=pO2, y=TesubO{2},]{dat/te_tet_10-3.dat}; 
        \addplot[no marks, dashed, draw=black] table [x=pO2, y=TesubO{3},]{dat/te_tet_10-3.dat}; 
        \addplot[no marks, dashed, draw=green] table [x=pO2, y=TesubZr{-3},]{dat/te_tet_10-3.dat}; 
        \addplot[no marks, dashed, draw=blue] table [x=pO2, y=TesubZr{-4},]{dat/te_tet_10-3.dat}; 
        \addplot[no marks, dashed, draw=red] table [x=pO2, y=TesubZr{-5},]{dat/te_tet_10-3.dat}; 
%        \addplot[no marks] table [x=pO2, y=Stoich,]{dat/te_tet_10-3.dat}; 
%\node at (-33.7,-0.5) {\textbf{b)}};
			\end{axis}            
\end{tikzpicture}
		\caption{Tetragonal phase Brouwer diagrams of point defects at Tellurium concentrations of a) $10^{-5}$ and b) $10^{-3}$, at a temperature of 1500 K. Space charge = 0}
		\label{figure:telluriumbrouwer-5-3}
	\end{center}
\end{figure}
\end{landscape}

\subsection{Iodine}

\begin{itemize}
\item Should we just reference the Brouwer diagram for iodine again? 
\end{itemize}

\subsection{Xenon}

\begin{itemize}
\item Xenon point defects showed a change in behaviour at high and low oxygen pressures
\end{itemize}

\begin{landscape}
\begin{figure}[htp] % XENON
\begin{center}
\begin{tikzpicture}
	\begin{axis}
		[width=11.22cm, xlabel={\ch{log_{10}}($p_{O_{2}}$) (atm)}, ylabel={\ch{log_{10}}([D]) (per f.u.)}, ymin=-10, ymax=0, xmin=-35, xmax=0, legend style={{draw=}, at={(0.30,1.47)}, anchor=north west, legend columns=3, nodes={scale=0.75, transform shape}}]
        \addplot[no marks, draw=blue!70!black] table [x=pO2, y=electrons,]{dat/xe_tet_10-5.dat}; \addlegendentry{\ch{e^{'}}}; %\node at (-26.0,-1.9) {\ch{e^{'}}};
        \addplot[no marks, draw=red!85!black] table [x=pO2, y=holes,]{dat/xe_tet_10-5.dat}; \addlegendentry{\ch{h^{\textperiodcentered}}}; %\node at (-7,-3.6) {\ch{h^{\textperiodcentered}}};
        \addplot[no marks, draw=black!70!green] table [x=pO2, y=VO{2},]{dat/xe_tet_10-5.dat}; \addlegendentry{\ch{V_{O}^{\textperiodcentered\textperiodcentered}}}; %\node at (-26.7,-3.3) {\ch{V_{O}^{\textperiodcentered\textperiodcentered}}};
         \addplot[no marks, draw=black!55!green] table [x=pO2, y=VO{1},]{dat/xe_tet_10-5.dat}; \addlegendentry{\ch{V_{O}^{\textperiodcentered}}};
         \addplot[no marks, draw=black!30!green] table [x=pO2, y=VO{0},]{dat/xe_tet_10-5.dat}; \addlegendentry{\ch{V_{O}^{x}}};
        \addplot[no marks, draw=yellow!85!blue] table [x=pO2, y=VM{-4},]{dat/xe_tet_10-5.dat}; \addlegendentry{\ch{V_{Zr}^{''''}}};
         \addplot[no marks, draw=yellow!75!blue] table [x=pO2, y=VM{-3},]{dat/xe_tet_10-5.dat}; \addlegendentry{\ch{V_{Zr}^{'''}}};
         \addplot[no marks, draw=yellow!65!blue] table [x=pO2, y=VM{-2},]{dat/xe_tet_10-5.dat}; \addlegendentry{\ch{V_{Zr}^{''}}};
         \addplot[no marks, draw=yellow!55!blue] table [x=pO2, y=VM{-1},]{dat/xe_tet_10-5.dat}; \addlegendentry{\ch{V_{Zr}^{'}}};
         \addplot[no marks, draw=yellow!45!blue] table [x=pO2, y=VM{0},]{dat/xe_tet_10-5.dat}; \addlegendentry{\ch{V_{Zr}^{x}}};
         \addplot[no marks, draw=red!60!yellow] table [x=pO2, y=Oi{-2},]{dat/xe_tet_10-5.dat}; \addlegendentry{\ch{O_{i}^{''}}};
         \addplot[no marks, draw=red!50!yellow] table [x=pO2, y=Oi{-1},]{dat/xe_tet_10-5.dat}; \addlegendentry{\ch{O_{i}^{'}}};
         \addplot[no marks, draw=red!40!yellow] table [x=pO2, y=Oi{0},]{dat/xe_tet_10-5.dat}; \addlegendentry{\ch{O_{i}^{x}}};
         \addplot[no marks, draw=green!80!pink] table [x=pO2, y=Mi{4},]{dat/xe_tet_10-5.dat}; \addlegendentry{\ch{Zr_{i}^{\textperiodcentered\textperiodcentered\textperiodcentered\textperiodcentered}}};
         \addplot[no marks, draw=green!70!pink] table [x=pO2, y=Mi{3},]{dat/xe_tet_10-5.dat}; \addlegendentry{\ch{Zr_{i}^{\textperiodcentered\textperiodcentered\textperiodcentered}}};
         \addplot[no marks, draw=green!60!pink] table [x=pO2, y=Mi{2},]{dat/xe_tet_10-5.dat}; \addlegendentry{\ch{Zr_{i}^{\textbf{\textperiodcentered\textperiodcentered}}}};
        \addplot[no marks, draw=green!50!pink] table [x=pO2, y=Mi{1},]{dat/xe_tet_10-5.dat}; \addlegendentry{\ch{Zr_{i}^{\textperiodcentered}}};
         \addplot[no marks, draw=green!40!pink] table [x=pO2, y=Mi{0},]{dat/xe_tet_10-5.dat}; \addlegendentry{\ch{Zr_{i}^{x}}};
         \addplot[no marks, dashed, draw=red!70!black] table [x=pO2, y=Xei{0},]{dat/xe_tet_10-5.dat}; \addlegendentry{\ch{Xe_{i}^{x}}};
         \addplot[no marks, dashed, draw=red!50!black] table [x=pO2, y=Xei{-1},]{dat/xe_tet_10-5.dat}; \addlegendentry{\ch{Xe_{i}^{'}}};
        \addplot[no marks, dashed, draw=purple] table [x=pO2, y=Xei{1},]{dat/xe_tet_10-5.dat}; \addlegendentry{\ch{Xe_{i}^{\textperiodcentered}}};
        \addplot[no marks, dashed, draw=blue!50!white] table [x=pO2, y=XesubO{1},]{dat/xe_tet_10-5.dat}; \addlegendentry{\ch{Xe_{O}^{\textperiodcentered}}};
        \addplot[no marks, dashed, draw=orange] table [x=pO2, y=XesubO{2},]{dat/xe_tet_10-5.dat}; \addlegendentry{\ch{Xe_{O}^{\textperiodcentered\textperiodcentered}}};
        \addplot[no marks, dashed, draw=black] table [x=pO2, y=XesubO{3},]{dat/xe_tet_10-5.dat}; \addlegendentry{\ch{Xe_{O}^{\textperiodcentered\textperiodcentered\textperiodcentered}}};
        \addplot[no marks, dashed, draw=green] table [x=pO2, y=XesubZr{-3},]{dat/xe_tet_10-5.dat}; \addlegendentry{\ch{Xe_{Zr}^{'''}}};
         \addplot[no marks, dashed, draw=blue] table [x=pO2, y=XesubZr{-4},]{dat/xe_tet_10-5.dat}; \addlegendentry{\ch{Xe_{Zr}^{''''}}};
         \addplot[no marks, dashed, draw=red] table [x=pO2, y=XesubZr{-5},]{dat/xe_tet_10-5.dat}; \addlegendentry{\ch{Xe_{Zr}^{'''''}}};
%         \addplot[no marks] table [x=pO2, y=Stoich,]{xe_tet.dat}; \addlegendentry{Stoich};
%\node at (-33.7,-0.5) {\textbf{a)}};
			\end{axis}            
\end{tikzpicture}
\begin{tikzpicture} % XENON 2
	\begin{axis} % change width to 8.22cm for portrait
		[width=11.22cm, xlabel={\ch{log_{10}}($p_{O_{2}}$) (atm)}, yticklabels={}, ymin=-10, ymax=0, xmin=-35, xmax=0]
        \addplot[no marks, draw=blue!70!black] table [x=pO2, y=electrons,]{dat/xe_tet_10-3.dat}; %\node at (-27,-1.7) {\ch{e^{'}}};
        \addplot[no marks, draw=red!85!black] table [x=pO2, y=holes,]{dat/xe_tet_10-3.dat}; %\node at (-2.5,-2.1) {\ch{h^{\textperiodcentered}}};
        \addplot[no marks, draw=black!70!green] table [x=pO2, y=VO{2},]{dat/xe_tet_10-3.dat}; 
         \addplot[no marks, draw=black!55!green] table [x=pO2, y=VO{1},]{dat/xe_tet_10-3.dat}; 
         \addplot[no marks, draw=black!30!green] table [x=pO2, y=VO{0},]{dat/xe_tet_10-3.dat}; 
        \addplot[no marks, draw=yellow!85!blue] table [x=pO2, y=VM{-4},]{dat/xe_tet_10-3.dat}; 
%         \addplot[no marks, draw=yellow!75!blue] table [x=pO2, y=VM{-3},]{dat/xe_tet_10-3.dat}; 
%         \addplot[no marks, draw=yellow!65!blue] table [x=pO2, y=VM{-2},]{dat/xe_tet_10-3.dat}; 
%         \addplot[no marks, draw=yellow!55!blue] table [x=pO2, y=VM{-1},]{dat/xe_tet_10-3.dat}; 
%         \addplot[no marks, draw=yellow!45!blue] table [x=pO2, y=VM{0},]{dat/xe_tet_10-3.dat}; 
%         \addplot[no marks, draw=red!60!yellow] table [x=pO2, y=Oi{-2},]{dat/xe_tet_10-3.dat}; 
%         \addplot[no marks, draw=red!50!yellow] table [x=pO2, y=Oi{-1},]{dat/xe_tet_10-3.dat}; 
%         \addplot[no marks, draw=red!40!yellow] table [x=pO2, y=Oi{0},]{dat/xe_tet_10-3.dat}; 
%         \addplot[no marks, draw=green!80!pink] table [x=pO2, y=Mi{4},]{dat/xe_tet_10-3.dat}; 
%         \addplot[no marks, draw=green!70!pink] table [x=pO2, y=Mi{3},]{dat/xe_tet_10-3.dat}; 
%         \addplot[no marks, draw=green!60!pink] table [x=pO2, y=Mi{2},]{dat/xe_tet_10-3.dat}; 
%         \addplot[no marks, draw=green!50!pink] table [x=pO2, y=Mi{1},]{dat/xe_tet_10-3.dat}; 
%         \addplot[no marks, draw=green!40!pink] table [x=pO2, y=Mi{0},]{dat/xe_tet_10-3.dat}; 
        \addplot[no marks, dashed, draw=red!70!black] table [x=pO2, y=Xei{0},]{dat/xe_tet_10-3.dat}; 
        \addplot[no marks, dashed, draw=red!50!black] table [x=pO2, y=Xei{-1},]{dat/xe_tet_10-3.dat}; 
        \addplot[no marks, dashed, draw=purple] table [x=pO2, y=Xei{1},]{dat/xe_tet_10-3.dat}; 
        \addplot[no marks, dashed, draw=blue!50!white] table [x=pO2, y=XesubO{1},]{dat/xe_tet_10-3.dat}; %\node at (-11,-2.6) {\ch{I_{O}^{\textperiodcentered}}};
        \addplot[no marks, dashed, draw=orange] table [x=pO2, y=XesubO{2},]{dat/xe_tet_10-3.dat}; 
        \addplot[no marks, dashed, draw=black] table [x=pO2, y=XesubO{3},]{dat/xe_tet_10-3.dat}; 
        \addplot[no marks, dashed, draw=green] table [x=pO2, y=XesubZr{-3},]{dat/xe_tet_10-3.dat}; 
        \addplot[no marks, dashed, draw=blue] table [x=pO2, y=XesubZr{-4},]{dat/xe_tet_10-3.dat}; 
        \addplot[no marks, dashed, draw=red] table [x=pO2, y=XesubZr{-5},]{dat/xe_tet_10-3.dat}; 
%        \addplot[no marks] table [x=pO2, y=Stoich,]{dat/xe_tet_10-3.dat}; 
%\node at (-33.7,-0.5) {\textbf{b)}};
			\end{axis}            
\end{tikzpicture}
		\caption{Tetragonal phase Brouwer diagrams of point defects at Xenon concentrations of a) $10^{-5}$ and b) $10^{-3}$, at a temperature of 1500 K. Space charge = 0}
		%\label{figure:tikzbrouwerconctet}
	\end{center}
\end{figure}
\end{landscape}

\subsection{Caesium}

\begin{itemize}
\item Cs point defects didn't show much change in behaviour
\item Defects behaviour strongly follows single ionisation preference (as expected).
\end{itemize}

\begin{landscape}
\begin{figure}[htp] % CAESIUM
\begin{center}
\begin{tikzpicture}
	\begin{axis}
		[width=11.22cm, xlabel={\ch{log_{10}}($p_{O_{2}}$) (atm)}, ylabel={\ch{log_{10}}([D]) (per f.u.)}, ymin=-10, ymax=0, xmin=-35, xmax=0, legend style={{draw=}, at={(0.30,1.47)}, anchor=north west, legend columns=3, nodes={scale=0.75, transform shape}}]
        \addplot[no marks, draw=blue!70!black] table [x=pO2, y=electrons,]{dat/cs_tet_10-5.dat}; \addlegendentry{\ch{e^{'}}}; %\node at (-26.0,-1.9) {\ch{e^{'}}};
        \addplot[no marks, draw=red!85!black] table [x=pO2, y=holes,]{dat/cs_tet_10-5.dat}; \addlegendentry{\ch{h^{\textperiodcentered}}}; %\node at (-7,-3.6) {\ch{h^{\textperiodcentered}}};
        \addplot[no marks, draw=black!70!green] table [x=pO2, y=VO{2},]{dat/cs_tet_10-5.dat}; \addlegendentry{\ch{V_{O}^{\textperiodcentered\textperiodcentered}}}; %\node at (-26.7,-3.3) {\ch{V_{O}^{\textperiodcentered\textperiodcentered}}};
         \addplot[no marks, draw=black!55!green] table [x=pO2, y=VO{1},]{dat/cs_tet_10-5.dat}; \addlegendentry{\ch{V_{O}^{\textperiodcentered}}};
         \addplot[no marks, draw=black!30!green] table [x=pO2, y=VO{0},]{dat/cs_tet_10-5.dat}; \addlegendentry{\ch{V_{O}^{x}}};
        \addplot[no marks, draw=yellow!85!blue] table [x=pO2, y=VM{-4},]{dat/cs_tet_10-5.dat}; \addlegendentry{\ch{V_{Zr}^{''''}}};
         \addplot[no marks, draw=yellow!75!blue] table [x=pO2, y=VM{-3},]{dat/cs_tet_10-5.dat}; \addlegendentry{\ch{V_{Zr}^{'''}}};
         \addplot[no marks, draw=yellow!65!blue] table [x=pO2, y=VM{-2},]{dat/cs_tet_10-5.dat}; \addlegendentry{\ch{V_{Zr}^{''}}};
         \addplot[no marks, draw=yellow!55!blue] table [x=pO2, y=VM{-1},]{dat/cs_tet_10-5.dat}; \addlegendentry{\ch{V_{Zr}^{'}}};
         \addplot[no marks, draw=yellow!45!blue] table [x=pO2, y=VM{0},]{dat/cs_tet_10-5.dat}; \addlegendentry{\ch{V_{Zr}^{x}}};
         \addplot[no marks, draw=red!60!yellow] table [x=pO2, y=Oi{-2},]{dat/cs_tet_10-5.dat}; \addlegendentry{\ch{O_{i}^{''}}};
         \addplot[no marks, draw=red!50!yellow] table [x=pO2, y=Oi{-1},]{dat/cs_tet_10-5.dat}; \addlegendentry{\ch{O_{i}^{'}}};
         \addplot[no marks, draw=red!40!yellow] table [x=pO2, y=Oi{0},]{dat/cs_tet_10-5.dat}; \addlegendentry{\ch{O_{i}^{x}}};
         \addplot[no marks, draw=green!80!pink] table [x=pO2, y=Mi{4},]{dat/cs_tet_10-5.dat}; \addlegendentry{\ch{Zr_{i}^{\textperiodcentered\textperiodcentered\textperiodcentered\textperiodcentered}}};
         \addplot[no marks, draw=green!70!pink] table [x=pO2, y=Mi{3},]{dat/cs_tet_10-5.dat}; \addlegendentry{\ch{Zr_{i}^{\textperiodcentered\textperiodcentered\textperiodcentered}}};
         \addplot[no marks, draw=green!60!pink] table [x=pO2, y=Mi{2},]{dat/cs_tet_10-5.dat}; \addlegendentry{\ch{Zr_{i}^{\textbf{\textperiodcentered\textperiodcentered}}}};
        \addplot[no marks, draw=green!50!pink] table [x=pO2, y=Mi{1},]{dat/cs_tet_10-5.dat}; \addlegendentry{\ch{Zr_{i}^{\textperiodcentered}}};
         \addplot[no marks, draw=green!40!pink] table [x=pO2, y=Mi{0},]{dat/cs_tet_10-5.dat}; \addlegendentry{\ch{Zr_{i}^{x}}};
         \addplot[no marks, dashed, draw=red!70!black] table [x=pO2, y=Csi{0},]{dat/cs_tet_10-5.dat}; \addlegendentry{\ch{Cs_{i}^{x}}};
         \addplot[no marks, dashed, draw=red!50!black] table [x=pO2, y=Csi{-1},]{dat/cs_tet_10-5.dat}; \addlegendentry{\ch{Cs_{i}^{'}}};
        \addplot[no marks, dashed, draw=purple] table [x=pO2, y=Csi{1},]{dat/cs_tet_10-5.dat}; \addlegendentry{\ch{Cs_{i}^{\textperiodcentered}}};
        \addplot[no marks, dashed, draw=blue!50!white] table [x=pO2, y=CssubO{1},]{dat/cs_tet_10-5.dat}; \addlegendentry{\ch{Cs_{O}^{\textperiodcentered}}};
        \addplot[no marks, dashed, draw=green!60!black] table [x=pO2, y=CssubO{2},]{dat/cs_tet_10-5.dat}; \addlegendentry{\ch{Cs_{O}^{\textperiodcentered\textperiodcentered}}};
        \addplot[no marks, dashed, draw=black] table [x=pO2, y=CssubO{3},]{dat/cs_tet_10-5.dat}; \addlegendentry{\ch{Cs_{O}^{\textperiodcentered\textperiodcentered\textperiodcentered}}};
        \addplot[no marks, dashed, draw=orange!80!black] table [x=pO2, y=CssubZr{-3},]{dat/cs_tet_10-5.dat}; \addlegendentry{\ch{Cs_{Zr}^{'''}}};
         \addplot[no marks, dashed, draw=pink] table [x=pO2, y=CssubZr{-4},]{dat/cs_tet_10-5.dat}; \addlegendentry{\ch{Cs_{Zr}^{''''}}};
         \addplot[no marks, dashed, draw=purple] table [x=pO2, y=CssubZr{-5},]{dat/cs_tet_10-5.dat}; \addlegendentry{\ch{Cs_{Zr}^{'''''}}};
%         \addplot[no marks] table [x=pO2, y=Stoich,]{cs_tet.dat}; \addlegendentry{Stoich};
%\node at (-33.7,-0.5) {\textbf{a)}};
			\end{axis}            
\end{tikzpicture}
\begin{tikzpicture} % CAESIUM 2
	\begin{axis}
		[width=11.22cm, xlabel={\ch{log_{10}}($p_{O_{2}}$) (atm)}, yticklabels={}, ymin=-10, ymax=0, xmin=-35, xmax=0]
        \addplot[no marks, draw=blue!70!black] table [x=pO2, y=electrons,]{dat/cs_tet_10-3.dat}; %\node at (-27,-1.7) {\ch{e^{'}}};
        \addplot[no marks, draw=red!85!black] table [x=pO2, y=holes,]{dat/cs_tet_10-3.dat}; %\node at (-2.5,-2.1) {\ch{h^{\textperiodcentered}}};
        \addplot[no marks, draw=black!70!green] table [x=pO2, y=VO{2},]{dat/cs_tet_10-3.dat}; 
         \addplot[no marks, draw=black!55!green] table [x=pO2, y=VO{1},]{dat/cs_tet_10-3.dat}; 
         \addplot[no marks, draw=black!30!green] table [x=pO2, y=VO{0},]{dat/cs_tet_10-3.dat}; 
        \addplot[no marks, draw=yellow!85!blue] table [x=pO2, y=VM{-4},]{dat/cs_tet_10-3.dat}; 
%         \addplot[no marks, draw=yellow!75!blue] table [x=pO2, y=VM{-3},]{dat/cs_tet_10-3.dat}; 
%         \addplot[no marks, draw=yellow!65!blue] table [x=pO2, y=VM{-2},]{dat/cs_tet_10-3.dat}; 
%         \addplot[no marks, draw=yellow!55!blue] table [x=pO2, y=VM{-1},]{dat/cs_tet_10-3.dat}; 
%         \addplot[no marks, draw=yellow!45!blue] table [x=pO2, y=VM{0},]{dat/cs_tet_10-3.dat}; 
%         \addplot[no marks, draw=red!60!yellow] table [x=pO2, y=Oi{-2},]{dat/cs_tet_10-3.dat}; 
%         \addplot[no marks, draw=red!50!yellow] table [x=pO2, y=Oi{-1},]{dat/cs_tet_10-3.dat}; 
%         \addplot[no marks, draw=red!40!yellow] table [x=pO2, y=Oi{0},]{dat/cs_tet_10-3.dat}; 
%         \addplot[no marks, draw=green!80!pink] table [x=pO2, y=Mi{4},]{dat/cs_tet_10-3.dat}; 
%         \addplot[no marks, draw=green!70!pink] table [x=pO2, y=Mi{3},]{dat/cs_tet_10-3.dat}; 
%         \addplot[no marks, draw=green!60!pink] table [x=pO2, y=Mi{2},]{dat/cs_tet_10-3.dat}; 
%         \addplot[no marks, draw=green!50!pink] table [x=pO2, y=Mi{1},]{dat/cs_tet_10-3.dat}; 
%         \addplot[no marks, draw=green!40!pink] table [x=pO2, y=Mi{0},]{dat/cs_tet_10-3.dat}; 
        \addplot[no marks, dashed, draw=red!70!black] table [x=pO2, y=Csi{0},]{dat/cs_tet_10-3.dat}; 
        \addplot[no marks, dashed, draw=red!50!black] table [x=pO2, y=Csi{-1},]{dat/cs_tet_10-3.dat}; 
        \addplot[no marks, dashed, draw=purple] table [x=pO2, y=Csi{1},]{dat/cs_tet_10-3.dat}; 
        \addplot[no marks, dashed, draw=blue!50!white] table [x=pO2, y=CssubO{1},]{dat/cs_tet_10-3.dat}; %\node at (-11,-2.6) {\ch{I_{O}^{\textperiodcentered}}};
        \addplot[no marks, dashed, draw=green!60!black] table [x=pO2, y=CssubO{2},]{dat/cs_tet_10-3.dat}; 
        \addplot[no marks, dashed, draw=black] table [x=pO2, y=CssubO{3},]{dat/cs_tet_10-3.dat}; 
        \addplot[no marks, dashed, draw=orange!80!black] table [x=pO2, y=CssubZr{-3},]{dat/cs_tet_10-3.dat}; 
        \addplot[no marks, dashed, draw=pink] table [x=pO2, y=CssubZr{-4},]{dat/cs_tet_10-3.dat}; 
        \addplot[no marks, dashed, draw=purple] table [x=pO2, y=CssubZr{-5},]{dat/cs_tet_10-3.dat}; 
%        \addplot[no marks] table [x=pO2, y=Stoich,]{dat/cs_tet_10-3.dat}; 
%\node at (-33.7,-0.5) {\textbf{b)}};
			\end{axis}            
\end{tikzpicture}
		\caption{Tetragonal phase Brouwer diagrams of point defects at caesium concentrations of a) $10^{-5}$ and b) $10^{-3}$, at a temperature of 1500 K. Space charge = 0}
		%\label{figure:tikzbrouwerconctet}
	\end{center}
\end{figure}
\end{landscape}

\section{Summary}

\chapter{Future work}

\label{ch:future}

\section{Iodine empirical potential}
\section{Grain boundary transport}
\section{Zr/ZrO/ZrO2 interface study}


\addcontentsline{toc}{chapter}{References}
\label{References}
\renewcommand\bibname{References}
\bibliographystyle{unsrt}
\bibliography{Mendeley}

\appendix
% Appendices come here
\addcontentsline{toc}{chapter}{Appendix}
\label{Appendix}

\chapter{ParaSweep}

ParaSweep is a generalised sensitivity analysis visualisation tool which was developed during this project. Initially, it was built to help visualise the effects of changing single parameters in Brouwer diagrams, such as temperature or concentration of defects. 


\chapter{CASTEP and HPC Scripts}

Throughout the course of this work, many useful scripts were created to help with preparing CASTEP jobs and analysing their outputs. These scripts have been made available online and for free at \href{https://github.com/v1thesource/CASTEP}{https://github.com/v1thesource/CASTEP}. The purpose of open-sourcing these scripts is to simplify the experience for new users of CASTEP and help them save a considerable amount of time.


\end{document}


\begin{document}

\title{\LARGE {\bf Atomistic Simulation of Fission Products in Zirconia Polymorphs}\\
 \vspace*{6mm}
}

\author{Alexandros Kenich}
\submitdate{April 2019}

\normallinespacing
\maketitle

\vspace*{140px}
\begin{center}
\textsc{\LARGE Declaration}
\end{center}
I declare that the work presented in this thesis is my own, and that all efforts from others are referenced. 

The copyright of this thesis rests with the author and is made available under a Creative Commons Attribution Non-Commercial No Derivatives licence. Researchers are free to copy, distribute or transmit the thesis on the condition that they attribute it, that they do not use it for commercial purposes and that they do not alter, transform or build upon it. For any reuse or redistribution, researchers must make clear to others the licence terms of this work. 

\begin{center}
\rule{125px}{0.2px}
\end{center}
\vfill
\pagebreak

\preface
\addcontentsline{toc}{chapter}{Abstract}

\begin{abstract}
Zirconium alloys are used as a cladding material in over 90\% of nuclear reactors worldwide due to properties which are uniquely suited to the operating environment of a reactor. In this thesis, density functional theory (DFT) simulations were conducted to investigate the behaviour of fission product dopants in the inner cladding oxide, and to examine the role this oxide layer plays in limiting corrosion in the context of pellet-cladding interaction (PCI). 

Simulations in undoped monoclinic, tetragonal and cubic \zirconia\ yielded non-defective structure properties in addition to intrinsic defect energies, volumes and defect equilibria. Fully-charged Schottky defects \{2\ch{V_{O}^{**}}:\ch{V_{Zr}^{''''}}\}$^{\times}$ were shown to have the smallest formation energies in each phase, followed by O Frenkel defects and then Zr Frenkel defects. Defective cubic \zirconia\ simulations are sensitive to finite-size effects, and would often break symmetry or collapse into the tetragonal phase when defect clusters were introduced. Free energy calculations predicted a transition from monoclinic to tetragonal as temperature is increased, but not from tetragonal to cubic. % as would be expected. 

Iodine adopts oxidation states of either +1 (\ch{I_{O}^{***}}, \ch{I_{i}^{*}} and \ch{I_{Zr}^{'''}}) or -1 (\ch{I_{O}^{*}}) when forming defects in \zirconia , with fewer defects in the 0 oxidation state (\ch{I_{O}^{**}}). At high oxygen partial pressures ($p_{O_{2}}$), iodine defects in tetragonal \zirconia\ fall significantly. In monoclinic \zirconia, iodine defects changed by only small amounts as $p_{O_{2}}$ was increased. This demonstrated competition between iodine and oxygen in \zirconia , and that it is dependent on both $p_{O_{2}}$ and phase. High $p_{O_{2}}$ in the tetragonal phase provides the greatest barrier to iodine ingress.

During reactor power ramps, the quantity of fission products implanted in the oxide layer will increase. Decay rates of major Te and I isotopes were found to be commensurate with time to failure in irradiation tests. Defect equilibria and volumes of Te, I, Xe and Cs were obtained in tetragonal \zirconia\ to investigate the effect of nuclear transmutation while dopant atoms are present. Defect evolution on the O site is predicted to be \ch{Te_{O}^{**}} $\rightarrow$ \ch{I_{O}^{*}} $\rightarrow$ \ch{Xe_{O}^{**}} $\rightarrow$ \ch{Cs_{O}^{**}}. On the Zr site, Brouwer diagrams predict \ch{Te_{Zr}^{'''}} $\rightarrow$ \ch{I_{Zr}^{'''}} $\rightarrow$ \ch{Xe_{Zr}^{''''}} $\rightarrow$ \ch{Cs_{Zr}^{'''}}. These defects have large defect volumes and will generate stresses which may promote crack formation.
\end{abstract}

%\begin{itemize}
%\item The third study was about tellurium, iodine, xenon and caesium in the tetragonal phase only.
%\item We propose a new initiation mechanism for PCI failures, whereby iodine diffuses deep into the \zirconia\ layer, past the monoclinic portion but short of the oxide-metal interface. \zirconia\ in this region of the oxide is predominantly tetragonal phase. The iodine nuclei then decay into xenon nuclei, which are larger and have less coherence with the \zirconia\ matrix. These xenon atoms impose a significant strain locally which will open cracks and initiate new ones. At a critical concentration of iodine, this effect bares enough fresh metal surface such that the corrosive effect of iodine outpaces the development of a passivating oxide layer, leading to failure of the clad.
%end{itemize}

\cleardoublepage

\addcontentsline{toc}{chapter}{Acknowledgements}

\begin{acknowledgements}

Firstly I would like to thank my supervisors Robin Grimes and Mark Wenman for giving me a chance (despite being a lowly mechanical engineer) and opening the door to pursue nuclear engineering at such a high level. I would also like to thank the EPSRC for funding my studentship through the ICO CDT, the Imperial College HPC team for their quick responses whenever something went wrong, Philipp Frankel and his research group at the University of Manchester for the fruitful discussions and insightful conferences over the years, the Department of Materials administration staff for all their help with my admin woes and the Centre for Nuclear Engineering for being my second home for half a decade.

In no particular order, I want to give a shoutout to the people who have left a strong impression on me and influenced my growth both as a scientist and as a person: Wael Al Jishi, Conor Galvin, Paul Fossati, Claudia Gasparrini, Navaratnarajah Kuganathan, Dhan-Sham Rana, Said El Chamaa, Filippo Vecchiato, Jana Smutna, Vlad Podgurschi, Matt Jackson, Lloyd Jones, Nipun Wickramasundara, Hussam Zaghal, Patrick Burr, William White, Mark Mawdsley, Richard Pearson, Sophie Morrison, Alan Charles, Jonathan Tate, Andy Wilson, John Brokx, Alexandru Paunoiu, Irina Dumitrescu, Julian Sutherland, Anca Semenescu and of course, Emma Warriss.

Finally, I give my everlasting gratitude and love to my parents, my wife Cristina and my daughter Livia-May.

\clearpage

\vfill
\begin{center}
\emph{In memory of Emma Warriss}
\end{center}
\vfill

\end{acknowledgements}
%\input{dedication/dedication}
%\input{quotes/quotes}

\body

% body of thesis comes here
\doublespacing

\chapter{Introduction} \label{introduction}

\section{Nuclear Power} % Complete

In the summer of 1956, the world's first commercial nuclear power plant was connected to the grid in the north of England. This marked a significant departure from previous forms of commercial energy production, which relied on relatively low energy density sources such as the combustion of coal, oil and gas. Before this, the closest anyone had come to utilising nuclear energy commercially was through geothermal power, where the thermal energy input is partly due to radiogenic heat from unstable isotopes in the Earth's mantle \cite{gando2011partial}. 
%which relied on the chemical reactions of coal oil and gas
%Combustion is a chemical process, where energy differences between reactants and products are exploited via electron exchange. Nuclear energy however, exploits the energy difference between nuclei. Both rely on the conversion of mass into energy, however, the amount of energy that can be extracted varies by several orders of magnitude

Combustion is a chemical process whereby energy differences between reactants and products are exploited via electron exchange. Nuclear energy exploits the energy difference between nuclei. Both rely on the conversion of mass into energy, however, the amount of energy that can be extracted from the nucleus is several orders of magnitude greater.
%Combustion is a chemical process, and its use in commercial energy production is fundamentally about exploiting the free energy difference when electrons are exchanged between some reactants to produce some products. Nuclear energy, however, is about the direct conversion of mass into energy. The difference between the two is staggering.

Consider methane, with an enthalpy of combustion of −887.2 kJ/mol \cite{thornton1917xv}. This is the equivalent of 9.14 eV per particle. By comparison, the total energy release from fission of one uranium-235 nucleus is at least $1.65 \times 10^{8}$ eV, as shown in Figure \ref{figure:fissionenergy}.

Combustion-based power as a technology has matured over hundreds of years, with modern optimisations only looking to offer fractional percent gains in efficiency. By comparison, nuclear power technology is far from mature, with large improvements yet to be realised. One such feature is load-following, an enormously useful feature for a power plant which is currently underutilised in nuclear reactors. Load-following, as currently practiced in some French and German nuclear power plants, is defined as operation where power output follows a variable load programme on a daily basis with several power changes (i.e. to follow the change in electricity demand over a 24 hour period). These power variations can be as large as 50\% of a reactor's rated power \cite{lokhov2011technical}. The biggest obstacle to load-following in nuclear reactors is the issue of pellet-cladding interaction (PCI), which is the basis of the work in this thesis.

\begin{figure}[ht]
\centering
\includegraphics[height=13cm]{images/fission_energy_total.png}
\caption[Energy from thermal fission of U$^{235}$ as a function of mass ratios of daughter nuclei. Total energy release includes contributions from gamma rays and subsequent radioactive decays.]{Energy from thermal fission of U$^{235}$ as a function of mass ratios of daughter nuclei. Total energy release includes contributions from gamma rays and subsequent radioactive decays. Taken from \cite{aras1965ranges}.}
\label{figure:fissionenergy}
\end{figure}

\subsection{Fission}

Commercial nuclear power plants extract energy through the process of fission, where a large nucleus is split into smaller nuclei. While it is also possible to extract energy from certain small nuclei by the process of fusing them into larger ones, no fusion reactor currently exists which achieves a net positive energy output. At a fundamental level, both fission and fusion rely upon mass-energy equivalence. The relationship between mass and energy is shown using Einstein's equation:
\begin{equation}
\label{emc2}
    E = mc^{2}
\end{equation}
where $E$ is the energy of the system, $m$ is the mass and $c$ is the speed of light in a vacuum. Using this equation we can analyse a typical fission reaction:
\begin{equation}
    \ch{U^{235}_{92}} + n^{1}_{0} \xrightarrow[]{absorption} \ch{U^{236}_{92}} \xrightarrow[]{fission} \ch{I^{132}_{53}} + \ch{Y^{101}_{39}} + 3n^{1}_{0}
\label{eqn:fission} 
\end{equation}
While the number of protons and neutrons are conserved throughout the reaction, a mass difference calculation will show that there is actually less mass in the products than the reactants by approximately 0.188 amu (3.127$\times 10^{-28}$ kg). This missing mass, known as the \emph{mass defect}, is converted to energy ($\sim$175 MeV). In this way, the total mass-energy of the system is conserved. Some of this energy is carried away as kinetic energy of the fission products (in Equation \ref{eqn:fission}, I and Y) and also the kinetic energy of the neutrons. The neutrons at this stage have energies \goodtilde{1} MeV and are known as \emph{fast} neutrons.

This change in mass arises due to the phenomenon of \emph{binding energy}. In order for two or more nucleons to be thermodynamically stable when bound together, the total free energy of the bound configuration must be less than the sum of constituent nucleon free energies. Much as with energy stored in a chemical bond, the binding energy represents the energy required to separate the nucleus into individual protons and neutrons. 

Larger nuclei will generally have a greater total binding energy value compared to smaller nuclei, but the mass defect per nucleon will not necessarily be the same in a larger nucleus. It is therefore useful to normalise the binding energy by the mass number. Different isotopes have different binding energies, and any nuclear reaction that increases the binding energy per nucleon will be exothermic, whether by fission or fusion. Figure \ref{figure:bindingenergy} shows a plot of binding energy per nucleon against mass number with the relevant isotopes from Equation \ref{eqn:fission}. 

%235.0439299 + 1.008664 (236.0525939) -> 131.907997 + 100.93031 + 3.025992 (235.864299)
\begin{figure}[ht]
\centering
\includegraphics[width=14cm]{images/Binding_energy_curve.png}
\caption[Plot of binding energy per nucleon against mass number. Arrows indicate the reaction shown in equation \ref{eqn:fission}.]{Plot of binding energy per nucleon against mass number. Arrows indicate the reaction shown in equation \ref{eqn:fission}. Adapted from \cite{Fastfission}.}
\label{figure:bindingenergy}
\end{figure}

\subsection{Reactor design} % Complete

Commercial nuclear reactors are large boilers in a Rankine cycle, designed to maximise heat transfer to a working fluid. All nuclear plants use steam turbines on the generation side, though the reactor coolant may be another fluid in a separate loop, such as carbon dioxide in gas-cooled reactors (GCRs), or even in a separate water loop such as in pressurised water reactors (PWRs). 

The most prevalent reactor type is the PWR, followed by the boiling water reactor (BWR). A schematic of a PWR power plant is shown in Figure \ref{figure:pwrschematic}. This design incorporates a primary coolant loop and heat exchanger to a secondary loop at a lower pressure. Steam is generated on the low pressure side of the heat exchanger which then drives a steam turbine. There are several other reactor types used around the world (enumerated in Table \ref{figure:world_reactors}). The work in this thesis is focused on zirconium-based claddings which are used worldwide in all commercial reactors except GCRs and sodium-cooled fast reactors. In total, zirconium fuel cladding is used in over 95\% of all nuclear reactor fuel pins, and so performance improvements in these cladding materials have an effect across the entire industry.

\begin{figure}[ht] % Schematic of a PWR
\centering
\includegraphics[width=\linewidth]{images/pwrschematic.png}
\caption[Schematic illustration of a PWR power plant.]{Schematic illustration of a PWR power plant. Taken from \cite{lokhov2011technical}.}
\label{figure:pwrschematic}
\end{figure}

The fission of uranium takes place inside a steel reactor pressure vessel (RPV) in PWRs and BWRs, which holds the fuel pins, control rods and other reactor internals. The working fluid in a nuclear reactor is typically under high pressure, with PWR RPV operating pressures between 150 and 160 bar, while BWRs operate at lower pressures of around 70 bar \cite{kok2016nuclear, Server2010, Durmayaz2001}. The pressure of the coolant acts on the fuel cladding, generating radial and hoop stresses which influence crack formation. 

The operating temperature of the coolant in a typical PWR is approximately 600 K. This is a low temperature relative to the melting points of Zr metal (2128 K) and ZrO$_{2}$ (2988 K). This temperature together with the high pressure is chosen in order to keep the coolant in the liquid phase for safety reasons, though this limits the thermodynamic efficiency of the plant. The highest temperature in a reactor will occur in the nuclear fuel, with PWR fuel pellets reaching centreline temperatures of up to 1673 K \cite{beyer1998review}.

\begin{table}[ht] % Reactors in the world
\centering
\caption[Type and number of different reactors operational worldwide at the end of 2017. Change from 2016 shown in parentheses.]{Type and number of different reactors operational worldwide at the end of 2017. Change from 2016 shown in parentheses. Taken from \cite{WNAreport2018}.}
\includegraphics[width=15cm]{images/WNA_report2018.png}
\label{figure:world_reactors}
\end{table}

For both PWRs and BWRs, the coolant is typically light water (as opposed to heavy water, D$_{2}$O). Water is used because it has many useful engineering properties. It has a high heat capacity (compared to the gaseous coolant in gas cooled reactors), has low activation in a free neutron environment and also serves as a good radiation shield. Furthermore, it is plentiful, cheap and easily purified.

In addition to its function as a coolant, water also acts as a neutron moderator, slowing down high-energy fast neutrons from fission events and nuclear decay processes. Moderation of neutrons is an important step in the nuclear reactor because slow thermal neutrons (i.e. neutrons at thermal equilibrium with the coolant) are significantly more likely to cause uranium nuclei to undergo fission than fast neutrons. Fast neutrons will typically escape from the fuel pin after they are generated, dispense most of their energy in the coolant via scattering with H and O nuclei, and then some will re-enter the fuel, where they may cause fission of a U$^{235}$ nucleus near the outer edge of the fuel pellet. Neutrons which do not end up fissioning fuel will either be absorbed parasitically by other nuclei (e.g. in the control rods or coolant), escape from the reactor entirely (neutron leakage), or decay into protons (free neutrons have a half-life of 10.61 minutes \cite{Christensen1972}).

\subsection{Fuel pellets and cladding} \label{ss_fuelpin}

Fuel assemblies in nuclear reactors are bundles of fuel pins (see Figure \ref{figure:fuelassembly}). In most commercial reactors, fuel pins are comprised of a zirconium-based cladding (tubes), which are filled with cylindrical UO$_{2}$ fuel pellets, each of which are approximately 1 cm$^{3}$ in volume (see Figure \ref{figure:fuelpellet}). Fuel pellets in PWRs and BWRs have dishes on the top and bottom faces of the cylinder as well as chamfered edges. The main function of the dishes is  to reduce the axial pellet-pellet stresses caused due to swelling of the pellet when irradiated \cite{marino2005crack}. Chamfers aid in the loading of fuel pellets into the cladding, as well as reducing the risk of chipping at the edges of the fuel pellet. This is important because chipping of the fuel pellet can lead to debris falling into the pellet-cladding gap where it can act as a stress raiser \cite{doerr2015nuclear}.

\begin{figure}[ht]
\centering
\includegraphics[width=11cm]{images/fuelassembly.png}
\caption[Schematic view of a PWR fuel assembly and a PWR fuel pin.]{Schematic view of a PWR fuel assembly and a PWR fuel pin. Adapted from \cite{Croff2003}.}
\label{figure:fuelassembly}
\end{figure} 

Once loaded with fuel pellets, the fuel pins are capped and filled with inert helium gas, pressurised to between 2 and 25 atm to improve heat transfer from the fuel pellets to the coolant as well as delaying inward creep deformation of the cladding due to the high coolant pressure \cite{King1980}. 

LWR fuel pellets are manufactured in a multi-stage process starting from enriched UF$_{6}$. The UF$_{6}$ must be converted into UO$_{2}$, which can be done using either a `dry' or `wet' process, referring to the use of liquid water in the process. The dry process, called the integrated dry route (IDR), is simpler and is described as follows:

\begin{itemize}
\item Enriched UF$_{6}$ (a solid at room temperature and pressure) is heated into vapour form using an autoclave.
\item UF$_{6}$ vapour is mixed with steam and fed into a rotary kiln.
\item Hydrogen gas is added to the mixture and the UF$_{6}$ is reduced to solid UO$_{2}$. The gaseous HF is recovered, leaving pure UO$_{2}$ crystals.
\end{itemize}

The UO$_{2}$ from this stage is then blended to homogenise the particle sizes and achieve a desired particle surface area. At this stage, additives may be introduced to the UO$_{2}$ (e.g. burnable poisons, lubricants, dopants to improve densification or to control microstructure). This powder is then fed into a pellet pressing die where it is pressed into a cylindrical shape, called a `green' pellet. Green pellets are then sintered in a furnace at temperatures of up to 2000 K in order to consolidate and increase the density of the pellets \cite{pramanik2010innovative}. These fired pellets are then machined to the appropriate dimensions, including chamfers and dishes, before final inspection and loading into a cladding tube.

\begin{figure}[ht]
\centering
\includegraphics[width=10cm]{images/fuelpellet.png}
\caption[UO$_{2}$ LWR fuel pellet showing dishes and chamfers.]{UO$_{2}$ LWR fuel pellet showing dishes and chamfers. Adapted from \cite{tulenko2013development}.}
\label{figure:fuelpellet}
\end{figure}

In the early stages of a fuel pin's life, there is a small gap between the fuel pellet and the cladding, known as the pellet-cladding gas gap. This gas gap slowly closes with increasing fuel burn-up due to swelling of the fuel pellets and inward creep deformation of the cladding due to the coolant pressure. The pellet-cladding system is shown using a schematic view of the cross section of a PWR fuel pin in Figure \ref{figure:gas_gap}. The cladding internal oxide layer covers the entire internal surface of the cladding and is the first barrier to corrosive species. 

\begin{figure}[ht]
\centering
\includegraphics[width=10cm]{images/gas_gap.png}
\caption{Schematic cross-section of a single PWR fuel pin with an expanded view of the pellet-gap-cladding system.}
\label{figure:gas_gap}
\end{figure}

\subsection{Effects of radiation on materials} 

While ionising radiation is always present in the environment as background radiation, the intensity of radiation in a nuclear reactor is so great that it causes significant engineering challenges because of how it affects change in reactor materials.

Radiation hardening (also known as radiation embrittlement) is a phenomenon which affects most materials subjected to ionising radiation. It is characterised by a loss of plasticity caused by radiation damage over time, leading to an increased risk of cracks and failure of components. While zirconium is a very useful nuclear material due to its neutron transparency, it is still susceptible to radiation damage \cite{Wisner1998}. Beyond certain levels of radiation damage, phase changes may also occur. %In \zirconia\ however,  

Amorphisation is another effect of radiation damage, which has been observed in the (Zr, U)O$_{2}$ bonding layer in fuel pins \cite{Nogita1997}. This is characterised by an overall loss of long-range order of atoms in a crystal. This typically occurs beyond a certain threshold of radiation damage depending on the material, called the critical amorphisation dose. Amorphisation causes a loss in long range crystallographic structure and a corresponding reduction in structural stability (amorphous materials have a higher Gibbs free energy than their crystalline counterparts), and causes swelling of the material \cite{Einfal2013}. In the literature, there is evidence of amorphisation in cubic stabilised \zirconia\ when bombarded with Cs$^{+}$ ions up to a fluence of $1 \times 10^{21}$ ions m$^{-2}$ \cite{amorphization2000wang}. However, no amorphisation is seen at an Xe$^{2+}$ fluence of $2 \times 10^{21}$ ions m$^{-2}$, or an I$^{+}$ fluence of $5 \times 10^{19}$ ions m$^{-2}$ \cite{sickafus1999radiation}.

One material phenomenon exclusive to nuclear reactor environments is neutron activation. The high free neutron environment leads to neutron capture in various nuclei within the reactor, including those of the fuel assemblies, coolant and RPV. There are many possible (n, x) reactions that may occur in materials experiencing a neutron flux, but of particular concern is transmutation of nuclei following a nuclear capture event. When a stable nucleus captures a neutron and becomes unstable, the nucleus may then emit particles to reduce its free energy, altering its atomic number in the process. This new element will have different chemical properties compared to the parent nucleus by virtue of a different electronic structure. This will change the elemental composition of a material, typically in an unfavourable way with dopants that negatively affect some desired material property. The extremely large number of nuclei relative to neutron flux means that this effect is small, though over time this becomes more significant due to the accumulation of these dopant elements.

In high-radiation environments, it is also possible for some molecules to be split by gamma photons above certain energies through the process of radiolysis. Corrosive fission products such as iodine will be present inside the fuel pin, but may exist in the form of (for example) CsI, which is not highly corrosive. Radiolysis however, decomposes CsI into Cs and I$_{2}$ vapour which can diffuse towards the cladding and promote cracking \cite{Konashi1983}.

\subsection{Fission products, their distribution and decay chains} \label{fissionyieldsection}

Nuclei which can undergo fission will produce daughter nuclei (fission products) with specific characteristics. At first, these nuclei will almost always be neutron-rich, as compared to their stable isotopes. This is the result of the higher neutron to proton (N/Z) ratios of larger nuclei. Figure \ref{figure:NZcurve} shows how the nuclei of abundant isotopes start with N/Z ratios of around 1 for small elements (e.g. \ch{He_{2}^{4}}, \ch{C_{6}^{12}}, \ch{O_{8}^{16}}), whereas larger elements will have isotopes with N/Z ratios approaching 1.6 (e.g. \ch{Pb_{82}^{208}}, \ch{Th_{90}^{232}}, \ch{U_{92}^{238}}).

\begin{figure}[ht]
\centering
\includegraphics[height=13cm]{images/Isotopes_and_half-life.png}
\caption[Plot of neutron number against proton number for nuclei with half-lives greater than 10${^{-8}}$ s.]{Plot of neutron number against proton number for nuclei with half-lives greater than 10${^{-8}}$ s. Taken from \cite{BenRG}.}
\label{figure:NZcurve}
\end{figure} 

Fissionable nuclei are typically very large and when fission occurs, the daughter nuclei will inherit a high N/Z ratio. These neutron-rich nuclei will generally decay by $\beta-$ particle emission to reduce their N/Z ratio and increase stability. One decay event is usually not enough to achieve a stable nucleus, so several decays as part of a decay chain are expected. An example is given for Te in Equations \ref{eqn:te_decay} and \ref{eqn:i_decay}.
\begin{gather}
\ch{Te^{134}_{52}} \xrightarrow[]{\beta-} \ch{I^{134}_{53}}+ e^{-} + \overline{\nu}_{e}
\label{eqn:te_decay} \\
\ch{I^{134}_{53}} \xrightarrow[]{\beta-} \ch{Xe^{134}_{54}} + e^{-} + \overline{\nu}_{e}
\label{eqn:i_decay}
\end{gather}
Another characteristic feature of fission products is that their masses are bi-modally distributed. Figure \ref{figure:fissionyield} shows calculated fission yields as a function of mass number, indicating a 40:60 rather than 50:50 mass distribution among daughter nuclei when heavy nuclei are fissioned. This is attributed to smaller nuclei (which have high binding energies per nucleon) separating from the nucleus first at the moment when fission occurs. This results in the majority of fission products centring around atomic masses of 95 and 135, producing a disproportionate number of nuclei such as Sr, Y and Zr on the low side and Te, I and Cs on the high side of the distribution. The distribution of fission products varies slightly depending on which nucleus is fissioned (e.g. U$^{233}$, U$^{235}$, Pu$^{239}$) and the energy of the fissioning neutron. Over time, transmutation of U$^{238}$ to Pu$^{239}$ and consumption of U$^{235}$ means that an increasing proportion of energy output will be due to fission of Pu$^{239}$, but fuel assemblies are typically retired before significant build-up of Pu$^{239}$ inventory and so the net yields of fission products will only change slightly over the life of the fuel.
% Iodine 0.1142684237, Tellurium 0.180582

\begin{figure}[ht!]
\centering
\includegraphics[height=12cm]{images/fissionyield.jpg}
\caption[Plot of the percentage yield of nuclei with a given mass following a fission event. Range of masses corresponding to isotopes of iodine shown in purple and isotopes of yttrium shown in green, based on Equation \ref{eqn:fission}.]{Plot of the percentage yield of nuclei with a given mass following a fission event. Range of masses corresponding to isotopes of iodine shown in purple and isotopes of yttrium shown in green, based on Equation \ref{eqn:fission}. Adapted from \cite{England1992}.}
% M. F. James, R. W. Mills and D. R. Weaver (1991) UKAEA Reports, AEA-TRS-1015, AEA-TRS-1018 
         %  and AEA-TRS-1019.
\label{figure:fissionyield}
\end{figure}

Iodine is an important fission product because it is known to corrode Zr metal. It is part of the Te $\rightarrow$ Cs decay chain, with most isotopes exhibiting half-lives ranging from a few seconds to several days. Select data on Te and I fission yields are presented in Table \ref{table:decaydata_chap1}. In total, the independent yield of I isotopes from U$^{235}$ fission is 10.4\%, while the independent yield of Te isotopes (I precursors) is 17.6\%, with \ch{Te_{52}^{134}} having the highest independent yield of all possible fission products (6.3\%). These particular elements will also usually be paired with Zr and Y fission products, the latter of which is a common phase stabiliser dopant in \zirconia .

\begin{table}[ht!] % Yields and half lives
\onehalfspacing
\caption{Independent fission product yields and half-lives for the major iodine isotopes and precursors in a thermal neutron reactor. Yields taken from the Joint Evaluated Fission and Fusion File (JEFF 3.3). All isotopes undergo single $\beta$- decay. Metastable states are included.}  \label{table:decaydata_chap1}
\begin{center}
\begin{tabular}{c c c}
\hline
Isotope & Independent Yield (\%) & Half-life \\
\hline
\ch{Te^{131}} & 0.126
 & 25.0 m \cite{auble1976nuclear} \\
\ch{I^{131}} & 0.00296
 & 8.02 d \cite{I131halflife} \\
\ch{Te^{132}} & 1.50 & 3.18 d \cite{Te132} \\
\ch{I^{132}} & 0.0284  & 2.30 h \cite{Te132} \\
\ch{Te^{133}} & 3.94 & 12.5 m \cite{khazov2011nuclear} \\
\ch{I^{133}} & 0.198  & 20.9 h \cite{I133} \\
\ch{Te^{134}} & 6.30 & 41.8 m \cite{Sonzogni2004} \\
\ch{I^{134}} & 0.767 & 52.5 m \cite{Sonzogni2004} \\
\ch{Te^{135}} & 3.48 & 19.0 s \cite{tellurium135halflife} \\
\ch{I^{135}} & 2.46 & 6.58 h \cite{tellurium135halflife} \\ 
\ch{Te^{136}} & 1.67 & 17.6  s \cite{Mccutchan2018} \\
\ch{I^{136}} & 3.04 & 83.4 s \cite{Mccutchan2018} \\
\ch{Te^{137}} & 0.484 & 2.49 s \cite{browne2007nuclear} \\
\ch{I^{137}} & 2.70 & 24.5 s \cite{browne2007nuclear} \\ 
\ch{Te^{138}} & 0.113 & 1.40 s \cite{chen2017nuclear} \\ 
\ch{I^{138}} & 1.24 & 6.26 s \cite{chen2017nuclear} \\ 
\hline
\end{tabular}
\end{center}
\end{table}

\subsection{Formation of plutonium and minor actinides}

Uranium fuel in LWRs is enriched to contain up to 5\% \ch{U^{235}}, which is a fissile isotope. The rest of the uranium is comprised of the more abundant \ch{U^{238}} which is a \emph{fertile} isotope. Fertile isotopes are not\footnote{\ch{U^{238}} has a thermal neutron fission cross section which is 7 orders of magnitude smaller than that of \ch{U^{235}}, so thermal fission of \ch{U^{238}} is insignificant.} fissioned directly, but can be converted into fissile isotopes through nuclear reactions. In the case of \ch{U^{238}}, the fissile isotope \ch{Pu^{239}} is produced through the following reactions:
\begin{gather}
\ch{U^{238}_{92}} + n^{1}_{0} \xrightarrow[]{absorption} \ch{U^{239}_{92}} \\
\ch{U^{239}_{92}} \xrightarrow[]{\beta-} \ch{Np^{239}_{93}} + e^{-} + \overline{\nu}_{e} \\
\ch{Np^{239}_{93}} \xrightarrow[]{\beta-} \ch{Pu^{239}_{94}} + e^{-} + \overline{\nu}_{e}
\end{gather}
UO$_{2}$ fuel pellets do not contain any plutonium before irradiation, however, production of \ch{Pu^{239}} and heavier isotopes during reactor operation is so significant that they can account for over 50\% of the fissile isotope inventory in a LWR \cite{OECD1989}. A fraction of the \ch{Pu^{239}} nuclei will capture additional neutrons, producing heavier isotopes of plutonium which may be fissile (e.g. \ch{Pu^{241}}). The production of fissile nuclei from fertile nuclei is known as \emph{breeding}.

Plutonium nuclei which are not burned through fission can either be extracted from spent fuel by reprocessing, or will undergo nuclear decay (specifically $\alpha$ or $\beta$- decay) and produce elements such as neptunium, americium and curium which are known as \emph{minor actinides}\footnote{In the nuclear power industry, uranium and plutonium are known as the major actinides.}. These minor actinides are produced from neutron irradiation of uranium (both \ch{U^{238}} and \ch{U^{235}}) and nuclear decay. The radiotoxicity of spent nuclear fuel is dominated by plutonium and the minor actinides, with global production rates of isotopes like \ch{Np^{237}}, \ch{Am^{241}} and \ch{Cm^{244}} measured in several metric tons per year \cite{Ewing2004}. 

In a nuclear fuel pin, the cladding serves as a physical barrier preventing the release of fission products and actinide wastes. The first barrier is the UO$_{2}$ matrix itself, however, volatile species can migrate to the pellet-cladding gap. This gap gradually closes during operation and the fuel pellet makes contact with the cladding, leading to mechanical and chemical interactions. This is discussed in detail in Chapter \ref{literature_review}.
\chapter{Crystallography and Point Defects}

\label{ch:crystallography}

\section{\zirconia\ phases and stabilisation}

\zirconia\ is unusual in exhibiting three commonly reported polytypes in its binary phase diagram (Figure \ref{figure:binary_phase_diagram}). Each will now be described and contrasted.

\begin{figure}[htp]
\centering
\includegraphics[height=9cm]{images/zro2_binary_phase.png}
\caption{Binary phase diagram of the Zr and O$_{2}$ system. Taken from \cite{Abriata1986TheSystem}.}
\label{figure:binary_phase_diagram}
\end{figure}

\subsection{Monoclinic}

A unit cell of monoclinic \zirconia\ is illustrated in Figure \ref{figure:coordination}. The dashed line (approximately 3.7\r{A} in length) shows the Zr-O bond which is broken when transitioning to monoclinic from the tetragonal phase.

\begin{figure}[htp] % Mono coordination figure
\centering
\includegraphics[height=6.2cm]{images/coordination.png}
\caption{A monoclinic zirconia unit cell indicating the two different oxygen bond coordinations. Small spheres represent oxygen ions while large spheres represent zirconium ions. Taken from \cite{Xia2010}.
\label{figure:coordination}}
\end{figure}

\begin{figure}[htp] % Mono Zr centre
\centering
\includegraphics[height=6cm]{images/zr_centre_mono.png}
\caption{Zirconium centre in monoclinic \zirconia\ showing nearest oxygen atoms and their respective bond co-ordinations. Zirconium atoms are shown in green and oxygen atoms in red.}
\label{figure:monoschottky}
\end{figure}


\begin{table}[htp]
\centering
\onehalfspacing
\caption{\zirconia\ crystal structures and their stable temperatures at 1 atm \cite{Howard1988}.}
\label{table:phases}
\begin{tabular}{ccc}
\hline
{Crystal Structure} & {Space Group}    & {Temperature Range (K)} \\ \hline
\multicolumn{1}{c}{Monoclinic} & \multicolumn{1}{c}{$P2_1/c$} & \multicolumn{1}{c}{$T$ \textless\ 1440}     \\
\multicolumn{1}{c}{Tetragonal} & \multicolumn{1}{c}{$P4_2/nmc$} & \multicolumn{1}{c}{1440 \textless\ $T$ \textless\ 2640}        \\
\multicolumn{1}{c}{Cubic} & \multicolumn{1}{c}{$Fm\overline{3}m$}     & \multicolumn{1}{c}{2640 \textless\ $T$ \textless\ 2950}      \\ \hline
\end{tabular}
\end{table}

\subsection{Tetragonal}

\subsection{Cubic}

\subsection{Other phases}

\subsubsection{Cotunnite}

Two orthorhombic phases of \zirconia\ have also been observed at high pressures.

\begin{figure}[htp]
  \centering
      \includegraphics[height=9cm]{images/cotunnite_structure.png}
  \caption{Illustration of the OII cotunnite crystal structure of \zirconia . Zirconium and oxygen ions are shaded dark and light respectively. Taken from \cite{Haines1997CharacterizationHafnia}.}
  \label{fig:cotunnite_structure}
\end{figure}

\subsubsection*{Volume expansion}

The phase transitions in \zirconia\ are accompanied by a change in volume, where the monoclinic phase is the least dense and the cubic phase is the most dense (see Figure \ref{figure:zrobonddistance}). This is especially significant in the case of the martensitic t-\zirconia\ to m-\zirconia\ transition, where the volume increases by around 9\% \cite{Gupta1977}. This has substantial implications for the creation and opening of cracks as \zirconia\ is a ceramic material with low toughness. This is especially relevant in a reactor scenario where temperature cycling (shutdown/startup or load-following behaviour) may lead to fatigue if the phase transition threshold is passed.

Another consequence of this large volume expansion is that a significant hysteresis effect is observed in the monoclinic/tetragonal phase transition, as shown in Figure \ref{fig:phasediagram}. 
%as the resulting coherency strain is likely to result in reduced mobility of fission products that have been embedded in the bulk crystal. 

\begin{figure}[htp]
  \centering
      \includegraphics[height=10cm]{images/zirconiaphasediagram.png}
  \caption{Pressure-temperature phase diagram for \zirconia . Dash-dotted lines represent more recent data. Diamonds mark transition points during an increase in pressure/temperature, while open circles are used for a decrease in pressure/temperature. Solid circles represent transition points for a fresh, single crystal sample. Taken from \cite{gando2011partial}. \label{fig:phasediagram}}
\end{figure}

\begin{figure}
\begin{center}
\begin{tikzpicture}
	\begin{axis}
		[width=12cm, xlabel={Nearest neighbour Zr-O bond distance (\r{A})}, ylabel={Relative occurrence}, ymin=0, ymax=140, xmin=2.0, xmax=2.50, legend style={{draw=}, at={(0.95,0.95)}, anchor=north east, legend columns=1}]
		\addplot[no marks] table [x=zr_o_dist, y=monoclinic,]{dat/zr_o_bond_distances.dat}; \addlegendentry{Monoclinic};
        \addplot[no marks, dashed] table [x=zr_o_dist, y=tetragonal, ]{dat/zr_o_bond_distances.dat}; \addlegendentry{Tetragonal};
        \addplot[no marks, densely dotted] table [x=zr_o_dist, y=cubic,]{dat/zr_o_bond_distances.dat}; \addlegendentry{Cubic};
			\end{axis}
		\end{tikzpicture}
		\caption{Density plot of the nearest neighbour Zr-O bond distances in \zirconia\ for each crystal structure. Specific volumes from DFT simulations are 11.99 \r{A}$^{3}$ion$^{-1}$, 11.51 \r{A}$^{3}$ion$^{-1}$, and 11.13 \r{A}$^{3}$ion$^{-1}$ for monoclinic, tetragonal, and cubic phases respectively.}
		\label{figure:zrobonddistance}
	\end{center}
\end{figure}


\subsection{Pressure stabilisation (isochoric + autostabilisation)}

The tetragonal and cubic phases of \zirconia\ are stabilised at high pressure. Since the oxide has a larger volume than the underlying metal (pilling-bedworth ratio of 1.5X), the growth of the oxide will itself impose stresses which may stabilise the tetragonal phase.

\subsection{Dopant stabilisation (lower valence cations)}

Some dopants will also stabilise the tetragonal and cubic phases of \zirconia. The most technologically significant of which is yttrium, which at concentrations of 15\% (atomic), fully stabilises the cubic phase. Zirconia stabilised this way is known as yttria-stabilised zirconia (YSZ). The way this works is by trivalent yttrium promoting the inclusion of charge compensating oxygen vacancy defects (see equation XXX). This works in a similar way with several other cation dopants such as trivalent scandium and divalent magnesium.

\section{Point Defects}

\subsection{Kr\"{o}ger-Vink notation}

Kr\"{o}ger-Vink notation \cite{kroger1956relations} is used throughout this thesis to describe defects. It is widely used in physical chemistry and is a useful shorthand for describing chemical reactions where conservation of mass, charge and lattice sites is required. The notation syntax is of the form \ch{x^{y}_{z}}, where x is the substituted atom or missing atom (i.e. a vacancy V), y is the charge of the defect (relative to the lattice species that originally occupied the site) and z is the site the defect occupies. Positive and negative charges are indicated with dots (\ch{^{*}}) and dashes (\ch{^{'}}) respectively, otherwise a cross (\ch{^{x}}) is used to denote a neutral defect. The site may be either a lattice site (such as Zr or O in \zirconia ) or an interstitial site ($i$). Table \ref{table:krogervink} shows examples of several different types of defects and their respective Kr\"{o}ger-Vink notation.

\begin{table}[htp] % Kroger-Vink notation table
\onehalfspacing
\centering
\caption{Examples of Kr\"{o}ger-Vink notation for several defects in \zirconia .}
\label{table:krogervink}
\begin{tabular}{cc}
\hline
Defect & Kr\"{o}ger-Vink Notation \\ \hline
Anion vacancy & \ch{V_{O}^{**}} \\
Cation vacancy & \ch{V_{Zr}^{''''}} \\
Anion interstitial & \ch{O_{i}^{''}} \\
Cation interstitial & \ch{Zr_{i}^{****}} \\
Iodine (I$^{-}$ anion) on oxygen site & \ch{I_{O}^{*}} \\ \hline
\end{tabular}
\end{table}


\chapter{Computational Methodology}

\label{ch:compmethodology}

\section{Density functional theory} \label{section:dft}

Quantum mechanics is currently the most complete modern theory which describes the behaviour of matter at the length scale of atoms. It can be used to predict things such as the energy levels of atoms, the interactions of light with matter, and the thermodynamic stability of systems of atoms. Ideally, the mathematical formalisms of quantum mechanics would be used to predict the properties and behaviour of all possible types of molecules and materials. In reality, this is very difficult to achieve, requiring several approximations and abstractions in order to produce methods which sacrifice some degree of physical accuracy in order to be computationally tractable. Currently, the most successful approach to predict the behaviour of most solids is provided by density functional theory.
% The mathematical formalisms of quantum mechanics must themselves be discretised to be used in computational simulation methods.

\subsection{The Schr\"{o}dinger equation}

The time-independent Schr\"{o}dinger equation is used to find the total energy of a system:
\begin{equation}
E\Psi(\textbf{r}) = \hat{H}\Psi(\textbf{r})
\label{equation:schrodinger}
\end{equation}

where $E$ is the total energy of the system, $\Psi$ is the wave function associated with the electrons, and $\hat{H}$ is the energy Hamiltonian operator. $\hat{H}$ includes the kinetic energy contributions ($\hat{T}$) and potential energy contributions ($\hat{V}$), shown in atomic units in equations \ref{equation:kineticcontribution} and \ref{equation:potentialcontribution} respectively:
\begin{gather}
\hat{H} = \hat{T} + \hat{V} \label{equation:hamiltonian}\\
\hat{T} = -\sum_i{\frac{1}{2}}\nabla^2_{r_i} - \sum_i{\frac{1}{2M_i}}\nabla^2_{R_i} \label{equation:kineticcontribution} \\
\hat{V} = \sum_{i,j=i+1}{\frac{1}{2|r_i - r_j|}} + \sum_{i,j=i+1}{\frac{Z_i Z_j}{2|R_i - R_j|}} - \sum_{i,j}{\frac{Z_i}{2|R_i - r_j|}} \label{equation:potentialcontribution}
\end{gather}

where $r_{i}$ is the position of electron $i$, $R_{i}$ is the position of nucleus $i$ and $M_{i}$ is the mass of nucleus $i$. Thus, the second term on the right of equation \ref{equation:kineticcontribution} relates to the kinetic energy of any associated nuclei, and the first term to electrons. 

If $\Psi(\textbf{r})$ is the wave function, the electron density at position \textbf{r} ($\rho(\textbf{r})$) is given by:
\begin{equation}
\rho(\textbf{r}) = \Psi(\textbf{r})^2
\end{equation}

\subsection{Kohn-Sham Method} \label{section:kohnsham}

DFT was developed by Kohn and Sham in 1964 \cite{Kohn1965} as an ab initio method for predicting $\rho(\textbf{r})$ associated with an ensemble of atoms. The Kohn-Sham Hamiltonian (Equation \ref{equation:kohnsham}) is used in the Schr\"odinger equation:
\begin{equation}
\hat{H}(\rho(\textbf{r})) = E_{KE}(\rho(\textbf{r})) + E_{P}(\rho(\textbf{r})) + E_{XC}(\rho(\textbf{r}))
\label{equation:kohnsham}
\end{equation}

where $E_{KE}$ and $E_{P}$ are the kinetic and potential energy functionals (functions of functions), $E_{XC}$ is the exchange correlation functional, and \textbf{r} is the position vector. The main approximation is to consider that the electrons only interact with nuclei and the average field generated by all other electrons, and not other electrons explicitly, thus allowing all the terms to be evaluated using the electron density rather than position. An exchange correlation term is then used to include the non-classical electron-electron interactions, namely electron exchange and correlation. Additionally, the exchange correlation term includes the difference in kinetic energy due to the use of non-interacting electrons. While Kohn and Sham proposed an exchange-correlation functional in the Hamiltonian, a general form of the functional has not yet been found. Several forms have been considered, each with strengths and weaknesses when applied to different systems. One basic form of the functional which is frequently used is the LDA \cite{Kohn1965}:
\begin{equation}
E_{LDA}(\rho(\textbf{r})) = \int\rho(\textbf{r})e_{uniform}(\rho(\textbf{r}))dr
\label{equation:LDA}
\end{equation}

where $e_{uniform}$ is the normalised exchange-correlation energy of a uniform electron gas (an idealised system). This exchange-correlation functional generates accurate results in materials such as metals where the electron density is relatively uniform, while systems with more rapidly changing electron densities (e.g. highly ionic materials) require more complex functionals. A natural extension of the LDA is to also take into account the gradient of the electron density, thus allowing a smoother functional fit when electron density is highly variable as a function of position. Such functionals are collectively referred to as GGAs \cite{Langreth1980, Langreth1983, Becke1988, perdew2008restoring}. One GGA which has enjoyed widespread use for many different types of systems is the Perdew-Burke-Ernzerhof (PBE) GGA \cite{Perdew1996}. The accuracy of this functional when modelling solid phase systems is well-established, and its frequent use in DFT studies provides ample reference material for comparing results. After conducting several convergence tests (see § \ref{section:convergence}), the PBE GGA was chosen as the exchange-correlation functional to be used for all calculations in this thesis.

\subsubsection{Born-Oppenheimer approximation}

The Born-Oppenheimer approximation is a two-step process for evaluating atomic forces which greatly reduces the computational costs of atomistic simulations. It exploits the large difference in mass between nuclei and electrons in order to separate their interactions. This allows us to decompose the total wave function into a product of an electronic wave function and a nuclear wave function via a separation of variables approach. The first step involves ignoring the kinetic energy contribution of nuclei by assuming they are stationary, thus the nuclear kinetic energy term in Equation \ref{equation:kineticcontribution} can be removed. The stationary nuclei assumption also simplifies the nuclear-nuclear Coulombic repulsion term in Equation \ref{equation:potentialcontribution} because $|R_i - R_j|$ becomes a constant throughout the calculation. An electronic Schr\"{o}dinger equation is then solved where electronic positions are variables and nuclear positions are fixed parameters. This solution contains information of the shape of the electronic orbitals. The next step is to take the electronic distribution and calculate the resultant forces on the nuclei. The nuclear positions are then modified to minimise these forces, followed by feeding these nuclear positions back into the electronic Schr\"{o}dinger equation to obtain the new electronic distribution. This process is repeated until the required convergence criterion (such as energy change per iteration and forces on nuclei) are satisfied.

\subsection{Pseudopotentials}

The electron-electron interaction component of the potential energy presents a problem when it comes to scaling experimental models. The number of terms in this interaction grows quadratically with the number of electrons in the system, quickly becoming computationally intractable for even small systems. However, it is known that in chemical reactions, the majority of chemical behaviour is determined by relatively few valence electrons, while the more numerous core electrons have a far smaller effect. 

Consider the zirconium atom with 40 electrons, of which 4 (4$d^2$5$s^2$) are typically involved in bonding and chemical reactions. By considering only these valence electrons for Coulombic-term calculations, we reduce the system size by 90\%, which provides a more than tenfold reduction in computational requirements.

Although the core electrons do not participate in chemical reactions, they still influence the properties of the atom, such as the atomic radius. Instead of modelling the core electrons explicitly, we can approximate their aggregate effect with a potential energy function. This is what we aim to achieve by using the pseudopotential method. An example indicative pseudopotential is shown in figure \ref{figure:pseudopotential}.

\begin{figure}[ht] % Pseudopotential Image
\begin{center}
\includegraphics[height=10cm]{images/pseudopotential.png}
\end{center}
\caption[Sketch of an all-electron potential V$_{AE}$ and a pseudopotential V$_{PS}$ with their corresponding wave functions. r$_{cut}$ indicates the radius beyond which both the potentials and their wave functions are the same.]{Sketch of an all-electron potential V$_{AE}$ and a pseudopotential V$_{PS}$ with their corresponding wave functions. r$_{cut}$ indicates the radius beyond which both the potentials and their wave functions are the same. Adapted from \cite{Payne1992}.}
\label{figure:pseudopotential}
\end{figure}

Figures \ref{figure:o_pp} and \ref{figure:zr_pp} show the actual pseudopotentials used throughout this work for oxygen and zirconium respectively. The potentials are shown broken down by the electronic sub-shells occupied by the valence electrons. The pseudopotentials are shown in order of increasing sub-shell energies, thus the 5s electron orbitals are filled before the 4d orbitals in zirconium. Two lines for the all-electron wavefunction are shown, corresponding to the different electron angular momenta.

\begin{figure}[ht] % Oxygen pseudopotentials
\begin{center}
\begin{tikzpicture}
	\begin{groupplot}[group style={group size=1 by 2}, width=14cm, height=7cm]
	\nextgroupplot[
    axis y line*=middle, axis x line*=bottom, y label style={at={(-0.03,-0.1)}}, ylabel=Wavefunction $\Psi$, axis y line shift=0.5cm,
    xtick=\empty, ymin=-1.1, ymax=1.1, ytick={-1, -0.5, 0, 0.5, 1}, xmax=2.5, 
    x axis line style={white}]
         \addplot[no marks, dashed, draw=red!80!white] table [x=distance, y=O_AE_s1,]{dat/o_pp.dat}; 
		\addplot[no marks, draw=red!80!white] table [x=distance, y=O_pp_s1,]{dat/o_pp.dat}; 
		\addplot[no marks, dashed, draw=blue!80!white] table [x=distance, y=O_AE_s2,]{dat/o_pp.dat}; 
		\addplot[no marks, draw=blue!80!white] table [x=distance, y=O_pp_s2,]{dat/o_pp.dat};
		\node at (0.1, 0.99) {\textbf{O (2s)}};
		\draw[black] % horizontal line
				(-0.11, 0)
				-- % = line-to
				++ % = calculate a vector sum
				(axis direction cs:2.61, 0);
		\draw[black, dotted] % Vertical line
				(1.299,1)
				-- % = line-to
				++ % = calculate a vector sum
				(axis direction cs:0,-2);
				
    \nextgroupplot[
    axis y line*=middle, axis x line*=bottom, axis x line shift=0, x axis line style={white}, axis y line shift=0.5cm,
    ymin=-1.1, ymax=1.1, ytick={-1, -0.5, 0, 0.5, 1}, xtick={0, 0.5, 1, 1.5, 2, 2.5}, xmax=2.5, xlabel=Distance from nucleus (Bohr)]
         \addplot[no marks, dashed, draw=red!80!white] table [x=distance, y=O_AE_p1,]{dat/o_pp.dat}; 
		\addplot[no marks, draw=red!80!white] table [x=distance, y=O_pp_p1,]{dat/o_pp.dat}; 
		\addplot[no marks, dashed, draw=blue!80!white] table [x=distance, y=O_AE_p2,]{dat/o_pp.dat}; 
		\addplot[no marks, draw=blue!80!white] table [x=distance, y=O_pp_p2,]{dat/o_pp.dat}; 
		\node at (0.1, 0.99) {\textbf{O (2p)}};
		\draw[black] % horizontal line
				(-0.11, 0)
				-- % = line-to
				++ % = calculate a vector sum
				(axis direction cs:2.61, 0);
		\draw[black, dotted] % vertical line
				(1.299,1)
				-- % = line-to
				++ % = calculate a vector sum
				(axis direction cs:0,-2);
	\end{groupplot}
		\end{tikzpicture}
		\caption{Plots of the valence $s$ and $p$ orbital potentials for oxygen with two projectors per angular momentum. Dashed lines indicate the all-electron potentials while solid lines indicate the corresponding pseudopotential. Dotted vertical line marks the radius beyond which the potentials match.}
		\label{figure:o_pp}
	\end{center}
\end{figure}

\clearpage

\begin{figure}[h!] % Zirconium pseudopotentials
\begin{center}
\begin{tikzpicture}
	\begin{groupplot}[group style={group size=1 by 3}, width=14cm, height=7cm]
	\nextgroupplot[ % 4p
    axis y line*=middle, axis x line*=bottom, axis y line shift=0.5cm,
    xtick=\empty, ymin=-1.2, ymax=1.2, ytick={-1, -0.5, 0, 0.5, 1}, xmax=2.5, 
    x axis line style={white}]
         \addplot[no marks, dashed, draw=red!80!white] table [x=distance, y=Zr_AE_p1,]{dat/zr_pp.dat}; 
		\addplot[no marks, draw=red!80!white] table [x=distance, y=Zr_pp_p1,]{dat/zr_pp.dat}; 
		\addplot[no marks, dashed, draw=blue!80!white] table [x=distance, y=Zr_AE_p2,]{dat/zr_pp.dat}; 
		\addplot[no marks, draw=blue!80!white] table [x=distance, y=Zr_pp_p2,]{dat/zr_pp.dat};
		\node at (0.1, 0.99) {\textbf{Zr (4p)}};
		\draw[black] % horizontal line
				(-0.11, 0)
				-- % = line-to
				++ % = calculate a vector sum
				(axis direction cs:2.61, 0);
		\draw[black, dotted] % Vertical line
				(2.105,1)
				-- % = line-to
				++ % = calculate a vector sum
				(axis direction cs:0,-2);
				
	\nextgroupplot[ % 5s
    axis y line*=middle, axis x line*=bottom, axis y line shift=0.5cm, ylabel=Wavefunction $\Psi$,
    xtick=\empty, ymin=-1.2, ymax=1.2, ytick={-1, -0.5, 0, 0.5, 1}, xmax=2.5, 
    x axis line style={white}]
         \addplot[no marks, dashed, draw=red!80!white] table [x=distance, y=Zr_AE_s1,]{dat/zr_pp.dat}; 
		\addplot[no marks, draw=red!80!white] table [x=distance, y=Zr_pp_s1,]{dat/zr_pp.dat}; 
		\addplot[no marks, dashed, draw=blue!80!white] table [x=distance, y=Zr_AE_s2,]{dat/zr_pp.dat}; 
		\addplot[no marks, draw=blue!80!white] table [x=distance, y=Zr_pp_s2,]{dat/zr_pp.dat};
		\node at (0.1, 0.99) {\textbf{Zr (5s)}};
		\draw[black] % horizontal line
				(-0.11, 0)
				-- % = line-to
				++ % = calculate a vector sum
				(axis direction cs:2.61, 0);
		\draw[black, dotted] % Vertical line
				(2.105,1)
				-- % = line-to
				++ % = calculate a vector sum
				(axis direction cs:0,-2);
				
    \nextgroupplot[ % 4d
    axis y line*=middle, axis x line*=bottom, axis x line shift=0, x axis line style={white}, axis y line shift=0.5cm,
    ymin=-1.2, ymax=1.2, ytick={-1, -0.5, 0, 0.5, 1}, xtick={0, 0.5, 1, 1.5, 2, 2.5}, xmax=2.5, xlabel=Distance from nucleus (Bohr)]
         \addplot[no marks, dashed, draw=red!80!white] table [x=distance, y=Zr_AE_d1,]{dat/zr_pp.dat}; 
		\addplot[no marks, draw=red!80!white] table [x=distance, y=Zr_pp_d1,]{dat/zr_pp.dat}; 
		\addplot[no marks, dashed, draw=blue!80!white] table [x=distance, y=Zr_AE_d2,]{dat/zr_pp.dat}; 
		\addplot[no marks, draw=blue!80!white] table [x=distance, y=Zr_pp_d2,]{dat/zr_pp.dat}; 
		\node at (0.1, 0.99) {\textbf{Zr (4d)}};
		\draw[black] % horizontal line
				(-0.11, 0)
				-- % = line-to
				++ % = calculate a vector sum
				(axis direction cs:2.61, 0);
		\draw[black, dotted] % vertical line
				(2.105,1)
				-- % = line-to
				++ % = calculate a vector sum
				(axis direction cs:0,-2);
	\end{groupplot}
		\end{tikzpicture}
		\caption{Plots of the valence $s$, $p$ and $d$ orbital potentials for zirconium with two projectors per angular momentum. Dashed lines indicate the all-electron potentials while solid lines indicate the corresponding pseudopotential. Dotted vertical line marks the radius beyond which the potentials match.}
		\label{figure:zr_pp}
	\end{center}
\end{figure}

%\begin{figure}[ht] % Oxygen pseudopotential
%\begin{center}
%\includegraphics[height=7.3cm]{images/oxygen_otf_pp.png}
%\end{center}
%\caption{Plots of the valence $s$ and $p$ orbital potentials for oxygen with two projectors per angular momentum. Dashed lines indicate the all-electron potentials while solid lines indicate the corresponding pseudopotential. Dotted vertical line marks the radius beyond which the potentials match.}
%\label{figure:oxygen_pseudopotential}
%\end{figure}

%\begin{figure}[ht] % Zirconium pseudopotential
%\begin{center}
%\includegraphics[height=12cm]{images/zirconium_otf_pp.png}
%\end{center}
%\caption{Plots of the valence $s$, $p$ and $d$ orbital potentials for zirconium with two projectors per angular momentum. Dashed lines show the all-electron potentials while solid lines indicate the corresponding pseudopotential. Dotted vertical line marks the radius beyond which the potentials match.}
%\label{figure:zirconium_pseudopotential}
%\end{figure}


\section{Periodic boundaries}

\subsection{Bloch's theorem}

The repeating nature of a crystal structure, defined by the lattice vectors plus a basis set of atoms that are repeated, is well-suited for computer models. It allows us to define periodicity in three dimensions for a given unit cell. An example of this periodicity is illustrated in Figure \ref{figure:periodicboundary} in two dimensions. A model based on this periodicity is justified as follows:

\begin{itemize}
\item Nuclei are arranged in a periodically repeating pattern, thus their potentials acting on electrons are also periodic.
\item If the potential is periodic, it follows that the electron density is also periodic.
\item The electron density is equivalent to the square of the wave function magnitude, thus the magnitude of the wave function is also periodic.
\end{itemize}

\begin{figure}[ht] % Periodic boundary image
\begin{center}
\includegraphics[width=\linewidth]{images/PeriodicBoundaryThesis.png}
\end{center}
\caption{Two dimensional illustration of periodic boundary around a primitive cell.}
\label{figure:periodicboundary}
\end{figure}

Knowing that the magnitude of the wave function is periodic greatly simplifies the calculation process; only one `period' of the function needs to be evaluated. However, the phase of the wave function can take any of an infinite number of values and still satisfy the periodicity condition. At this point, we consider Bloch's theorem which states that the possible wave functions are all quasi-periodic, and thus the wave function can be expressed as:  % Patrick's Fig 2.3 is really useful for describing this
\begin{equation}
\label{equation:bloch}
\psi_k(\textbf{r}) = e^{i\textbf{k}.\textbf{r}}u_k(\textbf{r})
\end{equation}

Where $\psi_k(\textbf{r})$ is the wave function evaluated at position \textbf{r}, $e^{i\textbf{k}.\textbf{r}}$ is an arbitrary phase factor, and $u_k(\textbf{r})$ is a periodic function with the same periodicity as the wave function. Solutions to this equation exist for any value of \textbf{k} and so the general solution can be expressed as an integral over the first Brillouin zone, the primitive lattice cell in reciprocal space. Instead of evaluating the integral over the range of \textbf{k} (a computationally costly task as it is done for many wave functions), a sum of values at discrete points, known as \textbf{k}-points, is used. This approximation is valid because the wave function varies slowly over \textbf{k}, thus allowing the integral to be approximated with several appropriately spaced \textbf{k}-points. In general, a finer \textbf{k}-point grid results in increased accuracy, but at an increased computational cost \cite{Hasnip2010}. For all DFT calculations in this thesis, a Monkhorst-Pack sampling scheme \cite{Monkhorst1976} was used for Brillouin zone integration, with a minimum \textbf{k}-point separation of 0.09 \r{A}$^{-1}$.

\subsection{Plane-waves}

The electron density of a system is described in the context of a basis set. A basis set is a collection of functions (known as basis functions) which can be combined to produce some relevant output, typically the mathematical description for the shape of an electron orbital. For example, any sound wave can be generated from a combination of sine functions (basis functions). 

The purpose of a basis set in DFT calculations is to describe the varying amplitude of the electron density in space. Any complete basis set (e.g. plane-wave, correlation-consistent, split-valence) may be used to represent the behaviour of electron orbitals, but a plane-wave method was chosen due to its greater suitability for periodic systems (plane-waves are intrinsically periodic). Since the electron densities are represented by a finite sum of plane-waves with different energies, a truncation error will be incurred. Plane-waves of higher energies provide a smaller contribution to the overall density, so only plane-wave up to a chosen cut-off energy value are considered in order to reduce computational requirements. An appropriate plane-wave cut-off energy must therefore be determined through a convergence test. 

%Figure \ref{Figure:cutoffconvergence} shows the first convergence study where the total energy of simulations with various values of $E_{cutoff}$ were compared to a highly converged value, and then plotted on a log scale to see how precision is improved at larger values.


\section{Computational details}
\subsection{Cell dimensions and initialisation}

A supercell method is used for the study of various defects. The first step is to create a unit cell of \zirconia\ in each of the three crystal structures. Each unit cell is then fully relaxed through a geometry optimisation process (see § \ref{geometry_optimisation_method}). The resulting cell is used to construct supercells through tessellation in three dimensions, before being fully relaxed again. In this way, we generate systems with up to ten times as many atoms as the unit cell (supercell details can be found in Table \ref{table:supercells}). This is necessary because introducing defects into a small unit cell will result in the defect interacting with itself across the periodic boundary. A supercell increases the distance between the defect and its periodic image, using the bulk material as an interaction buffer. 

When constructing a supercell, it is important to consider making the supercell equally large in all directions, such that any directional bias in defect-defect interaction is minimised. Larger supercells carry an increased computational cost when running calculations, limiting the sizes we can achieve. For example, a constant-volume defect calculation with 300 atom supercells will take upwards of 500 hours to complete (on the Imperial College HPC using four 32-core nodes), whereas the equivalent 100 atom supercell will take just 72 hours (fully relaxed calculations are even more computationally expensive).


\begin{table}[ht] % Supercell details
\doublespacing
\centering
\caption{Composition of the supercells in terms of the number of individual unit cells stacked in each direction.} % Unit cells were stacked in such a way as to produce the most cubic supercell in order to minimise directional defect-defect interactions.}
\vspace*{2mm}
\label{table:supercells}
\begin{tabular}{cccccccc}
\hline
\multirow{2}{*}{{\bf \begin{tabular}[c]{@{}c@{}}Crystal \\ Structure\end{tabular}}} & \multicolumn{3}{c}{{\bf No. unit cells}} & \multicolumn{3}{c}{{\bf Supercell size (\AA)}} & \multirow{2}{*}{{\bf \begin{tabular}[c]{@{}c@{}}No.\\ atoms\end{tabular}}} \\ \cline{2-7}
 & \hspace{0.25 cm} a \hspace{0.2 cm} & b & c & a \hspace{0.0 cm} & b & c \hspace{0.35 cm} &  \\ \hline
\begin{tabular}[c]{@{}c@{}}Monoclinic\\ ($P2_1/c$)\end{tabular} & 2 & 2 & 2 & 10.37 & 10.47 & 10.75 & 96 \\ \hline
\begin{tabular}[c]{@{}c@{}}Tetragonal\\ ($P4_2/nmc$)\end{tabular} & 3 & 3 & 2 & 10.85 & 10.85 & 10.56 & 108 \\ \hline
\begin{tabular}[c]{@{}c@{}}Cubic\\ ($Fm\overline{3}m$)\end{tabular} & 2 & 2 & 2 & 10.22 & 10.22 & 10.22 & 96 \\ \hline
\end{tabular}
\end{table}

\subsection{Geometry optimisation} \label{geometry_optimisation_method}

The geometry optimisation task in CASTEP follows a simple steepest-descent algorithm which attempts to satisfy certain convergence criteria, depending on the constraints applied to the system. This is an iterative process which takes an initial system state, modifies ion positions slightly and then calculates the difference in properties between the states to check for convergence. 

The variational principle in quantum mechanics tells us that the lowest system energy calculated is always an upper bound for the ground state energy, thus providing a way to check if modifications to the system are actually optimising the geometry. The exception is when the system converges upon a local minimum, which may not be an experimentally observed state. This can be avoided to some extent by having good initial ion placement from which to optimise.

\subsection{Convergence criteria for geometry optimisation} \label{convergence_criteria}

Four convergence criteria are used for the geometry optimisation tasks throughout this work, one of which is only used when performing constant-pressure calculations, such as when a supercell is being fully relaxed. These criteria are evaluated with respect to the previous iteration during the geometry optimisation task:

\begin{itemize}
\item \emph{Change in energy per ion}: The largest change in the energy per ion between iterations must be below $10^{-5}$ eV. Below this value, the total energy improvement towards the ground state for a 100 atom supercell is less than 0.001 eV, and is therefore considered converged.
\item \emph{Maximum force on an ion}: The maximum force requirement on any single ion in an iteration must be below $10^{-2}$ eV/\r{A}. This is required to make sure that the ion position will not change significantly in the following iteration, possibly bringing another convergence criterion above its threshold.
\item \emph{Maximum change in ion position}: This must be below $5 \times 10^{-4}$ \r{A} between iterations to be considered converged. This criterion specifies the maximum `rattle' of the ion that is tolerated once the minimum energy is reached (i.e. displacements above this value may still be important for achieving a correct atomic configuration). 
\item \emph{Maximum stress (constant-pressure only)}: During unconstrained relaxation, the maximum change in stress between iterations should be below 50 MPa. This is necessary to avoid large deviations which may distort the symmetry of the supercell, resulting in anomalous energy values.
\end{itemize}

Using these convergence criteria, non-defective supercells of \zirconia\ were relaxed under constant pressure. The resulting structure was used as the starting point to which defects were introduced, and subsequently relaxed again, this time under constant volume conditions to simulate low defect concentrations \cite{Murphy2014, Bell2015}. Finally, all DFT calculations on doped and defective structures in this thesis employed the Pulay method for density mixing \cite{Pulay1980} to take into consideration changes in electronic behaviour of the system caused by the defect and to speed up convergence.

\subsection{Charged cell correction} \label{charged_cell_correction}

When calculating the energy of a defect with an overall non-zero charge, this charge introduces a systematic error in the energy value which is a function of the charge magnitude. This is typically the case in high band-gap materials such as \zirconia\ where electron mobility is far lower than in metals or semiconductors, allowing defects such as \ch{V_{O}^{**}} and \ch{V_{Zr}^{''''}} to be thermodynamically stable in the lattice. 

The source of the error from charged defects is self-interaction across the periodic boundary, made necessary by the finite cell size. A common solution is to append a Makov-Payne correction term when calculating formation energies of defects \cite{Makov1995, Makov1996}. This works well in many cases, but does not take into consideration the anisotropy in the material's dielectric properties, as is the case in tetragonal \zirconia\ due to the non-unity lattice $c/a$ ratio. These effects are better captured when using a screened Madelung correction \cite{Murphy2013}. This method provides a more complete description of the dielectric properties by utilising a dielectric tensor rather than a single value of the dielectric constant (or relative permittivity). Dielectric tensors for the different phases of \zirconia\ were taken from the literature \cite{Zhao2002a, Zhao2002}. The screened Madelung correction is therefore used in preference to a Makov-Payne correction throughout this thesis.

\subsection{Helmholtz free energy} \label{helmholtz_method}

In order to examine the relationship between temperature and energy for the different \zirconia\ phases, phonon calculations were performed in CASTEP using a method outlined by Burr \emph{et al.} \cite{burr2015crystal,jackson2016resolving}. This entails using the harmonic approximation to determine the shape of the potential well that an atom sits in. The potential well is approximated by a spherically symmetric harmonic well, centred at an atom's equilibrium position in the lattice. At a temperature of 0 K, an atom will occupy the lowest region of its potential well, known as the ground state (though they will still have energy, known as zero-point energy). As temperature increases, the atom will sometimes occupy higher energy states in the potential well due to increased thermal vibrations, moving from its equilibrium position. The total energy ($A(T, V)$) of this system, known as the Helmholtz free energy, is calculated using the internal energy ($U(V)$), vibrational enthalpy ($H_{v}(T, V)$), vibrational entropy ($S_{v}(T, V)$) and configurational entropy ($S_{conf}$):
\begin{equation} \label{vibrational}
A(T, V) = U(V) + H_{v}(T, V) - TS_{v}(T, V) - TS_{conf} 
\end{equation}
where $T$ is temperature and $V$ is volume. The configurational entropy is calculated using Boltzmann statistics:
\begin{equation}
S_{conf} = k_{B}\ch{ln}(\Omega)
\end{equation}
where $k_{B}$ is the Boltzmann constant and $\Omega$ denotes the number of possible configurations (i.e. valid permutations of energy level occupancy). The vibrational terms in Equation \ref{vibrational} are obtained by performing a constant-volume phonon calculation in CASTEP and then integrating over the resulting phonon density of states (DOS). This is done over a range of temperatures for each crystal structure of \zirconia .

\subsection{Incorporation energies}

The inner oxide of the fuel cladding will be highly defective due to radiation damage, resulting in a high concentration of pre-existing intrinsic defect sites relative to the concentration of fission products. We therefore consider the energy of fission product incorporation on to these existing defect sites. The energies to incorporate atoms at interstitial and substitutional sites in \zirconia\ were calculated from the set of defective and perfect supercell DFT energies. For iodine, incorporation energies were established to place atoms into vacancy sites of different charge to generate defects from \ch{I_{O}^{x}} to \ch{I_{O}^{**}}, and \ch{I_{Zr}^{x}} to \ch{I_{Zr}^{''''}}. I was also incorporated onto the interstitial sites.

The incorporation energy equation for iodine uses $\frac{1}{2}$I$_{2}$ as the reference state of iodine, while Te, Xe and Cs use the DFT energy calculated as a single atom in a large cell:
\begin{equation}
\label{interstitial_incorp_equation}
E_{inc}(\ch{I_{$i$}^{x}}) = E_{DFT}(\ch{I_{$i$}^{x}}) - (E_{DFT}(ZrO_2) + \frac{1}{2}\mu_{I_{2}})  % - \frac{E_{I_2}}{2}
\end{equation}
where $E_{inc}(\ch{I_{$i$}^{x}})$ is the incorporation energy of a neutral iodine interstitial, $E_{DFT}(\ch{I_{$i$}^{x}})$ is the energy of a neutral iodine interstitial, $E_{DFT}(ZrO_2)$ is the energy of a non-defective \zirconia\ supercell and $\mu_{I_{2}}$ is the chemical potential of an I$_{2}$ molecule, taken from a single point DFT calculation of the I$_{2}$ dimer. For incorporation of a charged interstitial (e.g. $\ch{I_{i}^{*}}$), the energy required to add or remove an electron is included in the calculation:
\begin{equation}
\label{interstitial_incorp_equation_charged}
E_{inc}(\ch{I_{$i$}^{n}}) = E_{DFT}(\ch{I_{$i$}^{n}}) - (E_{DFT}(ZrO_2) + \frac{1}{2}\mu_{I_{2}} + n(E_{VBM} + \mu_{e}))
\end{equation}
Similarly, for a substitutional defect:
\begin{equation}
\label{o_sub_incorp_equation}
E_{inc}(\ch{I_{O}^{$n$}}) = E_{DFT}(\ch{I_{O}^{$n$}}) - (E_{DFT}(\ch{V_{O}^{$n$}}) + \frac{1}{2}\mu_{I_{2}})  % - \frac{E_{I_2}}{2}
\end{equation}
where $\ch{I_{O}^{$n$}}$ is an iodine substitutional defect at an oxygen site of charge $n$ and $\ch{V_{O}^{$n$}}$ is the corresponding oxygen vacancy.

\subsection{Stiffness matrix generation}

The elastic stiffness matrices for the pure monoclinic, tetragonal and cubic phases of \zirconia\ were calculated using CASTEP's \emph{elastic constants} task. The calculation of elastic constants is a multi-step process involving up to 36 individual DFT calculations. Several scripts have been made available to simplify this process (see Appendix \ref{castep_scripts}).

The first step is to generate multiple different \texttt{.cell} files, each with either a small deviation in the lattice parameter or an additional shear on the cell. This requires starting with a unit cell that has already been completely relaxed via the \emph{geometry optimisation} task. In total, 36 \texttt{.cell} files are generated, each corresponding to a single element of the eventual stiffness matrix. The next step is to run a single point DFT calculation on each \texttt{.cell} file with the \emph{calculate stress} parameter enabled to output the resulting stress matrix. The final step is to use Hooke's Law to calculate the elastic stiffness constants using the known stress and strain state. 

%\subsection{Strain method for defect volumes}
%
%The volumes of the defective supercells were kept constant because constant pressure calculations have been shown to sometimes break the symmetry of the supercell \cite{samanta2010thermodynamic}, leading to unreliable energy values. This is partly due to the assumed arrangement of the defects that may not be commensurate with the cell symmetry. This approach to calculating defect volumes then relies on calculating the elastic constants of the non-defective supercell, followed by extracting the resultant stress tensor from a defect simulation. The strain tensor of the defective cell can then be calculated using Hooke's law, giving the relaxation volume. 

\subsection{Defect relaxation volumes} \label{isobaricmethod}

Defect relaxation volumes of point defects were calculated using an isobaric method, requiring two calculations to be performed under constant-pressure using the geometry optimisation task in CASTEP. The defect relaxation volume ($\Delta$V) is defined as: 
\begin{equation}
\Delta V = V_{def} - V_{perf}
\end{equation}
where $V_{def}$ is the relaxation volume of the defective supercell and $V_{perf}$ is the relaxation volume of the non-defective (perfect) supercell. In this thesis, mentions of `volume' will refer to relaxation volumes unless stated otherwise.

After completing an energy calculation, CASTEP provides the volume of the resulting cell, defined as the space enclosed by the repeating unit of atoms within the calculated lattice parameters. By subtracting the volume of a non-defective cell from the volume of a defective cell, we obtain a value for the total defect volume. 

It is important to consider that if there is a non-zero charge on the system, this will affect the calculated volume. Two systems with the same type, amount and arrangement of atoms, but different overall charges, will have different energies (due to the number of electrons). Different electronic orbital occupancies will affect the inter-atomic forces and therefore the shape of the cell. In order to compensate for this effect, a `corrected' relaxation volume, as described by Goyal \emph{et al.} \cite{goyal2017conundrum} is calculated when the defect has a non-zero charge:
\begin{equation}
\Delta V = V_{def}^{q} - V_{perf}^{q}
\end{equation}
where $q$ is the defect charge. This formulation uses the volume of a non-defective supercell with equal charge magnitude as the reference structure. This method has been shown to yield more reasonable defect volumes than when using neutral non-defective supercells as the reference structure. Defect volumes without this correction applied are provided in Appendix \ref{uncorrected_volumes} for comparison.

\section{Defect energies and equilibria} 

\subsection{Defect formation energies}

Defect formation energies are calculated using equation \ref{equation:formation_energy}:
\begin{equation} \label{equation:formation_energy}
    E_{f} = E_{def} - (E_{perf} \pm \sum_{i} n_i\mu_i + q(E_{VBM} + \mu_{e})) + E_{corr}
\end{equation}
where $E_{f}$ is the formation energy, $E_{def}$ is the energy of the defective supercell, $E_{perf}$ is the energy of a non-defective supercell, $q$ is the defect charge, $E_{VBM}$ is the valence band maximum, $\mu_{e}$ is the Fermi level relative to the VBM and $E_{corr}$ is a charged-cell correction term (see § \ref{charged_cell_correction}). Since $\mu_{e}$ is not a fixed value, plots of formation energy against $\mu_{e}$ are produced to examine the behaviour of defects across the entire range of the band gap. These are reported in Figures \ref{figure:monovacancies}, \ref{figure:tetvacancies} and \ref{figure:cubicvacancies}.

\subsection{Defect equilibria} \label{brouwer_method} % 

Typically in materials, several types of defects will exist simultaneously. These defects will be present at an equilibrium concentration based on their thermodynamic stability. Predicting the defect equilibria is possible with statistical mechanics and some approximations. For example, it is expected that a crystal lattice will usually be overall charge-neutral (exceptions can be made under certain conditions, see § \ref{space_charge}), otherwise we would see a build-up of charge with a large Coulomb energy penalty which would be thermodynamically unsustainable.

Brouwer diagrams, also known as Kr{\"o}ger-Vink diagrams, were produced using a method outlined by Murphy \emph{et al}. \cite{Murphy2014, Murphy2014a} through which it is possible to determine defect concentrations as a function of oxygen partial pressure. We start from the statement that the chemical potential of \zirconia\ is equivalent to the sum of chemical potentials $\mu$ of its constituent species, Zr and O:
\begin{equation}
{\mu}_{ZrO_2(s)} = {\mu}_{Zr}(p_{O_2}, T) + {\mu}_{O_{2}}(p_{O_{2}}, T)
\label{mewZrO2compmethodology}
\end{equation}
where $T$ denotes temperature and $p_{O_2}$ denotes oxygen partial pressure. The chemical potential of \zirconia\ in the solid state is assumed to have negligible dependence on $T$ and $p_{O_2}$ relative to ${\mu}_{Zr}$ and ${\mu}_{O_2}$. Energies can be obtained for bulk \zirconia\ and Zr, but the ground state of oxygen is not correctly reproduced in DFT \cite{Batyrev2000,Lozovoi2001}. Instead, we use the approach of Finnis \emph{et al}. \cite{Finnis2005} to infer the oxygen chemical potential from standard state values. We can use the experimental Gibbs free energy to produce an equation where $\mu_{O_2}$ is the only unknown:
\begin{equation}
\Delta{G^{\plimsoll}_{f, ZrO_2}} = \mu_{ZrO_2(s)} - (\mu_{Zr(s)} + \mu^{\plimsoll}_{O_2})
\end{equation}
where $\Delta{G^{\plimsoll}_{f, ZrO_2}}$ is the experimental Gibbs energy at standard temperature and pressure and $\mu^{\plimsoll}_{O_2}$ is the oxygen chemical potential under the same conditions. Only monoclinic \zirconia\ is stable under standard conditions, with $\Delta{G^{\plimsoll}_{f, ZrO_2}}$ = -1042.746 kJ/mol (10.807 eV) \cite{brown2005chemical}. Values of the Gibbs free energy of formation for the tetragonal (10.697 eV) and cubic (10.595 eV) phases were obtained by adding the energy difference between the phases from DFT calculations. The values of $\mu_{ZrO_2(s)}$ and $\mu_{Zr(s)}$ are calculated using DFT. Once $\mu^{\plimsoll}_{O_2}$ is calculated, we can generalise the chemical potential of oxygen for any value of $T$ and $p_{O_2}$ by appending an ideal gas relationship $\Delta{\mu(T)}$ and a Boltzmann distribution:
\begin{gather}
\mu_{O_2}(p_{O_2},T) = \mu^{\plimsoll}_{O_2} + \Delta{\mu(T)} + \frac{1}{2}{k_B}log(\frac{p_{O_2}}{p^{\plimsoll}_{O_2}}) \\
\Delta \mu(T) = -\frac{1}{2}(S^{\plimsoll}_{O_{2}}- C^{\plimsoll}_{p})(T-T^{\plimsoll}) + C^{\plimsoll}_{p}T\textup{log}\left ( \frac{T}{T^{\plimsoll}} \right )
\end{gather} 
where $S^{\plimsoll}_{O_{2}}$ is the molecular entropy at standard temperature and pressure (T$^{\plimsoll}$ = 273.15 K, P$^{\plimsoll}$ = $10^{5}$ Pa), and $C^{\plimsoll}_{p}$ is the constant pressure heat capacity of oxygen. These quantities have values of $S^{\plimsoll}_{O_{2}}$ = 0.0021 eV/K and $C^{\plimsoll}_{p}$ = 0.000302 eV/K \cite{weast1984crc}. 

Using our generalised formula for $\mu_{O_2}$, we fix the temperature within the range of thermal phase-stabilisation (e.g. 1500 K for tetragonal \zirconia) and calculate $\mu_{O_2}$ for many different values of $p_{O_2}$ between $10^{-35}$ and 10$^{0}$ atm, corresponding to oxygen deficient and oxygen rich environments, respectively ($p_{O_2}$ in air is approximately 0.2 atm). While the tetragonal phase will be stress-stabilised in practice, thermal-stabilisation in such models has been shown to qualitatively approximate the effect of stress-stabilisation, while allowing a wider range of dopant behaviours to be predicted \cite{Bell2016}. 

Once a value of $\mu_{O_2}$ is calculated, defect concentrations can then be calculated using Boltzmann statistics. These concentrations were calculated using the method outlined by Kasamatsu \emph{et al}. \cite{Kasamatsu2012} whereby the effect of defects competing for the same lattice site is taken into account. The next step is to calculate the concentration of electron and hole defects. This is done by using the charge-neutrality condition to determine the Fermi level (electrochemical potential) in the system:
\begin{equation}
\sum_{i}q_{i}c_{i} - N_{c}\textrm{exp}{(-\frac{E_{g}-\mu_{e}}{k_{B}T})} + N_{v}\textrm{exp}{(-\frac{\mu_{e}}{k_{B}T})} = 0
\label{charge_neutrality}
\end{equation}

Where $c_{i}$ is the concentration of defect $i$, $q_{i}$ is its respective charge, $N_{c}$ and $N_{v}$ are the integrated density of states for the conduction and valence bands, $E_{g}$ is the band gap and $\mu_{e}$ is the Fermi level. 

\subsubsection{Temperature and pressure stabilisation}

As discussed in Chapter \ref{ch:crystallography}, the tetragonal and cubic phases are stabilised at elevated temperatures and pressures. However, DFT calculations of supercells under stress require significantly greater computational resources to yield sufficiently converged energy results, and so all DFT calculations in this thesis are performed on relaxed supercells. To account for this lack of stress stabilisation, Brouwer diagrams are generated at higher temperatures (where the tetragonal and cubic phases are thermally stabilised) rather than at 650 K, which is the expected temperature at the internal surface of the cladding. This approach to compensating for stress stabilisation follow that of similar studies published by other groups \cite{youssef2012intrinsic, Youssef2014, Otgonbaatar2014}.

\subsection{Effect of space charge} \label{space_charge}

Electrons have a higher rate of diffusion than oxygen vacancies in \zirconia , leading to a build-up of oxygen vacancies near the metal-oxide interface as corrosion progresses \cite{bojinov2010influence}. In the case of \zirconia , this effect will be pronounced because the layer is thin. This results in an overall positive charge (since the dominant oxygen vacancy is \ch{V_{O}^{**}}) referred to as a space charge. This effect can be taken into account when generating Brouwer diagrams by assuming an overall charge in the crystal structure instead of charge-neutrality:
\begin{equation}
\sum_{i}q_{i}c_{i} - N_{c}\textrm{exp}{(-\frac{E_{g}-\mu_{e}}{k_{B}T})} + N_{v}\textrm{exp}{(-\frac{\mu_{e}}{k_{B}T})} = q_{s}c_{s}
\label{charge_non_neutrality}
\end{equation}
where $q_{s}$ is the charge of a unit of the artificial space charge defect and $c_{s}$ is the concentration.

Figure \ref{figure:spacechargeexample} shows an example of the defect equilibria in tetragonal \zirconia\ with an overall positive space charge. In order for such a condition to be satisfied, higher concentrations of positively charged oxygen vacancy and hole defects are predicted to be present, while zirconium vacancy defects fall significantly. When extrinsic defects are also present in the lattice in significant concentrations, the space charge condition may influence which defect types are dominant at different oxygen pressures, as different oxidation states may be necessary to satisfy the charge condition.

%Another effect considered was the space charge of the system. Electrons have a higher rate of diffusion than oxygen vacancies in ZrO2, leading to a build-up of oxygen vacancies near the metal-oxide interface as corrosion progresses [44]. This results in an overall positive charge (since the dominant oxygen vacancy is referred to as a space charge. When included in our Brouwer diagrams, this space charge had a negligible effect on the concentration or charge state of iodine up to a charge of  holes per f.u. ZrO2. This corresponds to a high concentration of oxygen vacancies relative to the equilibrium concentration, predicting that a significant deviation from the equilibrium is not expected near the metal oxide interface as a result of a positive space charge.

\begin{figure}[htp] % Tet intrinsic no space charge
\begin{center}
\begin{tikzpicture}
	\begin{groupplot}[group style={group size=1 by 2}, width=14cm, height=11cm]
	\nextgroupplot[
		 ylabel={\ch{log_{10}}([D]) (per f.u.)}, ymin=-10, ymax=0, xmin=-35, xmax=0, legend style={{draw=}, at={(0.40,0.97)}, anchor=north west, legend columns=2, nodes={scale=1, transform shape}}]
        \addplot[no marks, draw=blue!70!black] table [x=pO2, y=electrons,]{dat/intrinsic_tet.dat}; \addlegendentry{\ch{e^{'}}}; \node at (-26.0,-2) {\ch{e^{'}}};
        \addplot[no marks, draw=red!85!black] table [x=pO2, y=holes,]{dat/intrinsic_tet.dat}; \addlegendentry{\ch{h^{\textperiodcentered}}}; \node at (-6.8,-3.6) {\ch{h^{\textperiodcentered}}};
        \addplot[no marks, draw=black!70!green] table [x=pO2, y=VO{2},]{dat/intrinsic_tet.dat}; \addlegendentry{\ch{V_{O}^{\textperiodcentered\textperiodcentered}}}; \node at (-28,-3) {\ch{V_{O}^{\textperiodcentered\textperiodcentered}}};
%         \addplot[no marks, draw=black!55!green] table [x=pO2, y=VO{1},]{dat/intrinsic_tet.dat}; \addlegendentry{\ch{V_{O}^{*}}};
%         \addplot[no marks, draw=black!30!green] table [x=pO2, y=VO{0},]{dat/intrinsic_tet.dat}; \addlegendentry{\ch{V_{O}^{x}}};
        \addplot[no marks, draw=yellow!85!blue] table [x=pO2, y=VM{-4},]{dat/intrinsic_tet.dat}; \addlegendentry{\ch{V_{Zr}^{''''}}}; \node at (-2.9,-4.6) {\ch{V_{Zr}^{''''}}};
%         \addplot[no marks, draw=yellow!75!blue] table [x=pO2, y=VM{-3},]{dat/intrinsic_tet.dat}; \addlegendentry{\ch{V_{Zr}^{'''}}};
%         \addplot[no marks, draw=yellow!65!blue] table [x=pO2, y=VM{-2},]{dat/intrinsic_tet.dat}; \addlegendentry{\ch{V_{Zr}^{''}}};
%         \addplot[no marks, draw=yellow!55!blue] table [x=pO2, y=VM{-1},]{dat/intrinsic_tet.dat}; \addlegendentry{\ch{V_{Zr}^{'}}};
%         \addplot[no marks, draw=yellow!45!blue] table [x=pO2, y=VM{0},]{dat/intrinsic_tet.dat}; \addlegendentry{\ch{V_{Zr}^{x}}};
%         \addplot[no marks, draw=red!60!yellow] table [x=pO2, y=Oi{-2},]{dat/intrinsic_tet.dat}; \addlegendentry{\ch{O_{i}^{''}}};
%         \addplot[no marks, draw=red!50!yellow] table [x=pO2, y=Oi{-1},]{dat/intrinsic_tet.dat}; \addlegendentry{\ch{O_{i}^{'}}};
%         \addplot[no marks, draw=red!40!yellow] table [x=pO2, y=Oi{0},]{dat/intrinsic_tet.dat}; \addlegendentry{\ch{O_{i}^{x}}};
%         \addplot[no marks, draw=green!80!pink] table [x=pO2, y=Mi{4},]{dat/intrinsic_tet.dat}; \addlegendentry{\ch{Zr_{i}^{****}}};
%         \addplot[no marks, draw=green!70!pink] table [x=pO2, y=Mi{3},]{dat/intrinsic_tet.dat}; \addlegendentry{\ch{Zr_{i}^{***}}};
%         \addplot[no marks, draw=green!60!pink] table [x=pO2, y=Mi{2},]{dat/intrinsic_tet.dat}; \addlegendentry{\ch{Zr_{i}^{\textbf{**}}}};
%         \addplot[no marks, draw=green!50!pink] table [x=pO2, y=Mi{1},]{dat/intrinsic_tet.dat}; \addlegendentry{\ch{Zr_{i}^{*}}};
%         \addplot[no marks, draw=green!40!pink] table [x=pO2, y=Mi{0},]{dat/intrinsic_tet.dat}; \addlegendentry{\ch{Zr_{i}^{x}}};
%         \addplot[no marks] table [x=pO2, y=Stoich,]{dat/intrinsic_tet.dat}; \addlegendentry{Stoich};
\node at (-33.7,-0.5) {\textbf{a)}}; 
			%\end{axis}     
%\end{tikzpicture}
%\begin{tikzpicture} % 1e-1
	\nextgroupplot[
		 xlabel={\ch{log_{10}}($p_{O_{2}}$) (atm)}, ylabel={\ch{log_{10}}([D]) (per f.u.)}, ymin=-10, ymax=0, xmin=-35, xmax=0, legend style={{draw=}, at={(0.40,0.97)}, anchor=north west, legend columns=4, nodes={scale=1, transform shape}}]
        \addplot[no marks, draw=blue!70!black] table [x=pO2, y=electrons,]{dat/intrinsic_spacecharge01.dat}; \node at (-26.5,-2.5) {\ch{e^{'}}};
        \addplot[no marks, draw=red!85!black] table [x=pO2, y=holes,]{dat/intrinsic_spacecharge01.dat}; \node at (-13,-3) {\ch{h^{\textperiodcentered}}};
        \addplot[no marks, draw=black!70!green] table [x=pO2, y=VO{2},]{dat/intrinsic_spacecharge01.dat}; \node at (-28,-0.8) {\ch{V_{O}^{\textperiodcentered\textperiodcentered}}};
%         \addplot[no marks, draw=black!55!green] table [x=pO2, y=VO{1},]{dat/intrinsic_spacecharge01.dat}; \addlegendentry{\ch{V_{O}^{*}}};
%         \addplot[no marks, draw=black!30!green] table [x=pO2, y=VO{0},]{dat/intrinsic_spacecharge01.dat}; \addlegendentry{\ch{V_{O}^{x}}};
        %\addplot[no marks, draw=yellow!85!blue] table [x=pO2, y=VM{-4},]{dat/intrinsic_spacecharge01.dat}; \node at (-3,-3) {\ch{V_{Zr}^{''''}}};
%         \addplot[no marks, draw=yellow!75!blue] table [x=pO2, y=VM{-3},]{dat/intrinsic_spacecharge01.dat}; \addlegendentry{\ch{V_{Zr}^{'''}}};
%         \addplot[no marks, draw=yellow!65!blue] table [x=pO2, y=VM{-2},]{dat/intrinsic_spacecharge01.dat}; \addlegendentry{\ch{V_{Zr}^{''}}};
%         \addplot[no marks, draw=yellow!55!blue] table [x=pO2, y=VM{-1},]{dat/intrinsic_spacecharge01.dat}; \addlegendentry{\ch{V_{Zr}^{'}}};
%         \addplot[no marks, draw=yellow!45!blue] table [x=pO2, y=VM{0},]{dat/intrinsic_spacecharge01.dat}; \addlegendentry{\ch{V_{Zr}^{x}}};
%         \addplot[no marks, draw=red!60!yellow] table [x=pO2, y=Oi{-2},]{dat/intrinsic_spacecharge01.dat}; \addlegendentry{\ch{O_{i}^{''}}};
%         \addplot[no marks, draw=red!50!yellow] table [x=pO2, y=Oi{-1},]{dat/intrinsic_spacecharge01.dat}; \addlegendentry{\ch{O_{i}^{'}}};
%         \addplot[no marks, draw=red!40!yellow] table [x=pO2, y=Oi{0},]{dat/intrinsic_spacecharge01.dat}; \addlegendentry{\ch{O_{i}^{x}}};
%         \addplot[no marks, draw=green!80!pink] table [x=pO2, y=Mi{4},]{dat/intrinsic_spacecharge01.dat}; \addlegendentry{\ch{Zr_{i}^{****}}};
%         \addplot[no marks, draw=green!70!pink] table [x=pO2, y=Mi{3},]{dat/intrinsic_spacecharge01.dat}; \addlegendentry{\ch{Zr_{i}^{***}}};
%         \addplot[no marks, draw=green!60!pink] table [x=pO2, y=Mi{2},]{dat/intrinsic_spacecharge01.dat}; \addlegendentry{\ch{Zr_{i}^{\textbf{**}}}};
%         \addplot[no marks, draw=green!50!pink] table [x=pO2, y=Mi{1},]{dat/intrinsic_spacecharge01.dat}; \addlegendentry{\ch{Zr_{i}^{*}}};
%         \addplot[no marks, draw=green!40!pink] table [x=pO2, y=Mi{0},]{dat/intrinsic_spacecharge01.dat}; \addlegendentry{\ch{Zr_{i}^{x}}};
%         \addplot[no marks] table [x=pO2, y=Stoich,]{dat/intrinsic_spacecharge01.dat}; \addlegendentry{Stoich};
\node at (-33.7,-0.5) {\textbf{b)}};
	\end{groupplot}
			%\end{axis}              
\end{tikzpicture}
		\caption{Tetragonal phase Brouwer diagrams of intrinsic point defects at a temperature of 1500 K \textbf{a)} without a space charge and \textbf{b)} with a space charge of $10^{-1}$ e$^{-1}$/fu.}
		\label{figure:spacechargeexample}
	\end{center}
\end{figure}

\section{Convergence testing} \label{section:convergence}

\subsection{Plane-wave cut-off energy}

In order to determine an appropriate value for the plane-wave cut-off energy, a convergence test was performed to determine the relative error in predicted energy compared to a highly converged value. This convergence test was conducted by running multiple geometry optimisation procedures under fully relaxed conditions on a unit cell of \zirconia\ for each phase. A small \textbf{k}-point spacing of 0.01 \r{A}$^{-1}$ was used for each task (highly converged), while increasing the plane-wave cut-off energy from 300 eV to 750 eV in 50 eV increments. The energy of each run was recorded and compared to the energy of a highly converged value taken when a cut-off energy of 900 eV was used. This provides a value for the truncation error at different cut-off energies. Figure \ref{Figure:cutoffconvergence} shows a log plot of the energy error for each phase of \zirconia\ as the cut-off energy is increased. 

The error is shown to be independent of phase, with all lines lying on a single path. This might be expected because the atoms in each phase are the same, and therefore the electrons involved in the calculations remain unchanged, however, interatomic distances are different in the different phases and thus so are the electron densities, so this equivalence in convergence does not necessarily have to follow. A cut-off energy of 600 eV was found to produce an error below 0.01 eV, and was subsequently used for future calculations as it provides a good compromise between computational cost and accuracy.

\begin{figure}[ht] % Plane-wave cut-off convergence
	\begin{center}
		\begin{tikzpicture}
			\begin{axis}
				[width=\linewidth*0.7, xlabel={E\textsubscript{cutoff} (eV)}, ylabel={log$_{10}$(error) / formula unit}, ymin=-3.5, legend style={{draw=}, at={(0.95,0.95)}, anchor=north east,}]
				\addplot[no marks] table [x=cutoffenergy, y=logerrormono,]{dat/convergence.dat}; \addlegendentry{Monoclinic};
			    \addplot[no marks, dashed] table [x=cutoffenergy, y=logerrortet,]{dat/convergence.dat}; \addlegendentry{Tetragonal};
			    \addplot[no marks, densely dotted] table [x=cutoffenergy, y=logerrorcubic,]{dat/convergence.dat}; \addlegendentry{Cubic};
                \draw[red,-stealth]
				(600,-1.96)
				-- % = line-to
				++ % = calculate a vector sum
				(axis direction cs:0,-1.46);
                \addplot [only marks,mark=*]
coordinates { (600,-1.95) };
			\end{axis}
		\end{tikzpicture}
		\caption{Plot of the log error of DFT energy against plane-wave cut-off energy for a perfect cell of each crystal structure. The error is calculated with respect to a highly converged value, calculated at a plane-wave cut-off energy of 900 eV. The red arrow indicates the cut-off energy beyond which the error is below 0.01 eV.}
		\label{Figure:cutoffconvergence}
	\end{center}
\end{figure}

\subsection{\textbf{k}-point convergence}

Too fine a grid in reciprocal space (i.e. a large number of \textbf{k}-points) results in prohibitively computationally expensive simulations, whereas too coarse a grid may have a large truncation error when energies are calculated. To find the optimum spacing of \textbf{k}-points, a convergence study was performed across a range of \textbf{k}-point spacings, with the output energies compared to a highly converged simulation to obtain a value for the error. 

Figure \ref{Figure:kpoint_convergence} shows the energy error for each phase of \zirconia\ as a function of the \textbf{k}-point spacing (given in reciprocal space as \r{A}$^{-1}$). The highly converged energy value was calculated with a \textbf{k}-point spacing of 0.01 \r{A}$^{-1}$ for error calculations. The plot shows a stepwise change in the error value as the grid spacing is reduced. This is because calculations demand an integer number of \textbf{k}-points, and larger spacings do not provide sufficient resolution to effectively fit an integer number of \textbf{k}-points into the reciprocal grid, so that the program snaps to the nearest appropriate grid number. An optimum \textbf{k}-point spacing was chosen at 0.09 \r{A}$^{-1}$, which was the largest spacing that kept the error below 0.01 eV for all phases, highlighted in the plot by the red arrow.

\begin{figure}[ht]
\begin{center}
\begin{tikzpicture}
	\begin{axis}
		[width=\linewidth*0.7, xlabel={\textbf{k}-point spacing (\r{A}\textsuperscript{-1})}, ylabel={log[error]}, ymin=-7, ymax=1, xmin=0, xmax=0.22, legend style={{draw=}, at={(0.05,0.95)}, anchor=north west, legend columns=1}, xticklabel
style={/pgf/number format/.cd,fixed,precision=5}]
		\addplot[no marks] table [x=kpoint_spacing, y=monoclinic,]{dat/kpoint_convergence.dat}; \addlegendentry{Monoclinic};
        \addplot[no marks, dashed] table [x=kpoint_spacing, y=tetragonal, ]{dat/kpoint_convergence.dat}; \addlegendentry{Tetragonal};
        \addplot[no marks, densely dotted] table [x=kpoint_spacing, y=cubic,]{dat/kpoint_convergence.dat}; \addlegendentry{Cubic};
        \draw[red,-stealth]
				(0.09,-2.35)
				-- % = line-to
				++ % = calculate a vector sum
				(axis direction cs:0,-4.6);
                \addplot [only marks,mark=*]
coordinates { (0.09,-2.35) };
			\end{axis}
		\end{tikzpicture}
		\caption{Log of the error in the total energy of the system as a function of \textbf{k}-point spacing. The error is calculated relative to a highly converged energy value at a \textbf{k}-point spacing of 0.01 \r{A}\textsuperscript{-1}. The red arrow indicates the \textbf{k}-point spacing which yields an error below 0.01 eV for all structures.}
		\label{Figure:kpoint_convergence}
	\end{center}
\end{figure}

\subsection{Exchange-correlation functionals}

There are a range of possible exchange-correlation functionals available in CASTEP, spanning both empirical and non-empirical types. Empirical exchange-correlation functionals are typically optimised to capture specific properties or systems particularly well, but perform less well for generalised systems. Non-empirical exchange-correlation functionals, while still not perfect, are preferred for modelling the widest range of properties. In a sense, non-empirical functions benefit from not being `over-fit' to experimental data. They are also more prevalent in the literature, thereby providing a rich corpus of work for comparison studies.

While the PBE-GGA exchange-correlation functional in this work had already been selected, it was helpful to conduct an energy convergence study of the systems across the different functionals available in CASTEP in order to determine how other functionals compared. Only 6 of the 14 functionals available in CASTEP were able to yield a converged energy calculation within a reasonable amount of time, as shown in Figure \ref{Figure:xc_test}. This is because several hybrid functionals partially incorporate the exact exchange using the Hartree-Fock method \cite{hartree1928wave}, significantly increasing the computational cost of an energy calculation. 

The calculated energies indicate that each functional correctly predicts the order of phase stability in \zirconia , though the magnitude of the energy difference between phases varied. These differences are small, approximately 0.1 eV/f.u., but their effects are compounded when defects are introduced into the cell. The total energies were more varied, with several eV differences between functionals, however, lower total energies across different exchange-correlation functionals do not necessarily suggest that a better minima has been found. For example, the PW91 functional resulted in even lower energies than PBE, despite PW91 preceding PBE and both having been developed by Perdew \emph{et al}. \cite{perdew1991unified, perdew1992atoms}. It is the energy difference between systems calculated with the same exchange-correlation functional which is important. 

To better gauge the performance of each functional, further studies across a much larger range of parameters, and even materials, would need to be conducted, however this is beyond the scope of this thesis.

\begin{figure}[ht!] % XC functional study
  \begin{center}
    \begin{tikzpicture}
      \begin{axis}
        [ybar, ymin=2350, ymax=2362, width=\linewidth*0.7, xtick=data, xlabel={XC Functional}, ylabel={Unit cell energy (-eV/f.u.)}, xticklabels from table={dat/xc_test.dat}{functional}, area legend, legend style={at={(0.04,0.96)},
anchor=north west, legend columns=1}] %axis x line=middle, ymin=5.1, ymax=5.35, xmin=0, xmax=12, legend style={{draw=}, at={(0.18,0.95)}, anchor=north east, legend columns=1}
        \addplot[style={black, fill=red!30!white, mark=none}] table [x expr=\coordindex, y=mono_energy]{dat/xc_test.dat};
        \addplot[style={black, fill=blue!30!white, mark=none}] table [x expr=\coordindex, y=tet_energy]{dat/xc_test.dat};
        \addplot[style={black, fill=green!30!white, mark=none}] table [x expr=\coordindex, y=cubic_energy]{dat/xc_test.dat};
        \legend{Monoclinic, Tetragonal, Cubic}
      \end{axis}
    \end{tikzpicture}
    \caption{Calculated energy of a unit cell of monoclinic, tetragonal and cubic \zirconia\ when using different exchange-correlation functionals.}
    \label{Figure:xc_test}
  \end{center}
\end{figure}

\subsection{On-the-fly pseudopotentials}

Ultra soft pseudopotentials are generated in CASTEP automatically (known as on-the-fly or OTF pseudopotentials) when none are specified for a particular element. Energies must be calculated and compared with the same set of pseudopotentials in order to keep simulations self-consistent. A single point calculation was performed on a unit cell of \zirconia\ and the resulting OTF pseudopotentials (one for oxygen and one for zirconium) were saved and used for all subsequent calculations. 

It is important to determine the variance in energy values of different pseudopotentials generated OTF in order to avoid systematic error. To assess error, 9 different pairs of OTF pseudopotentials were generated\footnote{OTF pseudopotential generation in CASTEP uses a random number seed which is generated whenever a calculation is run without specifying a pseudopotential. By default, these are ultra-soft pseudopotentials.} and used to calculate the total energy of a monoclinic \zirconia\ supercell. The difference in energy was then calculated with respect to the pseudopotential pair that resulted in the lowest energy. These deviations in total energy are shown in Figure \ref{Figure:otf_pp_test}. Across all calculations, the largest difference in total energy was 0.0012 eV, while the average difference was 0.0006 eV. Since here the only concern is with choosing other parameters to achieve a precision of 0.01 eV, and the largest deviation calculated is an order of magnitude below that, it is not necessary to take any special measures to correct any systematic error from randomly generated OTF pseudopotentials.

%\begin{figure}[ht] % +U cubic
%\begin{center}
%\begin{tikzpicture}
%	\begin{axis}
%		[width=11cm, xlabel={+U on Zr \emph{d} orbitals (eV)}, ylabel={Lattice parameter (\r{A})}, ymin=5.1, ymax=5.35, xmin=0, xmax=12, legend style={{draw=}, at={(0.18,0.95)}, anchor=north east, legend columns=1}]
%		\addplot[no marks] table [x=plusU, y=a,]{dat/plus_u_cubic.dat}; \addlegendentry{$a$};
%        %\addplot[no marks, dashed] table [x=plusU, y=b, ]{dat/plus_u_cubic.dat}; \addlegendentry{b};
%        %\addplot[no marks, densely dotted, black] table [x=plusU, y=c,]{dat/plus_u_cubic.dat}; \addlegendentry{c};
%			\end{axis}
%		\end{tikzpicture}
%		\caption{Individual lattice parameters as a function of +U term in cubic \zirconia .}
%		\label{Figure:plusucubic}
%	\end{center}
%\end{figure}

\begin{figure}[ht] % OTF PP study
  \begin{center}
    \begin{tikzpicture}
      \begin{axis}
        [ybar, width=\linewidth*0.7, xlabel={OTF pseudopotential pair}, ylabel={$\Delta$E with respect to lowest energy (meV)}, ] %axis x line=middle, ymin=5.1, ymax=5.35, xmin=0, xmax=12, legend style={{draw=}, at={(0.18,0.95)}, anchor=north east, legend columns=1}
        \addplot table [x=pp_pair, y=energy_diff_wrt_first,]{dat/otf_pp_test.dat};
      \end{axis}
    \end{tikzpicture}
    \caption{Energy deviation in meV of supercells with candidate OTF pseudopotential pairs. Energy deviations are shown with respect to the pseudopotential pair that resulted in the lowest total energy calculated.}
    \label{Figure:otf_pp_test}
  \end{center}
\end{figure}

\subsection{Chemical potential of oxygen}

The chemical potential of oxygen is required when performing any defect formation energy calculation where an atom of oxygen is added or removed (see Equation \ref{equation:formation_energy}). Calculating the chemical potential of oxygen requires special consideration of the electronic structure of O$_{2}$. The ground state of the O$_{2}$ molecule is known as triplet oxygen ($^{3}\Sigma^{-}_{g}$), an allotrope which exhibits a resultant spin magnetic moment (oxygen is paramagnetic). This is in contrast to singlet oxygen ($^{1}\Delta_{g}$) with a spin magnetic moment of zero. 

Two calculations, one for triplet and another for singlet oxygen, were performed using CASTEP. Large cells of 15 \r{A} x 15 \r{A} x 15 \r{A} were used to run geometry optimisation tasks on two oxygen atoms initially separated by 1.3 \r{A}. For the triplet oxygen calculation, a net electronic spin of +2 on the $p$ electrons was enforced, while the singlet oxygen calculation specified a net spin of 0.

The calculated bond lengths of triplet and singlet oxygen were 1.225 \r{A} and 1.227 \r{A} respectively. These bond lengths are within 2\% of the experimental value of 1.207 \cite{Lide2016}, with the triplet state prediction being slightly closer to this value. The calculated energies from DFT for triplet and singlet oxygen were -871.92 eV and -870.70 eV respectively. This gave an energy difference of 1.22 eV between the two forms of diatomic oxygen. While triplet oxygen was correctly predicted as the lower energy allotrope, the energy difference reported in the literature from microwave spectroscopy measurements is 0.9773 eV \cite{Atkins2006}, an almost 25\% difference compared to the DFT value. This large difference is attributed to the exchange-correlation functional and the inability to correctly model electron correlation effects in some cases. In this thesis, the DFT calculated energy of triplet oxygen was used only for formation energy against Fermi level plots, while defect equilibria calculations utilised a different method to calculate this value (see § \ref{brouwer_method}).

\subsection{Chemical potential of iodine}

To determine the chemical potential of iodine, an energy minimisation of the iodine dimer was performed. Unlike oxygen, iodine dimers do not exhibit a non-zero spin magnetic moment, thus avoiding a source of error in energy calculations with the PBE exchange-correlation functional. Similar to the \zirconia\ unit cell calculations, the lattice parameter after relaxation (bond length in this case) is compared to experimental data to assess the quality of the simulation parameters.

Figure \ref{figure:iodine_dimer} illustrates the energy minimisation of two iodine atoms in a cell of size 15 \r{A} x 15 \r{A} x 15 \r{A}, initially separated by 3.0 \r{A}. The geometry optimisation task finds an energy minima when the iodine atoms are bonded, at a separation of 2.69 \r{A}. This agrees well with the experimental value of 2.6745 \r{A} \cite{ukaji1966effect}.

\begin{figure}[ht] % Iodine dimer geometry optimisation
\centering
\includegraphics[width=14cm]{images/iodine_molecule.png}
\caption{Energy minimisation of two iodine atoms from an initial separation of 3.0 \r{A}.}
\label{figure:iodine_dimer}
\end{figure}

\subsection{+U study}
\label{subsection:plus_U}

In some DFT studies, an additional potential energy term (Hubbard U parameter or +U) is sometimes included to better capture the Coulomb interaction of localised electrons. An LDA or GGA functional alone will typically not describe this interaction correctly, especially for localised $d$ and $f$ electrons\footnote{Multiple occupation of $d$ and $f$ orbitals incurs an energy penalty which is not accurately modelled by the exchange-correlation functional.}. Of particular concern is the calculated value of the band gap from DFT simulations, as this value may deviate by up to 30\% from experimental values. Remedying this shortcoming with an appropriate +U parameter could therefore be valuable in obtaining accurate energies. 

In the literature, one GGA+U study of Fe-doped tetragonal \zirconia\ has shown that the inclusion of a +U term on Zr $d$ orbital electrons between 0 and 3.3 eV changes the electronic properties (in particular the electronic density of states) of the system significantly, and that the best agreement with experimental data occurs when U = 0 eV \cite{Sangalli2013}. Other GGA+U studies on bulk tetragonal \zirconia\ found that a +U term of 4 eV led to calculated lattice parameters which were in good agreement (within 0.05 \r{A}) with experimental values, but that the calculated band gap was still underestimated by 1.28 eV \cite{RuizPuigdollers2016, Chen2015, Puigdollers2015}. One LDA+U study in \zirconia\ used a +U parameter of 1 Ry (13.6 eV), and reproduced the correct order of stability of the monoclinic, tetragonal and cubic phase. This shows how the value of the +U parameter can vary significantly depending on the other approximations being used in the DFT calculations, and therefore an appropriate +U value for this thesis would have to be found independently. A +U study of the zirconium atom, with an electronic configuration of [Kr]$4d^{2}5s^{2}$, was performed to determine the response to and therefore the viability of this additional potential term for the $d$ electrons.

Figure \ref{Figure:plusubandgap} shows the effect on the calculated band gap when introducing a +U term. While the +U term does increase the band gap, the effect is not significant in bringing the prediction in-line with experimental values. Even with +U terms of 10 eV, the calculated band gap falls short of the experimental band gap by at least 1.5 eV. Moreover, with +U terms greater than 4 eV, we begin to see erratic behaviour in the development of both the band gap, and also in the predicted crystal structure. 

For the tetragonal phase, the calculated band gap (4.2 eV) does not agree with that calculated by Puigdollers \emph{et al.} \cite{RuizPuigdollers2016} (4.5 eV) when a +U term of 4 eV is used. The PBE GGA exchange-correlation functional and a plane-wave basis set was used in both studies, however, their study utilised the VASP 5.3 DFT software package while CASTEP 8.0 was used in this thesis. Different software packages may use different minimisation methods which could contribute to differing values, but determining the cause of this anomaly would require a separate study of the two codes at a low level which is beyond the scope of this thesis.

%Cubic is fine, it just keeps expanding with +U as we expect. The tetragonal phase expands in the short a&b directions but contracts in the long c direction (i.e. becomes more 'cubic') up until 6 eV, after which it grows in the same manner as cubic.

%The monoclinic phase is harder to explain. There is a cross-over in the length of the a and b parameter at around 4 eV, and then the beta angle (the ~99 deg between a and c) snaps into 90 deg at 11 eV, and when I look at the output structure at this energy, the coordination number of Zr is 6, down from 7. That's why the lattice parameters don't fall into a=b=c, because it doesn't become cubic.

%As you can see from the band gap plot below, just +U by itself is not enough to reproduce the experimental band gap, even for monoclinic. We're off by about 1.5 eV in each case.

\begin{figure}[ht] % +U band gaps
\begin{center}
\begin{tikzpicture}
	\begin{axis}
		[width=\linewidth*0.7, xlabel={+U on Zr \emph{d} orbitals (eV)}, ylabel={Band gap (eV)}, ymin=3.2, ymax=5, xmin=0, xmax=12, legend style={{draw=}, at={(0.35,0.95)}, anchor=north east, legend columns=1}]
		\addplot[no marks] table [x=plusU, y=bandgap,]{dat/plus_u_mono.dat}; \addlegendentry{Monoclinic};
        \addplot[no marks, dashed] table [x=plusU, y=bandgap, ]{dat/plus_u_tet.dat}; \addlegendentry{Tetragonal};
        \addplot[no marks, densely dotted, black] table [x=plusU, y=bandgap,]{dat/plus_u_cubic.dat}; \addlegendentry{Cubic};
			\end{axis}
		\end{tikzpicture}
		\caption{Calculated band gaps for different +U values in monoclinic, tetragonal and cubic \zirconia .}
		\label{Figure:plusubandgap}
	\end{center}
\end{figure}

\subsubsection{Monoclinic}

In monoclinic \zirconia , the use of a +U term causes the lattice parameters to change disproportionately to each other, as seen in Figure \ref{Figure:plusumono}. All lattice parameters increased with larger +U terms, however, expansion in the $a$ direction proceeded faster than in the $b$ direction, resulting in the $a$ lattice parameter becoming larger at a +U of 4 eV. +U terms larger than 10.5 eV caused the lattice parameters to snap suddenly onto new values. A  investigation of the atomic positions revealed that the monoclinic crystal structure had collapsed into an orthorhombic structure (i.e. the cell experienced a shear strain which resulted in a $\beta$ of 90\textdegree), with the co-ordination number of zirconium ions falling to 6 from 7.

\begin{figure}[ht] % +U mono
\begin{center}
\begin{tikzpicture}
	\begin{axis}
		[width=\linewidth*0.7, xlabel={+U on Zr \emph{d} orbitals (eV)}, ylabel={Lattice parameter (\r{A})}, ymin=4.9, ymax=6.3, xmin=0, xmax=12, legend style={{draw=}, at={(0.18,0.95)}, anchor=north east, legend columns=1}]
		\addplot[no marks] table [x=plusU, y=a,]{dat/plus_u_mono.dat}; \addlegendentry{$a$};
        \addplot[no marks, dashed] table [x=plusU, y=b, ]{dat/plus_u_mono.dat}; \addlegendentry{$b$};
        \addplot[no marks, densely dotted, black] table [x=plusU, y=c,]{dat/plus_u_mono.dat}; \addlegendentry{$c$};
			\end{axis}
		\end{tikzpicture}
		\caption{Individual lattice parameters as a function of +U term in monoclinic \zirconia .}
		\label{Figure:plusumono}
	\end{center}
\end{figure}

\subsubsection{Tetragonal}

In tetragonal \zirconia , increasing the +U term (Figure \ref{Figure:plusutet}) has a strong anisotropic effect on the lattice parameters. Unusually, the $c$ parameter falls (up to an +U energy of 6 eV) while the $a$ parameter increases. Typically it would be expected that both parameters would increase, perhaps at different rates, because the +U term increases the total energy (by increasing the Coulombic contribution) in the system. This increase in energy leads to higher stresses and therefore larger interatomic spacings (cells are permitted to relax in these calculations).

Systems will always tend towards the lowest energy configuration. Therefore the reduction of the $c$ parameter suggests that it is already in a high energy configuration in the $c$ direction (overextended) and can reduce its energy by being compressed in that direction (i.e. becoming more cubic). This is consistent with the interpretation that lower temperature phases of \zirconia\ are distortions of the cubic fluorite phase caused by a small cation radius. 

Above a +U parameter of 6 eV however, the $c$ parameter suddenly begins to increase. Upon further inspection of the resulting cell, it was found that it had transitioned completely to cubic fluorite from tetragonal. This can also be confirmed by observing that the relationship between the parameters becomes $2a^2 = c^2$ (i.e. the $c$ parameter is the same length as the unit cell's [110] diagonal, see Figure \ref{figure:tetvscubic}).

\begin{figure}[ht] % +U tet
\begin{center}
\begin{tikzpicture}
	\begin{axis}
		[width=\linewidth*0.7, xlabel={+U on Zr \emph{d} orbitals (eV)}, ylabel={$a$ parameter (\r{A})}, ymin=3.6, ymax=3.8, xmin=0, xmax=12, legend style={{draw=}, at={(0.18,0.95)}, anchor=north east, legend columns=1}, tick pos=left]
		\addplot[no marks] table [x=plusU, y=a,]{dat/plus_u_tet.dat}; \addlegendentry{$a$};
        %\addplot[no marks, dashed] table [x=plusU, y=b, ]{dat/plus_u_tet.dat}; \addlegendentry{b};
        \addplot[no marks, dashed, black] table [x=plusU, y=c,]{dat/plus_u_tet.dat}; \addlegendentry{$c$};
			\end{axis}
            \begin{axis}[width=\linewidth*0.7,
     xmin = 0, xmax = 12,
     ymin = 5.16, ymax = 5.32,
     hide x axis,
     hide y axis, tick pos=right]
     \addplot[no marks, dashed, black] table [x=plusU, y=c,]{dat/plus_u_tet.dat};
   			\end{axis}
            \pgfplotsset{every axis y label/.append style={rotate=180}}
   \begin{axis}[width=\linewidth*0.7,
         xmin=0, xmax=12,
         ymin=5.16, ymax=5.32,
         hide x axis,
         axis y line*=right,
         ylabel={$c$ parameter (\r{A})}
     ]
   \end{axis}
		\end{tikzpicture}
		\caption{Individual lattice parameters as a function of +U term in tetragonal \zirconia .}
		\label{Figure:plusutet}
	\end{center}
\end{figure}

\subsubsection{Cubic}

The effect of a +U term on a lattice parameter of cubic \zirconia\ is shown in Figure \ref{Figure:plusucubic}. Notably, the symmetry of the cell remains intact even up to a +U term of 12 eV, unlike in the monoclinic and tetragonal phases. The lattice parameter also increases superlinearly as the +U term is increased. This is the typical response that is expected when a +U term is introduced. 

\begin{figure}[ht] % +U cubic
\begin{center}
\begin{tikzpicture}
	\begin{axis}
		[width=\linewidth*0.7, xlabel={+U on Zr \emph{d} orbitals (eV)}, ylabel={Lattice parameter (\r{A})}, ymin=5.1, ymax=5.35, xmin=0, xmax=12, legend style={{draw=}, at={(0.18,0.95)}, anchor=north east, legend columns=1}]
		\addplot[no marks] table [x=plusU, y=a,]{dat/plus_u_cubic.dat}; \addlegendentry{$a$};
        %\addplot[no marks, dashed] table [x=plusU, y=b, ]{dat/plus_u_cubic.dat}; \addlegendentry{b};
        %\addplot[no marks, densely dotted, black] table [x=plusU, y=c,]{dat/plus_u_cubic.dat}; \addlegendentry{c};
			\end{axis}
		\end{tikzpicture}
		\caption{Lattice parameter as a function of +U term in cubic \zirconia .}
		\label{Figure:plusucubic}
	\end{center}
\end{figure}

There was however one instance of unexpected behaviour. Knowing that the tetragonal phase collapses to cubic at +U terms greater than 6 eV, we would expect both to exhibit the same band gap at this +U value. Looking at the band gap results from Figure \ref{Figure:plusubandgap}, we see that the band gaps of tetragonal and cubic \zirconia\ continue to be different above 6 eV, despite the crystal structures being the same in this region. One difference between the cells is that the cubic unit cell has 12 atoms while the tetragonal unit cell has 6 atoms. While this suggests a size effect, the band gap difference did not appear when comparing the band gaps of unit cells and supercells in the absence of a +U term. 

After considering the impact of a +U term in DFT calculations, the decision was made not to include the term. While it would provide a small improvement in the calculated band gap for the cubic phase, the effect on cell symmetry of the other phases would present a confounding variable, especially when placing defects into the structure. It is therefore more useful to run calculations without a +U term to maintain consistency of results between different phases in this thesis.
\chapter{Structure properties and intrinsic defects} \label{ch:results1}  

\label{ch:defects}

\section{Introduction}  

It is important to fully understand the behaviour of intrinsic defects in \zirconia\ before performing studies with dopant ions. In this chapter, intrinsic defects in monoclinic, tetragonal and cubic \zirconia\ are compared and contrasted, including values for formation energy, defect volumes and defect equilibria. Elastic constants, electronic density of states, band gaps and free energies of the non-defective structures are also reported. % because useful material properties may be exploited to improve performance, e.g. by doping with other ions to stabilise one crystal structure. For example, \zirconia\ doped with enough yttrium cations will stabilise the cubic phase and increase the concentration of oxygen vacancies. This would then affect the behaviour of dopant atoms also present in the lattice.

\subsection{Previous work} 

Previous works studying intrinsic defects in the \zirconia\ system have utilised quantum mechanical methods to determine defect formation energies in the monoclinic phase \cite{zheng2007first,foster2002modelling,foster2001structure} and defect equilibria in the tetragonal phase \cite{youssef2012intrinsic}. The cubic phase is mainly studied as a dopant-stabilised system \cite{orera1990intrinsic,jiang2011first}, with few undoped defect studies in the literature \cite{mackrodt1986theoretical,aarhammar2009energetics}. Building upon previous quantum mechanical studies, a comprehensive account of intrinsic defect energies, defect volumes, and defect equilibria for all three common crystal structures of \zirconia\ is provided, using state-of-the-art, accessible methods.

\section{Methodology}
\subsection{Simulation parameters}

As discussed in Chapter \ref{ch:compmethodology}, DFT calculations were performed using CASTEP 8.0 \cite{Clark2005}. Ultra-soft pseudo-potentials were used throughout, employing a 600 eV cut-off energy. The Perdew, Burke and Ernzerhof (PBE) \cite{Perdew1996} parameterisation of the generalised gradient approximation (GGA) was used to describe the exchange correlation functional. A Monkhorst-Pack sampling scheme \cite{Monkhorst1976} was used for Brillouin zone integration, with a minimum \emph{k}-point separation of 0.09 \r{A}\textsuperscript{-1}. The Pulay method for density mixing \cite{Pulay1980} was used to improve convergence of simulations. 

The electrical energy convergence criterion was set to $1\times10^{-6} $ eV. The maximum force between atoms was limited to $1\times10^{-2}$ eV \r{A}\textsuperscript{-1}. A gradient-descent geometry optimisation task was run on the cell until consecutive iterations differed in energy and atomic displacement by less than $1\times10^{-5}$ eV and $5\times10^{-4}$ \r{A}, respectively. 


\subsection{Helmholtz free energy}

To determine the temperature dependence of the ground states for the pure crystal structures, a harmonic approximation method as described by Burr et al. was used \cite{burr2015crystal,jackson2016resolving}. A constant-volume phonon calculation was performed for each structure, from which the vibrational enthalpy $H_{vib}(T, V)$ and entropy $S_{vib}(T, V)$ contributions to the Helmholtz free energy were calculated up to a temperature of 2500 K. The complete Helmholtz free energy $F(T, V)$ was then obtained by including the internal energy $U(V)$ and configurational entropy $S_{conf}$ of the system:
\begin{equation} \label{helmholtz_equation}
F(T, V) = U(V) + H_{vib}(T, V) - TS_{vib}(T, V) - TS_{conf}
\end{equation}

\subsection{Brouwer diagrams}

Using the method outlined in § \ref{brouwer_method}, Brouwer diagrams of the intrinsic defect equilibria for monoclinic, tetragonal and cubic \zirconia\ were generated. The monoclinic and tetragonal Brouwer diagrams were generated at temperatures of 650 K and 1500 K respectively, corresponding to temperatures at which these phases are thermally stabilised. Defect concentrations are reported in parts/fu (i.e. parts per formula unit \zirconia). The cubic Brouwer diagram was generated at 2000K despite being thermally stabilised at temperatures greater than 2400 K. This was because such high temperatures resulted in very large intrinsic defect concentrations such that defect behaviour could not be meaningfully examined (the system was completely defective). As discussed in § \ref{dis_form_energy_intrinsic} and § \ref{brouwer_discussion_intrinsic}, issues with modelling cubic \zirconia\ in DFT mean that calculated energy values may become unreliable, but are presented in this thesis for the purpose of completeness.

%\section{Cubic phase collapse}
%
%\begin{itemize}
%\item When some oxygen Frenkel defects were introduced to the cubic phase supercell, relaxation under constant volume conditions caused a collapse into a pseudo-tetragonal structure.
%\item This indicated that the cubic phase as modelled in DFT may not be fully stable.
%\item Further investigation indicated that the structure of a supercell of c-\zirconia\ broke down even with constrained symmetry, a result corroborated by Burr {et al}. \cite{burr2017importance}. 
%\end{itemize}

\subsection{Unit cells}

Having selected and optimised the parameters and functionals, unit cells of \zirconia\ in each phase were fully relaxed at constant pressure and the resulting structures were compared in detail to experimental data. Table \ref{lattice_params} shows the calculated lattice parameters and energy differences between the three \zirconia\ phases. 

The first thing to note is that the correct order of \zirconia\ phases is predicted in the total energy calculations, with monoclinic being the lowest energy phase and cubic being the highest. In addition, the energy difference between phases is small ($<$ 0.1 eV/fu). This is a good indication that the choice of exchange correlation functional can reproduce the energy landscape of the system accurately. This is especially important for when defects are introduced because they may promote stabilisation of one phase over another, and an inaccurate model will not capture this behaviour. In all cases the predicted cell volumes are consistently within approximately 2\% of experimental values. 

\begin{table}[ht] % Unit cell parameters
\onehalfspacing
\centering
\caption[Calculated unit cell parameters for the different crystal structures of \zirconia . Experimental data for pure monoclinic, yttria-stabilised tetragonal and magnesia-stabilised cubic phases at 295 K are shown in parentheses. Energy difference between structures is shown with respect to the cubic phase.]{Calculated unit cell parameters for the different crystal structures of \zirconia . Experimental data for monoclinic, tetragonal and cubic phases at 295 K are shown in parentheses \cite{Howard1988}. Energy difference between structures is shown with respect to the cubic phase.}
\label{lattice_params}
\resizebox{\textwidth}{!}{%
\begin{tabular}{ccccccc}
\hline Phase    & a (\AA) & b (\AA) & c (\AA) & $\beta$ ($\degree$) & Volume (\AA\textsuperscript{3}/fu) & $\Delta$E (eV/fu) \\ \hline
m-\zirconia   &    5.18 (5.15)          &    5.24 (5.21)         &    5.37 (5.32)         & 99.63 (99.23)             &       35.96 (35.22)                 &    -0.215              \\
t-\zirconia &    3.62 (3.61)         &              &    5.28  (5.18)        & 90             &   34.60 (33.75)                      &     -0.105             \\
c-\zirconia        &   5.11 (5.09)           &              &              & 90             &     33.36 (32.97)                   &      N/A     \\ \hline      
\end{tabular}}
\end{table}

\subsection{Electronic density of states} 

The electronic density of states for monoclinic, tetragonal and cubic \zirconia\ are generated in a two-step process. First, the non-defective structures are fully relaxed using the geometry optimisation task in CASTEP. This task will also calculate electronic eigenvalues for all k-points and save them to a \texttt{.bands} file. Second, the electronic band data is parsed from the \texttt{.bands} file using the OptaDOS code \cite{Nicholls2012, Morris2014} and the density of states is output to a text file. Further details on using OptaDOS to view the electronic density of states are given in Appendix \ref{castep_scripts}.

The electronic density of states for the three \zirconia\ phase are given in Figure \ref{figure:densityofstates}. In this figure, the valence and conduction bands can clearly be seen at 2-8 eV and 10-15 eV respectively. Most importantly, the energy values of the valence band maximum (VBM) and the conduction band minimum (CBM) for each phase can be obtained from this figure. These values are used to calculate the band gap in the different phases, shown in Figure \ref{table:bandgap} alongside experimental values. The VBM value is also used in the calculation of defect formation energies when electrons are added or removed from a system.

\begin{figure}[ht]
\begin{center}
\begin{tikzpicture}
	\begin{axis}
		[width=\linewidth*0.7, xlabel={Energy (eV)}, ylabel={Electronic density of states}, ymin=0, ymax=12, xmin=0, xmax=16, legend style={{draw=}, at={(0.05,0.95)}, anchor=north west, legend columns=1}]
       \addplot[no marks] table [x=mono_x, y=mono_y,]{dat/eDOS.dat}; \addlegendentry{Monoclinic};
       \addplot[no marks, dashed] table [x=tet_x, y=tet_y,]{dat/eDOS.dat}; \addlegendentry{Tetragonal};
       \addplot[no marks, densely dotted] table [x=cubic_x, y=cubic_y,]{dat/eDOS.dat}; \addlegendentry{Cubic};
			\end{axis}
		\end{tikzpicture}
		\caption{Electronic density of states for the different crystal structures of \zirconia\ showing the band gap predicted by DFT.}
		\label{figure:densityofstates}
	\end{center}
\end{figure}

The electronic density of states show that the VBM and CBM energies increase from monoclinic to tetragonal to cubic \zirconia . This means that at the same Fermi level, the total electronic energy will be smallest in the monoclinic phase and largest in the cubic phase. This corresponds to the correct order of thermal stability that is seen in experiments. Other features that can be seen are the band gaps of the different phases between 7 and 12 eV. These band gaps are significantly underestimated for each phase (see Table \ref{table:bandgap}), as is typical when using a GGA exchange-correlation functional.

\begin{table}[ht] % Band Gap
\onehalfspacing
\centering
\caption[Experimentally determined band gaps alongside values calculated from DFT simulations for each crystal structure of zirconia.]{Experimentally determined band gaps alongside values calculated from DFT simulations for each crystal structure of zirconia. Experimental values taken from \cite{French1994}.}
\begin{tabular}{ccc}
{\bf }                                       & \multicolumn{2}{c}{{\bf Band gap (eV)}}      \\ \hline
\multicolumn{1}{c}{{\bf Crystal Structure}} & \multicolumn{1}{c}{{\bf Expt.}} & {\bf DFT} \\ \hline
\multicolumn{1}{c}{Monoclinic}              & \multicolumn{1}{c}{5.83}        & 3.45      \\
\multicolumn{1}{c}{Tetragonal}              & \multicolumn{1}{c}{5.78}        & 4.00      \\
\multicolumn{1}{c}{Cubic}                   & \multicolumn{1}{c}{6.10}         &   3.55 \\ \hline
\label{table:bandgap}
\end{tabular}
\end{table}


\section{Frenkel and Schottky defects}

\subsection{Incorporation and defect formation energies}

\subsubsection*{Isolated Frenkel defects}

Zr and O Frenkel pair defect formation energies were determined via point defect DFT calculations for the three structures. The formation energies of the isolated Frenkel defect pairs were defined as: % Interstitial iodine defects were simulated in the neutral charge state at different interstitial sites in each phase. The incorporation energy of these defects, assuming a perfect lattice, was calculated using Equation \ref{equation_incorporation}:
\begin{equation}
\label{equation_frenkel}
E_{Frenkel} = E_{DFT}(V^{q}_{X}) + E_{DFT}(X^{-q}_{i}) - 2E_{DFT}(ZrO_2)% - \frac{E_{I_2}}{2}
\end{equation}

where $X$ is either Zr or O, $E_{DFT}(V^{q}_{X})$ is the energy of a supercell of \zirconia\ containing a single vacancy of charge $q$, $E_{DFT}(X^{-q}_{i})$ is the energy of a supercell of \zirconia\ containing a single interstitial with opposing charge $-q$, and $E_{DFT}(ZrO_2)$ is the energy of the non-defective supercell. Charges ranged from the fully charged case (+2 for oxygen vacancies, -4 for zirconium vacancies) to neutral. The interstitial sites, shown in Table \ref{table:interstitials}, were chosen based on standard vacant Wyckoff positions in each crystal structure \cite{theo1996international}.  In the case of oxygen vacancies in monoclinic \zirconia , a defect energy was obtained for both the (III) and (IV) co-ordinated oxygen sites, with the lowest energy value being used in the calculation of the Frenkel defect energy.

\begin{table}[ht] % Wyckoff positions of interstitials
\onehalfspacing
\centering
\caption{Wyckoff positions of interstitial sites used for each crystal structure.}
\label{table:interstitials}
\begin{tabular}{lcc}
\hline
\hspace{0.7 cm} {\bf Crystal Structure} \hspace{0.7 cm}                              & \hspace{0.7 cm} {\bf Interstitial Sites} \hspace{0.7 cm}                                               \\ \hline
\multicolumn{1}{c}{\textbf{Monoclinic}}              & $2a$, $2b$, $2c$, $2d$ \\
\multicolumn{1}{c}{\textbf{Tetragonal}}            & $2b$, $8e$                                   \\
\multicolumn{1}{c}{\textbf{Cubic}}       & $24d$, $4b$                                          \\ \hline
\end{tabular}
\end{table}

The isolated defect formation energies reported in Table \ref{isolated_defects} indicate that fully-charged Schottky defects have the lowest formation energy per atom (most energetically favourable) in all phases, followed by oxygen Frenkel defects and then zirconium Frenkel defects. A trend is seen where the high-temperature phases result in lower formation energies for both Schottky and oxygen Frenkel defects, whereas zirconium Frenkel defects have similar formation energies in all three phases. 

It has been suggested that the relatively small cation size leads to defect structures where oxygen vacancies are favoured over interstitial defects \cite{dwivedi1990computer}. As the zirconium ion is too small to maintain a strong 8-fold bond coordination with its neighbouring oxygen ions, the introduction of oxygen vacancies (which have the added effect of reducing cell volume) will have a stabilising effect.

\begin{table}[ht] % Isolated formation energies
\onehalfspacing
\centering
\caption{Formation energies in eV of isolated \zirconia\ defects.}
\label{isolated_defects}
\begin{tabular}{cccll}
\hline
\multirow{2}{*}{\textbf{Defect}}                      & \multirow{2}{*}{\textbf{Equation}}                                        & \multicolumn{3}{c}{\textbf{Formation Energy (eV)}} \\ \cline{3-5}
	&	& \multicolumn{1}{l}{Monoclinic} & Tetragonal & Cubic \\ \hline
\multirow{5}{*}{\textbf{Zr Frenkel}} & \ch{Zr_{Zr}^{x}} $\rightarrow$ \ch{V_{Zr}^{''''}} + \ch{Zr_{i}^{****}}              & 5.428 & 5.639 & 5.610                             \\
                                     & \ch{Zr_{Zr}^{x}} $\rightarrow$ \ch{V_{Zr}^{'''}} + \ch{Zr_{i}^{***}}               & 8.695 & 8.939 & 8.476                            \\
                                     & \ch{Zr_{Zr}^{x}} $\rightarrow$ \ch{V_{Zr}^{''}} + \ch{Zr_{i}^{**}}                & 12.118 & 12.058 & 11.628                             \\
                                     & \ch{Zr_{Zr}^{x}} $\rightarrow$ \ch{V_{Zr}^{'}} + \ch{Zr_{i}^{*}}                & 16.021 &	15.696 &	13.319                             \\
                                     & \ch{Zr_{Zr}^{x}} $\rightarrow$ \ch{V_{Zr}^{x}} + \ch{Zr_{i}^{x}}                  & 20.563	& 20.094 &	18.170                            \\ \hline
\multirow{3}{*}{\textbf{O Frenkel}}  & \ch{O_{O}^{x}} $\rightarrow$ \ch{V_{O}^{**}} + \ch{O_{i}^{''}}                   & 4.457 &	4.000 & 	3.728                             \\
                                     & \ch{O_{O}^{x}} $\rightarrow$ \ch{V_{O}^{*}} + \ch{O_{i}^{'}}                   & 6.432	& 6.588 &	7.055                             \\
                                     & \ch{O_{O}^{x}} $\rightarrow$ \ch{V_{O}^{x}} + \ch{O_{i}^{x}}                     & 7.518 &	7.452 &	8.477                             \\ \hline
\multirow{3}{*}{\textbf{Schottky}}   & $\varnothing$ $\rightarrow$ \ch{V_{Zr}^{''''}} + 2\ch{V_{O}^{**}} & 5.120 &	3.778	& 1.752                             \\
                                     & $\varnothing$ $\rightarrow$ \ch{V_{Zr}^{''}} + 2\ch{V_{O}^{*}} & 11.353 &	10.832 &	9.624                             \\
                                     & $\varnothing$ $\rightarrow$ \ch{V_{Zr}^{x}} + 2\ch{V_{O}^{x}}   & 18.554 &	18.232 &	17.073  \\ \hline                          
\end{tabular}
\end{table}

\subsubsection*{Bound Frenkel Defects}

Bound Zr and O Frenkel defect formation energies were calculated from DFT energies of supercells where a single ion was moved from its lattice site to an interstitial site. The formation energies of the bound Frenkel defect pairs were defined as:
% Interstitial iodine defects were simulated in the neutral charge state at different interstitial sites in each phase. The incorporation energy of these defects, assuming a perfect lattice, was calculated using Equation \ref{equation_incorporation}:

\begin{equation}
\label{equation_frenkel_bound}
E_{BoundFrenkel} = E_{DFT}(BoundFrenkel) - E_{DFT}(ZrO_2)% - \frac{E_{I_2}}{2}
\end{equation}

where $E_{DFT}(BoundFrenkel)$ is the energy of a supercell of \zirconia\ containing both a vacancy and interstitial defect of the same ion. The two defects were placed as far apart in the supercell as possible (7-8 \r{A}) to avoid recombination. The interstitial defect is assumed to fully compensate the charge of the vacancy defect, resulting in no overall charge on the supercell. The number and type of ions in the defective and non-defective supercell are the same, requiring no further steps to calculate the formation energy. The formation energies calculated for these defects in each crystal structure are presented in Table \ref{table:bound_defects}.

\begin{table}[ht] % Bound formation energies
%\setlength{\tabcolsep}{10pt} % Default value: 6pt
\onehalfspacing
\centering
\caption{Formation energies of bound defects in \zirconia.}
\label{table:bound_defects}
\begin{tabular}{cccc}
\hline
\multirow{2}{*}{\textbf{Defect}} & \multicolumn{3}{c}{\textbf{Formation Energy (eV)}} \\ \cline{2-4} 
 & \textbf{Monoclinic} & \textbf{Tetragonal} & \textbf{Cubic} \\ \hline
\textbf{O Frenkel} & 4.1212 & 4.0290 & 6.4397 \\
\textbf{Zr Frenkel} & 8.4232 & 7.8633 & 6.3274 \\
\textbf{NTV1} & 5.2272 & 3.5813 & 2.6961 \\
\textbf{NTV2} & 5.1405 & 4.2312 & 0.1798 \\
\textbf{NTV3} & 4.6620 & 3.3623 & 2.4089 \\ \hline
\end{tabular}
\end{table}

%Charges ranged from the fully charged case (+2 for oxygen, -4 for zirconium) to neutral. The interstitial sites, shown in Table \ref{table:interstitials}, were chosen based on standard vacant Wyckoff positions in each crystal structure \cite{theo1996international}.  In the case of oxygen vacancies in monoclinic \zirconia , a defect energy was obtained for both the (III) and (IV) co-ordinated oxygen sites. The lowest energies were used in the calculation of the Frenkel defect energy.

\subsubsection*{Isolated Schottky Defects}

Three Schottky energies were calculated for each structure, corresponding to fully charged, partially charged, and uncharged point defect energies. The Schottky formation energy was defined as:

\begin{equation}
\label{equation_schottky}
E_{Schottky} = E_{DFT}(V^{-2q}_{Zr}) + 2E_{DFT}(V^{q}_{O}) -\frac{3(n-1)}{n}E_{DFT}(ZrO_2)% - \frac{E_{I_2}}{2}
\end{equation}

where $n$ denotes the number of atoms in the supercell, $V^{q}_{O}$ denotes an oxygen vacancy with charge $q$, where $q$ varies from 2 to 0. This form maintains both the mass and charge balance of the Schottky defect description for \zirconia :

\begin{equation}
\label{generic_schottky}
Zr^{x}_{Zr} + 2O^{x}_{O} = V^{-2q}_{Zr} + 2V^{q}_{O} + ZrO_{2}
\end{equation}

This implies a rearrangement rather than complete removal of ions from the system. As with the Frenkel defects, the lowest energy vacancy energies were used to calculate Schottky formation energies. While there are multiple configurations of Schottky defects, such nuance cannot be accurately represented through a sum of individual vacancy defect energies. The values presented for Schottky defect formation energies should therefore be considered the lower bound for defect formation. 


\subsubsection*{Bound Schottky Defects}


\begin{figure}[ht] % Tet Zr centre
\centering
\includegraphics[width=8cm]{images/zr_centre_tet.png}
\caption{Zirconium centre  cell showing nearest oxygen atoms in tetragonal \zirconia. Schottky trios indicated by oxygen enumeration with Zr, O and a second oxygen in either the 1\textsuperscript{st}, 2\textsuperscript{nd} or 3\textsuperscript{rd} nearest neighbour with respect to the initial oxygen. Zirconium atoms are shown in green and oxygen atoms in red.}
\label{figure:tetschottky}
\end{figure}

\begin{figure}[ht] % Cubic Zr centre
\centering
\includegraphics[width=8cm]{images/sd_cubic_zro2.png}
\caption{Zirconium centre cell showing nearest oxygen atoms in cubic \zirconia. Schottky trios indicated by oxygen enumeration with Zr, O and a second oxygen in either the 1\textsuperscript{st}, 2\textsuperscript{nd} or 3\textsuperscript{rd} nearest neighbour with respect to the initial oxygen.. Zirconium atoms are shown in green and oxygen atoms in red.}
\label{figure:cubicschottky}
\end{figure}

Bound Schottky defects were modelled in a supercell of \zirconia\ by removing one Zr and two O atoms, in one of several possible nearest neighbour configurations as shown in Figures \ref{figure:monoschottky}, \ref{figure:tetschottky} and \ref{figure:cubicschottky}. Charge neutrality is maintained by the removal of a stoichiometric unit, therefore these defects were defined as neutral tri-vacancies (NTVs). The NTV formation energy was defined as:
\begin{equation}
\label{equation_NTV}
E_{NTV} = E_{DFT}(NTV) - \frac{n-3}{n}E_{DFT}(ZrO_2)% - \frac{E_{I_2}}{2}
\end{equation}

Where $E_{DFT}(NTV)$ is the energy of a supercell containing the NTV defect. As the defective supercell contains three fewer ions than the non-defective cell, the energy of the non-defective cell was adjusted by a proportional factor in our calculation. This form maintains both mass and charge balance of the Schottky defect description for \zirconia\ described in Equation \ref{generic_schottky}.




\section{Defect formation energies} \label{dis_form_energy_intrinsic}

\begin{figure}[ht] % Mono vacancies Fermi level
\begin{center}
\begin{tikzpicture}
	\begin{axis}
		[width=11cm, xlabel={Fermi level $\mu_{e}$ (eV)}, ylabel={Formation energy (eV) per \zirconia\ }, ymin=-10, ymax=18, xmin=0, xmax=6, legend style={{draw=}, at={(0.95,0.95)}, anchor=north east, legend columns=1}]
		\addplot[no marks, blue] table [x=ZRmono1, y=ZRmono2,]{dat/vacancies.dat}; \addlegendentry{Zr};
        \addplot[no marks, red, dashed] table [x=O3mono1, y=O3mono2,]{dat/vacancies.dat}; \addlegendentry{O (III)};
        \addplot[no marks, red] table [x=O4mono1, y=O4mono2,]{dat/vacancies.dat}; \addlegendentry{O (IV)};
			\end{axis}
		\end{tikzpicture}
		\caption{Monoclinic phase formation energies of intrinsic vacancy defects as a function of Fermi level. Gradient indicates defect charge. Oxygen coordination number shown in legend.}
		\label{figure:monovacancies}
	\end{center}
\end{figure}


\begin{figure}[ht] % Tet vacancies Fermi level
\begin{center}
\begin{tikzpicture}
	\begin{axis}
		[width=11cm, xlabel={Fermi level $\mu_{e}$ (eV)}, ylabel={Formation energy (eV) per \zirconia\ }, ymin=-10, ymax=18, xmin=0, xmax=6, legend style={{draw=}, at={(0.95,0.95)}, anchor=north east, legend columns=1}]
		\addplot[no marks, blue] table [x=ZRtet1, y=ZRtet2,]{dat/vacancies.dat}; \addlegendentry{Zr};
        \addplot[no marks, red] table [x=Otet1, y=Otet2,]{dat/vacancies.dat}; \addlegendentry{O};
			\end{axis}
		\end{tikzpicture}
		\caption{Tetragonal phase formation energies of intrinsic vacancy defects as a function of Fermi level. Gradient indicates defect charge.}
		\label{figure:tetvacancies}
	\end{center}
\end{figure}

\begin{figure}[ht]
\begin{center}
\begin{tikzpicture}
	\begin{axis}
		[width=11cm, xlabel={Fermi level $\mu_{e}$ (eV)}, ylabel={Formation energy (eV) per \zirconia\ }, ymin=-10, ymax=18, xmin=0, xmax=6, legend style={{draw=}, at={(0.95,0.95)}, anchor=north east, legend columns=1}]
		\addplot[no marks, blue] table [x=ZRcubic1, y=ZRcubic2,]{dat/vacancies.dat}; \addlegendentry{Zr};
        \addplot[no marks, red] table [x=Ocubic1, y=Ocubic2,]{dat/vacancies.dat}; \addlegendentry{O};
			\end{axis}
		\end{tikzpicture}
		\caption{Cubic phase formation energies of intrinsic vacancy defects as a function of Fermi level. Gradient indicates defect charge.}
		\label{figure:cubicvacancies}
	\end{center}
\end{figure}


\subsection{Defect Volumes}

Tables \ref{defect_volumes_raw} and \ref{defect_volumes_clusters_isolated} show the calculated point defect and cluster defect volumes respectively. The Frenkel and Schottky defect volumes are calculated from the sum of the relevant point defects that would result in an overall neutral charge, with clusters of fully-charged point defects being the expected defect structures in a real material.

\begin{table}[ht!] % Isolated defect volumes
\onehalfspacing
\centering
\caption{Isolated defect volumes in the three \zirconia\ structures.}
\label{defect_volumes_raw}
\begin{tabular}{cccc}
\hline
                      & \multicolumn{3}{c}{\textbf{Defect volume relative to non-defective cell (\r{A}\textsuperscript{3})}}  \\ \cline{2-4} 
\textbf{Defect}       & \textbf{Monoclinic} & \hspace{1cm} \textbf{Tetragonal} & \textbf{Cubic} \\ \hline
\ch{V_{Zr}^{''''}}             & 55.95             & 67.41             & 48.47         \\
\ch{V_{Zr}^{'''}}             & 42.48             &         51.08     &     36.94      \\
\ch{V_{Zr}^{''}}            & 29.90             &  34.28            &     25.93           \\
\ch{V_{Zr}^{'}}             & 17.10             &  18.43            &     15.08           \\
\ch{V_{Zr}^{x}}              & 4.06             &  4.70            &    4.32       \\
\ch{Zr_{i}^{****}}             & -34.62            & -41.94            & -27.34       \\
\ch{Zr_{i}^{***}}             &  -22.76           &	-27.74 		  &	-16.95         \\
\ch{Zr_{i}^{**}}             &  -11.79 	        &	-12.02 		  &	-6.24          \\
\ch{Zr_{i}^{*}}            &  2.68			& -0.02 		  & 	4.69             \\
\ch{Zr_{i}^{x}}              &  15.94		 	& 13.40	 		  & 15.97         \\
\ch{V_{O}^{**}} {[}4coord{]} & -22.52            & -37.53            & -22.76       \\
\ch{V_{O}^{*}} {[}4coord{]} &  -12.41           &    -19.53         &     -12.19           \\
\ch{V_{O}^{x}} {[}4coord{]}  &  -0.69          &  -2.80           &      -1.11          \\
\ch{V_{O}^{**}} {[}3coord{]} & -26.13            &                     &                \\
\ch{V_{O}^{*}} {[}3coord{]} &  -14.42           &                     &                \\
\ch{V_{O}^{x}} {[}3coord{]}  &   -1.71          &                     &                \\
\ch{O_{i}^{''}}              & 27.01             & 40.00              & 28.58        \\
\ch{O_{i}^{'}}              &  15.36            &    24.56         &  16.30              \\
\ch{O_{i}^{x}}               & 2.66             &    11.06          &   8.95        \\ \hline
\end{tabular}
\end{table}

\begin{table}[ht] % Isolated Frenkel volumes
\onehalfspacing
\centering
\caption{Isolated cluster defect volumes in the three \zirconia\ structures.}
\label{defect_volumes_clusters_isolated}
\begin{tabular}{cccc}
\hline
\multirow{2}{*}{\textbf{Defect}}   & \multicolumn{3}{c}{\textbf{Defect volume (\r{A}\textsuperscript{3})}}  \\ \cline{2-4} 
 & \textbf{Monoclinic} & \textbf{Tetragonal} & \textbf{Cubic} \\ \hline
\ch{V_{Zr}^{''''}} + \ch{Zr_{i}^{****}}          & 21.331	 & 25.4702 &	21.1309         \\
\ch{V_{Zr}^{'''}} + \ch{Zr_{i}^{***}}          & 19.7155 &	23.3463 &	19.9954      \\
\ch{V_{Zr}^{''}} + \ch{Zr_{i}^{**}}          & 18.1149 &	22.2525 &	19.68618           \\
\ch{V_{Zr}^{'}} + \ch{Zr_{i}^{*}}          & 19.78339 &	18.4096913 &	19.76396           \\
\ch{V_{Zr}^{x}} + \ch{Zr_{i}^{x}}          & 19.99485 &	18.1061 &	20.29223       \\
\ch{V_{O}^{**}} + \ch{O_{i}^{''}}           & 0.8839 &	2.4704 &	5.8217       \\
\ch{V_{O}^{*}} + \ch{O_{i}^{'}}           &  0.9486 &	5.032 &	4.1146        \\
\ch{V_{O}^{x}} + \ch{O_{i}^{x}}           &  0.9576 &	8.26065 &	7.83687          \\
\ch{V_{Zr}^{''''}} + 2\ch{V_{O}^{**}}       &  3.6979 &	-7.647 &	2.9448             \\
\ch{V_{Zr}^{''}} + 2\ch{V_{O}^{*}}       &  1.0707 &	-4.7866 &	1.5564         \\
\ch{V_{Zr}^{x}} + 2\ch{V_{O}^{x}}        & 0.64517 &	-0.8985 &	2.08973       \\ \hline
\end{tabular}
\end{table}

The oxygen Frenkel defect has the smallest defect volume in the monoclinic phase, followed by the tetragonal phase. This can be explained by the competition between phase density and matrix stiffness. As the monoclinic phase has the highest specific volume (see Table \ref{lattice_params}), we can argue that the monoclinic phase can best absorb the lattice strains imposed by the defect, despite having a lower stiffness than the cubic phase.

The zirconium Frenkel defect is significantly larger than the oxygen Frenkel, mostly due to the large positive strain contribution from the zirconium vacancy. This can explain the larger defect formation energy of zirconium Frenkel defects.

%\subsubsection*{Isolated Defects}
%
%The isolated defect formation energies reported in Table \ref{isolated_defects} indicate that fully-charged Schottky defects have the lowest formation energy per atom (most energetically favourable) in all phases, followed by oxygen Frenkel defects. A trend is seen where the high-temperature phases result in lower formation energies for both Schottky and oxygen Frenkel defects, whereas zirconium Frenkel defects have similar formation energies in all three phases. It has been suggested that the relatively small cation size leads to defect structures where oxygen vacancies are favoured over interstitial defects \cite{dwivedi1990computer}. As the zirconium ion is too small to maintain a strong 8-fold bond coordination with its neighbouring oxygen ions, the introduction of oxygen vacancies (which have the added effect of reducing cell volume) will have a stabilising effect.


\subsubsection*{Bound Defects}
The bound defect formation energies shown in Table \ref{table:bound_defects} show that NTV defects, on a per defect atom basis, are the most energetically favourable defects, followed by oxygen and zirconium Frenkel defects respectively. The NTV3 exhibited the smallest formation energy in all three crystal structures, with a single exception of the NTV2 in the cubic phase where a much smaller formation energy was observed due to collapse\footnote{Upon inspecting the output cell, it was found that all the oxygen atoms shifted positions along the [001] direction, becoming more like the tetragonal \zirconia\ structure. The cell size was constrained so the lattice parameters could not be changed, so this was not a true tetragonal cell.} of the supercell during geometry optimisation.

\section{Elastic constants and defect relaxation volumes}

Table \ref{stiffness_tensor} shows the calculated elastic constants for the monoclinic, tetragonal, and cubic phases of \zirconia . The cubic phase has the highest stiffness, likely due to the short Zr-O bond lengths in the energy-minimised structure (Figure \ref{figure:zrobonddistance}). It is expected that at high temperatures where the cubic phase is stable, the resulting increase in bond length would cause a reduction in stiffness. 

\begin{table}[ht] % Elastic constants
\onehalfspacing
\centering
\caption{Elastic constants for different phases of \zirconia\ from DFT calculations.}
\label{stiffness_tensor}
\begin{tabular}{cccc}
\hline
\multirow{2}{*}{\textbf{Elastic Component}} & \multicolumn{3}{c}{\textbf{Stiffness (GPa)}}               \\ \cline{2-4} 
                                            & \textbf{Monoclinic} & \textbf{Tetragonal} & \textbf{Cubic} \\ \hline
$C_{11}$                                         & 338.86        & 334.30               & 523.38    \\
$C_{12}$                                         & 151.80        & 207.30               & 92.93     \\
$C_{13}$                                         & 89.37         & 48.93               & 92.93     \\
$C_{22}$                                         & 348.37        & 334.20               & 523.39    \\
$C_{23}$                                         & 143.04        & 48.93               & 92.93    \\
$C_{33}$                                         & 262.17        & 250.50               & 523.38   \\
$C_{44}$                                         & 76.35         & 9.38                & 61.98    \\
$C_{55}$                                         & 71.65         & 9.38                & 61.98   \\
$C_{66}$                                         & 114.19        & 152.60               & 61.99     \\ \hline
\end{tabular}
\end{table}


The monoclinic and tetragonal phases have similar stiffness along the principal axes, but vary significantly under shearing conditions. In particular, the tetragonal phase exhibits much smaller $C_{44}$ and $C_{55}$ components. This may be attributed to the strong directional anisotropy of the tetragonal phase due to the larger $c$ parameter. 





\section{Helmholtz energies}

%The calculated Helmholtz energies plotted in Figure \ref{Figure:helmholtz} show the correct order of stability for the three phases of \zirconia\ at low temperatures (monoclinic $\rightarrow$ tetragonal $\rightarrow$ cubic) . However, as temperature is increased, only a transition from monoclinic to tetragonal is seen. The cubic phase Helmholtz energy does fall below the monoclinic curve, but never below the tetragonal curve, thus predicting no tetragonal to cubic phase transition. 

%It must also be noted that the transition temperatures are not predicted accurately. The tetragonal phase is predicted to have a lower energy than monoclinic at approximately 400 K, while experiments indicate that the transition temperature is above 1400 K. This difference is too large to attribute to a kinetic barrier.

The Helmholtz free energy results (Figure \ref{Figure:helmholtz}) show the correct order of crystal structure stability at low temperature. A transition from the monoclinic to tetragonal crystal structure is seen at 390K, but no further transition is seen from tetragonal to cubic. The low monoclinic-tetragonal transition temperature may be due to both the kinetic barrier \cite{bansal1972martensitic,bansal1974martensitic}, and the inability of the constant volume harmonic model to take into account the effects of thermal expansion. The lack of an observed transition to the cubic phase may indicate an inability to accurately simulate the high-temperature phase using DFT techniques. 

\begin{figure}[ht] % Helmholtz
\begin{center}
\begin{tikzpicture}
	\begin{axis}
		[width=\linewidth*0.7, xlabel={Temperature (K)}, ylabel={Helmholtz free energy (eV)}, ymin=-28, ymax=-10, xmin=0, xmax=3000, legend style={{draw=}, at={(0.95,0.95)}, anchor=north east, legend columns=1}]
		\addplot[no marks] table [x=temperature, y=monoclinic,]{dat/helmholtz.dat}; \addlegendentry{Monoclinic};
        \addplot[no marks, dashed] table [x=temperature, y=tetragonal, ]{dat/helmholtz.dat}; \addlegendentry{Tetragonal};
        \addplot[no marks, densely dotted, black] table [x=temperature, y=cubic,]{dat/helmholtz.dat}; \addlegendentry{Cubic};
			\end{axis}
		\end{tikzpicture}
		\caption{Helmholtz free energy as a function of temperature for the monoclinic, tetragonal, and cubic crystal structures of \zirconia .}
		\label{Figure:helmholtz}
	\end{center}
\end{figure}

\section{Defect equilibria} \label{brouwer_discussion_intrinsic}

\subsubsection*{Monoclinic}

The monoclinic Brouwer diagram (Figure \ref{figure:mono_intrinsic_brouwer}) predicts that at 635 K, few types of defects will be present and at very low (\textless 10 ppb \zirconia ) concentrations. This is typical of defect behaviour in a ceramics at temperatures far below their melting points \cite{kingery1997physical,ball2006computer}. Fully-charged zirconium vacancies, charge-compensated by holes, are the major defect type we expect to observe at $p_{O_{2}}$ \textgreater $10^{-15}$ atm. Below this, only electronic defects compensated by electron hole defects are expected. 

%We briefly see increased concentrations of uncharged oxygen interstitial defects at very high levels of $p_{O_{2}}$.
%The intrinsic defect equilibria are shown in Brouwer diagrams in Figures \ref{figure:mono_intrinsic_brouwer}, \ref{figure:tet_intrinsic_brouwer} and \ref{figure:cubic_intrinsic_brouwer}. The monoclinic phase exhibits the smallest overall concentration of intrinsic defects due to the low temperature (650 K) relative to the other tetragonal (1500 K) and cubic (XXX K) phases.

\begin{figure}[ht] % Mono intrinsic Brouwer
\begin{center}
\begin{tikzpicture}
	\begin{axis}
		[width=\linewidth*0.7, xlabel={\ch{log_{10}}($p_{O_{2}}$) (atm)}, ylabel={\ch{log_{10}}([D]) (per f.u.)}, ymin=-18, ymax=0, xmin=-35, xmax=0, legend style={{draw=}, at={(0.40,0.94)}, anchor=north west, legend columns=4, nodes={scale=1, transform shape}}]
        \addplot[no marks, draw=blue!70!black] table [x=pO2, y=electrons,]{dat/intrinsic_mono.dat}; \addlegendentry{\ch{e^{'}}}; \node at (-4.8,-13) {\ch{e^{'}}};
        \addplot[no marks, draw=red!85!black] table [x=pO2, y=holes,]{dat/intrinsic_mono.dat}; \addlegendentry{\ch{h^{\textperiodcentered}}}; \node at (-4.5,-8) {\ch{h^{\textperiodcentered}}};
        \addplot[no marks, draw=black!70!green] table [x=pO2, y=VO{2},]{dat/intrinsic_mono.dat}; \addlegendentry{\ch{V_{O}^{\textperiodcentered\textperiodcentered}}}; \node at (-33,-16.5) {\ch{V_{O}^{\textperiodcentered\textperiodcentered}}};
%         \addplot[no marks, draw=black!55!green] table [x=pO2, y=VO{1},]{dat/intrinsic_mono.dat}; \addlegendentry{\ch{V_{O}^{*}}};
%         \addplot[no marks, draw=black!30!green] table [x=pO2, y=VO{0},]{dat/intrinsic_mono.dat}; \addlegendentry{\ch{V_{O}^{x}}};
        \addplot[no marks, draw=yellow!85!blue] table [x=pO2, y=VM{-4},]{dat/intrinsic_mono.dat}; \addlegendentry{\ch{V_{Zr}^{''''}}}; \node at (-5,-10.5) {\ch{V_{Zr}^{''''}}};
%         \addplot[no marks, draw=yellow!75!blue] table [x=pO2, y=VM{-3},]{dat/intrinsic_mono.dat}; \addlegendentry{\ch{V_{Zr}^{'''}}};
%         \addplot[no marks, draw=yellow!65!blue] table [x=pO2, y=VM{-2},]{dat/intrinsic_mono.dat}; \addlegendentry{\ch{V_{Zr}^{''}}};
%         \addplot[no marks, draw=yellow!55!blue] table [x=pO2, y=VM{-1},]{dat/intrinsic_mono.dat}; \addlegendentry{\ch{V_{Zr}^{'}}};
%         \addplot[no marks, draw=yellow!45!blue] table [x=pO2, y=VM{0},]{dat/intrinsic_mono.dat}; \addlegendentry{\ch{V_{Zr}^{x}}};
%         \addplot[no marks, draw=red!60!yellow] table [x=pO2, y=Oi{-2},]{dat/intrinsic_mono.dat}; \addlegendentry{\ch{O_{i}^{''}}};
%         \addplot[no marks, draw=red!50!yellow] table [x=pO2, y=Oi{-1},]{dat/intrinsic_mono.dat}; \addlegendentry{\ch{O_{i}^{'}}};
%         \addplot[no marks, draw=red!40!yellow] table [x=pO2, y=Oi{0},]{dat/intrinsic_mono.dat}; \addlegendentry{\ch{O_{i}^{x}}};
%         \addplot[no marks, draw=green!80!pink] table [x=pO2, y=Mi{4},]{dat/intrinsic_mono.dat}; \addlegendentry{\ch{Zr_{i}^{****}}};
%         \addplot[no marks, draw=green!70!pink] table [x=pO2, y=Mi{3},]{dat/intrinsic_mono.dat}; \addlegendentry{\ch{Zr_{i}^{***}}};
%         \addplot[no marks, draw=green!60!pink] table [x=pO2, y=Mi{2},]{dat/intrinsic_mono.dat}; \addlegendentry{\ch{Zr_{i}^{\textbf{**}}}};
%         \addplot[no marks, draw=green!50!pink] table [x=pO2, y=Mi{1},]{dat/intrinsic_mono.dat}; \addlegendentry{\ch{Zr_{i}^{*}}};
%         \addplot[no marks, draw=green!40!pink] table [x=pO2, y=Mi{0},]{dat/intrinsic_mono.dat}; \addlegendentry{\ch{Zr_{i}^{x}}};
%         \addplot[no marks] table [x=pO2, y=Stoich,]{dat/intrinsic_mono.dat}; \addlegendentry{Stoich};
%\node at (-33.7,-0.5) {\textbf{a)}};
			\end{axis}            
\end{tikzpicture}
		\caption{Monoclinic phase Brouwer diagram of intrinsic defects at 650 K.}
		\label{figure:mono_intrinsic_brouwer}
	\end{center}
\end{figure}


\subsubsection*{Tetragonal}

Figure \ref{figure:tet_intrinsic_brouwer} shows a much greater concentration of defects across a wide range of $p_{O_{2}}$, mainly owing to an elevated temperature of 1500 K where the tetragonal crystal structure is fully stabilised. At low levels of $p_{O_{2}}$, electronic defects are again the dominant defect, but are now charge-compensated by the formation of fully-charged oxygen vacancies. A clear neutrality condition is seen at a $p_{O_{2}}$ of $10^{-11}$ atm where $[\ch{V_{O}^{**}}] = 2[\ch{V_{Zr}^{''''}}]$, with higher levels of $p_{O_{2}}$ being dominated by fully-charged zirconium vacancies charge-compensated by the formation of electron hole defects.

\begin{figure}[ht] % Tet intrinsic Brouwer
\begin{center}
\begin{tikzpicture}
	\begin{axis}
		[width=\linewidth*0.7, xlabel={\ch{log_{10}}($p_{O_{2}}$) (atm)}, ylabel={\ch{log_{10}}([D]) (per f.u.)}, ymin=-10, ymax=0, xmin=-35, xmax=0, legend style={{draw=}, at={(0.40,0.97)}, anchor=north west, legend columns=4, nodes={scale=1, transform shape}}]
        \addplot[no marks, draw=blue!70!black] table [x=pO2, y=electrons,]{dat/intrinsic_tet.dat}; \addlegendentry{\ch{e^{'}}}; \node at (-26.0,-2) {\ch{e^{'}}};
        \addplot[no marks, draw=red!85!black] table [x=pO2, y=holes,]{dat/intrinsic_tet.dat}; \addlegendentry{\ch{h^{\textperiodcentered}}}; \node at (-7,-3.6) {\ch{h^{\textperiodcentered}}};
        \addplot[no marks, draw=black!70!green] table [x=pO2, y=VO{2},]{dat/intrinsic_tet.dat}; \addlegendentry{\ch{V_{O}^{\textperiodcentered\textperiodcentered}}}; \node at (-28,-3) {\ch{V_{O}^{\textperiodcentered\textperiodcentered}}};
%         \addplot[no marks, draw=black!55!green] table [x=pO2, y=VO{1},]{dat/intrinsic_tet.dat}; \addlegendentry{\ch{V_{O}^{*}}};
%         \addplot[no marks, draw=black!30!green] table [x=pO2, y=VO{0},]{dat/intrinsic_tet.dat}; \addlegendentry{\ch{V_{O}^{x}}};
        \addplot[no marks, draw=yellow!85!blue] table [x=pO2, y=VM{-4},]{dat/intrinsic_tet.dat}; \addlegendentry{\ch{V_{Zr}^{''''}}}; \node at (-3,-4.5) {\ch{V_{Zr}^{''''}}};
%         \addplot[no marks, draw=yellow!75!blue] table [x=pO2, y=VM{-3},]{dat/intrinsic_tet.dat}; \addlegendentry{\ch{V_{Zr}^{'''}}};
%         \addplot[no marks, draw=yellow!65!blue] table [x=pO2, y=VM{-2},]{dat/intrinsic_tet.dat}; \addlegendentry{\ch{V_{Zr}^{''}}};
%         \addplot[no marks, draw=yellow!55!blue] table [x=pO2, y=VM{-1},]{dat/intrinsic_tet.dat}; \addlegendentry{\ch{V_{Zr}^{'}}};
%         \addplot[no marks, draw=yellow!45!blue] table [x=pO2, y=VM{0},]{dat/intrinsic_tet.dat}; \addlegendentry{\ch{V_{Zr}^{x}}};
%         \addplot[no marks, draw=red!60!yellow] table [x=pO2, y=Oi{-2},]{dat/intrinsic_tet.dat}; \addlegendentry{\ch{O_{i}^{''}}};
%         \addplot[no marks, draw=red!50!yellow] table [x=pO2, y=Oi{-1},]{dat/intrinsic_tet.dat}; \addlegendentry{\ch{O_{i}^{'}}};
%         \addplot[no marks, draw=red!40!yellow] table [x=pO2, y=Oi{0},]{dat/intrinsic_tet.dat}; \addlegendentry{\ch{O_{i}^{x}}};
%         \addplot[no marks, draw=green!80!pink] table [x=pO2, y=Mi{4},]{dat/intrinsic_tet.dat}; \addlegendentry{\ch{Zr_{i}^{****}}};
%         \addplot[no marks, draw=green!70!pink] table [x=pO2, y=Mi{3},]{dat/intrinsic_tet.dat}; \addlegendentry{\ch{Zr_{i}^{***}}};
%         \addplot[no marks, draw=green!60!pink] table [x=pO2, y=Mi{2},]{dat/intrinsic_tet.dat}; \addlegendentry{\ch{Zr_{i}^{\textbf{**}}}};
%         \addplot[no marks, draw=green!50!pink] table [x=pO2, y=Mi{1},]{dat/intrinsic_tet.dat}; \addlegendentry{\ch{Zr_{i}^{*}}};
%         \addplot[no marks, draw=green!40!pink] table [x=pO2, y=Mi{0},]{dat/intrinsic_tet.dat}; \addlegendentry{\ch{Zr_{i}^{x}}};
%         \addplot[no marks] table [x=pO2, y=Stoich,]{dat/intrinsic_tet.dat}; \addlegendentry{Stoich};
%\node at (-33.7,-0.5) {\textbf{a)}};
			\end{axis}            
\end{tikzpicture}
		\caption{Tetragonal phase Brouwer diagrams of intrinsic defects at 1500 K.}
		\label{figure:tet_intrinsic_brouwer}
	\end{center}
\end{figure}

\begin{figure}[ht] % cubic intrinsic Brouwer
\begin{center}
\begin{tikzpicture}
	\begin{axis}
		[width=\linewidth*0.7, xlabel={\ch{log_{10}}($p_{O_{2}}$) (atm)}, ylabel={\ch{log_{10}}([D]) (per f.u.)}, ymin=-10, ymax=0, xmin=-35, xmax=0, legend style={{draw=}, at={(0.60,0.97)}, anchor=north west, legend columns=3, nodes={scale=1, transform shape}}]
        \addplot[no marks, draw=blue!70!black] table [x=pO2, y=electrons,]{dat/intrinsic_cubic.dat}; \addlegendentry{\ch{e^{'}}}; \node at (-17.0,-1) {\ch{e^{'}}};
        \addplot[no marks, draw=red!85!black] table [x=pO2, y=holes,]{dat/intrinsic_cubic.dat}; \addlegendentry{\ch{h^{\textperiodcentered}}}; \node at (-8,-7.5) {\ch{h^{\textperiodcentered}}};
        \addplot[no marks, draw=black!70!green] table [x=pO2, y=VO{2},]{dat/intrinsic_cubic.dat}; \addlegendentry{\ch{V_{O}^{\textperiodcentered\textperiodcentered}}}; \node at (-20,-1.7) {\ch{V_{O}^{\textperiodcentered\textperiodcentered}}};
%         \addplot[no marks, draw=black!55!green] table [x=pO2, y=VO{1},]{dat/intrinsic_cubic.dat}; \addlegendentry{\ch{V_{O}^{*}}};
%         \addplot[no marks, draw=black!30!green] table [x=pO2, y=VO{0},]{dat/intrinsic_cubic.dat}; \addlegendentry{\ch{V_{O}^{x}}};
        \addplot[no marks, draw=yellow!85!blue] table [x=pO2, y=VM{-4},]{dat/intrinsic_cubic.dat}; \addlegendentry{\ch{V_{Zr}^{''''}}}; \node at (-20,-6) {\ch{V_{Zr}^{''''}}};
%         \addplot[no marks, draw=yellow!75!blue] table [x=pO2, y=VM{-3},]{dat/intrinsic_cubic.dat}; \addlegendentry{\ch{V_{Zr}^{'''}}};
%         \addplot[no marks, draw=yellow!65!blue] table [x=pO2, y=VM{-2},]{dat/intrinsic_cubic.dat}; \addlegendentry{\ch{V_{Zr}^{''}}};
%         \addplot[no marks, draw=yellow!55!blue] table [x=pO2, y=VM{-1},]{dat/intrinsic_cubic.dat}; \addlegendentry{\ch{V_{Zr}^{'}}};
%         \addplot[no marks, draw=yellow!45!blue] table [x=pO2, y=VM{0},]{dat/intrinsic_cubic.dat}; \addlegendentry{\ch{V_{Zr}^{x}}};
%         \addplot[no marks, draw=red!60!yellow] table [x=pO2, y=Oi{-2},]{dat/intrinsic_cubic.dat}; \addlegendentry{\ch{O_{i}^{''}}};
%         \addplot[no marks, draw=red!50!yellow] table [x=pO2, y=Oi{-1},]{dat/intrinsic_cubic.dat}; \addlegendentry{\ch{O_{i}^{'}}};
%         \addplot[no marks, draw=red!40!yellow] table [x=pO2, y=Oi{0},]{dat/intrinsic_cubic.dat}; \addlegendentry{\ch{O_{i}^{x}}};
%         \addplot[no marks, draw=green!80!pink] table [x=pO2, y=Mi{4},]{dat/intrinsic_cubic.dat}; \addlegendentry{\ch{Zr_{i}^{****}}};
%         \addplot[no marks, draw=green!70!pink] table [x=pO2, y=Mi{3},]{dat/intrinsic_cubic.dat}; \addlegendentry{\ch{Zr_{i}^{***}}};
%         \addplot[no marks, draw=green!60!pink] table [x=pO2, y=Mi{2},]{dat/intrinsic_cubic.dat}; \addlegendentry{\ch{Zr_{i}^{\textbf{**}}}};
%         \addplot[no marks, draw=green!50!pink] table [x=pO2, y=Mi{1},]{dat/intrinsic_cubic.dat}; \addlegendentry{\ch{Zr_{i}^{*}}};
%         \addplot[no marks, draw=green!40!pink] table [x=pO2, y=Mi{0},]{dat/intrinsic_cubic.dat}; \addlegendentry{\ch{Zr_{i}^{x}}};
%         \addplot[no marks, dashed, draw=red!70!black] table [x=pO2, y=Ii{0},]{dat/intrinsic_cubic.dat}; \addlegendentry{\ch{I_{i}^{x}}};
%         \addplot[no marks] table [x=pO2, y=Stoich,]{dat/intrinsic_cubic.dat}; \addlegendentry{Stoich};
%\node at (-33.7,-0.5) {\textbf{a)}};
			\end{axis}            
\end{tikzpicture} 
		\caption{Cubic phase Brouwer diagrams of intrinsic defects at 2000 K.}
		\label{figure:cubic_intrinsic_brouwer}
	\end{center}
\end{figure}

\section{Summary}

The main defects in \zirconia\ are oxygen vacancies and zirconium vacancies at low and high oxygen pressures respectively. 

The cubic phase cannot be modelled accurately due to instabilities outlined by Burr \emph{et al}. \cite{burr2017importance}. For this reason, it was decided that the cubic phase would not be considered when conducting extrinsic dopant simulation studies in \zirconia .

\chapter{Iodine defect energies and equilibria in \zirconia}

\emph{The work in this chapter has been published in:} \\ A. Kenich \emph{et al.} J. Nucl. Mater. \textbf{511} (2018) 390-395. Ref \cite{kenichiodine2018}.

\label{ch:results2}

\section{Introduction}

As discussed in Chapter \ref{introduction}, stress-corrosion cracking (SCC) in nuclear fuel pins is an issue related to early integrity of fuel assemblies in light water reactors (LWRs). SCC studies of the internal surface of zirconium-based fuel claddings have been conducted, which indicate that iodine is likely to be one of the main corrosive species involved in promoting crack growth \cite{rosenbaum1966interaction, bcoxpelletclad1990,fregonese2000failure,Sidky1998}. The exact mechanism for iodine SCC has not yet been determined due to difficulties observing the internal cladding surface in-situ, while experimental studies are not yet capable of reproducing the conditions under which such failures occur. The quantum-mechanical simulation approach is therefore particularly useful to model the behaviour of iodine within the oxide layer of the cladding, the layer preceding the zirconium metal. 

Iodine is produced in the fuel pellet directly from fission (see Chapter I for details) and also from the decay of tellurium precursors. As shown in Figure \ref{table:decaydata_chap1}, both iodine and tellurium are relatively common fission products, with combined independent yields from thermal fission of U$_{235}$ above 26\% \cite{kennett1956mass, iodine129fissionyield, imanishi1976independent, iodinefissionyields, iodine132, amiel1975odd, iaeafissionyield}. The majority of thermal fission events occur in the outer rim of the fuel pellet, and a fission product penetration depth of up to 8 $\mu$m in \zirconia\ \cite{degueldre2001behaviour} suggests a large degree of implantation within the oxide and the Zr metal into which the oxide grows, raising the concentration of I well above the equilibrium value. Iodine and many of its relevant compounds (as ZrI$_{4}$, CsI) are volatile and fuel pellets contain many cracks and spaces through which iodine may be rapidly transported to the cladding. When reactor power is increased during start-up, iodine is released in substantial quantities from the UO$_{2}$ pellet \cite{peehs1982experimental}. This is believed to cause crack propagation in the cladding when combined with stresses imposed on the cladding by the fuel pellet, and this contributes to pellet-cladding interaction (PCI), a phenomenon discussed in Chapter \ref{introduction}. Upper limits on power ramping and holding times have therefore been established by fuel suppliers to mitigate potential PCI failures \cite{yagnik2005effect}. While these restrictions have reduced or prevented the incidence of PCI failures, they also impose costs on the operator due to longer ramping periods. This also restricts the ability of the nuclear reactor to load-follow grid demand. Cladding/fuel materials resistant to PCI failure are therefore of great interest in the nuclear power industry, promoting research into solutions such as cladding liners and doped fuel pellets \cite{nonon2005pci,yang2012effect}. 

%\item Power ramping: Increasing power, such as during reactor start-up, can lead to cladding failure.   %power must be ramped up gradually in order to avoid excessive temperature gradients in the fuel pins, but also to manage fission product concentrations. Due to the different half-lives of various fission products, a power ramp will cause a transient increase in the iodine concentration within a fuel pin


Iodine is an oxidising agent, which, under standard conditions, will oxidise Zr metal to produce ZrI$_{4}$. However, oxygen is also present in the internal fuel pin environment, both from the native \zirconia\ layer on the cladding, and the evolution of oxygen from the fuel pellet during burnup. Liberated oxygen will compete with iodine in the oxidation of the Zr metal, but whereas iodine promotes crack growth under stress, oxygen provides a more protective effect, self-limiting its diffusion into the metal \cite{farina2002stress, causey2005review}. Furthermore, oxygen is a more powerful oxidising agent than iodine, reacting together to produce I${_2}$O$_{5}$. For these reasons, the internal oxide layer of the cladding is often considered a barrier to the ingress of iodine into the Zr metal. 

Unlike oxygen and hydrogen, which readily diffuse into Zr metal to occupy interstitial sites, iodine atoms have been predicted in atomistic studies to have very high energy barriers to bulk interstitial diffusion \cite{rossi2015first,legris2005ab,carlot2002energetically}. This is due to the relatively large radius of the iodine atom, which imposes large local strains when penetrating the Zr lattice. This suggests that iodine will instead be transported towards crack tips via grain boundaries. Indeed, intergranular cracking has been observed in PCI failures, but only for a few hundred nm before a more rapid transgranular crack propagation \cite{fregonese2000failure, une1984threshold, wood1983effects, lunde1981stress, vilpponen1981fuel}. Conversely, no atomic scale studies of iodine in \zirconia\ were found in the literature.  

As discussed in § \ref{section:outervsinner}, there is an oxide layer on the internal surface of the cladding consisting of monoclinic and tetragonal oxide grains. The effectiveness of the oxide layer as a barrier to iodine is debated, with one study presuming that the oxide is bypassed entirely by iodine due to fracturing, leaving the Zr metal underneath exposed \cite{rossi2015first}. The outermost part of the oxide, which is porous, exhibits networks of interconnected grain boundary diffusion pathways towards the oxide/metal interface which are certainly wide enough (1-3 nm) to allow iodine transport \cite{ni2010porosity}. The oxygen-saturated Zr at the oxide/metal interface is not, however, taken into account, and it is expected that this will influence the corrosion mechanism due to iodine-oxygen competition: even the much smaller hydrogen atom has its rate of diffusion into the metal reduced by the presence of oxygen, as shown in both computational \cite{glazoff2014oxidation} and experimental hydrogen pick-up studies \cite{couet2014hydrogen}. This means that some barrier to iodine ingress must already exist near the oxide/metal interface. The varying levels of oxygen across the oxide layer itself also have an effect on defect behaviour, and will therefore influence the initiation mechanisms behind PCI failures. Thus here, we predict iodine incorporation energies and defect equilibria in \zirconia\ as a function of oxygen pressure through Brouwer diagrams, in order to predict the resulting iodine defect response.


%\subsection{Pellet-cladding interaction}
%
%Pellet-cladding interaction refers to the interaction between the fuel and the cladding at higher burnups where the gas gap has been closed by the swelling of the fuel pellets. PCI has both a mechanical and a chemical component, sometimes referred to specifically as pellet cladding mechanical interaction (PCMI) and pellet cladding chemical interaction (PCCI) respectively.
%

\section{Methodology}
\subsection{Computational details}

As discussed in Chapter \ref{ch:compmethodology}, calculations were performed using CASTEP 8.0 \cite{Clark2005}. Ultra-soft pseudo-potentials with a cut-off energy of 600 eV were employed. The Perdew, Burke and Ernzerhof (PBE) \cite{Perdew1996} parameterisation of the generalised gradient approximation (GGA) was used to describe the exchange correlation functional. A Monkhorst-Pack sampling scheme \cite{Monkhorst1976} was used for Brillouin zone integration, with a minimum \emph{k}-point separation of 0.09 \r{A}\textsuperscript{-1}. The Pulay method for density mixing \cite{Pulay1980} was used to improve simulation convergence. 

The electronic energy convergence criterion was set to $1\times10^{-6}$ eV and the maximum force between atoms limited to $1\times10^{-2}$ eV \r{A}\textsuperscript{-1}, which are values demonstrated as appropriate in § \ref{convergence_criteria}. A gradient-descent geometry optimisation task was run on the cell until consecutive iterations differed in energy and atomic displacement by less than $1\times10^{-5}$ eV and $5\times10^{-4}$ \r{A} respectively, again demonstrated in § \ref{convergence_criteria}. 


\subsection{Defect Equilibrium Response to Oxygen Partial Pressure}

Brouwer diagrams were generated using the method previously outlined in § \ref{brouwer_method}. Defect concentrations for the monoclinic and tetragonal phases were calculated at 650 K and 1500 K respectively to reflect the temperatures at which these structures are stable. Brouwer diagrams at extrinsic defect concentrations of $10^{-5}$ and $10^{-3}$ parts/fu (i.e. parts per \zirconia\ formula unit) were generated to examine low and high dopant concentrations, respectively. These two concentrations were examined because the amount of fission products present at a particular point in a fuel pellet depends on macroscopic parameters, including its position in the core and the time since the last shutdown, but also microscopic parameters such as the radial position in the pellet. These two concentrations were selected because $10^{-3}$ parts/fu is high enough to model an aggregation of iodine (such as at a crack tip), and $10^{-5}$ parts/fu was found to be the concentration below which iodine did not have a significant effect on defect equilibria. 


%Brouwer diagrams, also known as Kr{\"o}ger-Vink diagrams, were produced using a method outlined by Murphy et al. \cite{Murphy2014} to determine defect concentrations as a function of oxygen partial pressure. We start from the statement that the chemical potential of \zirconia\ is equivalent to the sum of the chemical potentials $\mu$ of its constituent species, Zr and O:
%
%\begin{equation}
%{\mu}_{ZrO_2(s)} = {\mu}_{Zr}(p_{O_2}, T) + {\mu}_{O_2}(p_{O_2}, T)
%\label{mewZrO2results2}
%\end{equation}
%
%where $T$ denotes temperature and $p_{O_2}$ denotes oxygen partial pressure. The chemical potential of \zirconia\ in the solid state is assumed to have negligible dependence on $T$ and $p_{O_2}$ relative to ${\mu}_{Zr}$ and ${\mu}_{O_2}$. Energies can be obtained for bulk \zirconia\ and Zr, but the ground state of oxygen is not correctly reproduced in DFT \cite{Batyrev2000,Lozovoi2001}. Instead, we use the approach of Finnis et al. \cite{Finnis2005} to infer the oxygen chemical potential from standard state values. We can use the experimental Gibbs free energy to produce an equation where $\mu_{O_2}$ is the only unknown:
%
%\begin{equation}
%\Delta{G^{\plimsoll}_{f, ZrO_2}} = \mu_{ZrO_2(s)} - (\mu_{Zr(s)} + \mu^{\plimsoll}_{O_2})
%\end{equation}
%
%where $\Delta{G^{\plimsoll}_{f, ZrO_2}}$ is the experimental Gibbs energy at standard temperature and pressure and $\mu^{\plimsoll}_{O_2}$ is the oxygen chemical potential under the same conditions. The values of $\mu_{ZrO_2(s)}$ and $\mu_{Zr(s)}$ are calculated from the DFT energies. Once $\mu^{\plimsoll}_{O_2}$ is calculated, we can generalise the chemical potential of oxygen for any value of $T$ and $p_{O_2}$ by appending an ideal gas relationship $\Delta{\mu(T)}$ and a Boltzmann distribution:
%
%\begin{equation}
%\mu_{O_2}(p_{O_2},T) = \mu^{\plimsoll}_{O_2} + \Delta{\mu(T)} + \frac{1}{2}{k_B}log(\frac{p_{O_2}}{p^{\plimsoll}_{O_2}})
%\end{equation}
%
%Using our generalised formula for $\mu_{O_2}$, we fix the temperature within the range of thermal phase-stabilisation (1500 K for tetragonal \zirconia) and calculate $\mu_{O_2}$ for many different values of $p_{O_2}$ between $10^{-35}$ and 10$^{0}$ atm, corresponding to oxygen deficient and oxygen rich environments, respectively ($p_{O_2}$ in air is approximately 0.2 atm). While the tetragonal phase will be stress-stabilised in practice, thermal-stabilisation in such models has been shown to qualitatively approximate the effect of stress-stabilisation, while allowing a wider range of dopant behaviours to be predicted \cite{Bell2016}. 

\section{Results}

\subsection{Incorporation energies}

\subsubsection*{Interstitial Sites}
Neutral iodine incorporation energies at interstitial sites for each phase are reported in Table \ref{i_incorp_interstitial}. The $2a$ and $2c$ sites in monoclinic \zirconia\ provide the least unfavourable iodine incorporation energy, followed by the $2b$ and $8e$ sites in tetragonal \zirconia , although in all cases energies are positive and large, indicating a large energy penalty against interstitial incorporation. The difference in incorporation energies between monoclinic and tetragonal \zirconia\ is approximately 1 eV, whereas the difference between tetragonal and cubic is 3.5 eV, indicating especially unfavourable conditions in cubic \zirconia . These differences are likely due to the larger interstitial sites in the lower-temperature phases, as monoclinic \zirconia\ exhibits the least and cubic \zirconia\ the most dense cell, (see Chapter \ref{ch:crystallography}). 
%It is expected that these densities will be representative because the decreased density due to thermal expansion at 650 K will be countered by attractive dispersion forces. 

\begin{table}[ht]
\onehalfspacing
\centering
\caption{Incorporation energies of iodine interstitials in non-defective supercells.}
\label{i_incorp_interstitial}
\begin{tabular}{ccccc}
\hline
\multirow{2}{*}{\textbf{Structure}} & \multirow{2}{*}{\textbf{Site}} & \multicolumn{3}{c}{\textbf{Incorporation Energy (eV)}} \\ \cline{3-5} 
 &  &  \textbf{\ch{I^{x}_{i}}} & \textbf{\ch{I^{'}_{i}}} & \textbf{\ch{I^{*}_{i}}} \\ \hline % \hspace{0.7 cm}
\multirow{4}{*}{\textbf{Monoclinic}} & 2a & 8.55 & 12.10 & 6.55 \\
 & 2b & 10.81 & 16.40 & 5.63 \\
 & 2c & 8.79 & 12.20 & 4.62 \\
 & 2d & 10.94 & 13.66 & 6.92 \\ \hline
\multirow{2}{*}{\textbf{Tetragonal}} & 2b & 9.49 & 13.96 & 5.99 \\
 & 8e & 9.53 & 12.73 & 5.10 \\ \hline
\multirow{2}{*}{\textbf{Cubic}} & 24d & 13.02 & 18.24 & 7.62 \\
 & 4b & 13.08 & 16.46 & 9.82 \\ \hline
\end{tabular}%
\end{table}

While the incorporation energies of iodine in the interstitial sites of \zirconia\ are large, for a fixed iodine concentration, they become relevant as the intrinsic defect populations become small, such as at low temperatures relative to the melting point. This is because interstitial sites are always available, whereas at low intrinsic defect concentrations, substitutional sites become saturated and accommodation at a lattice site first requires the creation of a vacancy defect, which has a formation energy penalty associated with it. 

When Brouwer diagrams are generated, iodine will also be considered as a charged species at an interstitial site (see § \ref{results2_brouwer}). This includes I$^{+}$ and I$^{-}$, where I$^{+}$ is a smaller ion that is more easily accommodated at an interstitial site. Energy values for I$^{+}$ and I$^{-}$ are also reported in Table \ref{i_incorp_interstitial}, however, values for different charge states cannot be compared because an electron has been added or removed from the I atom to form the specific charge state. An interstitial site will always be uncharged (resultant charge is distributed onto nearby ions instead), meaning that there is no pre-existing charged interstitial site to incorporate an atom onto. It is therefore only appropriate to consider the \emph{formation} energy (i.e. including the addition or removal of an electron) of a charged interstitial defect. 

\subsubsection*{Oxygen Sites}

Table \ref{i_incorp_oxygen} reports incorporation energies of iodine at various oxygen sites. In each phase, the lowest incorporation energy was that for accommodation at a vacant oxygen site such that iodine is in the 1- oxidation state, resulting in the overall defect \ch{I_{O}^{*}}. This anionic behaviour is expected from a halogen atom in a highly reducing site as it promotes the filling of the \emph{p} shell. I is most readily accommodated in the monoclinic phase for all I charge states.

\begin{table}[ht] % Iodine O site incorporation
\onehalfspacing
\centering
\caption{Incorporation energies of iodine in oxygen sites of the monoclinic, tetragonal, and cubic \zirconia\ phases.}
\label{i_incorp_oxygen}
\begin{tabular}{cccc} % \ch{V_{O}^{x}}
\hline
\multirow{2}{*}{\textbf{Structure}} & \multicolumn{3}{c}{\textbf{Incorporation energy (eV)}} \\ \cline{2-4} 
                                    & \hspace{0.7 cm} \textbf{\ch{I_{O}^{**}}} \hspace{0.7 cm} & \textbf{\ch{I_{O}^{*}}} & \textbf{\ch{I_{O}^{x}}} \\ \hline
\textbf{Monoclinic (3 co-ord)}      & 4.54             & 2.90             & 3.67             \\
\textbf{Monoclinic (4 co-ord)}      & 5.63             & 3.77             & 4.87             \\
\textbf{Tetragonal}                 & 6.19             & 4.02             & 4.44             \\
\textbf{Cubic}                      & 8.37             & 5.74             & 6.66      \\     \hline  
\end{tabular}
\end{table}


 % and these atoms have large electron affinities since they require only one electron to achieve the relatively stable noble gas electron configuration. 

\subsubsection*{Zirconium Sites}

Incorporation energies of iodine on zirconium sites are reported in Table \ref{i_incorp_zirconium}. The  incorporation energy decreases as the charge of the defect decreases from -4 to 0 (i.e. nominally from I$^{0}$ to I$^{4+}$). This is due to the decrease in the size of the iodine species with increasing positive charge, fitting better into the small vacant Zr$^{4+}$ cation site. This alone does not guarantee the emergence of uncharged iodine defects (I$^{4+}$) on zirconium sites when all energy terms are considered. In particular, there is also an energy penalty incurred in the change in charge of iodine. A Mulliken population analysis revealed a charge localised on the iodine of +2.31 at the \ch{I_{Zr}^{x}} defect, and a +0.86 charge on the \ch{I_{Zr}^{''''}} defect, with the remaining charge accommodated by other ions in the lattice. Again, I is most readily accommodated in the monoclinic phase for all charge states.

\begin{table}[ht] % Iodine Zr site incorporation
\onehalfspacing
\centering
\caption{Incorporation energies of iodine in zirconium sites of \zirconia.}
\label{i_incorp_zirconium}
\begin{tabular}{ccllll}
\hline
\hspace{0.7 cm} \multirow{2}{*}{\textbf{Structure}} \hspace{0.7 cm} & \multicolumn{5}{c}{\hspace{0.7 cm} \textbf{Incorporation energy (eV)}} \hspace{0.7 cm}                                                          \\ \cline{2-6} 
                                    & \multicolumn{1}{l}  {\textbf{\hspace{0.45 cm} \ch{I_{Zr}^{''''}}}} \hspace{0.45 cm} & \textbf{\ch{I_{Zr}^{'''}}} \hspace{0.45 cm} & \textbf{\ch{I_{Zr}^{''}}} \hspace{0.45 cm} & \textbf{\ch{I_{Zr}^{'}}} \hspace{0.45 cm} & \textbf{\ch{I_{Zr}^{x}}} \\ \hline
\textbf{Monoclinic}                 & 6.78                             &       3.65            &        0.89          &         -2.84         &     -5.08             \\
\textbf{Tetragonal}                 & 7.58                            &         3.64         &        1.69          &      -2.13            &     -4.57             \\
\textbf{Cubic}                      & 9.70                            &         6.81         &        3.01          &        0.38          &      -3.14     \\      \hline
\end{tabular}
\end{table}

%\subsection{Temperature dependence}

%To examine the temperature dependence of the defect equilibria, the concentration of dopant iodine was held constant while temperature was changed.


%\subsection{Dopant concentration dependence}
%
%To examine the dependence of the defect equilibria on iodine dopant concentration, the temperature was held constant while iodine concentration was changed.

%\begin{figure}[ht] % Tet intrinsic no space charge
%\begin{center}
%\begin{tikzpicture}
%	\begin{groupplot}[group style={group size=1 by 2}, width=13cm, height=10.2cm]
%	\nextgroupplot[
%		 ylabel={\ch{log_{10}}([D]) (per f.u.)}, ymin=-10, ymax=0, xmin=-35, xmax=0, legend style={{draw=}, at={(0.40,0.97)}, anchor=north west, legend columns=2, nodes={scale=1, transform shape}}]
%
%	\nextgroupplot[
%		 xlabel={\ch{log_{10}}($p_{O_{2}}$) (atm)}, ylabel={\ch{log_{10}}([D]) (per f.u.)}, ymin=-10, ymax=0, xmin=-35, xmax=0, legend style={{draw=}, at={(0.40,0.97)}, anchor=north west, legend columns=4, nodes={scale=1, transform shape}}]
%      
%	\end{groupplot}           
%\end{tikzpicture}
%		\caption{Tetragonal phase Brouwer diagrams of intrinsic point defects at a temperature of 1500 K \textbf{a)} without a space charge and \textbf{b)} with a space charge of $10^{-1}$ e$^{-1}$ per f.u.}
%		\label{figure:spacechargeexample}
%	\end{center}
%\end{figure}

\subsection{Dopant interstitial defects}

The formation energies of iodine interstitial defects are useful not only to determine whether interstitial defects will form, but also which interstitial sites in particular they will occupy. As shown in Table \ref{table:interstitials}, ZrO$_{2}$ has four interstitial sites in the monoclinic phase, and two interstitial sites in the tetragonal and cubic phase (based on crystallographic data of the three phases). For each phase, the formation energies of each iodine interstitial defect as a function of Fermi level are calculated and provided below.

\subsubsection{Monoclinic interstitial defects}

Figure \ref{figure:monointer} shows how the formation energy of an iodine interstitial defect in monoclinic ZrO$_{2}$ varies based on site and Fermi level. In this phase, the 2$b$ Wyckoff position is the site of lowest formation energy across the entire range of Fermi levels that span the band gap, with a maximum of 8.1 eV at a Fermi level of 2.8 eV.  The next most favourable site is at 2$a$, with formation energies at least 0.4 eV greater at similar Fermi levels. Iodine at the 2$c$ and 2$d$ sites exhibits formation energies between 1 and 4 eV larger than at the 2$b$ site, making these sites significantly unfavourable. This shows that there are indeed four unique sites which iodine atoms can occupy in the monoclinic phase (as evidenced by the different formation energy evolutions at each site), and that of these sites, the 2$c$ is the most energetically favourable for iodine.

\begin{figure}[ht] % Mono iodine interstitials E vs Fermi
\begin{center}
\begin{tikzpicture}
	\begin{axis}
		[width=11.5cm, xlabel={Fermi level $\mu_{e}$ (eV)}, ylabel={Formation energy (eV) per \zirconia\ }, ymin=4, ymax=11, xmin=0, xmax=6, legend style={{draw=}, at={(0.5,0.05)}, anchor=south, legend columns=2}]
		\addplot[no marks, red] table [x=2a1, y=2a2,]{dat/monointer.dat}; \addlegendentry{$2a$};
        \addplot[no marks, red, dashed] table [x=2b1, y=2b2, ]{dat/monointer.dat}; \addlegendentry{$2b$};
        \addplot[no marks, blue] table [x=2c1, y=2c2,]{dat/monointer.dat}; \addlegendentry{$2c$};
        \addplot[no marks, blue, dashed] table [x=2d1, y=2d2,]{dat/monointer.dat}; \addlegendentry{$2d$};
			\end{axis}
		\end{tikzpicture}
		\caption{Iodine interstitial formation energies in monoclinic \zirconia\ as a function of Fermi level. Gradient indicates defect charge.}
		\label{figure:monointer}
	\end{center}
\end{figure}

It should also be noted that iodine will occupy these interstitial sites as either \ch{I^{*}_{i}} or \ch{I^{'}_{i}} (the slope of the curve indicates defect charge). \ch{I^{x}_{i}} appears for a small range of Fermi levels in the 2$c$ site, but the formation energy is so large compared to the other sites that this will not be present in the crystal. The preference for a charged state of +1 or -1 may be due to a combination of electron availability and ionic radius. \ch{I^{*}_{i}}, being positively charged, has a smaller ionic radius than \ch{I^{x}_{i}}. The smaller size of this defect comes with a smaller energy penalty when occupying an interstitial site, and at low Fermi levels the iodine is more susceptible to oxidation. At high Fermi levels, \ch{I^{'}_{i}} forms because electrons are more readily available and iodine has a high electron affinity, which compensates energetically for the increased ionic radius.

\subsubsection{Tetragonal interstitial defects}

\begin{figure}[ht] % Tet iodine interstitials E vs Fermi
\begin{center}
\begin{tikzpicture}
	\begin{axis}
		[width=11cm, xlabel={Fermi level $\mu_{e}$ (eV)}, ylabel={Formation energy (eV) per \zirconia\ }, ymin=6, ymax=11, xmin=0, xmax=6, legend style={{draw=}, at={(0.5,0.05)}, anchor=south, legend columns=1}]
		\addplot[no marks, red] table [x=2atet1, y=2atet2,]{dat/tetcubicinter.dat}; \addlegendentry{$2a$};
        \addplot[no marks, blue] table [x=8etet1, y=8etet2, ]{dat/tetcubicinter.dat}; \addlegendentry{$8e$};
			\end{axis}
		\end{tikzpicture}
		\caption{Iodine interstitial formation energies in tetragonal \zirconia\ as a function of Fermi level. Gradient indicates defect charge.}
		\label{figure:tetinter}
	\end{center}
\end{figure}

\subsubsection{Cubic interstitial defects}

\begin{figure}[ht] % Cubic iodine interstitials E vs Fermi
\begin{center}
\begin{tikzpicture}
	\begin{axis}
		[width=11cm, xlabel={Fermi level $\mu_{e}$ (eV)}, ylabel={Formation energy (eV) per \zirconia\ }, ymin=8, ymax=13, xmin=0, xmax=6, legend style={{draw=}, at={(0.5,0.05)}, anchor=south, legend columns=1}]
		\addplot[no marks, red] table [x=24cubic1, y=24cubic2,]{dat/tetcubicinter.dat}; \addlegendentry{$24d$};
        \addplot[no marks, blue] table [x=4bmcubic1, y=4bmcubic2, ]{dat/tetcubicinter.dat}; \addlegendentry{$4b$};
			\end{axis}
		\end{tikzpicture}
		\caption{Iodine interstitial formation energies in cubic \zirconia\ as a function of Fermi level. Gradient indicates defect charge.}
		\label{figure:cubicinter}
	\end{center}
\end{figure}

\subsection{Brouwer Diagrams}  \label{results2_brouwer}

\subsubsection*{Monoclinic Phase}

Brouwer diagrams associated with the monoclinic phase, at 650 K, at which temperature this \zirconia\ phase is stable, are shown in Figure \ref{figure:tikzbrouwerconcmono}. At 650 K, this phase exhibits a relatively low concentration of intrinsic defects; concentrations of \ch{V_{O}^{**}} and \ch{V_{Zr}^{''''}} remained below $10^{-10}$ parts/fu across the majority of oxygen pressures at both iodine concentrations and do not appear in the diagrams. At lower iodine concentrations, the intrinsic electronic defects, \ch{e^{'}} and \ch{h^{*}}, were more significant, with \ch{h^{*}} defects being a major fraction of the total defect population near stoichiometry (i.e. at an oxygen pressure of approximately $10^{-7.5}$ atm). 

\begin{figure}[ht!] % Mono conc sweep
\begin{center}
\begin{tikzpicture} % 10e-3 iodine conc in mono
	\begin{groupplot}[group style={group size=1 by 2}, width=13cm, height=10.2cm]
	\nextgroupplot[
		 ylabel={\ch{log_{10}}([D]) (per f.u.)}, ymin=-10, ymax=0, xmin=-35, xmax=0, legend style={{draw=}, at={(0.40,0.97)}, anchor=north west, legend columns=2, nodes={scale=1, transform shape}}]
        \addplot[no marks, draw=blue!70!black] table [x=pO2, y=electrons,]{dat/1e5iconcmono650.dat};  \node at (-28.1,-7.5) {\ch{e^{'}}};  \addlegendentry{\ch{e^{'}}};
        \addplot[no marks, draw=red!85!black] table [x=pO2, y=holes,]{dat/1e5iconcmono650.dat}; \addlegendentry{\ch{h^{\textperiodcentered}}}; % \node at (-1,-4.5) {\ch{h^{\textperiodcentered}}};
%         \addplot[no marks, draw=black!70!green] table [x=pO2, y=VO{2},]{dat/1e5iconcmono650.dat}; \addlegendentry{\ch{V_{O}^{\textperiodcentered\textperiodcentered}}};
%         \addplot[no marks, draw=black!55!green] table [x=pO2, y=VO{1},]{dat/1e5iconcmono650.dat}; \addlegendentry{\ch{V_{O}^{\textperiodcentered}}};
%         \addplot[no marks, draw=black!30!green] table [x=pO2, y=VO{0},]{dat/1e5iconcmono650.dat}; \addlegendentry{\ch{V_{O}^{x}}};
%         \addplot[no marks, draw=yellow!85!blue] table [x=pO2, y=VM{-4},]{dat/1e5iconcmono650.dat}; \addlegendentry{\ch{V_{Zr}^{''''}}};
%         \addplot[no marks, draw=yellow!75!blue] table [x=pO2, y=VM{-3},]{dat/1e5iconcmono650.dat}; \addlegendentry{\ch{V_{Zr}^{'''}}};
%         \addplot[no marks, draw=yellow!65!blue] table [x=pO2, y=VM{-2},]{dat/1e5iconcmono650.dat}; \addlegendentry{\ch{V_{Zr}^{''}}};
%         \addplot[no marks, draw=yellow!55!blue] table [x=pO2, y=VM{-1},]{dat/1e5iconcmono650.dat}; \addlegendentry{\ch{V_{Zr}^{'}}};
%         \addplot[no marks, draw=yellow!45!blue] table [x=pO2, y=VM{0},]{dat/1e5iconcmono650.dat}; \addlegendentry{\ch{V_{Zr}^{x}}};
%         \addplot[no marks, draw=red!60!yellow] table [x=pO2, y=Oi{-2},]{dat/1e5iconcmono650.dat}; \addlegendentry{\ch{O_{i}^{''}}};
%         \addplot[no marks, draw=red!50!yellow] table [x=pO2, y=Oi{-1},]{dat/1e5iconcmono650.dat}; \addlegendentry{\ch{O_{i}^{'}}};
%         \addplot[no marks, draw=red!40!yellow] table [x=pO2, y=Oi{0},]{dat/1e5iconcmono650.dat}; \addlegendentry{\ch{O_{i}^{x}}};
%         \addplot[no marks, draw=green!80!pink] table [x=pO2, y=Mi{4},]{dat/1e5iconcmono650.dat}; \addlegendentry{\ch{Zr_{i}^{\textperiodcentered\textperiodcentered\textperiodcentered\textperiodcentered}}};
%         \addplot[no marks, draw=green!70!pink] table [x=pO2, y=Mi{3},]{dat/1e5iconcmono650.dat}; \addlegendentry{\ch{Zr_{i}^{\textperiodcentered\textperiodcentered\textperiodcentered}}};
%         \addplot[no marks, draw=green!60!pink] table [x=pO2, y=Mi{2},]{dat/1e5iconcmono650.dat}; \addlegendentry{\ch{Zr_{i}^{\textbf{\textperiodcentered\textperiodcentered}}}};
%         \addplot[no marks, draw=green!50!pink] table [x=pO2, y=Mi{1},]{dat/1e5iconcmono650.dat}; \addlegendentry{\ch{Zr_{i}^{\textperiodcentered}}};
%         \addplot[no marks, draw=green!40!pink] table [x=pO2, y=Mi{0},]{dat/1e5iconcmono650.dat}; \addlegendentry{\ch{Zr_{i}^{x}}};
%         \addplot[no marks, dashed, draw=red!70!black] table [x=pO2, y=Ii{0},]{dat/1e5iconcmono650.dat}; \addlegendentry{\ch{I_{i}^{x}}};
%         \addplot[no marks, dashed, draw=red!50!black] table [x=pO2, y=Ii{-1},]{dat/1e5iconcmono650.dat}; \addlegendentry{\ch{I_{i}^{'}}};
        \addplot[no marks, dashed, draw=purple!60!white] table [x=pO2, y=Ii{1},]{dat/1e5iconcmono650.dat}; \addlegendentry{\ch{I_{i}^{\textperiodcentered}}}; 
        \addplot[no marks, dashed, draw=blue!50!white] table [x=pO2, y=IsubO{1},]{dat/1e5iconcmono650.dat}; \addlegendentry{\ch{I_{O}^{\textperiodcentered}}}; \node at (-22,-4.5) {\ch{I_{O}^{\textperiodcentered}}};
        \addplot[no marks, dashed, draw=green!60!black] table [x=pO2, y=IsubO{2},]{dat/1e5iconcmono650.dat}; \addlegendentry{\ch{I_{O}^{\textperiodcentered\textperiodcentered}}}; 
        \addplot[no marks, dashed, draw=black] table [x=pO2, y=IsubO{3},]{dat/1e5iconcmono650.dat}; \addlegendentry{\ch{I_{O}^{\textperiodcentered\textperiodcentered\textperiodcentered}}};
        \addplot[no marks, dashed, draw=orange!80!black] table [x=pO2, y=IsubZr{-3},]{dat/1e5iconcmono650.dat};  \node at (-22,-6.2) {\ch{I_{Zr}^{'''}}}; \addlegendentry{\ch{I_{Zr}^{'''}}};
%         \addplot[no marks, dashed, draw=green!50!black] table [x=pO2, y=IsubZr{-4},]{dat/1e5iconcmono650.dat}; \addlegendentry{\ch{I_{Zr}^{''''}}};
%         \addplot[no marks, dashed, draw=green!30!black] table [x=pO2, y=IsubZr{-5},]{dat/1e5iconcmono650.dat}; \addlegendentry{\ch{I_{Zr}^{'''''}}};
%         \addplot[no marks] table [x=pO2, y=Stoich,]{dat/1e5iconcmono650.dat}; \addlegendentry{Stoich};
\node at (-33.7,-0.5) {\textbf{a)}};
			\nextgroupplot[
		 xlabel={\ch{log_{10}}($p_{O_{2}}$) (atm)}, ylabel={\ch{log_{10}}([D]) (per f.u.)}, ymin=-10, ymax=0, xmin=-35, xmax=0, legend style={{draw=}, at={(0.40,0.97)}, anchor=north west, legend columns=4, nodes={scale=1, transform shape}}]
        \addplot[no marks, draw=blue!70!black] table [x=pO2, y=electrons,]{dat/1e3iconcmono650.dat}; \node at (-29,-7) {\ch{e^{'}}};
        \addplot[no marks, draw=red!85!black] table [x=pO2, y=holes,]{dat/1e3iconcmono650.dat}; 
%         \addplot[no marks, draw=black!70!green] table [x=pO2, y=VO{2},]{dat/1e3iconcmono650.dat}; 
%         \addplot[no marks, draw=black!55!green] table [x=pO2, y=VO{1},]{dat/1e3iconcmono650.dat}; 
%         \addplot[no marks, draw=black!30!green] table [x=pO2, y=VO{0},]{dat/1e3iconcmono650.dat}; 
%         \addplot[no marks, draw=yellow!85!blue] table [x=pO2, y=VM{-4},]{dat/1e3iconcmono650.dat}; 
%         \addplot[no marks, draw=yellow!75!blue] table [x=pO2, y=VM{-3},]{dat/1e3iconcmono650.dat}; 
%         \addplot[no marks, draw=yellow!65!blue] table [x=pO2, y=VM{-2},]{dat/1e3iconcmono650.dat}; 
%         \addplot[no marks, draw=yellow!55!blue] table [x=pO2, y=VM{-1},]{dat/1e3iconcmono650.dat}; 
%         \addplot[no marks, draw=yellow!45!blue] table [x=pO2, y=VM{0},]{dat/1e3iconcmono650.dat}; 
%         \addplot[no marks, draw=red!60!yellow] table [x=pO2, y=Oi{-2},]{dat/1e3iconcmono650.dat}; 
%         \addplot[no marks, draw=red!50!yellow] table [x=pO2, y=Oi{-1},]{dat/1e3iconcmono650.dat}; 
%         \addplot[no marks, draw=red!40!yellow] table [x=pO2, y=Oi{0},]{dat/1e3iconcmono650.dat}; 
%         \addplot[no marks, draw=green!80!pink] table [x=pO2, y=Mi{4},]{dat/1e3iconcmono650.dat}; 
%         \addplot[no marks, draw=green!70!pink] table [x=pO2, y=Mi{3},]{dat/1e3iconcmono650.dat}; 
%         \addplot[no marks, draw=green!60!pink] table [x=pO2, y=Mi{2},]{dat/1e3iconcmono650.dat}; 
%         \addplot[no marks, draw=green!50!pink] table [x=pO2, y=Mi{1},]{dat/1e3iconcmono650.dat}; 
%         \addplot[no marks, draw=green!40!pink] table [x=pO2, y=Mi{0},]{dat/1e3iconcmono650.dat}; 
%         \addplot[no marks, dashed, draw=red!70!black] table [x=pO2, y=Ii{0},]{dat/1e3iconcmono650.dat}; 
%         \addplot[no marks, dashed, draw=red!50!black] table [x=pO2, y=Ii{-1},]{dat/1e3iconcmono650.dat}; 
        \addplot[no marks, dashed, draw=purple!60!white] table [x=pO2, y=Ii{1},]{dat/1e3iconcmono650.dat}; 
        \addplot[no marks, dashed, draw=blue!50!white] table [x=pO2, y=IsubO{1},]{dat/1e3iconcmono650.dat}; \node at (-22,-2.5) {\ch{I_{O}^{\textperiodcentered}}};
        \addplot[no marks, dashed, draw=green!60!black] table [x=pO2, y=IsubO{2},]{dat/1e3iconcmono650.dat}; 
        \addplot[no marks, dashed, draw=black] table [x=pO2, y=IsubO{3},]{dat/1e3iconcmono650.dat}; 
        \addplot[no marks, dashed, draw=orange!80!black] table [x=pO2, y=IsubZr{-3},]{dat/1e3iconcmono650.dat}; \node at (-22,-4.4) {\ch{I_{Zr}^{'''}}}; 
%         \addplot[no marks, dashed, draw=green!50!black] table [x=pO2, y=IsubZr{-4},]{dat/1e3iconcmono650.dat}; 
%         \addplot[no marks, dashed, draw=green!30!black] table [x=pO2, y=IsubZr{-5},]{dat/1e3iconcmono650.dat}; 
%         \addplot[no marks] table [x=pO2, y=Stoich,]{dat/1e3iconcmono650.dat}; 
\node at (-33.7,-0.5) {\textbf{b)}};
			\end{groupplot}          
\end{tikzpicture}
		\caption{Monoclinic phase Brouwer diagrams of point defects at iodine concentrations of a) $10^{-5}$ and b) $10^{-3}$, at a temperature of 650 K.}
		\label{figure:tikzbrouwerconcmono}
	\end{center}
\end{figure} % 10e-3 iodine conc in mono

Between oxygen pressures of $10^{-35}$ and $10^{-10}$ atm, the dominant defects were \ch{I_{O}^{*}} charge-compensated by \ch{I_{Zr}^{'''}}. Above an oxygen pressure of $10^{-10}$ atm, a combination of \ch{I_{i}^{*}}, \ch{I_{Zr}^{'''}} and \ch{I_{O}^{***}} defects were dominant. This demonstrates that iodine will adopt a +1 oxidation state in order to facilitate iodine incorporation into the lattice. The effective ionic radius of I$^{-}$ is 2.20 \r{A} in VI-fold coordination, compared to 1.38 \r{A} for O$^{2-}$ in IV-fold coordination, as is the case in \zirconia\ \cite{Shannon1976}. Iodine with a higher positive charge state will have a smaller ionic radius, and thus impose less strain on the lattice (and therefore a smaller energy penalty) in each defect configuration, including substitution on a Zr site. At the highest oxygen pressures, the Brouwer diagrams show that oxidation of iodine, substituted at an oxygen site, to the +1 oxidation state (i.e. \ch{I_{O}^{***}}) becomes a necessary charge compensating defect. This is because the energy penalty to form hole defects in this broad band insulator is too great, as is the formation of other positive charge defects such as \ch{Zr_{i}^{****}}. This may translate to iodine out-competing oxygen for oxygen sites in monoclinic \zirconia , with higher oxygen pressures providing very little in terms of a barrier effect. 

% \begin{itemize}
% %\item Figure 1 shows monoclinic Brouwer diagrams generated at different assumed iodine concentrations.
% %\item The monoclinic Brouwer diagrams were generated at a temperature of 650 K. This is because the structure is stable at this temperature, and it is representative of the temperature which the \zirconia\ layer on the internal surface of the cladding would experience.
% %\item At an iodine concentration of 1e-5 and low oxygen pressures, the dominant defects were iodine -1 substitutional defects on the oxygen site, charged compensated by iodine +1 substitutional defects on the zirconium site. At higher oxygen pressures, the zirconium substitutional defects remained but were now charge compensated by iodine +1 substitutional defects on the oxygen site. At very high oxygen pressures, hole defects were preferred to oxygen substitutional defects.

% \item At an iodine concentration of 1e-3, intrinsic defects were found to be negligible compared to the extrinsic iodine defects. The same pattern was seen as with an iodine concentration of 1e-5, except hole defects were no longer significant at high oxygen pressures. Very low concentrations of iodine +1 interstitial defects began to appear at oxygen pressures greater than 1e-20.
% \end{itemize}
% The monoclinic Brouwer diagram (Figure \ref{figure:monoBrouwer}) predicts that at 635 K, few types of defects will be present and at very low (\textless 10 ppb \zirconia ) concentrations. This is typical of defect behaviour in a ceramics at temperatures far below their melting points \cite{kingery1997physical,ball2006computer}. Fully-charged zirconium vacancies, charge-compensated by holes, are the major defect type we expect to observe at $p_{O_{2}}$ \textgreater $10^{-15}$. Below this, only electronic defects compensated by electron hole defects are expected. We briefly see increased concentrations of uncharged oxygen interstitial defects at very high levels of $p_{O_{2}}$.

\subsubsection*{Tetragonal Phase}

Brouwer diagrams for the tetragonal phase are shown in Figure \ref{figure:tikzbrouwerconctet}. As these diagrams were generated at a temperature of 1500 K (at which the tetragonal phase becomes stable), intrinsic defect concentrations were significantly higher than in the monoclinic diagrams for all oxygen pressures (though trends remained the same). Intrinsic defects \ch{e^{'}}, \ch{h^{*}}, \ch{V_{O}^{**}} and \ch{V_{Zr}^{''''}} were dominant across most oxygen pressures at an iodine concentration of $10^{-5}$ parts/fu. Only around stoichiometry do extrinsic defect concentrations approach intrinsic values (which as mentioned earlier is why this concentration of iodine was chosen). Across all oxygen pressures, \ch{I_{O}^{*}} and \ch{I_{Zr}^{'''}} are the major iodine defects. Between $10^{-15}$ and $10^{-5}$ atm, Figure \ref{figure:tikzbrouwerconctet} illustrates that the major iodine defect swaps from being \ch{I_{O}^{*}} to \ch{I_{Zr}^{'''}}. 

\begin{figure}[ht!] % 10e-5 iodine conc in tet
\begin{center}
\begin{tikzpicture}
	\begin{groupplot}[group style={group size=1 by 2}, width=13cm, height=10.2cm]
	\nextgroupplot[
		 ylabel={\ch{log_{10}}([D]) (per f.u.)}, ymin=-10, ymax=0, xmin=-35, xmax=0, legend style={{draw=}, at={(0.40,0.97)}, anchor=north west, legend columns=2, nodes={scale=1, transform shape}}]
        \addplot[no marks, draw=blue!70!black] table [x=pO2, y=electrons,]{dat/1e5iconctet1500.dat}; \addlegendentry{\ch{e^{'}}}; \node at (-26.0,-1.9) {\ch{e^{'}}};
        \addplot[no marks, draw=red!85!black] table [x=pO2, y=holes,]{dat/1e5iconctet1500.dat}; \addlegendentry{\ch{h^{\textperiodcentered}}}; \node at (-7,-3.6) {\ch{h^{\textperiodcentered}}};
        \addplot[no marks, draw=black!70!green] table [x=pO2, y=VO{2},]{dat/1e5iconctet1500.dat}; \addlegendentry{\ch{V_{O}^{\textperiodcentered\textperiodcentered}}}; \node at (-26.7,-3.3) {\ch{V_{O}^{\textperiodcentered\textperiodcentered}}};
%         \addplot[no marks, draw=black!55!green] table [x=pO2, y=VO{1},]{dat/1e5iconctet1500.dat}; \addlegendentry{\ch{V_{O}^{\textperiodcentered}}};
%         \addplot[no marks, draw=black!30!green] table [x=pO2, y=VO{0},]{dat/1e5iconctet1500.dat}; \addlegendentry{\ch{V_{O}^{x}}};
        \addplot[no marks, draw=yellow!85!blue] table [x=pO2, y=VM{-4},]{dat/1e5iconctet1500.dat}; \addlegendentry{\ch{V_{Zr}^{''''}}};
%         \addplot[no marks, draw=yellow!75!blue] table [x=pO2, y=VM{-3},]{dat/1e5iconctet1500.dat}; \addlegendentry{\ch{V_{Zr}^{'''}}};
%         \addplot[no marks, draw=yellow!65!blue] table [x=pO2, y=VM{-2},]{dat/1e5iconctet1500.dat}; \addlegendentry{\ch{V_{Zr}^{''}}};
%         \addplot[no marks, draw=yellow!55!blue] table [x=pO2, y=VM{-1},]{dat/1e5iconctet1500.dat}; \addlegendentry{\ch{V_{Zr}^{'}}};
%         \addplot[no marks, draw=yellow!45!blue] table [x=pO2, y=VM{0},]{dat/1e5iconctet1500.dat}; \addlegendentry{\ch{V_{Zr}^{x}}};
%         \addplot[no marks, draw=red!60!yellow] table [x=pO2, y=Oi{-2},]{dat/1e5iconctet1500.dat}; \addlegendentry{\ch{O_{i}^{''}}};
%         \addplot[no marks, draw=red!50!yellow] table [x=pO2, y=Oi{-1},]{dat/1e5iconctet1500.dat}; \addlegendentry{\ch{O_{i}^{'}}};
%         \addplot[no marks, draw=red!40!yellow] table [x=pO2, y=Oi{0},]{dat/1e5iconctet1500.dat}; \addlegendentry{\ch{O_{i}^{x}}};
%         \addplot[no marks, draw=green!80!pink] table [x=pO2, y=Mi{4},]{dat/1e5iconctet1500.dat}; \addlegendentry{\ch{Zr_{i}^{\textperiodcentered\textperiodcentered\textperiodcentered\textperiodcentered}}};
%         \addplot[no marks, draw=green!70!pink] table [x=pO2, y=Mi{3},]{dat/1e5iconctet1500.dat}; \addlegendentry{\ch{Zr_{i}^{\textperiodcentered\textperiodcentered\textperiodcentered}}};
%         \addplot[no marks, draw=green!60!pink] table [x=pO2, y=Mi{2},]{dat/1e5iconctet1500.dat}; \addlegendentry{\ch{Zr_{i}^{\textbf{\textperiodcentered\textperiodcentered}}}};
%         \addplot[no marks, draw=green!50!pink] table [x=pO2, y=Mi{1},]{dat/1e5iconctet1500.dat}; \addlegendentry{\ch{Zr_{i}^{\textperiodcentered}}};
%         \addplot[no marks, draw=green!40!pink] table [x=pO2, y=Mi{0},]{dat/1e5iconctet1500.dat}; \addlegendentry{\ch{Zr_{i}^{x}}};
%         \addplot[no marks, dashed, draw=red!70!black] table [x=pO2, y=Ii{0},]{dat/1e5iconctet1500.dat}; \addlegendentry{\ch{I_{i}^{x}}};
%         \addplot[no marks, dashed, draw=red!50!black] table [x=pO2, y=Ii{-1},]{dat/1e5iconctet1500.dat}; \addlegendentry{\ch{I_{i}^{'}}};
        \addplot[no marks, dashed, draw=purple!60!white] table [x=pO2, y=Ii{1},]{dat/1e5iconctet1500.dat}; \addlegendentry{\ch{I_{i}^{\textperiodcentered}}};
        \addplot[no marks, dashed, draw=blue!50!white] table [x=pO2, y=IsubO{1},]{dat/1e5iconctet1500.dat}; \addlegendentry{\ch{I_{O}^{\textperiodcentered}}};
        \addplot[no marks, dashed, draw=green!60!black] table [x=pO2, y=IsubO{2},]{dat/1e5iconctet1500.dat}; \addlegendentry{\ch{I_{O}^{\textperiodcentered\textperiodcentered}}};
        \addplot[no marks, dashed, draw=black] table [x=pO2, y=IsubO{3},]{dat/1e5iconctet1500.dat}; \addlegendentry{\ch{I_{O}^{\textperiodcentered\textperiodcentered\textperiodcentered}}};
        \addplot[no marks, dashed, draw=orange!80!black] table [x=pO2, y=IsubZr{-3},]{dat/1e5iconctet1500.dat}; \addlegendentry{\ch{I_{Zr}^{'''}}};
%         \addplot[no marks, dashed, draw=pink] table [x=pO2, y=IsubZr{-4},]{dat/1e5iconctet1500.dat}; \addlegendentry{\ch{I_{Zr}^{''''}}};
%         \addplot[no marks, dashed, draw=purple] table [x=pO2, y=IsubZr{-5},]{dat/1e5iconctet1500.dat}; \addlegendentry{\ch{I_{Zr}^{'''''}}};
%         \addplot[no marks] table [x=pO2, y=Stoich,]{dat/1e5iconctet1500.dat}; \addlegendentry{Stoich};
\node at (-33.7,-0.5) {\textbf{a)}};
			\nextgroupplot[
		 xlabel={\ch{log_{10}}($p_{O_{2}}$) (atm)}, ylabel={\ch{log_{10}}([D]) (per f.u.)}, ymin=-10, ymax=0, xmin=-35, xmax=0, legend style={{draw=}, at={(0.40,0.97)}, anchor=north west, legend columns=4, nodes={scale=1, transform shape}}]
        \addplot[no marks, draw=blue!70!black] table [x=pO2, y=electrons,]{dat/1e3iconctet1500.dat}; \node at (-27,-1.7) {\ch{e^{'}}};
        \addplot[no marks, draw=red!85!black] table [x=pO2, y=holes,]{dat/1e3iconctet1500.dat}; \node at (-2.5,-2.1) {\ch{h^{\textperiodcentered}}};
        \addplot[no marks, draw=black!70!green] table [x=pO2, y=VO{2},]{dat/1e3iconctet1500.dat}; 
%         \addplot[no marks, draw=black!55!green] table [x=pO2, y=VO{1},]{dat/1e3iconctet1500.dat}; 
%         \addplot[no marks, draw=black!30!green] table [x=pO2, y=VO{0},]{dat/1e3iconctet1500.dat}; 
        \addplot[no marks, draw=yellow!85!blue] table [x=pO2, y=VM{-4},]{dat/1e3iconctet1500.dat}; 
%         \addplot[no marks, draw=yellow!75!blue] table [x=pO2, y=VM{-3},]{dat/1e3iconctet1500.dat}; 
%         \addplot[no marks, draw=yellow!65!blue] table [x=pO2, y=VM{-2},]{dat/1e3iconctet1500.dat}; 
%         \addplot[no marks, draw=yellow!55!blue] table [x=pO2, y=VM{-1},]{dat/1e3iconctet1500.dat}; 
%         \addplot[no marks, draw=yellow!45!blue] table [x=pO2, y=VM{0},]{dat/1e3iconctet1500.dat}; 
%         \addplot[no marks, draw=red!60!yellow] table [x=pO2, y=Oi{-2},]{dat/1e3iconctet1500.dat}; 
%         \addplot[no marks, draw=red!50!yellow] table [x=pO2, y=Oi{-1},]{dat/1e3iconctet1500.dat}; 
%         \addplot[no marks, draw=red!40!yellow] table [x=pO2, y=Oi{0},]{dat/1e3iconctet1500.dat}; 
%         \addplot[no marks, draw=green!80!pink] table [x=pO2, y=Mi{4},]{dat/1e3iconctet1500.dat}; 
%         \addplot[no marks, draw=green!70!pink] table [x=pO2, y=Mi{3},]{dat/1e3iconctet1500.dat}; 
%         \addplot[no marks, draw=green!60!pink] table [x=pO2, y=Mi{2},]{dat/1e3iconctet1500.dat}; 
%         \addplot[no marks, draw=green!50!pink] table [x=pO2, y=Mi{1},]{dat/1e3iconctet1500.dat}; 
%         \addplot[no marks, draw=green!40!pink] table [x=pO2, y=Mi{0},]{dat/1e3iconctet1500.dat}; 
%         \addplot[no marks, dashed, draw=red!70!black] table [x=pO2, y=Ii{0},]{dat/1e3iconctet1500.dat}; 
%         \addplot[no marks, dashed, draw=red!50!black] table [x=pO2, y=Ii{-1},]{dat/1e3iconctet1500.dat}; 
        \addplot[no marks, dashed, draw=purple!60!white] table [x=pO2, y=Ii{1},]{dat/1e3iconctet1500.dat}; 
        \addplot[no marks, dashed, draw=blue!50!white] table [x=pO2, y=IsubO{1},]{dat/1e3iconctet1500.dat}; \node at (-11,-2.6) {\ch{I_{O}^{\textperiodcentered}}};
        \addplot[no marks, dashed, draw=green!60!black] table [x=pO2, y=IsubO{2},]{dat/1e3iconctet1500.dat}; 
        \addplot[no marks, dashed, draw=black] table [x=pO2, y=IsubO{3},]{dat/1e3iconctet1500.dat}; 
        \addplot[no marks, dashed, draw=orange!80!black] table [x=pO2, y=IsubZr{-3},]{dat/1e3iconctet1500.dat}; 
%         \addplot[no marks, dashed, draw=pink] table [x=pO2, y=IsubZr{-4},]{dat/1e3iconctet1500.dat}; 
%         \addplot[no marks, dashed, draw=purple] table [x=pO2, y=IsubZr{-5},]{dat/1e3iconctet1500.dat}; 
%         \addplot[no marks] table [x=pO2, y=Stoich,]{dat/1e3iconctet1500.dat}; 
\node at (-33.7,-0.5) {\textbf{b)}};
			\end{groupplot}        
\end{tikzpicture} % 10e-3 iodine conc in tet
		\caption{Tetragonal phase Brouwer diagrams of point defects at iodine concentrations of a) $10^{-5}$ and b) $10^{-3}$, at a temperature of 1500 K.}
		\label{figure:tikzbrouwerconctet}
	\end{center}
\end{figure}

When the iodine concentration was increased to $10^{-3}$ parts/fu, a significant change in defect equilibria was predicted. The oxygen pressure at stoichiometry increased from $10^{-10}$ to $10^{-6.5}$ atm (for monoclinic \zirconia, it remained at $10^{-7.5}$ atm regardless of iodine concentration). Nevertheless, \ch{I_{O}^{*}} and \ch{I_{Zr}^{'''}} remain the dominant defect pair between oxygen pressures of $10^{-15}$ and $10^{-5}$ atm (as they are at the lower iodine concentration). However, \ch{I_{O}^{*}} and \ch{I_{Zr}^{'''}} became higher concentration defects than both intrinsic \ch{V_{O}^{**}} and \ch{V_{Zr}^{''''}} defects. We also observe that Zr vacancies no longer serve as the main negative charge-compensation defect near stoichiometry, leaving \ch{I_{Zr}^{'''}} as the most energetically favourable negatively-charged defect.  

Unlike in the Brouwer diagrams for the monoclinic phase, for the tetragonal phase, the concentration of iodine substitutional defects on oxygen sites decreases more steeply at high oxygen pressures, peaking near stoichiometry. \ch{I_{O}^{***}} in particular, which was the dominant defect at high oxygen pressures in monoclinic \zirconia , becomes insignificant under the same conditions in the tetragonal phase, with iodine confined to Zr sites. This behaviour is indicative of a `barrier' effect against iodine at high oxygen partial pressures, with oxygen out-competing iodine for oxygen sites. Given that the inner oxide is likely to have a higher tetragonal phase fraction than the external oxide, due to the incorporation of fission products, this result could help to explain why there appears to be an oxygen effect on PCI-related SCC of zirconium alloys \cite{hofmann1984stress}. 

Another effect considered was the space charge of the system. Electrons have a higher rate of diffusion than oxygen vacancies in \zirconia , leading to a build-up of oxygen vacancies near the metal-oxide interface as corrosion progresses \cite{bojinov2010influence}. This results in an overall positive charge (since the dominant oxygen vacancy is \ch{V_{O}^{**}}) referred to as a space charge. When included in our Brouwer diagrams, this space charge had a negligible effect on the concentration or charge state of iodine up to a charge of $10^{-1}$ holes per f.u. \zirconia . This corresponds to a high concentration of oxygen vacancies relative to the equilibrium concentration, predicting that a significant deviation from equilibrium is not expected near the metal oxide interface as a result of a positive space charge.

\section{Conclusions}

Iodine exhibits lower incorporation energies when occupying defects in monoclinic \zirconia\ than in the tetragonal phase. However, as monoclinic is the low-temperature phase, intrinsic defect concentrations will also be low, thereby requiring additional energy input to produce vacancies when the concentration of iodine is much larger than that of the intrinsic defects. This leads to relatively large concentrations of iodine interstitial defects predicted in the monoclinic Brouwer diagrams, as interstitial sites are always available in the lattice. 

Defects involving iodine in the +1 oxidation state are present in significant concentrations, especially in monoclinic \zirconia , indicating that filling of the $p$ electronic sub-shell is not always energetically favourable compared to forming the smaller iodine ionic radius developed through oxidation. 

The competition between iodine and oxygen for anion sites in \zirconia\ is phase and oxygen pressure dependent. At high oxygen pressures in monoclinic \zirconia , iodine in the +1 oxidation state is predicted to occupy oxygen sites and remains the dominant defect. In tetragonal \zirconia\ at high oxygen pressures, however, the concentration of iodine defects on anion sites decreases steeply, indicating a preference for iodine accommodated at zirconium cation sites. This is indicative of a barrier effect in the tetragonal phase with oxygen out-competing iodine for anion sites.

\chapter{Radioparagenesis of fission products in tetragonal \zirconia}

\label{ch:results3}

\section{Introduction}
\subsection{Radioparagenesis}

The nuclei of fission products immediately after a fission event are typically neutron-rich and unstable. In the case of iodine, the stable isotope is I-127, yet isotopes up to I-143 are produced during fission. This is the true for the fission of all large nuclei, including U-233 (thorium cycle), U-235 (conventional) and Pu-239 (breeder/MOX)

Stress-corrosion cracking (SCC) in nuclear fuel pins is an issue related to early loss of structural integrity of fuel assemblies in light water reactors (LWRs). In particular, the phenomenon of pellet-cladding interaction (PCI) in combination with SCC can lead to failures where the cladding is breached, exposing fuel to the coolant \cite{bcoxpelletclad1990}.     

This study follows previous work on defect equilibria in \zirconia\ to determine the oxide layer's effectiveness as a barrier to iodine \cite{kenichiodine2018}. It was found that the tetragonal phase of \zirconia\ is a greater barrier to iodine ingress than monoclinic \zirconia\ as the partial pressure of oxygen is increased. It is also known that tetragonal \zirconia\ will always be present on the inner surface of the cladding in significant quantities because it is self-stabilised by the stresses imposed as the oxide grows into the zirconium metal, in addition to compressive residual stresses induced by radiation damage. The iodine defect study, however, only informs us about one part of the SCC process. For a more holistic understanding, the life cycle of the iodine must be taken into account as well.     

%SCC studies of the internal surface of zirconium-based fuel claddings have been conducted, which indicate that iodine is likely to be one of the main corrosive species involved in promoting crack growth \cite{rosenbaum1966interaction, Cox1990Pellet-cladReview,Fregonese1998AmountIodine,Sidky1998IodineReview}. The exact mechanism for iodine SCC has not yet been determined due to difficulties observing the internal cladding surface in-situ, while experimental studies are not yet capable of reproducing the conditions under which such failures occur. This study focuses on the oxide on the internal surface of the fuel cladding, following from a previous study on iodine in the oxide layer. \\

Nuclear fuel claddings have unique materials challenges associated with them owing to the highly active environment and creation of unstable isotopes. Corrosive species in the pin such as iodine can be produced directly as a result of fission of uranium fuel. While it is known that iodine plays a role in SCC, one must also consider that these iodine nuclei are unstable. Fission of uranium will produce iodine precursors, mainly unstable isotopes of tellurium. Both iodine and tellurium are relatively common fission products, with combined independent yields from thermal fission of U$_{235}$ above 5\% \cite{kennett1956mass, iodine129fissionyield, imanishi1976independent, iodinefissionyields, iodine132, amiel1975odd}. 

Nuclei produced during fission are typically neutron-rich, resulting in decay modes such as $\beta-$ or neutron emission. In the case of tellurium, the vast majority of unstable isotopes will decay into iodine, which then decays into xenon with varying half-lives depending on the isotope. The decay chain continues with xenon nuclei decaying into caesium, many isotopes of which have half lives measured in years. At this point, fuel is typically retired long before a significant quantity of caesium decays into barium. For this reason we only consider the elements tellurium through caesium in this study. It should also be noted that the majority of thermal fission events occur in the outer rim of the fuel pellet, and a fission product penetration depth of up to 8 $\mu$m in \zirconia\ \cite{degueldre2001behaviour} suggests a large degree of fission product implantation within the oxide. With each nuclear decay comes a change in the chemical and therefore physical behaviour of the atom with its immediate environment. For example, an iodine dopant in \zirconia\ may decay into xenon which will then have a significantly different thermodynamic equilibrium site from the one it inherited.   

Determining the effect of each of these elements in the oxide layer may provide information about the initiation of SCC in fuel cladding. We have therefore adopted a quantum-mechanical calculation approach to model the behaviour of the decay chain elements tellurium through caesium within tetragonal phase zirconia. 

\begin{itemize}
\item We propose that crack initiation on the internal surface of the cladding may be in part due to radioparagenesis of fission products
\item One mechanism is neutron-rich iodine making its way through the monoclinic \zirconia\ before being stopped by the highly passivating tetragonal \zirconia\ closer to the metal interface.
\item The iodine nucleus then decays by beta- particle emission, converting from an iodine to a xenon nucleus.
\item This xenon ion quickly fills its valence shell to the noble gas configuration.
\item The uncharged xenon atom then imposes a large strain on the surrounding \zirconia\ due to the volume mismatch.
\item This strain weakens the monoclinic \zirconia , and promotes crack initiation (new surface relieves the strain imposed by the xenon).
\item The tetragonal \zirconia , now less constrained by the monoclinic layer, expands and becomes less inhibiting to iodine and oxygen ingress.
\item If the iodine partial pressure is high enough relative to the oxygen pressure, the \zirconia\ layer will fail to impede iodine corrosive attack on the zirconium metal.
\end{itemize}

Xenon in a reactor will also eventually decay by beta- emission into caesium, a much more chemically reactive element.

\subsection{Site preference of fission products}

\begin{itemize}

\item \textbf{Tellurium} is a group 6 element like oxygen, but it displays some metallic behaviour.
\item Because of its electronic structure, it may be expected to display preference for the oxygen site in \zirconia .
\item It's metallic properties and low electronegativity, however, suggest that it may be able to fill a cation site instead, but this would require the creation of oxygen vacancies since it has a lower valence than zirconium.
\item \textbf{Iodine} was shown in Chapter 4 to adopt either oxygen and zirconium sites under the right conditions
\item \textbf{Xenon} is a noble gas, but is still able to form compounds with very strong oxidising agents (e.g. XeF4). It's large size (comparison here) may make it unfavourable in both cation and anion sites, thus imposing a large lattice strain.
\item \textbf{Caesium} is a group 1 metal. Its second ionisation energy is very large (removing an electron from a full $p$ sub-shell), likely making it very unfavourable on a zirconium site, only made worse by its size.
\end{itemize}

\subsection{Fission product penetration}

\begin{itemize}
\item Fission products can penetrate up to 10 microns into the cladding, with most deposition occurring at 5 microns (REF)
\item This means we can expect some existing fission products in the cladding before crack-assisted diffusion becomes relevant
\item Therefore some defects will already exist, and the Brouwer diagrams lets us predict what the thermodynamically stable (most likely) ones will be.
\end{itemize}

\section{Methodology}
\subsection{Simulation parameters}

\begin{itemize}
\item energy per atom convergence
\item displacement per atom convergence
\item plane-wave cutoff
\item k-point spacing
\item PBE GGA exchange correlation functional
\end{itemize}

\subsection{Brouwer diagram generation}

\begin{itemize}
\item Defect concentration against oxygen partial pressure 
\item Find Fermi level that leads to charge neutrality
\end{itemize}

\subsection{Defect Volumes}

\begin{itemize}
\item compare constant pressure relaxation of defective to perfect supercell
\end{itemize}

\section{Defect equilibria}
\subsection{Tellurium}

\begin{landscape}
\begin{figure}[htp] % Tellurium
\begin{center}
\begin{tikzpicture}
	\begin{axis}
		[width=11.22cm, xlabel={\ch{log_{10}}($p_{O_{2}}$) (atm)}, ylabel={\ch{log_{10}}([D]) (per f.u.)}, ymin=-10, ymax=0, xmin=-35, xmax=0, legend style={{draw=}, at={(0.30,1.47)}, anchor=north west, legend columns=3, nodes={scale=0.75, transform shape}}]
        \addplot[no marks, draw=blue!70!black] table [x=pO2, y=electrons,]{dat/te_tet_10-5.dat}; \addlegendentry{\ch{e^{'}}}; %\node at (-26.0,-1.9) {\ch{e^{'}}};
        \addplot[no marks, draw=red!85!black] table [x=pO2, y=holes,]{dat/te_tet_10-5.dat}; \addlegendentry{\ch{h^{\textperiodcentered}}}; %\node at (-7,-3.6) {\ch{h^{\textperiodcentered}}};
        \addplot[no marks, draw=black!70!green] table [x=pO2, y=VO{2},]{dat/te_tet_10-5.dat}; \addlegendentry{\ch{V_{O}^{\textperiodcentered\textperiodcentered}}}; %\node at (-26.7,-3.3) {\ch{V_{O}^{\textperiodcentered\textperiodcentered}}};
         \addplot[no marks, draw=black!55!green] table [x=pO2, y=VO{1},]{dat/te_tet_10-5.dat}; \addlegendentry{\ch{V_{O}^{\textperiodcentered}}};
         \addplot[no marks, draw=black!30!green] table [x=pO2, y=VO{0},]{dat/te_tet_10-5.dat}; \addlegendentry{\ch{V_{O}^{x}}};
        \addplot[no marks, draw=yellow!85!blue] table [x=pO2, y=VM{-4},]{dat/te_tet_10-5.dat}; \addlegendentry{\ch{V_{Zr}^{''''}}};
         \addplot[no marks, draw=yellow!75!blue] table [x=pO2, y=VM{-3},]{dat/te_tet_10-5.dat}; \addlegendentry{\ch{V_{Zr}^{'''}}};
         \addplot[no marks, draw=yellow!65!blue] table [x=pO2, y=VM{-2},]{dat/te_tet_10-5.dat}; \addlegendentry{\ch{V_{Zr}^{''}}};
         \addplot[no marks, draw=yellow!55!blue] table [x=pO2, y=VM{-1},]{dat/te_tet_10-5.dat}; \addlegendentry{\ch{V_{Zr}^{'}}};
         \addplot[no marks, draw=yellow!45!blue] table [x=pO2, y=VM{0},]{dat/te_tet_10-5.dat}; \addlegendentry{\ch{V_{Zr}^{x}}};
         \addplot[no marks, draw=red!60!yellow] table [x=pO2, y=Oi{-2},]{dat/te_tet_10-5.dat}; \addlegendentry{\ch{O_{i}^{''}}};
         \addplot[no marks, draw=red!50!yellow] table [x=pO2, y=Oi{-1},]{dat/te_tet_10-5.dat}; \addlegendentry{\ch{O_{i}^{'}}};
         \addplot[no marks, draw=red!40!yellow] table [x=pO2, y=Oi{0},]{dat/te_tet_10-5.dat}; \addlegendentry{\ch{O_{i}^{x}}};
         \addplot[no marks, draw=green!80!pink] table [x=pO2, y=Mi{4},]{dat/te_tet_10-5.dat}; \addlegendentry{\ch{Zr_{i}^{\textperiodcentered\textperiodcentered\textperiodcentered\textperiodcentered}}};
         \addplot[no marks, draw=green!70!pink] table [x=pO2, y=Mi{3},]{dat/te_tet_10-5.dat}; \addlegendentry{\ch{Zr_{i}^{\textperiodcentered\textperiodcentered\textperiodcentered}}};
         \addplot[no marks, draw=green!60!pink] table [x=pO2, y=Mi{2},]{dat/te_tet_10-5.dat}; \addlegendentry{\ch{Zr_{i}^{\textbf{\textperiodcentered\textperiodcentered}}}};
        \addplot[no marks, draw=green!50!pink] table [x=pO2, y=Mi{1},]{dat/te_tet_10-5.dat}; \addlegendentry{\ch{Zr_{i}^{\textperiodcentered}}};
         \addplot[no marks, draw=green!40!pink] table [x=pO2, y=Mi{0},]{dat/te_tet_10-5.dat}; \addlegendentry{\ch{Zr_{i}^{x}}};
         \addplot[no marks, dashed, draw=red!70!black] table [x=pO2, y=Tei{0},]{dat/te_tet_10-5.dat}; \addlegendentry{\ch{Te_{i}^{x}}};
         \addplot[no marks, dashed, draw=red!50!black] table [x=pO2, y=Tei{-1},]{dat/te_tet_10-5.dat}; \addlegendentry{\ch{Te_{i}^{'}}};
        \addplot[no marks, dashed, draw=purple] table [x=pO2, y=Tei{1},]{dat/te_tet_10-5.dat}; \addlegendentry{\ch{Te_{i}^{\textperiodcentered}}};
        \addplot[no marks, dashed, draw=blue!50!white] table [x=pO2, y=TesubO{1},]{dat/te_tet_10-5.dat}; \addlegendentry{\ch{Te_{O}^{\textperiodcentered}}};
        \addplot[no marks, dashed, draw=orange] table [x=pO2, y=TesubO{2},]{dat/te_tet_10-5.dat}; \addlegendentry{\ch{Te_{O}^{\textperiodcentered\textperiodcentered}}};
        \addplot[no marks, dashed, draw=black] table [x=pO2, y=TesubO{3},]{dat/te_tet_10-5.dat}; \addlegendentry{\ch{Te_{O}^{\textperiodcentered\textperiodcentered\textperiodcentered}}};
        \addplot[no marks, dashed, draw=green] table [x=pO2, y=TesubZr{-3},]{dat/te_tet_10-5.dat}; \addlegendentry{\ch{Te_{Zr}^{'''}}};
         \addplot[no marks, dashed, draw=blue] table [x=pO2, y=TesubZr{-4},]{dat/te_tet_10-5.dat}; \addlegendentry{\ch{Te_{Zr}^{''''}}};
         \addplot[no marks, dashed, draw=red] table [x=pO2, y=TesubZr{-5},]{dat/te_tet_10-5.dat}; \addlegendentry{\ch{Te_{Zr}^{'''''}}};
%         \addplot[no marks] table [x=pO2, y=Stoich,]{Te_tet.dat}; \addlegendentry{Stoich};
%\node at (-33.7,-0.5) {\textbf{a)}};
			\end{axis}            
\end{tikzpicture}
\begin{tikzpicture} % TELLURIUM 2
	\begin{axis} % change width to 8.22cm for portrait
		[width=11.22cm, xlabel={\ch{log_{10}}($p_{O_{2}}$) (atm)}, yticklabels={}, ymin=-10, ymax=0, xmin=-35, xmax=0]
        \addplot[no marks, draw=blue!70!black] table [x=pO2, y=electrons,]{dat/te_tet_10-3.dat}; %\node at (-27,-1.7) {\ch{e^{'}}};
        \addplot[no marks, draw=red!85!black] table [x=pO2, y=holes,]{dat/te_tet_10-3.dat}; %\node at (-2.5,-2.1) {\ch{h^{\textperiodcentered}}};
        \addplot[no marks, draw=black!70!green] table [x=pO2, y=VO{2},]{dat/te_tet_10-3.dat}; 
         \addplot[no marks, draw=black!55!green] table [x=pO2, y=VO{1},]{dat/te_tet_10-3.dat}; 
         \addplot[no marks, draw=black!30!green] table [x=pO2, y=VO{0},]{dat/te_tet_10-3.dat}; 
        \addplot[no marks, draw=yellow!85!blue] table [x=pO2, y=VM{-4},]{dat/te_tet_10-3.dat}; 
%         \addplot[no marks, draw=yellow!75!blue] table [x=pO2, y=VM{-3},]{dat/te_tet_10-3.dat}; 
%         \addplot[no marks, draw=yellow!65!blue] table [x=pO2, y=VM{-2},]{dat/te_tet_10-3.dat}; 
%         \addplot[no marks, draw=yellow!55!blue] table [x=pO2, y=VM{-1},]{dat/te_tet_10-3.dat}; 
%         \addplot[no marks, draw=yellow!45!blue] table [x=pO2, y=VM{0},]{dat/te_tet_10-3.dat}; 
%         \addplot[no marks, draw=red!60!yellow] table [x=pO2, y=Oi{-2},]{dat/te_tet_10-3.dat}; 
%         \addplot[no marks, draw=red!50!yellow] table [x=pO2, y=Oi{-1},]{dat/te_tet_10-3.dat}; 
%         \addplot[no marks, draw=red!40!yellow] table [x=pO2, y=Oi{0},]{dat/te_tet_10-3.dat}; 
%         \addplot[no marks, draw=green!80!pink] table [x=pO2, y=Mi{4},]{dat/te_tet_10-3.dat}; 
%         \addplot[no marks, draw=green!70!pink] table [x=pO2, y=Mi{3},]{dat/te_tet_10-3.dat}; 
%         \addplot[no marks, draw=green!60!pink] table [x=pO2, y=Mi{2},]{dat/te_tet_10-3.dat}; 
%         \addplot[no marks, draw=green!50!pink] table [x=pO2, y=Mi{1},]{dat/te_tet_10-3.dat}; 
%         \addplot[no marks, draw=green!40!pink] table [x=pO2, y=Mi{0},]{dat/te_tet_10-3.dat}; 
        \addplot[no marks, dashed, draw=red!70!black] table [x=pO2, y=Tei{0},]{dat/te_tet_10-3.dat}; 
        \addplot[no marks, dashed, draw=red!50!black] table [x=pO2, y=Tei{-1},]{dat/te_tet_10-3.dat}; 
        \addplot[no marks, dashed, draw=purple] table [x=pO2, y=Tei{1},]{dat/te_tet_10-3.dat}; 
        \addplot[no marks, dashed, draw=blue!50!white] table [x=pO2, y=TesubO{1},]{dat/te_tet_10-3.dat}; %\node at (-11,-2.6) {\ch{I_{O}^{\textperiodcentered}}};
        \addplot[no marks, dashed, draw=orange] table [x=pO2, y=TesubO{2},]{dat/te_tet_10-3.dat}; 
        \addplot[no marks, dashed, draw=black] table [x=pO2, y=TesubO{3},]{dat/te_tet_10-3.dat}; 
        \addplot[no marks, dashed, draw=green] table [x=pO2, y=TesubZr{-3},]{dat/te_tet_10-3.dat}; 
        \addplot[no marks, dashed, draw=blue] table [x=pO2, y=TesubZr{-4},]{dat/te_tet_10-3.dat}; 
        \addplot[no marks, dashed, draw=red] table [x=pO2, y=TesubZr{-5},]{dat/te_tet_10-3.dat}; 
%        \addplot[no marks] table [x=pO2, y=Stoich,]{dat/te_tet_10-3.dat}; 
%\node at (-33.7,-0.5) {\textbf{b)}};
			\end{axis}            
\end{tikzpicture}
		\caption{Tetragonal phase Brouwer diagrams of point defects at Tellurium concentrations of a) $10^{-5}$ and b) $10^{-3}$, at a temperature of 1500 K. Space charge = 0}
		\label{figure:telluriumbrouwer-5-3}
	\end{center}
\end{figure}
\end{landscape}

\subsection{Iodine}

\begin{itemize}
\item Should we just reference the Brouwer diagram for iodine again? 
\end{itemize}

\subsection{Xenon}

\begin{itemize}
\item Xenon point defects showed a change in behaviour at high and low oxygen pressures
\end{itemize}

\begin{landscape}
\begin{figure}[htp] % XENON
\begin{center}
\begin{tikzpicture}
	\begin{axis}
		[width=11.22cm, xlabel={\ch{log_{10}}($p_{O_{2}}$) (atm)}, ylabel={\ch{log_{10}}([D]) (per f.u.)}, ymin=-10, ymax=0, xmin=-35, xmax=0, legend style={{draw=}, at={(0.30,1.47)}, anchor=north west, legend columns=3, nodes={scale=0.75, transform shape}}]
        \addplot[no marks, draw=blue!70!black] table [x=pO2, y=electrons,]{dat/xe_tet_10-5.dat}; \addlegendentry{\ch{e^{'}}}; %\node at (-26.0,-1.9) {\ch{e^{'}}};
        \addplot[no marks, draw=red!85!black] table [x=pO2, y=holes,]{dat/xe_tet_10-5.dat}; \addlegendentry{\ch{h^{\textperiodcentered}}}; %\node at (-7,-3.6) {\ch{h^{\textperiodcentered}}};
        \addplot[no marks, draw=black!70!green] table [x=pO2, y=VO{2},]{dat/xe_tet_10-5.dat}; \addlegendentry{\ch{V_{O}^{\textperiodcentered\textperiodcentered}}}; %\node at (-26.7,-3.3) {\ch{V_{O}^{\textperiodcentered\textperiodcentered}}};
         \addplot[no marks, draw=black!55!green] table [x=pO2, y=VO{1},]{dat/xe_tet_10-5.dat}; \addlegendentry{\ch{V_{O}^{\textperiodcentered}}};
         \addplot[no marks, draw=black!30!green] table [x=pO2, y=VO{0},]{dat/xe_tet_10-5.dat}; \addlegendentry{\ch{V_{O}^{x}}};
        \addplot[no marks, draw=yellow!85!blue] table [x=pO2, y=VM{-4},]{dat/xe_tet_10-5.dat}; \addlegendentry{\ch{V_{Zr}^{''''}}};
         \addplot[no marks, draw=yellow!75!blue] table [x=pO2, y=VM{-3},]{dat/xe_tet_10-5.dat}; \addlegendentry{\ch{V_{Zr}^{'''}}};
         \addplot[no marks, draw=yellow!65!blue] table [x=pO2, y=VM{-2},]{dat/xe_tet_10-5.dat}; \addlegendentry{\ch{V_{Zr}^{''}}};
         \addplot[no marks, draw=yellow!55!blue] table [x=pO2, y=VM{-1},]{dat/xe_tet_10-5.dat}; \addlegendentry{\ch{V_{Zr}^{'}}};
         \addplot[no marks, draw=yellow!45!blue] table [x=pO2, y=VM{0},]{dat/xe_tet_10-5.dat}; \addlegendentry{\ch{V_{Zr}^{x}}};
         \addplot[no marks, draw=red!60!yellow] table [x=pO2, y=Oi{-2},]{dat/xe_tet_10-5.dat}; \addlegendentry{\ch{O_{i}^{''}}};
         \addplot[no marks, draw=red!50!yellow] table [x=pO2, y=Oi{-1},]{dat/xe_tet_10-5.dat}; \addlegendentry{\ch{O_{i}^{'}}};
         \addplot[no marks, draw=red!40!yellow] table [x=pO2, y=Oi{0},]{dat/xe_tet_10-5.dat}; \addlegendentry{\ch{O_{i}^{x}}};
         \addplot[no marks, draw=green!80!pink] table [x=pO2, y=Mi{4},]{dat/xe_tet_10-5.dat}; \addlegendentry{\ch{Zr_{i}^{\textperiodcentered\textperiodcentered\textperiodcentered\textperiodcentered}}};
         \addplot[no marks, draw=green!70!pink] table [x=pO2, y=Mi{3},]{dat/xe_tet_10-5.dat}; \addlegendentry{\ch{Zr_{i}^{\textperiodcentered\textperiodcentered\textperiodcentered}}};
         \addplot[no marks, draw=green!60!pink] table [x=pO2, y=Mi{2},]{dat/xe_tet_10-5.dat}; \addlegendentry{\ch{Zr_{i}^{\textbf{\textperiodcentered\textperiodcentered}}}};
        \addplot[no marks, draw=green!50!pink] table [x=pO2, y=Mi{1},]{dat/xe_tet_10-5.dat}; \addlegendentry{\ch{Zr_{i}^{\textperiodcentered}}};
         \addplot[no marks, draw=green!40!pink] table [x=pO2, y=Mi{0},]{dat/xe_tet_10-5.dat}; \addlegendentry{\ch{Zr_{i}^{x}}};
         \addplot[no marks, dashed, draw=red!70!black] table [x=pO2, y=Xei{0},]{dat/xe_tet_10-5.dat}; \addlegendentry{\ch{Xe_{i}^{x}}};
         \addplot[no marks, dashed, draw=red!50!black] table [x=pO2, y=Xei{-1},]{dat/xe_tet_10-5.dat}; \addlegendentry{\ch{Xe_{i}^{'}}};
        \addplot[no marks, dashed, draw=purple] table [x=pO2, y=Xei{1},]{dat/xe_tet_10-5.dat}; \addlegendentry{\ch{Xe_{i}^{\textperiodcentered}}};
        \addplot[no marks, dashed, draw=blue!50!white] table [x=pO2, y=XesubO{1},]{dat/xe_tet_10-5.dat}; \addlegendentry{\ch{Xe_{O}^{\textperiodcentered}}};
        \addplot[no marks, dashed, draw=orange] table [x=pO2, y=XesubO{2},]{dat/xe_tet_10-5.dat}; \addlegendentry{\ch{Xe_{O}^{\textperiodcentered\textperiodcentered}}};
        \addplot[no marks, dashed, draw=black] table [x=pO2, y=XesubO{3},]{dat/xe_tet_10-5.dat}; \addlegendentry{\ch{Xe_{O}^{\textperiodcentered\textperiodcentered\textperiodcentered}}};
        \addplot[no marks, dashed, draw=green] table [x=pO2, y=XesubZr{-3},]{dat/xe_tet_10-5.dat}; \addlegendentry{\ch{Xe_{Zr}^{'''}}};
         \addplot[no marks, dashed, draw=blue] table [x=pO2, y=XesubZr{-4},]{dat/xe_tet_10-5.dat}; \addlegendentry{\ch{Xe_{Zr}^{''''}}};
         \addplot[no marks, dashed, draw=red] table [x=pO2, y=XesubZr{-5},]{dat/xe_tet_10-5.dat}; \addlegendentry{\ch{Xe_{Zr}^{'''''}}};
%         \addplot[no marks] table [x=pO2, y=Stoich,]{xe_tet.dat}; \addlegendentry{Stoich};
%\node at (-33.7,-0.5) {\textbf{a)}};
			\end{axis}            
\end{tikzpicture}
\begin{tikzpicture} % XENON 2
	\begin{axis} % change width to 8.22cm for portrait
		[width=11.22cm, xlabel={\ch{log_{10}}($p_{O_{2}}$) (atm)}, yticklabels={}, ymin=-10, ymax=0, xmin=-35, xmax=0]
        \addplot[no marks, draw=blue!70!black] table [x=pO2, y=electrons,]{dat/xe_tet_10-3.dat}; %\node at (-27,-1.7) {\ch{e^{'}}};
        \addplot[no marks, draw=red!85!black] table [x=pO2, y=holes,]{dat/xe_tet_10-3.dat}; %\node at (-2.5,-2.1) {\ch{h^{\textperiodcentered}}};
        \addplot[no marks, draw=black!70!green] table [x=pO2, y=VO{2},]{dat/xe_tet_10-3.dat}; 
         \addplot[no marks, draw=black!55!green] table [x=pO2, y=VO{1},]{dat/xe_tet_10-3.dat}; 
         \addplot[no marks, draw=black!30!green] table [x=pO2, y=VO{0},]{dat/xe_tet_10-3.dat}; 
        \addplot[no marks, draw=yellow!85!blue] table [x=pO2, y=VM{-4},]{dat/xe_tet_10-3.dat}; 
%         \addplot[no marks, draw=yellow!75!blue] table [x=pO2, y=VM{-3},]{dat/xe_tet_10-3.dat}; 
%         \addplot[no marks, draw=yellow!65!blue] table [x=pO2, y=VM{-2},]{dat/xe_tet_10-3.dat}; 
%         \addplot[no marks, draw=yellow!55!blue] table [x=pO2, y=VM{-1},]{dat/xe_tet_10-3.dat}; 
%         \addplot[no marks, draw=yellow!45!blue] table [x=pO2, y=VM{0},]{dat/xe_tet_10-3.dat}; 
%         \addplot[no marks, draw=red!60!yellow] table [x=pO2, y=Oi{-2},]{dat/xe_tet_10-3.dat}; 
%         \addplot[no marks, draw=red!50!yellow] table [x=pO2, y=Oi{-1},]{dat/xe_tet_10-3.dat}; 
%         \addplot[no marks, draw=red!40!yellow] table [x=pO2, y=Oi{0},]{dat/xe_tet_10-3.dat}; 
%         \addplot[no marks, draw=green!80!pink] table [x=pO2, y=Mi{4},]{dat/xe_tet_10-3.dat}; 
%         \addplot[no marks, draw=green!70!pink] table [x=pO2, y=Mi{3},]{dat/xe_tet_10-3.dat}; 
%         \addplot[no marks, draw=green!60!pink] table [x=pO2, y=Mi{2},]{dat/xe_tet_10-3.dat}; 
%         \addplot[no marks, draw=green!50!pink] table [x=pO2, y=Mi{1},]{dat/xe_tet_10-3.dat}; 
%         \addplot[no marks, draw=green!40!pink] table [x=pO2, y=Mi{0},]{dat/xe_tet_10-3.dat}; 
        \addplot[no marks, dashed, draw=red!70!black] table [x=pO2, y=Xei{0},]{dat/xe_tet_10-3.dat}; 
        \addplot[no marks, dashed, draw=red!50!black] table [x=pO2, y=Xei{-1},]{dat/xe_tet_10-3.dat}; 
        \addplot[no marks, dashed, draw=purple] table [x=pO2, y=Xei{1},]{dat/xe_tet_10-3.dat}; 
        \addplot[no marks, dashed, draw=blue!50!white] table [x=pO2, y=XesubO{1},]{dat/xe_tet_10-3.dat}; %\node at (-11,-2.6) {\ch{I_{O}^{\textperiodcentered}}};
        \addplot[no marks, dashed, draw=orange] table [x=pO2, y=XesubO{2},]{dat/xe_tet_10-3.dat}; 
        \addplot[no marks, dashed, draw=black] table [x=pO2, y=XesubO{3},]{dat/xe_tet_10-3.dat}; 
        \addplot[no marks, dashed, draw=green] table [x=pO2, y=XesubZr{-3},]{dat/xe_tet_10-3.dat}; 
        \addplot[no marks, dashed, draw=blue] table [x=pO2, y=XesubZr{-4},]{dat/xe_tet_10-3.dat}; 
        \addplot[no marks, dashed, draw=red] table [x=pO2, y=XesubZr{-5},]{dat/xe_tet_10-3.dat}; 
%        \addplot[no marks] table [x=pO2, y=Stoich,]{dat/xe_tet_10-3.dat}; 
%\node at (-33.7,-0.5) {\textbf{b)}};
			\end{axis}            
\end{tikzpicture}
		\caption{Tetragonal phase Brouwer diagrams of point defects at Xenon concentrations of a) $10^{-5}$ and b) $10^{-3}$, at a temperature of 1500 K. Space charge = 0}
		%\label{figure:tikzbrouwerconctet}
	\end{center}
\end{figure}
\end{landscape}

\subsection{Caesium}

\begin{itemize}
\item Cs point defects didn't show much change in behaviour
\item Defects behaviour strongly follows single ionisation preference (as expected).
\end{itemize}

\begin{landscape}
\begin{figure}[htp] % CAESIUM
\begin{center}
\begin{tikzpicture}
	\begin{axis}
		[width=11.22cm, xlabel={\ch{log_{10}}($p_{O_{2}}$) (atm)}, ylabel={\ch{log_{10}}([D]) (per f.u.)}, ymin=-10, ymax=0, xmin=-35, xmax=0, legend style={{draw=}, at={(0.30,1.47)}, anchor=north west, legend columns=3, nodes={scale=0.75, transform shape}}]
        \addplot[no marks, draw=blue!70!black] table [x=pO2, y=electrons,]{dat/cs_tet_10-5.dat}; \addlegendentry{\ch{e^{'}}}; %\node at (-26.0,-1.9) {\ch{e^{'}}};
        \addplot[no marks, draw=red!85!black] table [x=pO2, y=holes,]{dat/cs_tet_10-5.dat}; \addlegendentry{\ch{h^{\textperiodcentered}}}; %\node at (-7,-3.6) {\ch{h^{\textperiodcentered}}};
        \addplot[no marks, draw=black!70!green] table [x=pO2, y=VO{2},]{dat/cs_tet_10-5.dat}; \addlegendentry{\ch{V_{O}^{\textperiodcentered\textperiodcentered}}}; %\node at (-26.7,-3.3) {\ch{V_{O}^{\textperiodcentered\textperiodcentered}}};
         \addplot[no marks, draw=black!55!green] table [x=pO2, y=VO{1},]{dat/cs_tet_10-5.dat}; \addlegendentry{\ch{V_{O}^{\textperiodcentered}}};
         \addplot[no marks, draw=black!30!green] table [x=pO2, y=VO{0},]{dat/cs_tet_10-5.dat}; \addlegendentry{\ch{V_{O}^{x}}};
        \addplot[no marks, draw=yellow!85!blue] table [x=pO2, y=VM{-4},]{dat/cs_tet_10-5.dat}; \addlegendentry{\ch{V_{Zr}^{''''}}};
         \addplot[no marks, draw=yellow!75!blue] table [x=pO2, y=VM{-3},]{dat/cs_tet_10-5.dat}; \addlegendentry{\ch{V_{Zr}^{'''}}};
         \addplot[no marks, draw=yellow!65!blue] table [x=pO2, y=VM{-2},]{dat/cs_tet_10-5.dat}; \addlegendentry{\ch{V_{Zr}^{''}}};
         \addplot[no marks, draw=yellow!55!blue] table [x=pO2, y=VM{-1},]{dat/cs_tet_10-5.dat}; \addlegendentry{\ch{V_{Zr}^{'}}};
         \addplot[no marks, draw=yellow!45!blue] table [x=pO2, y=VM{0},]{dat/cs_tet_10-5.dat}; \addlegendentry{\ch{V_{Zr}^{x}}};
         \addplot[no marks, draw=red!60!yellow] table [x=pO2, y=Oi{-2},]{dat/cs_tet_10-5.dat}; \addlegendentry{\ch{O_{i}^{''}}};
         \addplot[no marks, draw=red!50!yellow] table [x=pO2, y=Oi{-1},]{dat/cs_tet_10-5.dat}; \addlegendentry{\ch{O_{i}^{'}}};
         \addplot[no marks, draw=red!40!yellow] table [x=pO2, y=Oi{0},]{dat/cs_tet_10-5.dat}; \addlegendentry{\ch{O_{i}^{x}}};
         \addplot[no marks, draw=green!80!pink] table [x=pO2, y=Mi{4},]{dat/cs_tet_10-5.dat}; \addlegendentry{\ch{Zr_{i}^{\textperiodcentered\textperiodcentered\textperiodcentered\textperiodcentered}}};
         \addplot[no marks, draw=green!70!pink] table [x=pO2, y=Mi{3},]{dat/cs_tet_10-5.dat}; \addlegendentry{\ch{Zr_{i}^{\textperiodcentered\textperiodcentered\textperiodcentered}}};
         \addplot[no marks, draw=green!60!pink] table [x=pO2, y=Mi{2},]{dat/cs_tet_10-5.dat}; \addlegendentry{\ch{Zr_{i}^{\textbf{\textperiodcentered\textperiodcentered}}}};
        \addplot[no marks, draw=green!50!pink] table [x=pO2, y=Mi{1},]{dat/cs_tet_10-5.dat}; \addlegendentry{\ch{Zr_{i}^{\textperiodcentered}}};
         \addplot[no marks, draw=green!40!pink] table [x=pO2, y=Mi{0},]{dat/cs_tet_10-5.dat}; \addlegendentry{\ch{Zr_{i}^{x}}};
         \addplot[no marks, dashed, draw=red!70!black] table [x=pO2, y=Csi{0},]{dat/cs_tet_10-5.dat}; \addlegendentry{\ch{Cs_{i}^{x}}};
         \addplot[no marks, dashed, draw=red!50!black] table [x=pO2, y=Csi{-1},]{dat/cs_tet_10-5.dat}; \addlegendentry{\ch{Cs_{i}^{'}}};
        \addplot[no marks, dashed, draw=purple] table [x=pO2, y=Csi{1},]{dat/cs_tet_10-5.dat}; \addlegendentry{\ch{Cs_{i}^{\textperiodcentered}}};
        \addplot[no marks, dashed, draw=blue!50!white] table [x=pO2, y=CssubO{1},]{dat/cs_tet_10-5.dat}; \addlegendentry{\ch{Cs_{O}^{\textperiodcentered}}};
        \addplot[no marks, dashed, draw=green!60!black] table [x=pO2, y=CssubO{2},]{dat/cs_tet_10-5.dat}; \addlegendentry{\ch{Cs_{O}^{\textperiodcentered\textperiodcentered}}};
        \addplot[no marks, dashed, draw=black] table [x=pO2, y=CssubO{3},]{dat/cs_tet_10-5.dat}; \addlegendentry{\ch{Cs_{O}^{\textperiodcentered\textperiodcentered\textperiodcentered}}};
        \addplot[no marks, dashed, draw=orange!80!black] table [x=pO2, y=CssubZr{-3},]{dat/cs_tet_10-5.dat}; \addlegendentry{\ch{Cs_{Zr}^{'''}}};
         \addplot[no marks, dashed, draw=pink] table [x=pO2, y=CssubZr{-4},]{dat/cs_tet_10-5.dat}; \addlegendentry{\ch{Cs_{Zr}^{''''}}};
         \addplot[no marks, dashed, draw=purple] table [x=pO2, y=CssubZr{-5},]{dat/cs_tet_10-5.dat}; \addlegendentry{\ch{Cs_{Zr}^{'''''}}};
%         \addplot[no marks] table [x=pO2, y=Stoich,]{cs_tet.dat}; \addlegendentry{Stoich};
%\node at (-33.7,-0.5) {\textbf{a)}};
			\end{axis}            
\end{tikzpicture}
\begin{tikzpicture} % CAESIUM 2
	\begin{axis}
		[width=11.22cm, xlabel={\ch{log_{10}}($p_{O_{2}}$) (atm)}, yticklabels={}, ymin=-10, ymax=0, xmin=-35, xmax=0]
        \addplot[no marks, draw=blue!70!black] table [x=pO2, y=electrons,]{dat/cs_tet_10-3.dat}; %\node at (-27,-1.7) {\ch{e^{'}}};
        \addplot[no marks, draw=red!85!black] table [x=pO2, y=holes,]{dat/cs_tet_10-3.dat}; %\node at (-2.5,-2.1) {\ch{h^{\textperiodcentered}}};
        \addplot[no marks, draw=black!70!green] table [x=pO2, y=VO{2},]{dat/cs_tet_10-3.dat}; 
         \addplot[no marks, draw=black!55!green] table [x=pO2, y=VO{1},]{dat/cs_tet_10-3.dat}; 
         \addplot[no marks, draw=black!30!green] table [x=pO2, y=VO{0},]{dat/cs_tet_10-3.dat}; 
        \addplot[no marks, draw=yellow!85!blue] table [x=pO2, y=VM{-4},]{dat/cs_tet_10-3.dat}; 
%         \addplot[no marks, draw=yellow!75!blue] table [x=pO2, y=VM{-3},]{dat/cs_tet_10-3.dat}; 
%         \addplot[no marks, draw=yellow!65!blue] table [x=pO2, y=VM{-2},]{dat/cs_tet_10-3.dat}; 
%         \addplot[no marks, draw=yellow!55!blue] table [x=pO2, y=VM{-1},]{dat/cs_tet_10-3.dat}; 
%         \addplot[no marks, draw=yellow!45!blue] table [x=pO2, y=VM{0},]{dat/cs_tet_10-3.dat}; 
%         \addplot[no marks, draw=red!60!yellow] table [x=pO2, y=Oi{-2},]{dat/cs_tet_10-3.dat}; 
%         \addplot[no marks, draw=red!50!yellow] table [x=pO2, y=Oi{-1},]{dat/cs_tet_10-3.dat}; 
%         \addplot[no marks, draw=red!40!yellow] table [x=pO2, y=Oi{0},]{dat/cs_tet_10-3.dat}; 
%         \addplot[no marks, draw=green!80!pink] table [x=pO2, y=Mi{4},]{dat/cs_tet_10-3.dat}; 
%         \addplot[no marks, draw=green!70!pink] table [x=pO2, y=Mi{3},]{dat/cs_tet_10-3.dat}; 
%         \addplot[no marks, draw=green!60!pink] table [x=pO2, y=Mi{2},]{dat/cs_tet_10-3.dat}; 
%         \addplot[no marks, draw=green!50!pink] table [x=pO2, y=Mi{1},]{dat/cs_tet_10-3.dat}; 
%         \addplot[no marks, draw=green!40!pink] table [x=pO2, y=Mi{0},]{dat/cs_tet_10-3.dat}; 
        \addplot[no marks, dashed, draw=red!70!black] table [x=pO2, y=Csi{0},]{dat/cs_tet_10-3.dat}; 
        \addplot[no marks, dashed, draw=red!50!black] table [x=pO2, y=Csi{-1},]{dat/cs_tet_10-3.dat}; 
        \addplot[no marks, dashed, draw=purple] table [x=pO2, y=Csi{1},]{dat/cs_tet_10-3.dat}; 
        \addplot[no marks, dashed, draw=blue!50!white] table [x=pO2, y=CssubO{1},]{dat/cs_tet_10-3.dat}; %\node at (-11,-2.6) {\ch{I_{O}^{\textperiodcentered}}};
        \addplot[no marks, dashed, draw=green!60!black] table [x=pO2, y=CssubO{2},]{dat/cs_tet_10-3.dat}; 
        \addplot[no marks, dashed, draw=black] table [x=pO2, y=CssubO{3},]{dat/cs_tet_10-3.dat}; 
        \addplot[no marks, dashed, draw=orange!80!black] table [x=pO2, y=CssubZr{-3},]{dat/cs_tet_10-3.dat}; 
        \addplot[no marks, dashed, draw=pink] table [x=pO2, y=CssubZr{-4},]{dat/cs_tet_10-3.dat}; 
        \addplot[no marks, dashed, draw=purple] table [x=pO2, y=CssubZr{-5},]{dat/cs_tet_10-3.dat}; 
%        \addplot[no marks] table [x=pO2, y=Stoich,]{dat/cs_tet_10-3.dat}; 
%\node at (-33.7,-0.5) {\textbf{b)}};
			\end{axis}            
\end{tikzpicture}
		\caption{Tetragonal phase Brouwer diagrams of point defects at caesium concentrations of a) $10^{-5}$ and b) $10^{-3}$, at a temperature of 1500 K. Space charge = 0}
		%\label{figure:tikzbrouwerconctet}
	\end{center}
\end{figure}
\end{landscape}

\section{Summary}

\chapter{Future work}

\label{ch:future}

\section{Iodine empirical potential}
\section{Grain boundary transport}
\section{Zr/ZrO/ZrO2 interface study}


\addcontentsline{toc}{chapter}{References}
\label{References}
\renewcommand\bibname{References}
\bibliographystyle{unsrt}
\bibliography{Mendeley}

\appendix
% Appendices come here
\addcontentsline{toc}{chapter}{Appendix}
\label{Appendix}

\chapter{ParaSweep}

ParaSweep is a generalised sensitivity analysis visualisation tool which was developed during this project. Initially, it was built to help visualise the effects of changing single parameters in Brouwer diagrams, such as temperature or concentration of defects. 


\chapter{CASTEP and HPC Scripts}

Throughout the course of this work, many useful scripts were created to help with preparing CASTEP jobs and analysing their outputs. These scripts have been made available online and for free at \href{https://github.com/v1thesource/CASTEP}{https://github.com/v1thesource/CASTEP}. The purpose of open-sourcing these scripts is to simplify the experience for new users of CASTEP and help them save a considerable amount of time.


\end{document}
\DeclareUnicodeCharacter{2212}{-}

\begin{document}
%\tracingall

\title{\LARGE {\bf Atomistic Simulation of Fission Products in Zirconia Polymorphs}
 %\vspace*{6mm}
}

\author{Alexandros Kenich}
\submitdate{October 2019}

\normallinespacing
\maketitle

\vspace*{140px}
\begin{center}
\textsc{\LARGE Declaration}
\end{center}
I declare that the work presented in this thesis is my own, and that all efforts from others are referenced. 

The copyright of this thesis rests with the author and is made available under a Creative Commons Attribution Non-Commercial No Derivatives licence. Researchers are free to copy, distribute or transmit the thesis on the condition that they attribute it, that they do not use it for commercial purposes and that they do not alter, transform or build upon it. For any reuse or redistribution, researchers must make clear to others the licence terms of this work. 

\begin{center}
\rule{125px}{0.2px}
\end{center}
\vfill
\pagebreak

\preface
\setstretch{1.8}
\addcontentsline{toc}{chapter}{Abstract}

\begin{abstract}
Zirconium alloys are used as a cladding material in over 90\% of nuclear reactors worldwide due to properties which are uniquely suited to the operating environment of a reactor. In this thesis, density functional theory (DFT) simulations were conducted to investigate the behaviour of fission product dopants in the inner cladding oxide, and to examine the role this oxide layer plays in limiting corrosion in the context of pellet-cladding interaction (PCI). 

Simulations in undoped monoclinic, tetragonal and cubic \zirconia\ yielded non-defective structure properties in addition to intrinsic defect energies, volumes and defect equilibria. Fully-charged Schottky defects \{2\ch{V_{O}^{**}}:\ch{V_{Zr}^{''''}}\}$^{\times}$ were shown to have the smallest formation energies in each phase, followed by O Frenkel defects and then Zr Frenkel defects. Defective cubic \zirconia\ simulations are sensitive to finite-size effects, and would often break symmetry or collapse into the tetragonal phase when defect clusters were introduced. Free energy calculations predicted a transition from monoclinic to tetragonal as temperature is increased, but not from tetragonal to cubic. % as would be expected. 

Iodine adopts oxidation states of either +1 (\ch{I_{O}^{***}}, \ch{I_{i}^{*}} and \ch{I_{Zr}^{'''}}) or -1 (\ch{I_{O}^{*}}) when forming defects in \zirconia , with fewer defects in the 0 oxidation state (\ch{I_{O}^{**}}). At high oxygen partial pressures ($p_{O_{2}}$), iodine defects in tetragonal \zirconia\ fall significantly. In monoclinic \zirconia, iodine defects changed by only small amounts as $p_{O_{2}}$ was increased. This demonstrated competition between iodine and oxygen in \zirconia , and that it is dependent on both $p_{O_{2}}$ and phase. High $p_{O_{2}}$ in the tetragonal phase provides the greatest barrier to iodine ingress.

During reactor power ramps, the quantity of fission products implanted in the oxide layer will increase. Decay rates of major Te and I isotopes were found to be commensurate with time to failure in irradiation tests. Defect equilibria and volumes of Te, I, Xe and Cs were obtained in tetragonal \zirconia\ to investigate the effect of nuclear transmutation while dopant atoms are present. Defect evolution on the O site is predicted to be \ch{Te_{O}^{**}} $\rightarrow$ \ch{I_{O}^{*}} $\rightarrow$ \ch{Xe_{O}^{**}} $\rightarrow$ \ch{Cs_{O}^{**}}. On the Zr site, Brouwer diagrams predict \ch{Te_{Zr}^{'''}} $\rightarrow$ \ch{I_{Zr}^{'''}} $\rightarrow$ \ch{Xe_{Zr}^{''''}} $\rightarrow$ \ch{Cs_{Zr}^{'''}}. These defects have large defect volumes and will generate stresses which may promote crack formation.
\end{abstract}

%\begin{itemize}
%\item The third study was about tellurium, iodine, xenon and caesium in the tetragonal phase only.
%\item We propose a new initiation mechanism for PCI failures, whereby iodine diffuses deep into the \zirconia\ layer, past the monoclinic portion but short of the oxide-metal interface. \zirconia\ in this region of the oxide is predominantly tetragonal phase. The iodine nuclei then decay into xenon nuclei, which are larger and have less coherence with the \zirconia\ matrix. These xenon atoms impose a significant strain locally which will open cracks and initiate new ones. At a critical concentration of iodine, this effect bares enough fresh metal surface such that the corrosive effect of iodine outpaces the development of a passivating oxide layer, leading to failure of the clad.
%end{itemize}
 % Good
\cleardoublepage

\addcontentsline{toc}{chapter}{Acknowledgements}

\begin{acknowledgements}

Firstly I would like to thank my supervisors Robin Grimes and Mark Wenman for giving me a chance (despite being a lowly mechanical engineer) and opening the door to pursue nuclear engineering at such a high level. I would also like to thank the EPSRC for funding my studentship through the ICO CDT, the Imperial College HPC team for their quick responses whenever something went wrong, Philipp Frankel and his research group at the University of Manchester for the fruitful discussions and insightful conferences over the years, the Department of Materials administration staff for all their help with my admin woes and the Centre for Nuclear Engineering for being my second home for half a decade.

In no particular order, I want to give a shoutout to the people who have left a strong impression on me and influenced my growth both as a scientist and as a person: Wael Al Jishi, Conor Galvin, Paul Fossati, Claudia Gasparrini, Navaratnarajah Kuganathan, Dhan-Sham Rana, Said El Chamaa, Filippo Vecchiato, Jana Smutna, Vlad Podgurschi, Matt Jackson, Lloyd Jones, Nipun Wickramasundara, Hussam Zaghal, Patrick Burr, William White, Mark Mawdsley, Richard Pearson, Sophie Morrison, Alan Charles, Jonathan Tate, Andy Wilson, John Brokx, Alexandru Paunoiu, Irina Dumitrescu, Julian Sutherland, Anca Semenescu and of course, Emma Warriss.

Finally, I give my everlasting gratitude and love to my parents, my wife Cristina and my daughter Livia-May.

\clearpage

\vfill
\begin{center}
\emph{In memory of Emma Warriss}
\end{center}
\vfill

\end{acknowledgements} % Complete
\setstretch{1}
%\input{dedication/dedication}
%\input{quotes/quotes}

\body

% body of thesis comes here
\doublespacing

\setstretch{2.1}
\chapter{Introduction} \label{introduction}

\section{Nuclear Power} % Complete

In the summer of 1956, the world's first commercial nuclear power plant was connected to the grid in the north of England. This marked a significant departure from previous forms of commercial energy production, which relied on relatively low energy density sources such as the combustion of coal, oil and gas. Before this, the closest anyone had come to utilising nuclear energy commercially was through geothermal power, where the thermal energy input is partly due to radiogenic heat from unstable isotopes in the Earth's mantle \cite{gando2011partial}. 
%which relied on the chemical reactions of coal oil and gas
%Combustion is a chemical process, where energy differences between reactants and products are exploited via electron exchange. Nuclear energy however, exploits the energy difference between nuclei. Both rely on the conversion of mass into energy, however, the amount of energy that can be extracted varies by several orders of magnitude

Combustion is a chemical process whereby energy differences between reactants and products are exploited via electron exchange. Nuclear energy exploits the energy difference between nuclei. Both rely on the conversion of mass into energy, however, the amount of energy that can be extracted from the nucleus is several orders of magnitude greater.
%Combustion is a chemical process, and its use in commercial energy production is fundamentally about exploiting the free energy difference when electrons are exchanged between some reactants to produce some products. Nuclear energy, however, is about the direct conversion of mass into energy. The difference between the two is staggering.

Consider methane, with an enthalpy of combustion of −887.2 kJ/mol \cite{thornton1917xv}. This is the equivalent of 9.14 eV per particle. By comparison, the total energy release from fission of one uranium-235 nucleus is at least $1.65 \times 10^{8}$ eV, as shown in Figure \ref{figure:fissionenergy}.

Combustion-based power as a technology has matured over hundreds of years, with modern optimisations only looking to offer fractional percent gains in efficiency. By comparison, nuclear power technology is far from mature, with large improvements yet to be realised. One such feature is load-following, an enormously useful feature for a power plant which is currently underutilised in nuclear reactors. Load-following, as currently practiced in some French and German nuclear power plants, is defined as operation where power output follows a variable load programme on a daily basis with several power changes (i.e. to follow the change in electricity demand over a 24 hour period). These power variations can be as large as 50\% of a reactor's rated power \cite{lokhov2011technical}. The biggest obstacle to load-following in nuclear reactors is the issue of pellet-cladding interaction (PCI), which is the basis of the work in this thesis.

\begin{figure}[ht]
\centering
\includegraphics[height=13cm]{images/fission_energy_total.png}
\caption[Energy from thermal fission of U$^{235}$ as a function of mass ratios of daughter nuclei. Total energy release includes contributions from gamma rays and subsequent radioactive decays.]{Energy from thermal fission of U$^{235}$ as a function of mass ratios of daughter nuclei. Total energy release includes contributions from gamma rays and subsequent radioactive decays. Taken from \cite{aras1965ranges}.}
\label{figure:fissionenergy}
\end{figure}

\subsection{Fission}

Commercial nuclear power plants extract energy through the process of fission, where a large nucleus is split into smaller nuclei. While it is also possible to extract energy from certain small nuclei by the process of fusing them into larger ones, no fusion reactor currently exists which achieves a net positive energy output. At a fundamental level, both fission and fusion rely upon mass-energy equivalence. The relationship between mass and energy is shown using Einstein's equation:
\begin{equation}
\label{emc2}
    E = mc^{2}
\end{equation}
where $E$ is the energy of the system, $m$ is the mass and $c$ is the speed of light in a vacuum. Using this equation we can analyse a typical fission reaction:
\begin{equation}
    \ch{U^{235}_{92}} + n^{1}_{0} \xrightarrow[]{absorption} \ch{U^{236}_{92}} \xrightarrow[]{fission} \ch{I^{132}_{53}} + \ch{Y^{101}_{39}} + 3n^{1}_{0}
\label{eqn:fission} 
\end{equation}
While the number of protons and neutrons are conserved throughout the reaction, a mass difference calculation will show that there is actually less mass in the products than the reactants by approximately 0.188 amu (3.127$\times 10^{-28}$ kg). This missing mass, known as the \emph{mass defect}, is converted to energy ($\sim$175 MeV). In this way, the total mass-energy of the system is conserved. Some of this energy is carried away as kinetic energy of the fission products (in Equation \ref{eqn:fission}, I and Y) and also the kinetic energy of the neutrons. The neutrons at this stage have energies \goodtilde{1} MeV and are known as \emph{fast} neutrons.

This change in mass arises due to the phenomenon of \emph{binding energy}. In order for two or more nucleons to be thermodynamically stable when bound together, the total free energy of the bound configuration must be less than the sum of constituent nucleon free energies. Much as with energy stored in a chemical bond, the binding energy represents the energy required to separate the nucleus into individual protons and neutrons. 

Larger nuclei will generally have a greater total binding energy value compared to smaller nuclei, but the mass defect per nucleon will not necessarily be the same in a larger nucleus. It is therefore useful to normalise the binding energy by the mass number. Different isotopes have different binding energies, and any nuclear reaction that increases the binding energy per nucleon will be exothermic, whether by fission or fusion. Figure \ref{figure:bindingenergy} shows a plot of binding energy per nucleon against mass number with the relevant isotopes from Equation \ref{eqn:fission}. 

%235.0439299 + 1.008664 (236.0525939) -> 131.907997 + 100.93031 + 3.025992 (235.864299)
\begin{figure}[ht]
\centering
\includegraphics[width=14cm]{images/Binding_energy_curve.png}
\caption[Plot of binding energy per nucleon against mass number. Arrows indicate the reaction shown in equation \ref{eqn:fission}.]{Plot of binding energy per nucleon against mass number. Arrows indicate the reaction shown in equation \ref{eqn:fission}. Adapted from \cite{Fastfission}.}
\label{figure:bindingenergy}
\end{figure}

\subsection{Reactor design} % Complete

Commercial nuclear reactors are large boilers in a Rankine cycle, designed to maximise heat transfer to a working fluid. All nuclear plants use steam turbines on the generation side, though the reactor coolant may be another fluid in a separate loop, such as carbon dioxide in gas-cooled reactors (GCRs), or even in a separate water loop such as in pressurised water reactors (PWRs). 

The most prevalent reactor type is the PWR, followed by the boiling water reactor (BWR). A schematic of a PWR power plant is shown in Figure \ref{figure:pwrschematic}. This design incorporates a primary coolant loop and heat exchanger to a secondary loop at a lower pressure. Steam is generated on the low pressure side of the heat exchanger which then drives a steam turbine. There are several other reactor types used around the world (enumerated in Table \ref{figure:world_reactors}). The work in this thesis is focused on zirconium-based claddings which are used worldwide in all commercial reactors except GCRs and sodium-cooled fast reactors. In total, zirconium fuel cladding is used in over 95\% of all nuclear reactor fuel pins, and so performance improvements in these cladding materials have an effect across the entire industry.

\begin{figure}[ht] % Schematic of a PWR
\centering
\includegraphics[width=\linewidth]{images/pwrschematic.png}
\caption[Schematic illustration of a PWR power plant.]{Schematic illustration of a PWR power plant. Taken from \cite{lokhov2011technical}.}
\label{figure:pwrschematic}
\end{figure}

The fission of uranium takes place inside a steel reactor pressure vessel (RPV) in PWRs and BWRs, which holds the fuel pins, control rods and other reactor internals. The working fluid in a nuclear reactor is typically under high pressure, with PWR RPV operating pressures between 150 and 160 bar, while BWRs operate at lower pressures of around 70 bar \cite{kok2016nuclear, Server2010, Durmayaz2001}. The pressure of the coolant acts on the fuel cladding, generating radial and hoop stresses which influence crack formation. 

The operating temperature of the coolant in a typical PWR is approximately 600 K. This is a low temperature relative to the melting points of Zr metal (2128 K) and ZrO$_{2}$ (2988 K). This temperature together with the high pressure is chosen in order to keep the coolant in the liquid phase for safety reasons, though this limits the thermodynamic efficiency of the plant. The highest temperature in a reactor will occur in the nuclear fuel, with PWR fuel pellets reaching centreline temperatures of up to 1673 K \cite{beyer1998review}.

\begin{table}[ht] % Reactors in the world
\centering
\caption[Type and number of different reactors operational worldwide at the end of 2017. Change from 2016 shown in parentheses.]{Type and number of different reactors operational worldwide at the end of 2017. Change from 2016 shown in parentheses. Taken from \cite{WNAreport2018}.}
\includegraphics[width=15cm]{images/WNA_report2018.png}
\label{figure:world_reactors}
\end{table}

For both PWRs and BWRs, the coolant is typically light water (as opposed to heavy water, D$_{2}$O). Water is used because it has many useful engineering properties. It has a high heat capacity (compared to the gaseous coolant in gas cooled reactors), has low activation in a free neutron environment and also serves as a good radiation shield. Furthermore, it is plentiful, cheap and easily purified.

In addition to its function as a coolant, water also acts as a neutron moderator, slowing down high-energy fast neutrons from fission events and nuclear decay processes. Moderation of neutrons is an important step in the nuclear reactor because slow thermal neutrons (i.e. neutrons at thermal equilibrium with the coolant) are significantly more likely to cause uranium nuclei to undergo fission than fast neutrons. Fast neutrons will typically escape from the fuel pin after they are generated, dispense most of their energy in the coolant via scattering with H and O nuclei, and then some will re-enter the fuel, where they may cause fission of a U$^{235}$ nucleus near the outer edge of the fuel pellet. Neutrons which do not end up fissioning fuel will either be absorbed parasitically by other nuclei (e.g. in the control rods or coolant), escape from the reactor entirely (neutron leakage), or decay into protons (free neutrons have a half-life of 10.61 minutes \cite{Christensen1972}).

\subsection{Fuel pellets and cladding} \label{ss_fuelpin}

Fuel assemblies in nuclear reactors are bundles of fuel pins (see Figure \ref{figure:fuelassembly}). In most commercial reactors, fuel pins are comprised of a zirconium-based cladding (tubes), which are filled with cylindrical UO$_{2}$ fuel pellets, each of which are approximately 1 cm$^{3}$ in volume (see Figure \ref{figure:fuelpellet}). Fuel pellets in PWRs and BWRs have dishes on the top and bottom faces of the cylinder as well as chamfered edges. The main function of the dishes is  to reduce the axial pellet-pellet stresses caused due to swelling of the pellet when irradiated \cite{marino2005crack}. Chamfers aid in the loading of fuel pellets into the cladding, as well as reducing the risk of chipping at the edges of the fuel pellet. This is important because chipping of the fuel pellet can lead to debris falling into the pellet-cladding gap where it can act as a stress raiser \cite{doerr2015nuclear}.

\begin{figure}[ht]
\centering
\includegraphics[width=11cm]{images/fuelassembly.png}
\caption[Schematic view of a PWR fuel assembly and a PWR fuel pin.]{Schematic view of a PWR fuel assembly and a PWR fuel pin. Adapted from \cite{Croff2003}.}
\label{figure:fuelassembly}
\end{figure} 

Once loaded with fuel pellets, the fuel pins are capped and filled with inert helium gas, pressurised to between 2 and 25 atm to improve heat transfer from the fuel pellets to the coolant as well as delaying inward creep deformation of the cladding due to the high coolant pressure \cite{King1980}. 

LWR fuel pellets are manufactured in a multi-stage process starting from enriched UF$_{6}$. The UF$_{6}$ must be converted into UO$_{2}$, which can be done using either a `dry' or `wet' process, referring to the use of liquid water in the process. The dry process, called the integrated dry route (IDR), is simpler and is described as follows:

\begin{itemize}
\item Enriched UF$_{6}$ (a solid at room temperature and pressure) is heated into vapour form using an autoclave.
\item UF$_{6}$ vapour is mixed with steam and fed into a rotary kiln.
\item Hydrogen gas is added to the mixture and the UF$_{6}$ is reduced to solid UO$_{2}$. The gaseous HF is recovered, leaving pure UO$_{2}$ crystals.
\end{itemize}

The UO$_{2}$ from this stage is then blended to homogenise the particle sizes and achieve a desired particle surface area. At this stage, additives may be introduced to the UO$_{2}$ (e.g. burnable poisons, lubricants, dopants to improve densification or to control microstructure). This powder is then fed into a pellet pressing die where it is pressed into a cylindrical shape, called a `green' pellet. Green pellets are then sintered in a furnace at temperatures of up to 2000 K in order to consolidate and increase the density of the pellets \cite{pramanik2010innovative}. These fired pellets are then machined to the appropriate dimensions, including chamfers and dishes, before final inspection and loading into a cladding tube.

\begin{figure}[ht]
\centering
\includegraphics[width=10cm]{images/fuelpellet.png}
\caption[UO$_{2}$ LWR fuel pellet showing dishes and chamfers.]{UO$_{2}$ LWR fuel pellet showing dishes and chamfers. Adapted from \cite{tulenko2013development}.}
\label{figure:fuelpellet}
\end{figure}

In the early stages of a fuel pin's life, there is a small gap between the fuel pellet and the cladding, known as the pellet-cladding gas gap. This gas gap slowly closes with increasing fuel burn-up due to swelling of the fuel pellets and inward creep deformation of the cladding due to the coolant pressure. The pellet-cladding system is shown using a schematic view of the cross section of a PWR fuel pin in Figure \ref{figure:gas_gap}. The cladding internal oxide layer covers the entire internal surface of the cladding and is the first barrier to corrosive species. 

\begin{figure}[ht]
\centering
\includegraphics[width=10cm]{images/gas_gap.png}
\caption{Schematic cross-section of a single PWR fuel pin with an expanded view of the pellet-gap-cladding system.}
\label{figure:gas_gap}
\end{figure}

\subsection{Effects of radiation on materials} 

While ionising radiation is always present in the environment as background radiation, the intensity of radiation in a nuclear reactor is so great that it causes significant engineering challenges because of how it affects change in reactor materials.

Radiation hardening (also known as radiation embrittlement) is a phenomenon which affects most materials subjected to ionising radiation. It is characterised by a loss of plasticity caused by radiation damage over time, leading to an increased risk of cracks and failure of components. While zirconium is a very useful nuclear material due to its neutron transparency, it is still susceptible to radiation damage \cite{Wisner1998}. Beyond certain levels of radiation damage, phase changes may also occur. %In \zirconia\ however,  

Amorphisation is another effect of radiation damage, which has been observed in the (Zr, U)O$_{2}$ bonding layer in fuel pins \cite{Nogita1997}. This is characterised by an overall loss of long-range order of atoms in a crystal. This typically occurs beyond a certain threshold of radiation damage depending on the material, called the critical amorphisation dose. Amorphisation causes a loss in long range crystallographic structure and a corresponding reduction in structural stability (amorphous materials have a higher Gibbs free energy than their crystalline counterparts), and causes swelling of the material \cite{Einfal2013}. In the literature, there is evidence of amorphisation in cubic stabilised \zirconia\ when bombarded with Cs$^{+}$ ions up to a fluence of $1 \times 10^{21}$ ions m$^{-2}$ \cite{amorphization2000wang}. However, no amorphisation is seen at an Xe$^{2+}$ fluence of $2 \times 10^{21}$ ions m$^{-2}$, or an I$^{+}$ fluence of $5 \times 10^{19}$ ions m$^{-2}$ \cite{sickafus1999radiation}.

One material phenomenon exclusive to nuclear reactor environments is neutron activation. The high free neutron environment leads to neutron capture in various nuclei within the reactor, including those of the fuel assemblies, coolant and RPV. There are many possible (n, x) reactions that may occur in materials experiencing a neutron flux, but of particular concern is transmutation of nuclei following a nuclear capture event. When a stable nucleus captures a neutron and becomes unstable, the nucleus may then emit particles to reduce its free energy, altering its atomic number in the process. This new element will have different chemical properties compared to the parent nucleus by virtue of a different electronic structure. This will change the elemental composition of a material, typically in an unfavourable way with dopants that negatively affect some desired material property. The extremely large number of nuclei relative to neutron flux means that this effect is small, though over time this becomes more significant due to the accumulation of these dopant elements.

In high-radiation environments, it is also possible for some molecules to be split by gamma photons above certain energies through the process of radiolysis. Corrosive fission products such as iodine will be present inside the fuel pin, but may exist in the form of (for example) CsI, which is not highly corrosive. Radiolysis however, decomposes CsI into Cs and I$_{2}$ vapour which can diffuse towards the cladding and promote cracking \cite{Konashi1983}.

\subsection{Fission products, their distribution and decay chains} \label{fissionyieldsection}

Nuclei which can undergo fission will produce daughter nuclei (fission products) with specific characteristics. At first, these nuclei will almost always be neutron-rich, as compared to their stable isotopes. This is the result of the higher neutron to proton (N/Z) ratios of larger nuclei. Figure \ref{figure:NZcurve} shows how the nuclei of abundant isotopes start with N/Z ratios of around 1 for small elements (e.g. \ch{He_{2}^{4}}, \ch{C_{6}^{12}}, \ch{O_{8}^{16}}), whereas larger elements will have isotopes with N/Z ratios approaching 1.6 (e.g. \ch{Pb_{82}^{208}}, \ch{Th_{90}^{232}}, \ch{U_{92}^{238}}).

\begin{figure}[ht]
\centering
\includegraphics[height=13cm]{images/Isotopes_and_half-life.png}
\caption[Plot of neutron number against proton number for nuclei with half-lives greater than 10${^{-8}}$ s.]{Plot of neutron number against proton number for nuclei with half-lives greater than 10${^{-8}}$ s. Taken from \cite{BenRG}.}
\label{figure:NZcurve}
\end{figure} 

Fissionable nuclei are typically very large and when fission occurs, the daughter nuclei will inherit a high N/Z ratio. These neutron-rich nuclei will generally decay by $\beta-$ particle emission to reduce their N/Z ratio and increase stability. One decay event is usually not enough to achieve a stable nucleus, so several decays as part of a decay chain are expected. An example is given for Te in Equations \ref{eqn:te_decay} and \ref{eqn:i_decay}.
\begin{gather}
\ch{Te^{134}_{52}} \xrightarrow[]{\beta-} \ch{I^{134}_{53}}+ e^{-} + \overline{\nu}_{e}
\label{eqn:te_decay} \\
\ch{I^{134}_{53}} \xrightarrow[]{\beta-} \ch{Xe^{134}_{54}} + e^{-} + \overline{\nu}_{e}
\label{eqn:i_decay}
\end{gather}
Another characteristic feature of fission products is that their masses are bi-modally distributed. Figure \ref{figure:fissionyield} shows calculated fission yields as a function of mass number, indicating a 40:60 rather than 50:50 mass distribution among daughter nuclei when heavy nuclei are fissioned. This is attributed to smaller nuclei (which have high binding energies per nucleon) separating from the nucleus first at the moment when fission occurs. This results in the majority of fission products centring around atomic masses of 95 and 135, producing a disproportionate number of nuclei such as Sr, Y and Zr on the low side and Te, I and Cs on the high side of the distribution. The distribution of fission products varies slightly depending on which nucleus is fissioned (e.g. U$^{233}$, U$^{235}$, Pu$^{239}$) and the energy of the fissioning neutron. Over time, transmutation of U$^{238}$ to Pu$^{239}$ and consumption of U$^{235}$ means that an increasing proportion of energy output will be due to fission of Pu$^{239}$, but fuel assemblies are typically retired before significant build-up of Pu$^{239}$ inventory and so the net yields of fission products will only change slightly over the life of the fuel.
% Iodine 0.1142684237, Tellurium 0.180582

\begin{figure}[ht!]
\centering
\includegraphics[height=12cm]{images/fissionyield.jpg}
\caption[Plot of the percentage yield of nuclei with a given mass following a fission event. Range of masses corresponding to isotopes of iodine shown in purple and isotopes of yttrium shown in green, based on Equation \ref{eqn:fission}.]{Plot of the percentage yield of nuclei with a given mass following a fission event. Range of masses corresponding to isotopes of iodine shown in purple and isotopes of yttrium shown in green, based on Equation \ref{eqn:fission}. Adapted from \cite{England1992}.}
% M. F. James, R. W. Mills and D. R. Weaver (1991) UKAEA Reports, AEA-TRS-1015, AEA-TRS-1018 
         %  and AEA-TRS-1019.
\label{figure:fissionyield}
\end{figure}

Iodine is an important fission product because it is known to corrode Zr metal. It is part of the Te $\rightarrow$ Cs decay chain, with most isotopes exhibiting half-lives ranging from a few seconds to several days. Select data on Te and I fission yields are presented in Table \ref{table:decaydata_chap1}. In total, the independent yield of I isotopes from U$^{235}$ fission is 10.4\%, while the independent yield of Te isotopes (I precursors) is 17.6\%, with \ch{Te_{52}^{134}} having the highest independent yield of all possible fission products (6.3\%). These particular elements will also usually be paired with Zr and Y fission products, the latter of which is a common phase stabiliser dopant in \zirconia .

\begin{table}[ht!] % Yields and half lives
\onehalfspacing
\caption{Independent fission product yields and half-lives for the major iodine isotopes and precursors in a thermal neutron reactor. Yields taken from the Joint Evaluated Fission and Fusion File (JEFF 3.3). All isotopes undergo single $\beta$- decay. Metastable states are included.}  \label{table:decaydata_chap1}
\begin{center}
\begin{tabular}{c c c}
\hline
Isotope & Independent Yield (\%) & Half-life \\
\hline
\ch{Te^{131}} & 0.126
 & 25.0 m \cite{auble1976nuclear} \\
\ch{I^{131}} & 0.00296
 & 8.02 d \cite{I131halflife} \\
\ch{Te^{132}} & 1.50 & 3.18 d \cite{Te132} \\
\ch{I^{132}} & 0.0284  & 2.30 h \cite{Te132} \\
\ch{Te^{133}} & 3.94 & 12.5 m \cite{khazov2011nuclear} \\
\ch{I^{133}} & 0.198  & 20.9 h \cite{I133} \\
\ch{Te^{134}} & 6.30 & 41.8 m \cite{Sonzogni2004} \\
\ch{I^{134}} & 0.767 & 52.5 m \cite{Sonzogni2004} \\
\ch{Te^{135}} & 3.48 & 19.0 s \cite{tellurium135halflife} \\
\ch{I^{135}} & 2.46 & 6.58 h \cite{tellurium135halflife} \\ 
\ch{Te^{136}} & 1.67 & 17.6  s \cite{Mccutchan2018} \\
\ch{I^{136}} & 3.04 & 83.4 s \cite{Mccutchan2018} \\
\ch{Te^{137}} & 0.484 & 2.49 s \cite{browne2007nuclear} \\
\ch{I^{137}} & 2.70 & 24.5 s \cite{browne2007nuclear} \\ 
\ch{Te^{138}} & 0.113 & 1.40 s \cite{chen2017nuclear} \\ 
\ch{I^{138}} & 1.24 & 6.26 s \cite{chen2017nuclear} \\ 
\hline
\end{tabular}
\end{center}
\end{table}

\subsection{Formation of plutonium and minor actinides}

Uranium fuel in LWRs is enriched to contain up to 5\% \ch{U^{235}}, which is a fissile isotope. The rest of the uranium is comprised of the more abundant \ch{U^{238}} which is a \emph{fertile} isotope. Fertile isotopes are not\footnote{\ch{U^{238}} has a thermal neutron fission cross section which is 7 orders of magnitude smaller than that of \ch{U^{235}}, so thermal fission of \ch{U^{238}} is insignificant.} fissioned directly, but can be converted into fissile isotopes through nuclear reactions. In the case of \ch{U^{238}}, the fissile isotope \ch{Pu^{239}} is produced through the following reactions:
\begin{gather}
\ch{U^{238}_{92}} + n^{1}_{0} \xrightarrow[]{absorption} \ch{U^{239}_{92}} \\
\ch{U^{239}_{92}} \xrightarrow[]{\beta-} \ch{Np^{239}_{93}} + e^{-} + \overline{\nu}_{e} \\
\ch{Np^{239}_{93}} \xrightarrow[]{\beta-} \ch{Pu^{239}_{94}} + e^{-} + \overline{\nu}_{e}
\end{gather}
UO$_{2}$ fuel pellets do not contain any plutonium before irradiation, however, production of \ch{Pu^{239}} and heavier isotopes during reactor operation is so significant that they can account for over 50\% of the fissile isotope inventory in a LWR \cite{OECD1989}. A fraction of the \ch{Pu^{239}} nuclei will capture additional neutrons, producing heavier isotopes of plutonium which may be fissile (e.g. \ch{Pu^{241}}). The production of fissile nuclei from fertile nuclei is known as \emph{breeding}.

Plutonium nuclei which are not burned through fission can either be extracted from spent fuel by reprocessing, or will undergo nuclear decay (specifically $\alpha$ or $\beta$- decay) and produce elements such as neptunium, americium and curium which are known as \emph{minor actinides}\footnote{In the nuclear power industry, uranium and plutonium are known as the major actinides.}. These minor actinides are produced from neutron irradiation of uranium (both \ch{U^{238}} and \ch{U^{235}}) and nuclear decay. The radiotoxicity of spent nuclear fuel is dominated by plutonium and the minor actinides, with global production rates of isotopes like \ch{Np^{237}}, \ch{Am^{241}} and \ch{Cm^{244}} measured in several metric tons per year \cite{Ewing2004}. 

In a nuclear fuel pin, the cladding serves as a physical barrier preventing the release of fission products and actinide wastes. The first barrier is the UO$_{2}$ matrix itself, however, volatile species can migrate to the pellet-cladding gap. This gap gradually closes during operation and the fuel pellet makes contact with the cladding, leading to mechanical and chemical interactions. This is discussed in detail in Chapter \ref{literature_review}. % Complete
\chapter{Literature Review}

\label{literature_review}

\section{Pellet-cladding interaction (PCI)}

When a fuel pin is first inpile, there is a gap between the fuel pellet and the cladding (see § \ref{ss_fuelpin}). This gas gap slowly closes over time, mostly due to thermal expansion and swelling of the fuel pellet (illustrated in Figure \ref{figure:pcmi}) due to radiation damage and the accumulation of fission products. At a high enough burnup, the fuel pellet finally makes contact with the cladding. This phenomenon is called pellet-cladding interaction (PCI) and involves both mechanical and chemical interactions which contribute to observed fuel failures.

\begin{figure}[ht] % PCMI bambooing
\centering
\includegraphics[width=13cm]{images/pcmi.png}
\caption[Illustration of fuel swelling and clad deformation due to PCI. \textbf{a)} Fresh fuel before irradiation. \textbf{b)} Thermal expansion and swelling of fuel pellets and closing of fuel-pellet gap during operation. \textbf{c)} Fuel contact with cladding and subsequent `bambooing' of the cladding.]{Illustration of fuel swelling and clad deformation due to PCI. \textbf{a)} Fresh fuel before irradiation. \textbf{b)} Thermal expansion and swelling of fuel pellets and closing of fuel-pellet gap during operation. \textbf{c)} Fuel contact with cladding and subsequent `bambooing' of the cladding. Adapted from \cite{alam2011review}.}
\label{figure:pcmi}
\end{figure}

PCI-related failure of nuclear fuel pins has been known about since 1963, when the first reported failure occurred in a highly rated fuel pin during reactor start-up \cite{lyons1963high}. Many studies have since been made regarding the topic \cite{alam2011review, bcoxpelletclad1990}. The exact cause of PCI failures has not yet been determined, however it is likely that both the mechanical and chemical effects of contact between the fuel pellet and the cladding are necessary for it to occur. It is also known that PCI failures are typically preceded by power transients, such as during reactor startup where several power ramps are performed over many hours.

\subsection{Fuel pellet relocation}

Irradiated fuel pellets will sometimes crack and break into fragments while in the cladding. These fragments are unconstrained and are able to move radially outwards (i.e. towards the pellet-cladding gap). This phenomenon is called `relocation', although this term is also used to refer to movement of the pellet fragments in the axial direction within the cladding \cite{sheppard1982data}. Relocation of fuel fragments means that firm contact between the fuel and cladding can occur earlier in the fuel pin's life (i.e. before fuel swelling closes the fuel-cladding gap).

The conceptual model of fuel pellet relocation is shown in Figure \ref{figure:relocation} as a plot of the `effective' gap (based on cladding temperature and fuel centreline temperature) against rod power. Initially, the fuel pellet has no fragmentation and a wide fuel-cladding gap (region I). The gap decreases slowly due to thermal expansion as rod power is increased until point A where the fuel pellet cracks and fragments. The fuel pellet fragments move radially outward and the gap decreases significantly (region II). Point B marks the onset of `soft' PCI, meaning that the fuel pellet fragments make some contact with the cladding, but are not yet fully constrained (e.g. radial or azimuthal motion is still possible). The fragments' lack of mechanical constraints prevent a stress from being imparted to the cladding, thus the mechanical interaction is considered `soft' (region III). At point C, thermal expansion due to rod power is large enough that some pellet fragments become mechanically constrained, marking the onset of `hard' PCI (region IV). As discussed in § \ref{ss_fuelpin}, chipping of fuel pellets can cause debris to occupy the pellet-cladding gap, thereby reducing the rod power at which onset of hard PCI occurs.

\begin{figure}[ht] % Relocation
\centering
\includegraphics[width=15cm]{images/relocation.png}
\caption[Concept of pellet-cladding gap model showing stages of fuel pellet fragment relocation with onset of soft and hard PCI. ]{Concept of pellet-cladding gap model showing stages of fuel pellet fragment relocation with onset of soft and hard PCI. Adapted from \cite{Oguma1983} with fuel pin cross-sections from \cite{walton1983fuel}.}
\label{figure:relocation}
\end{figure}

\subsection{Pellet-clad gap and bonding}

The gap between the fuel and the cladding allows fuel pellets to be inserted into the fuel pin easily during manufacture, but this clearance is also designed to accommodate some increase in fuel pellet volume. It is important to consider the thermal expansion of the fuel pellet, as the centreline temperature of a PWR fuel pellet during a power transient can vary between 1500 and 1800 $^{\circ}$C, depending on the burnup of the fuel and magnitude of the reactivity insertion \cite{Bagger1994}. In addition, fuel pellets will swell due to radiation damage throughout their operational lifetime. Once the pellet-clad gap has closed entirely, any pellet expansion during a power transient will translate to a force exerted on the cladding, generating hoop stresses which open cracks on the inner cladding surface. This is the mechanical component of PCI.

When the fuel pellet makes contact with the cladding, there is also a chemical interaction between the UO$_{2}$ (and fission products) and the internal surface oxide of the cladding. UO$_{2}$ has some solid solubility in ZrO$_{2}$, and we therefore see a bonded reaction layer, which has a chemical composition (U, Zr)O$_{2}$.  Due to the large U atom, cation substitution allows this mixed layer to adopt the crystal structure of cubic fluorite, the high temperature phase of ZrO$_{2}$. The uranium nuclei in both this bonding layer and the outer rim of the fuel pellet experience the highest number of fission events (due to the proximity to the moderator), and therefore this region contains fission products that contribute to the chemical degradation of the fuel cladding, in particular iodine. This is the chemical component of PCI.

\subsection{Reactor power ramps}

It has been established that power ramping of the reactor is associated with PCI failures \cite{penn1977candu, MacDonald1979, Hardy1977198, Knaab1987}. This presents a problem for reactors when it comes to events such as start-up, load-following and any other power transients experienced by the fuel pins. Figure \ref{figure:reactor_startup} shows how reactor power varies over time during a typical PWR start-up procedure. A combination of low ramping rates and long holds at low power (to remain below ramping limits, to condition fuel \cite{billaux2005pellet}\footnote{Fuel is considered `conditioned' after it has operated at a specified power level for a certain period of time.}, and conduct coolant chemistry checks) require the entire procedure to be completed in a period of 90 hours, with several operator switch-overs in between. This is a costly procedure for the utility owner to perform, with millions of US\$ foregone in electricity sales for larger reactors.

\begin{figure}[ht]
\centering
\includegraphics[height=10cm]{images/reactor_startup.png}
\caption[Reactor startup procedure for a typical US PWR. Dashed line indicates \% withdrawal of control rods.]{Reactor startup procedure for a typical US PWR. Dashed line indicates \% withdrawal of control rods. Adapted from \cite{ramping}.}
% M. F. James, R. W. Mills and D. R. Weaver (1991) UKAEA Reports, AEA-TRS-1015, AEA-TRS-1018 
         %  and AEA-TRS-1019.
\label{figure:reactor_startup}
\end{figure}

Scheduled reactor shutdowns or extended reduced power operation (ERPO) events occur whenever refuelling or maintenance of the reactor is required (and in some cases there may be a grid demand to reduce power output).  Refuelling is typically performed every 1 to 2 years, while maintenance may be required at any time. Emergency shutdowns may also occur and have their own challenges to consider (e.g. xenon poisoning, decay heat removal), though they are more rare. In each case, going through the lengthy power ramping procedure is required, and since these shutdowns cannot be avoided, being able to ramp up power faster would be a significant improvement. Ramp rates in reactors are restricted to between 3-4\% of full power per hour above a certain threshold level to avoid PCI failures \cite{ramping}. Additionally, fuel conditioning holds (operation at certain power levels for long periods) are performed to further reduce the incidence of PCI failures. 

These limits present a challenge not just when restarting reactors, but also for the implementation of load-following in reactors. PWRs have thermal feedback loops which provide some level of intrinsic load-following behaviour. For example, an increase in steam demand leads to increased boiling in the steam generators. The subsequent decrease in coolant temperature in the primary circuit leaving the steam generator causes a reactivity increase and therefore a power increase in the reactor. The reactor returns to critical (reactivity = 0) after some fluctuation, and the average temperature of the coolant remains unchanged.

\subsection{PCI Failures}

PCI failures are typically associated with cracks which span the thickness of the cladding (so-called \emph{pinhole} defects), subsequently causing fission product contamination of the primary loop coolant. A fuel cladding breach is detected when a sharp spike in radioactivity is registered by sensors in the primary loop (i.e. a signal that is distinguishable above the previous background). This means that the time to detection of a failure from the failure-inducing event (e.g. a power ramp) is known, but that the time taken for the PCI failure to occur is unknown \cite{bcoxpelletclad1990}. 

Figure \ref{figure:fuelcladdingcrack} shows the typical cracking behaviour of fuel pellets and cladding from a PWR fuel pin subjected to two annual operating cycles. % TODO

\begin{figure}[ht]
\centering
\includegraphics[width=\linewidth]{images/fuelcladdingcrack.png}
\caption[Cracking of PWR fuel pellets and cladding. \textbf{a)} Transversal macrography of a fuel rod irradiated for two annual PWR operating cycles. \textbf{b)} Axial macrography of a fuel rod irradiated for two annual PWR operating cycles. \textbf{c)} SCC cladding failure.]{Cracking of PWR fuel pellets and cladding. \textbf{a)} Transversal macrography of a fuel rod irradiated for two annual PWR operating cycles. \textbf{b)} Axial macrography of a fuel rod irradiated for two annual PWR operating cycles. \textbf{c)} SCC cladding failure. Adapted from \cite{brochard2001modelling}.}
\label{figure:fuelcladdingcrack}
\end{figure}

\subsection{Fission products and SCC}

Although PCI failures were found to occur during power ramping, it was not yet known whether these failures were due to fission product induced SCC or tensile failure of the cladding due to radiation embrittlement. 

In 1971, scientists at the Chalk River Nuclear Laboratories conducted a series of tests to determine if fission products were necessary \cite{MacDonald1979}. The experiments involved taking highly irradiated zirconium fuel pins (fluence of $8 \times 10^{24}$ n/m$^{2}$ with 1 MeV neutrons) and then inserting fresh, unirradiated UO$_{2}$ fuel pellets into them. These fuel pins were then inserted back into a reactor and subjected to large power ramps, as shown in Figure \ref{figure:fueltests}. These ramps (phase I and II) would typically cause failures in fuel pins with similar irradiation histories. In the initial ramp tests however, all the fuel pins survived the ramps intact. Six fuel pins were then irradiated in the reactor at low power to a burnup in excess of 50 MWh/kg U in order to build up fission products in the fuel (phase III). In a subsequent ramp test (phase IV), two of the high burnup fuel pins failed in the reactor. This finding provided the strongest evidence to date that fission products are necessary for PCI failure of zirconium-based claddings. The fission product most likely to cause cracking, based on known SCC susceptibility of Zr metal, is iodine.

\begin{landscape}
\begin{figure}[ht]
\centering
\includegraphics[width=\linewidth]{images/fueltests.png}
\caption[Power ramping test data for the different phases of the Chalk River Nuclear Lab experiment.]{Power ramping test data for the different phases of the Chalk River Nuclear Lab experiment. Taken from \cite{MacDonald1979}.}
\label{figure:fueltests}
\end{figure}
\end{landscape}

As discussed earlier, iodine is one of the most prevalent fission products and it is known to corrode zirconium metal \cite{iodinezrmetal, Sidky1998, rosenbaum1966interaction, Fregonese1998, Lewis2011, anghel2010experimental}. The exact mechanism by which this occurs in fuel pins is not yet known, though a combination of radiolysis, I$_{2}$ diffusion and chemical attack (I-SCC) are considered to be most likely. The most commonly proposed mechanism is illustrated in Figure \ref{figure:vanarkel}. In the first step, iodine and caesium are produced through fission of the fuel and diffuse towards the outer surface of the fuel pellet (though not necessarily together \cite{Grimes1992}). A thin film of CsI is deposited on the outer surface of the fuel pellet and subsequently decomposes via radiolysis, liberating iodine in vapour form. The iodine vapour then diffuses towards a crack site in the cladding and reacts with Zr to produce ZrI$_{4}$. The ZrI$_{4}$ then breaks away from the metal due to the high surface energies, causing pitting and progressing the crack tip further into the metal. This model however, fails to consider the effect of the oxide on the internal layer of the cladding (as a barrier layer to ingress of corrosive species), and it does not take into account the presence of oxygen in the pellet-cladding gap and in the metal matrix (in solution) near the metal-oxide interface. Both these factors are important as the oxide provides a protective effect (otherwise I-SCC failures would be far more prevalent, regardless of ramp rate limitations and conditioning holds) and where the oxide has been ruptured, repassivation and competition between iodine and oxygen will occur.

\begin{figure}[ht]
\centering
\includegraphics[width=\linewidth]{images/vanarkel.png}
\caption[Schematic of a proposed I-SCC process for cladding crack penetration.]{Schematic of a proposed I-SCC process for cladding crack penetration. Taken from \cite{Lewis2011}.}
\label{figure:vanarkel}
\end{figure}


\subsection{Iodine availability}

The amount of free iodine available in the fuel pin is dependent on many different factors (e.g. temperature, pressure, power history, axial location in fuel pin, diffusion rate through fuel), making it difficult to measure. Additionally, iodine nuclei (and fission products in general) will likely implant in either the fuel pellet or the cladding due to their large kinetic energies following a fission event. 

\subsubsection{Radiolysis}

Thermodynamic calculations performed by Konashi \emph{et al}. estimate the equilibrium partial pressure of iodine in the fuel-cladding gap to be as low as 10$^{-17}$ atm when there is no radiolysis of CsI, and up to 10$^{-8}$ atm with the effect of CsI radiolysis included \cite{Konashi1983}. For comparison, mandrel tests of irradiated Zr claddings at 350 $^{\circ}$C show that susceptibility to I-SCC is reduced when the iodine partial pressure is below 60 Pa (approximately 6$\times 10^{-4}$ atm) \cite{anghel2010experimental}. While these calculated values of the iodine pressure are too low to induce corrosion in Zr, they demonstrate that radiation will increase the partial pressure of iodine by several orders of magnitude. Increasing reactor power (e.g. during a ramp) will increase radiation flux and therefore dissociation of CsI, however, no figures are yet available which demonstrate a clear link between this contribution to the iodine pressure and PCI failures. 

\subsubsection{Implantation}

Fission products immediately following a fission event have kinetic energies of up to 90 MeV. These are large, highly ionising particles which deposit their energy to the surrounding atoms within several microns of where they are produced, due to the large electronic and nuclear stopping power of crystalline solids such as UO$_{2}$ and ZrO$_{2}$. Figure \ref{figure:srimtrim}a shows TRIM calculations of the amount of iodine that is implanted in ZrO$_{2}$ at different incident ion energy levels. 

\begin{figure}[ht] % SRIM/TRIM values for iodine in ZrO2
\centering
\includegraphics[width=\linewidth]{images/srimtrim.png}
\caption[\textbf{a)} TRIM distributions for iodine implantation in ZrO$_{2}$ at 50 keV, 800 keV and 6.5 MeV. \textbf{b)} Rutherford backscattering spectrometry profile of 50 keV iodine implanted in ZrO$_{2}$ samples.]{\textbf{a)} TRIM distributions for iodine implantation in ZrO$_{2}$ at 50 keV, 800 keV and 6.5 MeV. \textbf{b)} Rutherford backscattering spectrometry profile of 50 keV iodine implanted in ZrO$_{2}$ samples. Taken from \cite{brossard1998use}.}
\label{figure:srimtrim}
\end{figure}

Further to the mechanism described above, it is important to consider the role of oxygen in the I-SCC process. Fuel pins do not regularly fail during normal operation, despite iodine being produced continuously from fission of the fuel. As previously mentioned, this is because the inner surface of the cladding is not pure Zr metal, but rather a protective oxide which provides an effective barrier against corrosive species such as iodine.

\section{Oxidation of zirconium}

The oxidation of zirconium to produce \zirconia\ occurs during manufacture of the fuel cladding when the Zr metal is exposed to oxygen in air. \zirconia\ is a ceramic with material properties that make it desirable in many industrial applications, including solid-oxide fuel cells \cite{radford1979zirconia}, refractory linings \cite{whittemore1952fused}, and nuclear waste storage \cite{wang2012ceramics}. However, in the context of nuclear fuel cladding, the most important function of \zirconia\ is the barrier it provides against the ingress of corrosive species. 

\zirconia\ grown thermally on Zr metal exists mainly in either the monoclinic or tetragonal phase \cite{Howard1988,teufer1962crystal}. We can expect the internal \zirconia\ layer of the cladding to be mostly monoclinic in early life, with the stress-stabilised tetragonal phase appearing near the oxide/metal interface due to cohesive strains resulting from the lattice mismatch. With increasing burnup, it is expected that more tetragonal and possibly even the cubic phase of \zirconia\ forms due to anion vacancy formation and residual stresses in the lattice from radiation damage \cite{sickafus1999radiation}. Amorphisation due to radiation damage has also been observed in the cubic phase from Cs$^{+}$ implantation \cite{amorphization2000wang}. In this thesis however, while defect energies for the cubic phase are reported (see Figures \ref{isolated_defects}, \ref{table:bound_defects}, \ref{figure:cubicinter}), the focus is on monoclinic and tetragonal \zirconia\ phases, partly due to difficulties predicting the behaviour of the pure high-temperature cubic phase using energies calculated from a static energy technique. 

\subsection{Oxide growth mechanism}

An oxide layer will form on the surface of zirconium metal even at very low oxygen partial pressures \cite{causey2005review}. The oxidation process is mainly driven by the ingress of oxygen ions. Initially, the oxide is highly protective, growing slowly into the metal until it reaches a thickness of approximately 2-3 $\mu$m \cite{garzarolli1991oxide, dawson1968kinetics}, after which the oxide growth mechanism enters a `post-transition' stage where the oxidation kinetics follow a cubic-rate law  \cite{porte1960oxidation}. At low temperatures relative to the melting point or high pressures, and after reaching a critical thickness (called the transition point), parts of the initial oxide will fail and the oxidation rate will increase again. This process is illustrated in Figure \ref{figure:oxide_weight_gain}. 

\begin{figure}[ht]
\centering
\includegraphics[height=10.5cm]{images/zro2_oxide_weight_gain.png}
\caption[Diagrammatic representation of the cyclical oxidation behaviour of \zirconia .]{Diagrammatic representation of the cyclical oxidation behaviour of \zirconia . Taken from \cite{cox1963some}.}
\label{figure:oxide_weight_gain}
\end{figure}

\subsection{Oxygen solubility of zirconium}

Considering the Zr-O binary phase diagram in Figure \ref{figure:binary_phase_diagram}, oxygen is soluble in zirconium up to 29\% of total molar mass when below 400 $^{\circ}$C, commensurate with the operating temperature of a typical PWR (330 \textdegree C at the outer surface of the cladding). Solubility increases slightly up to 35\% of total molar mass as temperature is increased to the liquidus line at 2065 $^{\circ}$C. This is important to note because in the literature, it is assumed that there is pure Zr metal immediately beneath the \zirconia\ layer \cite{rossi2015first}, which is an assumption that will underestimate the extent to which repassivation occurs when the oxide layer is breached, and disregards the effect of the thin ZrO interface that can precede the metal. The molar mass of oxygen required to grow more oxide near the interface will therefore be at least 37\% lower than expected when using this assumption.

The presence of oxygen in the Zr metal will also have an effect on thermodynamic calculations of extrinsic defect formation. Atoms such as Te and I will have to compete with O atoms (and potentially, self-interstitial Zr atoms) for interstitial sites in the metal. This increases the energy barrier to diffusion because of the lower availability of sites. An energy input is required to remove O or Zr atoms occupying these sites, making them a less favourable diffusion path.


\begin{figure}[ht]
\centering
\includegraphics[width=14cm]{images/zro2_binary_phase.png}
\caption[Binary phase diagram of the Zr and O$_{2}$ system.]{Binary phase diagram of the Zr and O$_{2}$ system. Taken from \cite{Abriata1986}.}
\label{figure:binary_phase_diagram}
\end{figure}

\subsection{Outer oxide vs inner oxide} \label{section:outervsinner}

As mentioned previously, the cladding of an LWR fuel pin develops an oxide on both the inner and outer surfaces due to exposure to oxygen in air during manufacture. Both the outer and inner oxide layers provide protection against corrosion, though the corrosive environment is different. 

The outer oxide layer is in contact with the coolant, which is mostly light water with some additional dissolved species such as boron and hydrogen to control reactivity and pH. Figure \ref{figure:outer_oxide} shows a section of the cladding with the outer oxide visible. This layer is mostly monoclinic \zirconia\ with small (nano) grains of tetragonal \zirconia\ distributed uniformly throughout. These grains of tetragonal \zirconia\ are autostabilised during growth of the oxide because of the large volume expansion associated with oxidation (Zr has a Pilling-Bedworth ratio of 1.57). Of course, transmission electron microscope (TEM) foils under examination are always stress-relieved, whereas the oxide in reactor conditions will be under 1-2 GPa of residual stress due to the growth of the oxide (see § \ref{section:tet_stress_stabilisation}).

\begin{figure}[ht]
\centering
\includegraphics[height=10.5cm]{images/outer_oxide.png}
\caption[Scanning transmission electron microscrope (STEM) image of the outer oxide layer formed in an autoclave under simulated PWR water conditions showing the prevalence of different \zirconia\ phases.]{Scanning transmission electron microscrope (STEM) image of the outer oxide layer formed in an autoclave under simulated PWR water conditions showing the prevalence of different \zirconia\ phases. Taken from \cite{Hu2016}.}
\label{figure:outer_oxide}
\end{figure}

The internal oxide layer is much more challenging to examine due to the need to prepare samples in hot cells. This layer is typically very brittle due to radiation damage and implantation of fission products. At a high enough burnup, the \zirconia\ layer makes contact with the UO$_{2}$ fuel, with which it will bond. Figure \ref{figure:inner_oxide} shows a section of the cladding with the inner oxide layer bonded to the pellet. The crystal structure of the \zirconia\ in this layer is debated. Some studies report no monoclinic phase in high burnup fuels, with cubic phase \zirconia\ throughout most of the layer and an amorphous phase nearer the pellet side \cite{Nogita1997}, while other studies report mostly tetragonal phase in this layer \cite{ciszak2017etude, gibert1998influence}. 
%After removal from the reactor and the subsequent cooling period, the inner oxide

\begin{figure}[ht]
\centering
\includegraphics[height=10.5cm]{images/pci_bondinglayer.png}
\caption[High resolution scanning electron microscrope (SEM) image of the bonding layer between a PWR UO$_{2}$ fuel pellet and Zr cladding in a fuel pin at an approximate burnup of 60 GWd/tU.]{High resolution SEM image of the bonding layer between a PWR UO$_{2}$ fuel pellet and Zr cladding in a fuel pin at an approximate burnup of 60 GWd/tU. Adapted from \cite{Lozano1998}.}
\label{figure:inner_oxide}
\end{figure}

Figure \ref{figure:bonding_layer_composition} shows the composition of the inner oxide of a high burnup fuel pin. The fission product (Ba, Mo, Nd) content is highest at the beginning of the \zirconia\ layer and decreases almost linearly with distance towards the Zr metal. This is due to fission product implantation rather than diffusion in the oxide as these elements have low volatility and low cation diffusion rates in UO$_{2}$ where they are produced \cite{S.G.PrussinD.R.OlanderP.Goubeault1984, Prussin1988}. % can you use the weight composition and fission yields to estimate what percentage of iodine implantation will be?

\begin{figure}[ht!]
\centering
\includegraphics[width=14cm]{images/bonding_layer_composition.png}
\caption[Elemental composition of the bonded UO$_{2}$/ZrO$_{2}$ layer from a PWR UO$_{2}$ fuel pellet with a burnup of 61 GWd/tU.]{Elemental composition of the bonded UO$_{2}$/ZrO$_{2}$ layer from a PWR UO$_{2}$ fuel pellet with a burnup of 61 GWd/tU. Taken from \cite{Lozano1998}.}
\label{figure:bonding_layer_composition}
\end{figure}


\subsection{Sources of oxygen}

The internal oxide layer is present before fuel claddings are pressurised with helium gas and sealed. This is from the normal oxidation of Zr in air, where the oxygen pressure is 0.21 atm. After capping of the fuel rods, the only other available oxygen is from the UO$_{2}$ fuel pellets.

Uranium oxides have a wide range of non-stoichiometric compositions, with U/O ratios ranging from 1.67 to 2.24 in solid UO$_{2 \pm x}$, as shown in Figure \ref{figure:U_O_phase_diagram}. The oxide form U$_{3}$O$_{8-y}$ also exists and is more kinetically and thermodynamically stable than UO$_{2}$, but has lower density, making it less suitable for use as a fuel form. The different stoichiometries have different equilibrium O$_{2}$ pressures at constant temperature, allowing some level of internal cladding environment control depending on whether more or less oxygen is desired. 

Oxygen and oxygen precursors may also be produced directly from fission of U$_{235}$, but this contribution is insignificant compared to changing the stoichiometry of the fuel pellet. That is, liberation of oxygen from UO$_{2 \pm x}$ due to fission (which is also a function of the fuel's stoichiometry) is a more significant contributor to the oxygen pressure than direct production via fission.

\begin{figure}[ht!]
\centering
\includegraphics[height=12cm]{images/UO_phase_diagram.png}
\caption[Partial U-O temperature phase diagram between O/U ratios of 1.2 and 2.25.]{Partial U-O temperature phase diagram between O/U ratios of 1.2 and 2.25. Figure taken from \cite{katz2007chemistry}, with phase boundaries from \cite{rand1978thermodynamic, chevalier2002progress, gueneau2002thermodynamic}.}
\label{figure:U_O_phase_diagram}
\end{figure}


\section{Atomistic Simulation}

Conducting experiments on active nuclear materials is a difficult undertaking. Handling irradiated materials is an expensive process, and materials such as uranium are highly controlled (though they are relatively benign before irradiation compared to many typical chemical laboratory materials and solvents). Furthermore, experiments which require samples to be irradiated must be left to cool-down (due to material activation) for up to a year before they can be analysed in a specialised lab \cite{efthymiopoulos2011hiradmat}. Thus, any errors in the experimental procedure or problems with samples are not revealed until months later when material analysis is performed. This makes it difficult to study many material phenomena, especially if they are time-dependent. In-situ reactor experiments are also problematic, requiring sensor equipment to be made tolerant to the high radiation environments as well as being acceptable to and consistent with the reactor operation safety case. The risks and costs mean that experimental work is mostly restricted to the largest labs and the few researchers with enough funding to support it.

\subsection{Classical approach - molecular dynamics}

Molecular dynamics (MD) uses classical mechanics as the basis for calculations. These types of simulations typically use pair potentials (though many-bodied potentials are also used). These are mathematical functions which effectively describe the energy of interaction between two particles. Pair potentials are created by fitting functions to several parameters from empirical data, such as equilibrium bond lengths, thermal properties or even values from quantum mechanical calculations. The simple form of pair potentials allows MD simulations to scale up to billions of atoms, corresponding to a length scale of approximately 0.1 $\mu$m. 

In the literature, many molecular dynamics studies of \zirconia\ exist. However, these studies typically focus on dopant-stabilised zirconias (i.e. cubic ones as empirical potentials don't capture monoclinic or tetragonal phases and their transitions accurately). The large system sizes possible in molecular dynamics simulations are often necessary for examining properties such as ion diffusion, thermal conductivity or melting points \cite{Davis2010}. Studying fission products in \zirconia\ however, requires potentials which can accurately model interactions of atoms such as Zr, O, and I in the solid phase. A good potential for such a system has not yet been published and so a quantum mechanical study of the \zirconia\ system is the focus of this thesis. The work herein may then be used in the future development of such potentials.

%These relatively large system sizes make MD a useful tool for studying a range of materials phenomena which are difficult to model at smaller scales, such as dislocations and long-range diffusion. 

%Add a figure here showing lennard jones. Also show the basic equations
%These are mainly pair potentials (although many-bodied potentials are also used) which are some combination of a short-range repulsion term (Pauli exclusion, nuclear repulsion if van der waals is taken into account) and a long range Coulombic attraction term

\subsection{Quantum mechanical approach - DFT}

Another method for modelling materials at the atomic scale is to use a quantum mechanical approach. In this thesis, the framework of density functional theory (DFT) is used throughout for quantum mechanical calculations (see § \ref{section:dft}). These techniques use a more fundamental approach than molecular dynamics, and are sometimes referred to as \emph{ab initio} methods (although several empirical approximations are often used in DFT). The CASTEP 8.0 software package was used for all DFT calculations \cite{Clark2005}.

System sizes are far more constrained when using DFT. The equations being solved scale combinatorially with the number of electrons and ions in the system, quickly becoming intractable even for simple molecules with a few atoms before applying DFT methods. There are several ways to significantly reduce the computational complexity without sacrificing too much accuracy (e.g. the pseudopotential method, periodic boundaries, cell constraints and symmetry). This allows system sizes on the order of hundreds of atoms to be studied, corresponding to a length scale of approximately 1 nm. While this length scale is much smaller than what can be achieved using MD, the use of a more fundamental parameter (electron density) in calculations provides a stronger scientific basis when material properties are derived from DFT models. Additionally, DFT allows electronic properties such as electron orbital occupancy and band gaps to be studied.

In the literature, DFT studies of \zirconia\ are predominantly focused on the dopant-stabilised cubic phase because of its use in fuel cells and medical applications \cite{orera1990intrinsic,jiang2011first}, with few studies on the undoped system \cite{mackrodt1986theoretical,aarhammar2009energetics}. Pure oxide studies also tend to focus on only one of the three common phase, typically the monoclinic \cite{zheng2007first,foster2002modelling,foster2001structure} and tetragonal phases \cite{Gionco2013, Eichler2004, Zhang2014}. Two notable studies have looked at all three phases. The first focused on the electronic structure and optical properties of \zirconia\ \cite{French1994}, while the second examined the structural properties and band structure of \zirconia\ \cite{Kralik1998}. Both studies utilised the LDA exchange-correlation functional. This functional has since been improved upon (see § \ref{section:kohnsham}), improving the accuracy of more recent models. However, it is always useful to compare to data from older studies. 

Lattice dielectric properties in the three phases have been calculated using DFT \cite{Zhao2002a}, and these have been used in this thesis to predict energies and defect equilibria. Various data from these studies have been used either for comparison or to aid in new calculations which have then been published.

\subsection{Band gap}

Conductors and insulators are two common ways to describe materials. While this binary characterisation may work as an approximation for many materials, in reality there is more of a continuum between these two states (e.g. semiconductors), and at the heart of this lies the band gap.

It is known that for ions, electron energy levels are quantised, restricting the range of possible electron energies to discrete quantities. More specifically, electrons can only occupy unique quantum states, defined by parameters such as quantum spin and angular momentum. In a crystal, where there are large numbers of electrons and many possible configurations of them, we refer to energy \emph{bands} which are comprised of many quantised energies. There are sometimes gaps between energy bands in crystals (illustrated in Figure \ref{figure:band_gap}) corresponding to energy levels that cannot be occupied, meaning that if electrons were to be added at the lowest energy levels one by one, there would occasionally be relatively large jumps in energy as an electron is forced to enter a higher energy band. 

\begin{figure}[ht]
\centering
\includegraphics[width=\linewidth]{images/band_gap.png}
\caption[Illustration of the band gap in diamond as a function of interatomic spacing.]{Illustration of the band gap in diamond as a function of interatomic spacing. Taken from \cite{Chetvorno2017}.}
\label{figure:band_gap}
\end{figure}

Two energy bands, the \emph{valence band} and \emph{conduction band}, help determine a material's metallic or non-metallic character. The valence band contains energy levels occupied by the valence electrons (at absolute zero), while the conduction band contains energy levels which are high enough that electrons may freely move throughout the crystal. In metals, the valence and conduction bands have some amount of overlap, meaning that once the valence band is full, the highest occupied electron energy states are within the conduction band and so the material acts as a conductor. Materials like \zirconia , however, have large energy gaps between the valence and conduction bands, known as the band gap. These materials are called \emph{band insulators} (as opposed to \emph{Mott insulators} where there is no conventional band gap, but electron-electron interactions impede electron promotion to higher energies), because the band gap is an energy barrier preventing the valence electrons from moving freely around the crystal. 

In addition to the valence and conduction bands, a value for the electron chemical potential or Fermi level of the material is needed to determine how the energy bands are filled. If the Fermi level is exactly halfway between the valence band maximum (VBM) and the conduction band minimum (CBM), then the additional energy input required to promote an electron to the conduction band is half of the band gap. The Fermi level is strongly dependent on extrinsic defects and temperature. Extrinsic defects (called dopants) can be introduced to materials such as semiconductors in order to change the concentration of electronic defects (electrons and holes), while an increase in temperature will result in an increase in the Fermi level because of the larger quantity of thermal energy available.

It is important to note that band gaps reported in DFT studies using classical LDA/GGA methods are significantly lower than those obtained experimentally. This is a known problem in DFT, and an exchange-correlation functional which reproduces correct band gap energies in semiconductors and insulators (without overfitting to experimental data) has not yet been discovered. The GW method, which uses a self-energy energy term in place of an exchange-correlation functional, allows more accurate\footnote{The GW approximation still has inaccuracies when modelling strongly correlated systems, but works well with $s$-$p$ systems.} estimates of the band gap than with DFT, but at a significantly higher computational expense. In many cases, the band gap from DFT calculations may be increased by appending an additional potential term, known as a +U parameter, to certain valence electron orbitals (discussed further in § \ref{subsection:plus_U}), or by using hybrid potentials which can incorporate the exact exchange energy.








%\section{Fission product empirical potential}
%
%In order to study the interaction of fission products with larger features in the cladding microstructure, such as dislocations and grain boundaries, it is necessary to develop empirical potentials for use in molecular dynamics simulations. Grain boundary transport is of particular interest, and this would require something on the order of 10$^{4}$ atoms to simulate to a reasonable degree of accuracy. This cannot be done using DFT currently due to the significant amount of computing resources required to run such a simulation. 
%
%The development of an iodine and xenon potential with \zirconia\ should be prioritised in order to run simulations to determine the migration of iodine within \zirconia, followed by the behaviour of xenon at the equilibrium iodine sites.
%
%\section{Grain boundary transport}
%
%Grain boundaries are interesting areas for studying species migration because diffusion towards the metal is expected to be more rapid through them than through bulk \zirconia .
%
%\section{Zr/ZrO/\zirconia\ interface study}
%
%The inner oxide is not a homogeneous structure, as described in § \ref{ch:crystallography}. Figure \ref{fig:zro_interface} clearly shows the existence of a ZrO phase up to 200 nm thick at the interface between \zirconia\ and Zr metal. The presence of ZrO and even oxygen-saturated Zr metal will have an effect on the thermodynamic equilibria of different fission products. An interface study can be conducted using DFT, to determine stresses at the interfaces of Zr and ZrO, and ZrO and \zirconia . Studying the aggregate effect of these interfaces on fission product behaviour may require larger molecular dynamics simulations, however. The crystal structure of the ZrO phase has been studied using both simulation and high-resolution electron microscopy, with two likely crystal structures being proposed \cite{Nicholls2015}. Further atomistic studies must be conducted to determine the stability of each crystal structure of ZrO when constrained by \zirconia\ and oxygen-saturated Zr metal interfaces.
%
%
%\begin{figure}[ht]
%    \centering
%    \includegraphics[height=9cm]{images/zro_interface.png}
%    \caption[STEM image of a Zr-1.0\%Nb sample oxidised in simulated PWR water at 360 C for 120 days.]{STEM image of a Zr-1.0\%Nb sample oxidised in simulated PWR water at 360 C for 120 days. Taken from \cite{inproceedings}.}
%    \label{fig:zro_interface}
%\end{figure} % Good
\chapter{Crystallography and Point Defects}

\label{ch:crystallography}

\section{\zirconia\ phases and stabilisation}

\zirconia\ is unusual in exhibiting three commonly reported polytypes in its binary phase diagram (Figure \ref{figure:binary_phase_diagram}). Each will now be described and contrasted.

\begin{figure}[htp]
\centering
\includegraphics[height=9cm]{images/zro2_binary_phase.png}
\caption{Binary phase diagram of the Zr and O$_{2}$ system. Taken from \cite{Abriata1986TheSystem}.}
\label{figure:binary_phase_diagram}
\end{figure}

\subsection{Monoclinic}

A unit cell of monoclinic \zirconia\ is illustrated in Figure \ref{figure:coordination}. The dashed line (approximately 3.7\r{A} in length) shows the Zr-O bond which is broken when transitioning to monoclinic from the tetragonal phase.

\begin{figure}[htp] % Mono coordination figure
\centering
\includegraphics[height=6.2cm]{images/coordination.png}
\caption{A monoclinic zirconia unit cell indicating the two different oxygen bond coordinations. Small spheres represent oxygen ions while large spheres represent zirconium ions. Taken from \cite{Xia2010}.
\label{figure:coordination}}
\end{figure}

\begin{figure}[htp] % Mono Zr centre
\centering
\includegraphics[height=6cm]{images/zr_centre_mono.png}
\caption{Zirconium centre in monoclinic \zirconia\ showing nearest oxygen atoms and their respective bond co-ordinations. Zirconium atoms are shown in green and oxygen atoms in red.}
\label{figure:monoschottky}
\end{figure}


\begin{table}[htp]
\centering
\onehalfspacing
\caption{\zirconia\ crystal structures and their stable temperatures at 1 atm \cite{Howard1988}.}
\label{table:phases}
\begin{tabular}{ccc}
\hline
{Crystal Structure} & {Space Group}    & {Temperature Range (K)} \\ \hline
\multicolumn{1}{c}{Monoclinic} & \multicolumn{1}{c}{$P2_1/c$} & \multicolumn{1}{c}{$T$ \textless\ 1440}     \\
\multicolumn{1}{c}{Tetragonal} & \multicolumn{1}{c}{$P4_2/nmc$} & \multicolumn{1}{c}{1440 \textless\ $T$ \textless\ 2640}        \\
\multicolumn{1}{c}{Cubic} & \multicolumn{1}{c}{$Fm\overline{3}m$}     & \multicolumn{1}{c}{2640 \textless\ $T$ \textless\ 2950}      \\ \hline
\end{tabular}
\end{table}

\subsection{Tetragonal}

\subsection{Cubic}

\subsection{Other phases}

\subsubsection{Cotunnite}

Two orthorhombic phases of \zirconia\ have also been observed at high pressures.

\begin{figure}[htp]
  \centering
      \includegraphics[height=9cm]{images/cotunnite_structure.png}
  \caption{Illustration of the OII cotunnite crystal structure of \zirconia . Zirconium and oxygen ions are shaded dark and light respectively. Taken from \cite{Haines1997CharacterizationHafnia}.}
  \label{fig:cotunnite_structure}
\end{figure}

\subsubsection*{Volume expansion}

The phase transitions in \zirconia\ are accompanied by a change in volume, where the monoclinic phase is the least dense and the cubic phase is the most dense (see Figure \ref{figure:zrobonddistance}). This is especially significant in the case of the martensitic t-\zirconia\ to m-\zirconia\ transition, where the volume increases by around 9\% \cite{Gupta1977}. This has substantial implications for the creation and opening of cracks as \zirconia\ is a ceramic material with low toughness. This is especially relevant in a reactor scenario where temperature cycling (shutdown/startup or load-following behaviour) may lead to fatigue if the phase transition threshold is passed.

Another consequence of this large volume expansion is that a significant hysteresis effect is observed in the monoclinic/tetragonal phase transition, as shown in Figure \ref{fig:phasediagram}. 
%as the resulting coherency strain is likely to result in reduced mobility of fission products that have been embedded in the bulk crystal. 

\begin{figure}[htp]
  \centering
      \includegraphics[height=10cm]{images/zirconiaphasediagram.png}
  \caption{Pressure-temperature phase diagram for \zirconia . Dash-dotted lines represent more recent data. Diamonds mark transition points during an increase in pressure/temperature, while open circles are used for a decrease in pressure/temperature. Solid circles represent transition points for a fresh, single crystal sample. Taken from \cite{gando2011partial}. \label{fig:phasediagram}}
\end{figure}

\begin{figure}
\begin{center}
\begin{tikzpicture}
	\begin{axis}
		[width=12cm, xlabel={Nearest neighbour Zr-O bond distance (\r{A})}, ylabel={Relative occurrence}, ymin=0, ymax=140, xmin=2.0, xmax=2.50, legend style={{draw=}, at={(0.95,0.95)}, anchor=north east, legend columns=1}]
		\addplot[no marks] table [x=zr_o_dist, y=monoclinic,]{dat/zr_o_bond_distances.dat}; \addlegendentry{Monoclinic};
        \addplot[no marks, dashed] table [x=zr_o_dist, y=tetragonal, ]{dat/zr_o_bond_distances.dat}; \addlegendentry{Tetragonal};
        \addplot[no marks, densely dotted] table [x=zr_o_dist, y=cubic,]{dat/zr_o_bond_distances.dat}; \addlegendentry{Cubic};
			\end{axis}
		\end{tikzpicture}
		\caption{Density plot of the nearest neighbour Zr-O bond distances in \zirconia\ for each crystal structure. Specific volumes from DFT simulations are 11.99 \r{A}$^{3}$ion$^{-1}$, 11.51 \r{A}$^{3}$ion$^{-1}$, and 11.13 \r{A}$^{3}$ion$^{-1}$ for monoclinic, tetragonal, and cubic phases respectively.}
		\label{figure:zrobonddistance}
	\end{center}
\end{figure}


\subsection{Pressure stabilisation (isochoric + autostabilisation)}

The tetragonal and cubic phases of \zirconia\ are stabilised at high pressure. Since the oxide has a larger volume than the underlying metal (pilling-bedworth ratio of 1.5X), the growth of the oxide will itself impose stresses which may stabilise the tetragonal phase.

\subsection{Dopant stabilisation (lower valence cations)}

Some dopants will also stabilise the tetragonal and cubic phases of \zirconia. The most technologically significant of which is yttrium, which at concentrations of 15\% (atomic), fully stabilises the cubic phase. Zirconia stabilised this way is known as yttria-stabilised zirconia (YSZ). The way this works is by trivalent yttrium promoting the inclusion of charge compensating oxygen vacancy defects (see equation XXX). This works in a similar way with several other cation dopants such as trivalent scandium and divalent magnesium.

\section{Point Defects}

\subsection{Kr\"{o}ger-Vink notation}

Kr\"{o}ger-Vink notation \cite{kroger1956relations} is used throughout this thesis to describe defects. It is widely used in physical chemistry and is a useful shorthand for describing chemical reactions where conservation of mass, charge and lattice sites is required. The notation syntax is of the form \ch{x^{y}_{z}}, where x is the substituted atom or missing atom (i.e. a vacancy V), y is the charge of the defect (relative to the lattice species that originally occupied the site) and z is the site the defect occupies. Positive and negative charges are indicated with dots (\ch{^{*}}) and dashes (\ch{^{'}}) respectively, otherwise a cross (\ch{^{x}}) is used to denote a neutral defect. The site may be either a lattice site (such as Zr or O in \zirconia ) or an interstitial site ($i$). Table \ref{table:krogervink} shows examples of several different types of defects and their respective Kr\"{o}ger-Vink notation.

\begin{table}[htp] % Kroger-Vink notation table
\onehalfspacing
\centering
\caption{Examples of Kr\"{o}ger-Vink notation for several defects in \zirconia .}
\label{table:krogervink}
\begin{tabular}{cc}
\hline
Defect & Kr\"{o}ger-Vink Notation \\ \hline
Anion vacancy & \ch{V_{O}^{**}} \\
Cation vacancy & \ch{V_{Zr}^{''''}} \\
Anion interstitial & \ch{O_{i}^{''}} \\
Cation interstitial & \ch{Zr_{i}^{****}} \\
Iodine (I$^{-}$ anion) on oxygen site & \ch{I_{O}^{*}} \\ \hline
\end{tabular}
\end{table}

 % Complete 
\chapter{Computational Methodology}

\label{ch:compmethodology}

\section{Density functional theory} \label{section:dft}

Quantum mechanics is currently the most complete modern theory which describes the behaviour of matter at the length scale of atoms. It can be used to predict things such as the energy levels of atoms, the interactions of light with matter, and the thermodynamic stability of systems of atoms. Ideally, the mathematical formalisms of quantum mechanics would be used to predict the properties and behaviour of all possible types of molecules and materials. In reality, this is very difficult to achieve, requiring several approximations and abstractions in order to produce methods which sacrifice some degree of physical accuracy in order to be computationally tractable. Currently, the most successful approach to predict the behaviour of most solids is provided by density functional theory.
% The mathematical formalisms of quantum mechanics must themselves be discretised to be used in computational simulation methods.

\subsection{The Schr\"{o}dinger equation}

The time-independent Schr\"{o}dinger equation is used to find the total energy of a system:
\begin{equation}
E\Psi(\textbf{r}) = \hat{H}\Psi(\textbf{r})
\label{equation:schrodinger}
\end{equation}

where $E$ is the total energy of the system, $\Psi$ is the wave function associated with the electrons, and $\hat{H}$ is the energy Hamiltonian operator. $\hat{H}$ includes the kinetic energy contributions ($\hat{T}$) and potential energy contributions ($\hat{V}$), shown in atomic units in equations \ref{equation:kineticcontribution} and \ref{equation:potentialcontribution} respectively:
\begin{gather}
\hat{H} = \hat{T} + \hat{V} \label{equation:hamiltonian}\\
\hat{T} = -\sum_i{\frac{1}{2}}\nabla^2_{r_i} - \sum_i{\frac{1}{2M_i}}\nabla^2_{R_i} \label{equation:kineticcontribution} \\
\hat{V} = \sum_{i,j=i+1}{\frac{1}{2|r_i - r_j|}} + \sum_{i,j=i+1}{\frac{Z_i Z_j}{2|R_i - R_j|}} - \sum_{i,j}{\frac{Z_i}{2|R_i - r_j|}} \label{equation:potentialcontribution}
\end{gather}

where $r_{i}$ is the position of electron $i$, $R_{i}$ is the position of nucleus $i$ and $M_{i}$ is the mass of nucleus $i$. Thus, the second term on the right of equation \ref{equation:kineticcontribution} relates to the kinetic energy of any associated nuclei, and the first term to electrons. 

If $\Psi(\textbf{r})$ is the wave function, the electron density at position \textbf{r} ($\rho(\textbf{r})$) is given by:
\begin{equation}
\rho(\textbf{r}) = \Psi(\textbf{r})^2
\end{equation}

\subsection{Kohn-Sham Method} \label{section:kohnsham}

DFT was developed by Kohn and Sham in 1964 \cite{Kohn1965} as an ab initio method for predicting $\rho(\textbf{r})$ associated with an ensemble of atoms. The Kohn-Sham Hamiltonian (Equation \ref{equation:kohnsham}) is used in the Schr\"odinger equation:
\begin{equation}
\hat{H}(\rho(\textbf{r})) = E_{KE}(\rho(\textbf{r})) + E_{P}(\rho(\textbf{r})) + E_{XC}(\rho(\textbf{r}))
\label{equation:kohnsham}
\end{equation}

where $E_{KE}$ and $E_{P}$ are the kinetic and potential energy functionals (functions of functions), $E_{XC}$ is the exchange correlation functional, and \textbf{r} is the position vector. The main approximation is to consider that the electrons only interact with nuclei and the average field generated by all other electrons, and not other electrons explicitly, thus allowing all the terms to be evaluated using the electron density rather than position. An exchange correlation term is then used to include the non-classical electron-electron interactions, namely electron exchange and correlation. Additionally, the exchange correlation term includes the difference in kinetic energy due to the use of non-interacting electrons. While Kohn and Sham proposed an exchange-correlation functional in the Hamiltonian, a general form of the functional has not yet been found. Several forms have been considered, each with strengths and weaknesses when applied to different systems. One basic form of the functional which is frequently used is the LDA \cite{Kohn1965}:
\begin{equation}
E_{LDA}(\rho(\textbf{r})) = \int\rho(\textbf{r})e_{uniform}(\rho(\textbf{r}))dr
\label{equation:LDA}
\end{equation}

where $e_{uniform}$ is the normalised exchange-correlation energy of a uniform electron gas (an idealised system). This exchange-correlation functional generates accurate results in materials such as metals where the electron density is relatively uniform, while systems with more rapidly changing electron densities (e.g. highly ionic materials) require more complex functionals. A natural extension of the LDA is to also take into account the gradient of the electron density, thus allowing a smoother functional fit when electron density is highly variable as a function of position. Such functionals are collectively referred to as GGAs \cite{Langreth1980, Langreth1983, Becke1988, perdew2008restoring}. One GGA which has enjoyed widespread use for many different types of systems is the Perdew-Burke-Ernzerhof (PBE) GGA \cite{Perdew1996}. The accuracy of this functional when modelling solid phase systems is well-established, and its frequent use in DFT studies provides ample reference material for comparing results. After conducting several convergence tests (see § \ref{section:convergence}), the PBE GGA was chosen as the exchange-correlation functional to be used for all calculations in this thesis.

\subsubsection{Born-Oppenheimer approximation}

The Born-Oppenheimer approximation is a two-step process for evaluating atomic forces which greatly reduces the computational costs of atomistic simulations. It exploits the large difference in mass between nuclei and electrons in order to separate their interactions. This allows us to decompose the total wave function into a product of an electronic wave function and a nuclear wave function via a separation of variables approach. The first step involves ignoring the kinetic energy contribution of nuclei by assuming they are stationary, thus the nuclear kinetic energy term in Equation \ref{equation:kineticcontribution} can be removed. The stationary nuclei assumption also simplifies the nuclear-nuclear Coulombic repulsion term in Equation \ref{equation:potentialcontribution} because $|R_i - R_j|$ becomes a constant throughout the calculation. An electronic Schr\"{o}dinger equation is then solved where electronic positions are variables and nuclear positions are fixed parameters. This solution contains information of the shape of the electronic orbitals. The next step is to take the electronic distribution and calculate the resultant forces on the nuclei. The nuclear positions are then modified to minimise these forces, followed by feeding these nuclear positions back into the electronic Schr\"{o}dinger equation to obtain the new electronic distribution. This process is repeated until the required convergence criterion (such as energy change per iteration and forces on nuclei) are satisfied.

\subsection{Pseudopotentials}

The electron-electron interaction component of the potential energy presents a problem when it comes to scaling experimental models. The number of terms in this interaction grows quadratically with the number of electrons in the system, quickly becoming computationally intractable for even small systems. However, it is known that in chemical reactions, the majority of chemical behaviour is determined by relatively few valence electrons, while the more numerous core electrons have a far smaller effect. 

Consider the zirconium atom with 40 electrons, of which 4 (4$d^2$5$s^2$) are typically involved in bonding and chemical reactions. By considering only these valence electrons for Coulombic-term calculations, we reduce the system size by 90\%, which provides a more than tenfold reduction in computational requirements.

Although the core electrons do not participate in chemical reactions, they still influence the properties of the atom, such as the atomic radius. Instead of modelling the core electrons explicitly, we can approximate their aggregate effect with a potential energy function. This is what we aim to achieve by using the pseudopotential method. An example indicative pseudopotential is shown in figure \ref{figure:pseudopotential}.

\begin{figure}[ht] % Pseudopotential Image
\begin{center}
\includegraphics[height=10cm]{images/pseudopotential.png}
\end{center}
\caption[Sketch of an all-electron potential V$_{AE}$ and a pseudopotential V$_{PS}$ with their corresponding wave functions. r$_{cut}$ indicates the radius beyond which both the potentials and their wave functions are the same.]{Sketch of an all-electron potential V$_{AE}$ and a pseudopotential V$_{PS}$ with their corresponding wave functions. r$_{cut}$ indicates the radius beyond which both the potentials and their wave functions are the same. Adapted from \cite{Payne1992}.}
\label{figure:pseudopotential}
\end{figure}

Figures \ref{figure:o_pp} and \ref{figure:zr_pp} show the actual pseudopotentials used throughout this work for oxygen and zirconium respectively. The potentials are shown broken down by the electronic sub-shells occupied by the valence electrons. The pseudopotentials are shown in order of increasing sub-shell energies, thus the 5s electron orbitals are filled before the 4d orbitals in zirconium. Two lines for the all-electron wavefunction are shown, corresponding to the different electron angular momenta.

\begin{figure}[ht] % Oxygen pseudopotentials
\begin{center}
\begin{tikzpicture}
	\begin{groupplot}[group style={group size=1 by 2}, width=14cm, height=7cm]
	\nextgroupplot[
    axis y line*=middle, axis x line*=bottom, y label style={at={(-0.03,-0.1)}}, ylabel=Wavefunction $\Psi$, axis y line shift=0.5cm,
    xtick=\empty, ymin=-1.1, ymax=1.1, ytick={-1, -0.5, 0, 0.5, 1}, xmax=2.5, 
    x axis line style={white}]
         \addplot[no marks, dashed, draw=red!80!white] table [x=distance, y=O_AE_s1,]{dat/o_pp.dat}; 
		\addplot[no marks, draw=red!80!white] table [x=distance, y=O_pp_s1,]{dat/o_pp.dat}; 
		\addplot[no marks, dashed, draw=blue!80!white] table [x=distance, y=O_AE_s2,]{dat/o_pp.dat}; 
		\addplot[no marks, draw=blue!80!white] table [x=distance, y=O_pp_s2,]{dat/o_pp.dat};
		\node at (0.1, 0.99) {\textbf{O (2s)}};
		\draw[black] % horizontal line
				(-0.11, 0)
				-- % = line-to
				++ % = calculate a vector sum
				(axis direction cs:2.61, 0);
		\draw[black, dotted] % Vertical line
				(1.299,1)
				-- % = line-to
				++ % = calculate a vector sum
				(axis direction cs:0,-2);
				
    \nextgroupplot[
    axis y line*=middle, axis x line*=bottom, axis x line shift=0, x axis line style={white}, axis y line shift=0.5cm,
    ymin=-1.1, ymax=1.1, ytick={-1, -0.5, 0, 0.5, 1}, xtick={0, 0.5, 1, 1.5, 2, 2.5}, xmax=2.5, xlabel=Distance from nucleus (Bohr)]
         \addplot[no marks, dashed, draw=red!80!white] table [x=distance, y=O_AE_p1,]{dat/o_pp.dat}; 
		\addplot[no marks, draw=red!80!white] table [x=distance, y=O_pp_p1,]{dat/o_pp.dat}; 
		\addplot[no marks, dashed, draw=blue!80!white] table [x=distance, y=O_AE_p2,]{dat/o_pp.dat}; 
		\addplot[no marks, draw=blue!80!white] table [x=distance, y=O_pp_p2,]{dat/o_pp.dat}; 
		\node at (0.1, 0.99) {\textbf{O (2p)}};
		\draw[black] % horizontal line
				(-0.11, 0)
				-- % = line-to
				++ % = calculate a vector sum
				(axis direction cs:2.61, 0);
		\draw[black, dotted] % vertical line
				(1.299,1)
				-- % = line-to
				++ % = calculate a vector sum
				(axis direction cs:0,-2);
	\end{groupplot}
		\end{tikzpicture}
		\caption{Plots of the valence $s$ and $p$ orbital potentials for oxygen with two projectors per angular momentum. Dashed lines indicate the all-electron potentials while solid lines indicate the corresponding pseudopotential. Dotted vertical line marks the radius beyond which the potentials match.}
		\label{figure:o_pp}
	\end{center}
\end{figure}

\clearpage

\begin{figure}[h!] % Zirconium pseudopotentials
\begin{center}
\begin{tikzpicture}
	\begin{groupplot}[group style={group size=1 by 3}, width=14cm, height=7cm]
	\nextgroupplot[ % 4p
    axis y line*=middle, axis x line*=bottom, axis y line shift=0.5cm,
    xtick=\empty, ymin=-1.2, ymax=1.2, ytick={-1, -0.5, 0, 0.5, 1}, xmax=2.5, 
    x axis line style={white}]
         \addplot[no marks, dashed, draw=red!80!white] table [x=distance, y=Zr_AE_p1,]{dat/zr_pp.dat}; 
		\addplot[no marks, draw=red!80!white] table [x=distance, y=Zr_pp_p1,]{dat/zr_pp.dat}; 
		\addplot[no marks, dashed, draw=blue!80!white] table [x=distance, y=Zr_AE_p2,]{dat/zr_pp.dat}; 
		\addplot[no marks, draw=blue!80!white] table [x=distance, y=Zr_pp_p2,]{dat/zr_pp.dat};
		\node at (0.1, 0.99) {\textbf{Zr (4p)}};
		\draw[black] % horizontal line
				(-0.11, 0)
				-- % = line-to
				++ % = calculate a vector sum
				(axis direction cs:2.61, 0);
		\draw[black, dotted] % Vertical line
				(2.105,1)
				-- % = line-to
				++ % = calculate a vector sum
				(axis direction cs:0,-2);
				
	\nextgroupplot[ % 5s
    axis y line*=middle, axis x line*=bottom, axis y line shift=0.5cm, ylabel=Wavefunction $\Psi$,
    xtick=\empty, ymin=-1.2, ymax=1.2, ytick={-1, -0.5, 0, 0.5, 1}, xmax=2.5, 
    x axis line style={white}]
         \addplot[no marks, dashed, draw=red!80!white] table [x=distance, y=Zr_AE_s1,]{dat/zr_pp.dat}; 
		\addplot[no marks, draw=red!80!white] table [x=distance, y=Zr_pp_s1,]{dat/zr_pp.dat}; 
		\addplot[no marks, dashed, draw=blue!80!white] table [x=distance, y=Zr_AE_s2,]{dat/zr_pp.dat}; 
		\addplot[no marks, draw=blue!80!white] table [x=distance, y=Zr_pp_s2,]{dat/zr_pp.dat};
		\node at (0.1, 0.99) {\textbf{Zr (5s)}};
		\draw[black] % horizontal line
				(-0.11, 0)
				-- % = line-to
				++ % = calculate a vector sum
				(axis direction cs:2.61, 0);
		\draw[black, dotted] % Vertical line
				(2.105,1)
				-- % = line-to
				++ % = calculate a vector sum
				(axis direction cs:0,-2);
				
    \nextgroupplot[ % 4d
    axis y line*=middle, axis x line*=bottom, axis x line shift=0, x axis line style={white}, axis y line shift=0.5cm,
    ymin=-1.2, ymax=1.2, ytick={-1, -0.5, 0, 0.5, 1}, xtick={0, 0.5, 1, 1.5, 2, 2.5}, xmax=2.5, xlabel=Distance from nucleus (Bohr)]
         \addplot[no marks, dashed, draw=red!80!white] table [x=distance, y=Zr_AE_d1,]{dat/zr_pp.dat}; 
		\addplot[no marks, draw=red!80!white] table [x=distance, y=Zr_pp_d1,]{dat/zr_pp.dat}; 
		\addplot[no marks, dashed, draw=blue!80!white] table [x=distance, y=Zr_AE_d2,]{dat/zr_pp.dat}; 
		\addplot[no marks, draw=blue!80!white] table [x=distance, y=Zr_pp_d2,]{dat/zr_pp.dat}; 
		\node at (0.1, 0.99) {\textbf{Zr (4d)}};
		\draw[black] % horizontal line
				(-0.11, 0)
				-- % = line-to
				++ % = calculate a vector sum
				(axis direction cs:2.61, 0);
		\draw[black, dotted] % vertical line
				(2.105,1)
				-- % = line-to
				++ % = calculate a vector sum
				(axis direction cs:0,-2);
	\end{groupplot}
		\end{tikzpicture}
		\caption{Plots of the valence $s$, $p$ and $d$ orbital potentials for zirconium with two projectors per angular momentum. Dashed lines indicate the all-electron potentials while solid lines indicate the corresponding pseudopotential. Dotted vertical line marks the radius beyond which the potentials match.}
		\label{figure:zr_pp}
	\end{center}
\end{figure}

%\begin{figure}[ht] % Oxygen pseudopotential
%\begin{center}
%\includegraphics[height=7.3cm]{images/oxygen_otf_pp.png}
%\end{center}
%\caption{Plots of the valence $s$ and $p$ orbital potentials for oxygen with two projectors per angular momentum. Dashed lines indicate the all-electron potentials while solid lines indicate the corresponding pseudopotential. Dotted vertical line marks the radius beyond which the potentials match.}
%\label{figure:oxygen_pseudopotential}
%\end{figure}

%\begin{figure}[ht] % Zirconium pseudopotential
%\begin{center}
%\includegraphics[height=12cm]{images/zirconium_otf_pp.png}
%\end{center}
%\caption{Plots of the valence $s$, $p$ and $d$ orbital potentials for zirconium with two projectors per angular momentum. Dashed lines show the all-electron potentials while solid lines indicate the corresponding pseudopotential. Dotted vertical line marks the radius beyond which the potentials match.}
%\label{figure:zirconium_pseudopotential}
%\end{figure}


\section{Periodic boundaries}

\subsection{Bloch's theorem}

The repeating nature of a crystal structure, defined by the lattice vectors plus a basis set of atoms that are repeated, is well-suited for computer models. It allows us to define periodicity in three dimensions for a given unit cell. An example of this periodicity is illustrated in Figure \ref{figure:periodicboundary} in two dimensions. A model based on this periodicity is justified as follows:

\begin{itemize}
\item Nuclei are arranged in a periodically repeating pattern, thus their potentials acting on electrons are also periodic.
\item If the potential is periodic, it follows that the electron density is also periodic.
\item The electron density is equivalent to the square of the wave function magnitude, thus the magnitude of the wave function is also periodic.
\end{itemize}

\begin{figure}[ht] % Periodic boundary image
\begin{center}
\includegraphics[width=\linewidth]{images/PeriodicBoundaryThesis.png}
\end{center}
\caption{Two dimensional illustration of periodic boundary around a primitive cell.}
\label{figure:periodicboundary}
\end{figure}

Knowing that the magnitude of the wave function is periodic greatly simplifies the calculation process; only one `period' of the function needs to be evaluated. However, the phase of the wave function can take any of an infinite number of values and still satisfy the periodicity condition. At this point, we consider Bloch's theorem which states that the possible wave functions are all quasi-periodic, and thus the wave function can be expressed as:  % Patrick's Fig 2.3 is really useful for describing this
\begin{equation}
\label{equation:bloch}
\psi_k(\textbf{r}) = e^{i\textbf{k}.\textbf{r}}u_k(\textbf{r})
\end{equation}

Where $\psi_k(\textbf{r})$ is the wave function evaluated at position \textbf{r}, $e^{i\textbf{k}.\textbf{r}}$ is an arbitrary phase factor, and $u_k(\textbf{r})$ is a periodic function with the same periodicity as the wave function. Solutions to this equation exist for any value of \textbf{k} and so the general solution can be expressed as an integral over the first Brillouin zone, the primitive lattice cell in reciprocal space. Instead of evaluating the integral over the range of \textbf{k} (a computationally costly task as it is done for many wave functions), a sum of values at discrete points, known as \textbf{k}-points, is used. This approximation is valid because the wave function varies slowly over \textbf{k}, thus allowing the integral to be approximated with several appropriately spaced \textbf{k}-points. In general, a finer \textbf{k}-point grid results in increased accuracy, but at an increased computational cost \cite{Hasnip2010}. For all DFT calculations in this thesis, a Monkhorst-Pack sampling scheme \cite{Monkhorst1976} was used for Brillouin zone integration, with a minimum \textbf{k}-point separation of 0.09 \r{A}$^{-1}$.

\subsection{Plane-waves}

The electron density of a system is described in the context of a basis set. A basis set is a collection of functions (known as basis functions) which can be combined to produce some relevant output, typically the mathematical description for the shape of an electron orbital. For example, any sound wave can be generated from a combination of sine functions (basis functions). 

The purpose of a basis set in DFT calculations is to describe the varying amplitude of the electron density in space. Any complete basis set (e.g. plane-wave, correlation-consistent, split-valence) may be used to represent the behaviour of electron orbitals, but a plane-wave method was chosen due to its greater suitability for periodic systems (plane-waves are intrinsically periodic). Since the electron densities are represented by a finite sum of plane-waves with different energies, a truncation error will be incurred. Plane-waves of higher energies provide a smaller contribution to the overall density, so only plane-wave up to a chosen cut-off energy value are considered in order to reduce computational requirements. An appropriate plane-wave cut-off energy must therefore be determined through a convergence test. 

%Figure \ref{Figure:cutoffconvergence} shows the first convergence study where the total energy of simulations with various values of $E_{cutoff}$ were compared to a highly converged value, and then plotted on a log scale to see how precision is improved at larger values.


\section{Computational details}
\subsection{Cell dimensions and initialisation}

A supercell method is used for the study of various defects. The first step is to create a unit cell of \zirconia\ in each of the three crystal structures. Each unit cell is then fully relaxed through a geometry optimisation process (see § \ref{geometry_optimisation_method}). The resulting cell is used to construct supercells through tessellation in three dimensions, before being fully relaxed again. In this way, we generate systems with up to ten times as many atoms as the unit cell (supercell details can be found in Table \ref{table:supercells}). This is necessary because introducing defects into a small unit cell will result in the defect interacting with itself across the periodic boundary. A supercell increases the distance between the defect and its periodic image, using the bulk material as an interaction buffer. 

When constructing a supercell, it is important to consider making the supercell equally large in all directions, such that any directional bias in defect-defect interaction is minimised. Larger supercells carry an increased computational cost when running calculations, limiting the sizes we can achieve. For example, a constant-volume defect calculation with 300 atom supercells will take upwards of 500 hours to complete (on the Imperial College HPC using four 32-core nodes), whereas the equivalent 100 atom supercell will take just 72 hours (fully relaxed calculations are even more computationally expensive).


\begin{table}[ht] % Supercell details
\doublespacing
\centering
\caption{Composition of the supercells in terms of the number of individual unit cells stacked in each direction.} % Unit cells were stacked in such a way as to produce the most cubic supercell in order to minimise directional defect-defect interactions.}
\vspace*{2mm}
\label{table:supercells}
\begin{tabular}{cccccccc}
\hline
\multirow{2}{*}{{\bf \begin{tabular}[c]{@{}c@{}}Crystal \\ Structure\end{tabular}}} & \multicolumn{3}{c}{{\bf No. unit cells}} & \multicolumn{3}{c}{{\bf Supercell size (\AA)}} & \multirow{2}{*}{{\bf \begin{tabular}[c]{@{}c@{}}No.\\ atoms\end{tabular}}} \\ \cline{2-7}
 & \hspace{0.25 cm} a \hspace{0.2 cm} & b & c & a \hspace{0.0 cm} & b & c \hspace{0.35 cm} &  \\ \hline
\begin{tabular}[c]{@{}c@{}}Monoclinic\\ ($P2_1/c$)\end{tabular} & 2 & 2 & 2 & 10.37 & 10.47 & 10.75 & 96 \\ \hline
\begin{tabular}[c]{@{}c@{}}Tetragonal\\ ($P4_2/nmc$)\end{tabular} & 3 & 3 & 2 & 10.85 & 10.85 & 10.56 & 108 \\ \hline
\begin{tabular}[c]{@{}c@{}}Cubic\\ ($Fm\overline{3}m$)\end{tabular} & 2 & 2 & 2 & 10.22 & 10.22 & 10.22 & 96 \\ \hline
\end{tabular}
\end{table}

\subsection{Geometry optimisation} \label{geometry_optimisation_method}

The geometry optimisation task in CASTEP follows a simple steepest-descent algorithm which attempts to satisfy certain convergence criteria, depending on the constraints applied to the system. This is an iterative process which takes an initial system state, modifies ion positions slightly and then calculates the difference in properties between the states to check for convergence. 

The variational principle in quantum mechanics tells us that the lowest system energy calculated is always an upper bound for the ground state energy, thus providing a way to check if modifications to the system are actually optimising the geometry. The exception is when the system converges upon a local minimum, which may not be an experimentally observed state. This can be avoided to some extent by having good initial ion placement from which to optimise.

\subsection{Convergence criteria for geometry optimisation} \label{convergence_criteria}

Four convergence criteria are used for the geometry optimisation tasks throughout this work, one of which is only used when performing constant-pressure calculations, such as when a supercell is being fully relaxed. These criteria are evaluated with respect to the previous iteration during the geometry optimisation task:

\begin{itemize}
\item \emph{Change in energy per ion}: The largest change in the energy per ion between iterations must be below $10^{-5}$ eV. Below this value, the total energy improvement towards the ground state for a 100 atom supercell is less than 0.001 eV, and is therefore considered converged.
\item \emph{Maximum force on an ion}: The maximum force requirement on any single ion in an iteration must be below $10^{-2}$ eV/\r{A}. This is required to make sure that the ion position will not change significantly in the following iteration, possibly bringing another convergence criterion above its threshold.
\item \emph{Maximum change in ion position}: This must be below $5 \times 10^{-4}$ \r{A} between iterations to be considered converged. This criterion specifies the maximum `rattle' of the ion that is tolerated once the minimum energy is reached (i.e. displacements above this value may still be important for achieving a correct atomic configuration). 
\item \emph{Maximum stress (constant-pressure only)}: During unconstrained relaxation, the maximum change in stress between iterations should be below 50 MPa. This is necessary to avoid large deviations which may distort the symmetry of the supercell, resulting in anomalous energy values.
\end{itemize}

Using these convergence criteria, non-defective supercells of \zirconia\ were relaxed under constant pressure. The resulting structure was used as the starting point to which defects were introduced, and subsequently relaxed again, this time under constant volume conditions to simulate low defect concentrations \cite{Murphy2014, Bell2015}. Finally, all DFT calculations on doped and defective structures in this thesis employed the Pulay method for density mixing \cite{Pulay1980} to take into consideration changes in electronic behaviour of the system caused by the defect and to speed up convergence.

\subsection{Charged cell correction} \label{charged_cell_correction}

When calculating the energy of a defect with an overall non-zero charge, this charge introduces a systematic error in the energy value which is a function of the charge magnitude. This is typically the case in high band-gap materials such as \zirconia\ where electron mobility is far lower than in metals or semiconductors, allowing defects such as \ch{V_{O}^{**}} and \ch{V_{Zr}^{''''}} to be thermodynamically stable in the lattice. 

The source of the error from charged defects is self-interaction across the periodic boundary, made necessary by the finite cell size. A common solution is to append a Makov-Payne correction term when calculating formation energies of defects \cite{Makov1995, Makov1996}. This works well in many cases, but does not take into consideration the anisotropy in the material's dielectric properties, as is the case in tetragonal \zirconia\ due to the non-unity lattice $c/a$ ratio. These effects are better captured when using a screened Madelung correction \cite{Murphy2013}. This method provides a more complete description of the dielectric properties by utilising a dielectric tensor rather than a single value of the dielectric constant (or relative permittivity). Dielectric tensors for the different phases of \zirconia\ were taken from the literature \cite{Zhao2002a, Zhao2002}. The screened Madelung correction is therefore used in preference to a Makov-Payne correction throughout this thesis.

\subsection{Helmholtz free energy} \label{helmholtz_method}

In order to examine the relationship between temperature and energy for the different \zirconia\ phases, phonon calculations were performed in CASTEP using a method outlined by Burr \emph{et al.} \cite{burr2015crystal,jackson2016resolving}. This entails using the harmonic approximation to determine the shape of the potential well that an atom sits in. The potential well is approximated by a spherically symmetric harmonic well, centred at an atom's equilibrium position in the lattice. At a temperature of 0 K, an atom will occupy the lowest region of its potential well, known as the ground state (though they will still have energy, known as zero-point energy). As temperature increases, the atom will sometimes occupy higher energy states in the potential well due to increased thermal vibrations, moving from its equilibrium position. The total energy ($A(T, V)$) of this system, known as the Helmholtz free energy, is calculated using the internal energy ($U(V)$), vibrational enthalpy ($H_{v}(T, V)$), vibrational entropy ($S_{v}(T, V)$) and configurational entropy ($S_{conf}$):
\begin{equation} \label{vibrational}
A(T, V) = U(V) + H_{v}(T, V) - TS_{v}(T, V) - TS_{conf} 
\end{equation}
where $T$ is temperature and $V$ is volume. The configurational entropy is calculated using Boltzmann statistics:
\begin{equation}
S_{conf} = k_{B}\ch{ln}(\Omega)
\end{equation}
where $k_{B}$ is the Boltzmann constant and $\Omega$ denotes the number of possible configurations (i.e. valid permutations of energy level occupancy). The vibrational terms in Equation \ref{vibrational} are obtained by performing a constant-volume phonon calculation in CASTEP and then integrating over the resulting phonon density of states (DOS). This is done over a range of temperatures for each crystal structure of \zirconia .

\subsection{Incorporation energies}

The inner oxide of the fuel cladding will be highly defective due to radiation damage, resulting in a high concentration of pre-existing intrinsic defect sites relative to the concentration of fission products. We therefore consider the energy of fission product incorporation on to these existing defect sites. The energies to incorporate atoms at interstitial and substitutional sites in \zirconia\ were calculated from the set of defective and perfect supercell DFT energies. For iodine, incorporation energies were established to place atoms into vacancy sites of different charge to generate defects from \ch{I_{O}^{x}} to \ch{I_{O}^{**}}, and \ch{I_{Zr}^{x}} to \ch{I_{Zr}^{''''}}. I was also incorporated onto the interstitial sites.

The incorporation energy equation for iodine uses $\frac{1}{2}$I$_{2}$ as the reference state of iodine, while Te, Xe and Cs use the DFT energy calculated as a single atom in a large cell:
\begin{equation}
\label{interstitial_incorp_equation}
E_{inc}(\ch{I_{$i$}^{x}}) = E_{DFT}(\ch{I_{$i$}^{x}}) - (E_{DFT}(ZrO_2) + \frac{1}{2}\mu_{I_{2}})  % - \frac{E_{I_2}}{2}
\end{equation}
where $E_{inc}(\ch{I_{$i$}^{x}})$ is the incorporation energy of a neutral iodine interstitial, $E_{DFT}(\ch{I_{$i$}^{x}})$ is the energy of a neutral iodine interstitial, $E_{DFT}(ZrO_2)$ is the energy of a non-defective \zirconia\ supercell and $\mu_{I_{2}}$ is the chemical potential of an I$_{2}$ molecule, taken from a single point DFT calculation of the I$_{2}$ dimer. For incorporation of a charged interstitial (e.g. $\ch{I_{i}^{*}}$), the energy required to add or remove an electron is included in the calculation:
\begin{equation}
\label{interstitial_incorp_equation_charged}
E_{inc}(\ch{I_{$i$}^{n}}) = E_{DFT}(\ch{I_{$i$}^{n}}) - (E_{DFT}(ZrO_2) + \frac{1}{2}\mu_{I_{2}} + n(E_{VBM} + \mu_{e}))
\end{equation}
Similarly, for a substitutional defect:
\begin{equation}
\label{o_sub_incorp_equation}
E_{inc}(\ch{I_{O}^{$n$}}) = E_{DFT}(\ch{I_{O}^{$n$}}) - (E_{DFT}(\ch{V_{O}^{$n$}}) + \frac{1}{2}\mu_{I_{2}})  % - \frac{E_{I_2}}{2}
\end{equation}
where $\ch{I_{O}^{$n$}}$ is an iodine substitutional defect at an oxygen site of charge $n$ and $\ch{V_{O}^{$n$}}$ is the corresponding oxygen vacancy.

\subsection{Stiffness matrix generation}

The elastic stiffness matrices for the pure monoclinic, tetragonal and cubic phases of \zirconia\ were calculated using CASTEP's \emph{elastic constants} task. The calculation of elastic constants is a multi-step process involving up to 36 individual DFT calculations. Several scripts have been made available to simplify this process (see Appendix \ref{castep_scripts}).

The first step is to generate multiple different \texttt{.cell} files, each with either a small deviation in the lattice parameter or an additional shear on the cell. This requires starting with a unit cell that has already been completely relaxed via the \emph{geometry optimisation} task. In total, 36 \texttt{.cell} files are generated, each corresponding to a single element of the eventual stiffness matrix. The next step is to run a single point DFT calculation on each \texttt{.cell} file with the \emph{calculate stress} parameter enabled to output the resulting stress matrix. The final step is to use Hooke's Law to calculate the elastic stiffness constants using the known stress and strain state. 

%\subsection{Strain method for defect volumes}
%
%The volumes of the defective supercells were kept constant because constant pressure calculations have been shown to sometimes break the symmetry of the supercell \cite{samanta2010thermodynamic}, leading to unreliable energy values. This is partly due to the assumed arrangement of the defects that may not be commensurate with the cell symmetry. This approach to calculating defect volumes then relies on calculating the elastic constants of the non-defective supercell, followed by extracting the resultant stress tensor from a defect simulation. The strain tensor of the defective cell can then be calculated using Hooke's law, giving the relaxation volume. 

\subsection{Defect relaxation volumes} \label{isobaricmethod}

Defect relaxation volumes of point defects were calculated using an isobaric method, requiring two calculations to be performed under constant-pressure using the geometry optimisation task in CASTEP. The defect relaxation volume ($\Delta$V) is defined as: 
\begin{equation}
\Delta V = V_{def} - V_{perf}
\end{equation}
where $V_{def}$ is the relaxation volume of the defective supercell and $V_{perf}$ is the relaxation volume of the non-defective (perfect) supercell. In this thesis, mentions of `volume' will refer to relaxation volumes unless stated otherwise.

After completing an energy calculation, CASTEP provides the volume of the resulting cell, defined as the space enclosed by the repeating unit of atoms within the calculated lattice parameters. By subtracting the volume of a non-defective cell from the volume of a defective cell, we obtain a value for the total defect volume. 

It is important to consider that if there is a non-zero charge on the system, this will affect the calculated volume. Two systems with the same type, amount and arrangement of atoms, but different overall charges, will have different energies (due to the number of electrons). Different electronic orbital occupancies will affect the inter-atomic forces and therefore the shape of the cell. In order to compensate for this effect, a `corrected' relaxation volume, as described by Goyal \emph{et al.} \cite{goyal2017conundrum} is calculated when the defect has a non-zero charge:
\begin{equation}
\Delta V = V_{def}^{q} - V_{perf}^{q}
\end{equation}
where $q$ is the defect charge. This formulation uses the volume of a non-defective supercell with equal charge magnitude as the reference structure. This method has been shown to yield more reasonable defect volumes than when using neutral non-defective supercells as the reference structure. Defect volumes without this correction applied are provided in Appendix \ref{uncorrected_volumes} for comparison.

\section{Defect energies and equilibria} 

\subsection{Defect formation energies}

Defect formation energies are calculated using equation \ref{equation:formation_energy}:
\begin{equation} \label{equation:formation_energy}
    E_{f} = E_{def} - (E_{perf} \pm \sum_{i} n_i\mu_i + q(E_{VBM} + \mu_{e})) + E_{corr}
\end{equation}
where $E_{f}$ is the formation energy, $E_{def}$ is the energy of the defective supercell, $E_{perf}$ is the energy of a non-defective supercell, $q$ is the defect charge, $E_{VBM}$ is the valence band maximum, $\mu_{e}$ is the Fermi level relative to the VBM and $E_{corr}$ is a charged-cell correction term (see § \ref{charged_cell_correction}). Since $\mu_{e}$ is not a fixed value, plots of formation energy against $\mu_{e}$ are produced to examine the behaviour of defects across the entire range of the band gap. These are reported in Figures \ref{figure:monovacancies}, \ref{figure:tetvacancies} and \ref{figure:cubicvacancies}.

\subsection{Defect equilibria} \label{brouwer_method} % 

Typically in materials, several types of defects will exist simultaneously. These defects will be present at an equilibrium concentration based on their thermodynamic stability. Predicting the defect equilibria is possible with statistical mechanics and some approximations. For example, it is expected that a crystal lattice will usually be overall charge-neutral (exceptions can be made under certain conditions, see § \ref{space_charge}), otherwise we would see a build-up of charge with a large Coulomb energy penalty which would be thermodynamically unsustainable.

Brouwer diagrams, also known as Kr{\"o}ger-Vink diagrams, were produced using a method outlined by Murphy \emph{et al}. \cite{Murphy2014, Murphy2014a} through which it is possible to determine defect concentrations as a function of oxygen partial pressure. We start from the statement that the chemical potential of \zirconia\ is equivalent to the sum of chemical potentials $\mu$ of its constituent species, Zr and O:
\begin{equation}
{\mu}_{ZrO_2(s)} = {\mu}_{Zr}(p_{O_2}, T) + {\mu}_{O_{2}}(p_{O_{2}}, T)
\label{mewZrO2compmethodology}
\end{equation}
where $T$ denotes temperature and $p_{O_2}$ denotes oxygen partial pressure. The chemical potential of \zirconia\ in the solid state is assumed to have negligible dependence on $T$ and $p_{O_2}$ relative to ${\mu}_{Zr}$ and ${\mu}_{O_2}$. Energies can be obtained for bulk \zirconia\ and Zr, but the ground state of oxygen is not correctly reproduced in DFT \cite{Batyrev2000,Lozovoi2001}. Instead, we use the approach of Finnis \emph{et al}. \cite{Finnis2005} to infer the oxygen chemical potential from standard state values. We can use the experimental Gibbs free energy to produce an equation where $\mu_{O_2}$ is the only unknown:
\begin{equation}
\Delta{G^{\plimsoll}_{f, ZrO_2}} = \mu_{ZrO_2(s)} - (\mu_{Zr(s)} + \mu^{\plimsoll}_{O_2})
\end{equation}
where $\Delta{G^{\plimsoll}_{f, ZrO_2}}$ is the experimental Gibbs energy at standard temperature and pressure and $\mu^{\plimsoll}_{O_2}$ is the oxygen chemical potential under the same conditions. Only monoclinic \zirconia\ is stable under standard conditions, with $\Delta{G^{\plimsoll}_{f, ZrO_2}}$ = -1042.746 kJ/mol (10.807 eV) \cite{brown2005chemical}. Values of the Gibbs free energy of formation for the tetragonal (10.697 eV) and cubic (10.595 eV) phases were obtained by adding the energy difference between the phases from DFT calculations. The values of $\mu_{ZrO_2(s)}$ and $\mu_{Zr(s)}$ are calculated using DFT. Once $\mu^{\plimsoll}_{O_2}$ is calculated, we can generalise the chemical potential of oxygen for any value of $T$ and $p_{O_2}$ by appending an ideal gas relationship $\Delta{\mu(T)}$ and a Boltzmann distribution:
\begin{gather}
\mu_{O_2}(p_{O_2},T) = \mu^{\plimsoll}_{O_2} + \Delta{\mu(T)} + \frac{1}{2}{k_B}log(\frac{p_{O_2}}{p^{\plimsoll}_{O_2}}) \\
\Delta \mu(T) = -\frac{1}{2}(S^{\plimsoll}_{O_{2}}- C^{\plimsoll}_{p})(T-T^{\plimsoll}) + C^{\plimsoll}_{p}T\textup{log}\left ( \frac{T}{T^{\plimsoll}} \right )
\end{gather} 
where $S^{\plimsoll}_{O_{2}}$ is the molecular entropy at standard temperature and pressure (T$^{\plimsoll}$ = 273.15 K, P$^{\plimsoll}$ = $10^{5}$ Pa), and $C^{\plimsoll}_{p}$ is the constant pressure heat capacity of oxygen. These quantities have values of $S^{\plimsoll}_{O_{2}}$ = 0.0021 eV/K and $C^{\plimsoll}_{p}$ = 0.000302 eV/K \cite{weast1984crc}. 

Using our generalised formula for $\mu_{O_2}$, we fix the temperature within the range of thermal phase-stabilisation (e.g. 1500 K for tetragonal \zirconia) and calculate $\mu_{O_2}$ for many different values of $p_{O_2}$ between $10^{-35}$ and 10$^{0}$ atm, corresponding to oxygen deficient and oxygen rich environments, respectively ($p_{O_2}$ in air is approximately 0.2 atm). While the tetragonal phase will be stress-stabilised in practice, thermal-stabilisation in such models has been shown to qualitatively approximate the effect of stress-stabilisation, while allowing a wider range of dopant behaviours to be predicted \cite{Bell2016}. 

Once a value of $\mu_{O_2}$ is calculated, defect concentrations can then be calculated using Boltzmann statistics. These concentrations were calculated using the method outlined by Kasamatsu \emph{et al}. \cite{Kasamatsu2012} whereby the effect of defects competing for the same lattice site is taken into account. The next step is to calculate the concentration of electron and hole defects. This is done by using the charge-neutrality condition to determine the Fermi level (electrochemical potential) in the system:
\begin{equation}
\sum_{i}q_{i}c_{i} - N_{c}\textrm{exp}{(-\frac{E_{g}-\mu_{e}}{k_{B}T})} + N_{v}\textrm{exp}{(-\frac{\mu_{e}}{k_{B}T})} = 0
\label{charge_neutrality}
\end{equation}

Where $c_{i}$ is the concentration of defect $i$, $q_{i}$ is its respective charge, $N_{c}$ and $N_{v}$ are the integrated density of states for the conduction and valence bands, $E_{g}$ is the band gap and $\mu_{e}$ is the Fermi level. 

\subsubsection{Temperature and pressure stabilisation}

As discussed in Chapter \ref{ch:crystallography}, the tetragonal and cubic phases are stabilised at elevated temperatures and pressures. However, DFT calculations of supercells under stress require significantly greater computational resources to yield sufficiently converged energy results, and so all DFT calculations in this thesis are performed on relaxed supercells. To account for this lack of stress stabilisation, Brouwer diagrams are generated at higher temperatures (where the tetragonal and cubic phases are thermally stabilised) rather than at 650 K, which is the expected temperature at the internal surface of the cladding. This approach to compensating for stress stabilisation follow that of similar studies published by other groups \cite{youssef2012intrinsic, Youssef2014, Otgonbaatar2014}.

\subsection{Effect of space charge} \label{space_charge}

Electrons have a higher rate of diffusion than oxygen vacancies in \zirconia , leading to a build-up of oxygen vacancies near the metal-oxide interface as corrosion progresses \cite{bojinov2010influence}. In the case of \zirconia , this effect will be pronounced because the layer is thin. This results in an overall positive charge (since the dominant oxygen vacancy is \ch{V_{O}^{**}}) referred to as a space charge. This effect can be taken into account when generating Brouwer diagrams by assuming an overall charge in the crystal structure instead of charge-neutrality:
\begin{equation}
\sum_{i}q_{i}c_{i} - N_{c}\textrm{exp}{(-\frac{E_{g}-\mu_{e}}{k_{B}T})} + N_{v}\textrm{exp}{(-\frac{\mu_{e}}{k_{B}T})} = q_{s}c_{s}
\label{charge_non_neutrality}
\end{equation}
where $q_{s}$ is the charge of a unit of the artificial space charge defect and $c_{s}$ is the concentration.

Figure \ref{figure:spacechargeexample} shows an example of the defect equilibria in tetragonal \zirconia\ with an overall positive space charge. In order for such a condition to be satisfied, higher concentrations of positively charged oxygen vacancy and hole defects are predicted to be present, while zirconium vacancy defects fall significantly. When extrinsic defects are also present in the lattice in significant concentrations, the space charge condition may influence which defect types are dominant at different oxygen pressures, as different oxidation states may be necessary to satisfy the charge condition.

%Another effect considered was the space charge of the system. Electrons have a higher rate of diffusion than oxygen vacancies in ZrO2, leading to a build-up of oxygen vacancies near the metal-oxide interface as corrosion progresses [44]. This results in an overall positive charge (since the dominant oxygen vacancy is referred to as a space charge. When included in our Brouwer diagrams, this space charge had a negligible effect on the concentration or charge state of iodine up to a charge of  holes per f.u. ZrO2. This corresponds to a high concentration of oxygen vacancies relative to the equilibrium concentration, predicting that a significant deviation from the equilibrium is not expected near the metal oxide interface as a result of a positive space charge.

\begin{figure}[htp] % Tet intrinsic no space charge
\begin{center}
\begin{tikzpicture}
	\begin{groupplot}[group style={group size=1 by 2}, width=14cm, height=11cm]
	\nextgroupplot[
		 ylabel={\ch{log_{10}}([D]) (per f.u.)}, ymin=-10, ymax=0, xmin=-35, xmax=0, legend style={{draw=}, at={(0.40,0.97)}, anchor=north west, legend columns=2, nodes={scale=1, transform shape}}]
        \addplot[no marks, draw=blue!70!black] table [x=pO2, y=electrons,]{dat/intrinsic_tet.dat}; \addlegendentry{\ch{e^{'}}}; \node at (-26.0,-2) {\ch{e^{'}}};
        \addplot[no marks, draw=red!85!black] table [x=pO2, y=holes,]{dat/intrinsic_tet.dat}; \addlegendentry{\ch{h^{\textperiodcentered}}}; \node at (-6.8,-3.6) {\ch{h^{\textperiodcentered}}};
        \addplot[no marks, draw=black!70!green] table [x=pO2, y=VO{2},]{dat/intrinsic_tet.dat}; \addlegendentry{\ch{V_{O}^{\textperiodcentered\textperiodcentered}}}; \node at (-28,-3) {\ch{V_{O}^{\textperiodcentered\textperiodcentered}}};
%         \addplot[no marks, draw=black!55!green] table [x=pO2, y=VO{1},]{dat/intrinsic_tet.dat}; \addlegendentry{\ch{V_{O}^{*}}};
%         \addplot[no marks, draw=black!30!green] table [x=pO2, y=VO{0},]{dat/intrinsic_tet.dat}; \addlegendentry{\ch{V_{O}^{x}}};
        \addplot[no marks, draw=yellow!85!blue] table [x=pO2, y=VM{-4},]{dat/intrinsic_tet.dat}; \addlegendentry{\ch{V_{Zr}^{''''}}}; \node at (-2.9,-4.6) {\ch{V_{Zr}^{''''}}};
%         \addplot[no marks, draw=yellow!75!blue] table [x=pO2, y=VM{-3},]{dat/intrinsic_tet.dat}; \addlegendentry{\ch{V_{Zr}^{'''}}};
%         \addplot[no marks, draw=yellow!65!blue] table [x=pO2, y=VM{-2},]{dat/intrinsic_tet.dat}; \addlegendentry{\ch{V_{Zr}^{''}}};
%         \addplot[no marks, draw=yellow!55!blue] table [x=pO2, y=VM{-1},]{dat/intrinsic_tet.dat}; \addlegendentry{\ch{V_{Zr}^{'}}};
%         \addplot[no marks, draw=yellow!45!blue] table [x=pO2, y=VM{0},]{dat/intrinsic_tet.dat}; \addlegendentry{\ch{V_{Zr}^{x}}};
%         \addplot[no marks, draw=red!60!yellow] table [x=pO2, y=Oi{-2},]{dat/intrinsic_tet.dat}; \addlegendentry{\ch{O_{i}^{''}}};
%         \addplot[no marks, draw=red!50!yellow] table [x=pO2, y=Oi{-1},]{dat/intrinsic_tet.dat}; \addlegendentry{\ch{O_{i}^{'}}};
%         \addplot[no marks, draw=red!40!yellow] table [x=pO2, y=Oi{0},]{dat/intrinsic_tet.dat}; \addlegendentry{\ch{O_{i}^{x}}};
%         \addplot[no marks, draw=green!80!pink] table [x=pO2, y=Mi{4},]{dat/intrinsic_tet.dat}; \addlegendentry{\ch{Zr_{i}^{****}}};
%         \addplot[no marks, draw=green!70!pink] table [x=pO2, y=Mi{3},]{dat/intrinsic_tet.dat}; \addlegendentry{\ch{Zr_{i}^{***}}};
%         \addplot[no marks, draw=green!60!pink] table [x=pO2, y=Mi{2},]{dat/intrinsic_tet.dat}; \addlegendentry{\ch{Zr_{i}^{\textbf{**}}}};
%         \addplot[no marks, draw=green!50!pink] table [x=pO2, y=Mi{1},]{dat/intrinsic_tet.dat}; \addlegendentry{\ch{Zr_{i}^{*}}};
%         \addplot[no marks, draw=green!40!pink] table [x=pO2, y=Mi{0},]{dat/intrinsic_tet.dat}; \addlegendentry{\ch{Zr_{i}^{x}}};
%         \addplot[no marks] table [x=pO2, y=Stoich,]{dat/intrinsic_tet.dat}; \addlegendentry{Stoich};
\node at (-33.7,-0.5) {\textbf{a)}}; 
			%\end{axis}     
%\end{tikzpicture}
%\begin{tikzpicture} % 1e-1
	\nextgroupplot[
		 xlabel={\ch{log_{10}}($p_{O_{2}}$) (atm)}, ylabel={\ch{log_{10}}([D]) (per f.u.)}, ymin=-10, ymax=0, xmin=-35, xmax=0, legend style={{draw=}, at={(0.40,0.97)}, anchor=north west, legend columns=4, nodes={scale=1, transform shape}}]
        \addplot[no marks, draw=blue!70!black] table [x=pO2, y=electrons,]{dat/intrinsic_spacecharge01.dat}; \node at (-26.5,-2.5) {\ch{e^{'}}};
        \addplot[no marks, draw=red!85!black] table [x=pO2, y=holes,]{dat/intrinsic_spacecharge01.dat}; \node at (-13,-3) {\ch{h^{\textperiodcentered}}};
        \addplot[no marks, draw=black!70!green] table [x=pO2, y=VO{2},]{dat/intrinsic_spacecharge01.dat}; \node at (-28,-0.8) {\ch{V_{O}^{\textperiodcentered\textperiodcentered}}};
%         \addplot[no marks, draw=black!55!green] table [x=pO2, y=VO{1},]{dat/intrinsic_spacecharge01.dat}; \addlegendentry{\ch{V_{O}^{*}}};
%         \addplot[no marks, draw=black!30!green] table [x=pO2, y=VO{0},]{dat/intrinsic_spacecharge01.dat}; \addlegendentry{\ch{V_{O}^{x}}};
        %\addplot[no marks, draw=yellow!85!blue] table [x=pO2, y=VM{-4},]{dat/intrinsic_spacecharge01.dat}; \node at (-3,-3) {\ch{V_{Zr}^{''''}}};
%         \addplot[no marks, draw=yellow!75!blue] table [x=pO2, y=VM{-3},]{dat/intrinsic_spacecharge01.dat}; \addlegendentry{\ch{V_{Zr}^{'''}}};
%         \addplot[no marks, draw=yellow!65!blue] table [x=pO2, y=VM{-2},]{dat/intrinsic_spacecharge01.dat}; \addlegendentry{\ch{V_{Zr}^{''}}};
%         \addplot[no marks, draw=yellow!55!blue] table [x=pO2, y=VM{-1},]{dat/intrinsic_spacecharge01.dat}; \addlegendentry{\ch{V_{Zr}^{'}}};
%         \addplot[no marks, draw=yellow!45!blue] table [x=pO2, y=VM{0},]{dat/intrinsic_spacecharge01.dat}; \addlegendentry{\ch{V_{Zr}^{x}}};
%         \addplot[no marks, draw=red!60!yellow] table [x=pO2, y=Oi{-2},]{dat/intrinsic_spacecharge01.dat}; \addlegendentry{\ch{O_{i}^{''}}};
%         \addplot[no marks, draw=red!50!yellow] table [x=pO2, y=Oi{-1},]{dat/intrinsic_spacecharge01.dat}; \addlegendentry{\ch{O_{i}^{'}}};
%         \addplot[no marks, draw=red!40!yellow] table [x=pO2, y=Oi{0},]{dat/intrinsic_spacecharge01.dat}; \addlegendentry{\ch{O_{i}^{x}}};
%         \addplot[no marks, draw=green!80!pink] table [x=pO2, y=Mi{4},]{dat/intrinsic_spacecharge01.dat}; \addlegendentry{\ch{Zr_{i}^{****}}};
%         \addplot[no marks, draw=green!70!pink] table [x=pO2, y=Mi{3},]{dat/intrinsic_spacecharge01.dat}; \addlegendentry{\ch{Zr_{i}^{***}}};
%         \addplot[no marks, draw=green!60!pink] table [x=pO2, y=Mi{2},]{dat/intrinsic_spacecharge01.dat}; \addlegendentry{\ch{Zr_{i}^{\textbf{**}}}};
%         \addplot[no marks, draw=green!50!pink] table [x=pO2, y=Mi{1},]{dat/intrinsic_spacecharge01.dat}; \addlegendentry{\ch{Zr_{i}^{*}}};
%         \addplot[no marks, draw=green!40!pink] table [x=pO2, y=Mi{0},]{dat/intrinsic_spacecharge01.dat}; \addlegendentry{\ch{Zr_{i}^{x}}};
%         \addplot[no marks] table [x=pO2, y=Stoich,]{dat/intrinsic_spacecharge01.dat}; \addlegendentry{Stoich};
\node at (-33.7,-0.5) {\textbf{b)}};
	\end{groupplot}
			%\end{axis}              
\end{tikzpicture}
		\caption{Tetragonal phase Brouwer diagrams of intrinsic point defects at a temperature of 1500 K \textbf{a)} without a space charge and \textbf{b)} with a space charge of $10^{-1}$ e$^{-1}$/fu.}
		\label{figure:spacechargeexample}
	\end{center}
\end{figure}

\section{Convergence testing} \label{section:convergence}

\subsection{Plane-wave cut-off energy}

In order to determine an appropriate value for the plane-wave cut-off energy, a convergence test was performed to determine the relative error in predicted energy compared to a highly converged value. This convergence test was conducted by running multiple geometry optimisation procedures under fully relaxed conditions on a unit cell of \zirconia\ for each phase. A small \textbf{k}-point spacing of 0.01 \r{A}$^{-1}$ was used for each task (highly converged), while increasing the plane-wave cut-off energy from 300 eV to 750 eV in 50 eV increments. The energy of each run was recorded and compared to the energy of a highly converged value taken when a cut-off energy of 900 eV was used. This provides a value for the truncation error at different cut-off energies. Figure \ref{Figure:cutoffconvergence} shows a log plot of the energy error for each phase of \zirconia\ as the cut-off energy is increased. 

The error is shown to be independent of phase, with all lines lying on a single path. This might be expected because the atoms in each phase are the same, and therefore the electrons involved in the calculations remain unchanged, however, interatomic distances are different in the different phases and thus so are the electron densities, so this equivalence in convergence does not necessarily have to follow. A cut-off energy of 600 eV was found to produce an error below 0.01 eV, and was subsequently used for future calculations as it provides a good compromise between computational cost and accuracy.

\begin{figure}[ht] % Plane-wave cut-off convergence
	\begin{center}
		\begin{tikzpicture}
			\begin{axis}
				[width=\linewidth*0.7, xlabel={E\textsubscript{cutoff} (eV)}, ylabel={log$_{10}$(error) / formula unit}, ymin=-3.5, legend style={{draw=}, at={(0.95,0.95)}, anchor=north east,}]
				\addplot[no marks] table [x=cutoffenergy, y=logerrormono,]{dat/convergence.dat}; \addlegendentry{Monoclinic};
			    \addplot[no marks, dashed] table [x=cutoffenergy, y=logerrortet,]{dat/convergence.dat}; \addlegendentry{Tetragonal};
			    \addplot[no marks, densely dotted] table [x=cutoffenergy, y=logerrorcubic,]{dat/convergence.dat}; \addlegendentry{Cubic};
                \draw[red,-stealth]
				(600,-1.96)
				-- % = line-to
				++ % = calculate a vector sum
				(axis direction cs:0,-1.46);
                \addplot [only marks,mark=*]
coordinates { (600,-1.95) };
			\end{axis}
		\end{tikzpicture}
		\caption{Plot of the log error of DFT energy against plane-wave cut-off energy for a perfect cell of each crystal structure. The error is calculated with respect to a highly converged value, calculated at a plane-wave cut-off energy of 900 eV. The red arrow indicates the cut-off energy beyond which the error is below 0.01 eV.}
		\label{Figure:cutoffconvergence}
	\end{center}
\end{figure}

\subsection{\textbf{k}-point convergence}

Too fine a grid in reciprocal space (i.e. a large number of \textbf{k}-points) results in prohibitively computationally expensive simulations, whereas too coarse a grid may have a large truncation error when energies are calculated. To find the optimum spacing of \textbf{k}-points, a convergence study was performed across a range of \textbf{k}-point spacings, with the output energies compared to a highly converged simulation to obtain a value for the error. 

Figure \ref{Figure:kpoint_convergence} shows the energy error for each phase of \zirconia\ as a function of the \textbf{k}-point spacing (given in reciprocal space as \r{A}$^{-1}$). The highly converged energy value was calculated with a \textbf{k}-point spacing of 0.01 \r{A}$^{-1}$ for error calculations. The plot shows a stepwise change in the error value as the grid spacing is reduced. This is because calculations demand an integer number of \textbf{k}-points, and larger spacings do not provide sufficient resolution to effectively fit an integer number of \textbf{k}-points into the reciprocal grid, so that the program snaps to the nearest appropriate grid number. An optimum \textbf{k}-point spacing was chosen at 0.09 \r{A}$^{-1}$, which was the largest spacing that kept the error below 0.01 eV for all phases, highlighted in the plot by the red arrow.

\begin{figure}[ht]
\begin{center}
\begin{tikzpicture}
	\begin{axis}
		[width=\linewidth*0.7, xlabel={\textbf{k}-point spacing (\r{A}\textsuperscript{-1})}, ylabel={log[error]}, ymin=-7, ymax=1, xmin=0, xmax=0.22, legend style={{draw=}, at={(0.05,0.95)}, anchor=north west, legend columns=1}, xticklabel
style={/pgf/number format/.cd,fixed,precision=5}]
		\addplot[no marks] table [x=kpoint_spacing, y=monoclinic,]{dat/kpoint_convergence.dat}; \addlegendentry{Monoclinic};
        \addplot[no marks, dashed] table [x=kpoint_spacing, y=tetragonal, ]{dat/kpoint_convergence.dat}; \addlegendentry{Tetragonal};
        \addplot[no marks, densely dotted] table [x=kpoint_spacing, y=cubic,]{dat/kpoint_convergence.dat}; \addlegendentry{Cubic};
        \draw[red,-stealth]
				(0.09,-2.35)
				-- % = line-to
				++ % = calculate a vector sum
				(axis direction cs:0,-4.6);
                \addplot [only marks,mark=*]
coordinates { (0.09,-2.35) };
			\end{axis}
		\end{tikzpicture}
		\caption{Log of the error in the total energy of the system as a function of \textbf{k}-point spacing. The error is calculated relative to a highly converged energy value at a \textbf{k}-point spacing of 0.01 \r{A}\textsuperscript{-1}. The red arrow indicates the \textbf{k}-point spacing which yields an error below 0.01 eV for all structures.}
		\label{Figure:kpoint_convergence}
	\end{center}
\end{figure}

\subsection{Exchange-correlation functionals}

There are a range of possible exchange-correlation functionals available in CASTEP, spanning both empirical and non-empirical types. Empirical exchange-correlation functionals are typically optimised to capture specific properties or systems particularly well, but perform less well for generalised systems. Non-empirical exchange-correlation functionals, while still not perfect, are preferred for modelling the widest range of properties. In a sense, non-empirical functions benefit from not being `over-fit' to experimental data. They are also more prevalent in the literature, thereby providing a rich corpus of work for comparison studies.

While the PBE-GGA exchange-correlation functional in this work had already been selected, it was helpful to conduct an energy convergence study of the systems across the different functionals available in CASTEP in order to determine how other functionals compared. Only 6 of the 14 functionals available in CASTEP were able to yield a converged energy calculation within a reasonable amount of time, as shown in Figure \ref{Figure:xc_test}. This is because several hybrid functionals partially incorporate the exact exchange using the Hartree-Fock method \cite{hartree1928wave}, significantly increasing the computational cost of an energy calculation. 

The calculated energies indicate that each functional correctly predicts the order of phase stability in \zirconia , though the magnitude of the energy difference between phases varied. These differences are small, approximately 0.1 eV/f.u., but their effects are compounded when defects are introduced into the cell. The total energies were more varied, with several eV differences between functionals, however, lower total energies across different exchange-correlation functionals do not necessarily suggest that a better minima has been found. For example, the PW91 functional resulted in even lower energies than PBE, despite PW91 preceding PBE and both having been developed by Perdew \emph{et al}. \cite{perdew1991unified, perdew1992atoms}. It is the energy difference between systems calculated with the same exchange-correlation functional which is important. 

To better gauge the performance of each functional, further studies across a much larger range of parameters, and even materials, would need to be conducted, however this is beyond the scope of this thesis.

\begin{figure}[ht!] % XC functional study
  \begin{center}
    \begin{tikzpicture}
      \begin{axis}
        [ybar, ymin=2350, ymax=2362, width=\linewidth*0.7, xtick=data, xlabel={XC Functional}, ylabel={Unit cell energy (-eV/f.u.)}, xticklabels from table={dat/xc_test.dat}{functional}, area legend, legend style={at={(0.04,0.96)},
anchor=north west, legend columns=1}] %axis x line=middle, ymin=5.1, ymax=5.35, xmin=0, xmax=12, legend style={{draw=}, at={(0.18,0.95)}, anchor=north east, legend columns=1}
        \addplot[style={black, fill=red!30!white, mark=none}] table [x expr=\coordindex, y=mono_energy]{dat/xc_test.dat};
        \addplot[style={black, fill=blue!30!white, mark=none}] table [x expr=\coordindex, y=tet_energy]{dat/xc_test.dat};
        \addplot[style={black, fill=green!30!white, mark=none}] table [x expr=\coordindex, y=cubic_energy]{dat/xc_test.dat};
        \legend{Monoclinic, Tetragonal, Cubic}
      \end{axis}
    \end{tikzpicture}
    \caption{Calculated energy of a unit cell of monoclinic, tetragonal and cubic \zirconia\ when using different exchange-correlation functionals.}
    \label{Figure:xc_test}
  \end{center}
\end{figure}

\subsection{On-the-fly pseudopotentials}

Ultra soft pseudopotentials are generated in CASTEP automatically (known as on-the-fly or OTF pseudopotentials) when none are specified for a particular element. Energies must be calculated and compared with the same set of pseudopotentials in order to keep simulations self-consistent. A single point calculation was performed on a unit cell of \zirconia\ and the resulting OTF pseudopotentials (one for oxygen and one for zirconium) were saved and used for all subsequent calculations. 

It is important to determine the variance in energy values of different pseudopotentials generated OTF in order to avoid systematic error. To assess error, 9 different pairs of OTF pseudopotentials were generated\footnote{OTF pseudopotential generation in CASTEP uses a random number seed which is generated whenever a calculation is run without specifying a pseudopotential. By default, these are ultra-soft pseudopotentials.} and used to calculate the total energy of a monoclinic \zirconia\ supercell. The difference in energy was then calculated with respect to the pseudopotential pair that resulted in the lowest energy. These deviations in total energy are shown in Figure \ref{Figure:otf_pp_test}. Across all calculations, the largest difference in total energy was 0.0012 eV, while the average difference was 0.0006 eV. Since here the only concern is with choosing other parameters to achieve a precision of 0.01 eV, and the largest deviation calculated is an order of magnitude below that, it is not necessary to take any special measures to correct any systematic error from randomly generated OTF pseudopotentials.

%\begin{figure}[ht] % +U cubic
%\begin{center}
%\begin{tikzpicture}
%	\begin{axis}
%		[width=11cm, xlabel={+U on Zr \emph{d} orbitals (eV)}, ylabel={Lattice parameter (\r{A})}, ymin=5.1, ymax=5.35, xmin=0, xmax=12, legend style={{draw=}, at={(0.18,0.95)}, anchor=north east, legend columns=1}]
%		\addplot[no marks] table [x=plusU, y=a,]{dat/plus_u_cubic.dat}; \addlegendentry{$a$};
%        %\addplot[no marks, dashed] table [x=plusU, y=b, ]{dat/plus_u_cubic.dat}; \addlegendentry{b};
%        %\addplot[no marks, densely dotted, black] table [x=plusU, y=c,]{dat/plus_u_cubic.dat}; \addlegendentry{c};
%			\end{axis}
%		\end{tikzpicture}
%		\caption{Individual lattice parameters as a function of +U term in cubic \zirconia .}
%		\label{Figure:plusucubic}
%	\end{center}
%\end{figure}

\begin{figure}[ht] % OTF PP study
  \begin{center}
    \begin{tikzpicture}
      \begin{axis}
        [ybar, width=\linewidth*0.7, xlabel={OTF pseudopotential pair}, ylabel={$\Delta$E with respect to lowest energy (meV)}, ] %axis x line=middle, ymin=5.1, ymax=5.35, xmin=0, xmax=12, legend style={{draw=}, at={(0.18,0.95)}, anchor=north east, legend columns=1}
        \addplot table [x=pp_pair, y=energy_diff_wrt_first,]{dat/otf_pp_test.dat};
      \end{axis}
    \end{tikzpicture}
    \caption{Energy deviation in meV of supercells with candidate OTF pseudopotential pairs. Energy deviations are shown with respect to the pseudopotential pair that resulted in the lowest total energy calculated.}
    \label{Figure:otf_pp_test}
  \end{center}
\end{figure}

\subsection{Chemical potential of oxygen}

The chemical potential of oxygen is required when performing any defect formation energy calculation where an atom of oxygen is added or removed (see Equation \ref{equation:formation_energy}). Calculating the chemical potential of oxygen requires special consideration of the electronic structure of O$_{2}$. The ground state of the O$_{2}$ molecule is known as triplet oxygen ($^{3}\Sigma^{-}_{g}$), an allotrope which exhibits a resultant spin magnetic moment (oxygen is paramagnetic). This is in contrast to singlet oxygen ($^{1}\Delta_{g}$) with a spin magnetic moment of zero. 

Two calculations, one for triplet and another for singlet oxygen, were performed using CASTEP. Large cells of 15 \r{A} x 15 \r{A} x 15 \r{A} were used to run geometry optimisation tasks on two oxygen atoms initially separated by 1.3 \r{A}. For the triplet oxygen calculation, a net electronic spin of +2 on the $p$ electrons was enforced, while the singlet oxygen calculation specified a net spin of 0.

The calculated bond lengths of triplet and singlet oxygen were 1.225 \r{A} and 1.227 \r{A} respectively. These bond lengths are within 2\% of the experimental value of 1.207 \cite{Lide2016}, with the triplet state prediction being slightly closer to this value. The calculated energies from DFT for triplet and singlet oxygen were -871.92 eV and -870.70 eV respectively. This gave an energy difference of 1.22 eV between the two forms of diatomic oxygen. While triplet oxygen was correctly predicted as the lower energy allotrope, the energy difference reported in the literature from microwave spectroscopy measurements is 0.9773 eV \cite{Atkins2006}, an almost 25\% difference compared to the DFT value. This large difference is attributed to the exchange-correlation functional and the inability to correctly model electron correlation effects in some cases. In this thesis, the DFT calculated energy of triplet oxygen was used only for formation energy against Fermi level plots, while defect equilibria calculations utilised a different method to calculate this value (see § \ref{brouwer_method}).

\subsection{Chemical potential of iodine}

To determine the chemical potential of iodine, an energy minimisation of the iodine dimer was performed. Unlike oxygen, iodine dimers do not exhibit a non-zero spin magnetic moment, thus avoiding a source of error in energy calculations with the PBE exchange-correlation functional. Similar to the \zirconia\ unit cell calculations, the lattice parameter after relaxation (bond length in this case) is compared to experimental data to assess the quality of the simulation parameters.

Figure \ref{figure:iodine_dimer} illustrates the energy minimisation of two iodine atoms in a cell of size 15 \r{A} x 15 \r{A} x 15 \r{A}, initially separated by 3.0 \r{A}. The geometry optimisation task finds an energy minima when the iodine atoms are bonded, at a separation of 2.69 \r{A}. This agrees well with the experimental value of 2.6745 \r{A} \cite{ukaji1966effect}.

\begin{figure}[ht] % Iodine dimer geometry optimisation
\centering
\includegraphics[width=14cm]{images/iodine_molecule.png}
\caption{Energy minimisation of two iodine atoms from an initial separation of 3.0 \r{A}.}
\label{figure:iodine_dimer}
\end{figure}

\subsection{+U study}
\label{subsection:plus_U}

In some DFT studies, an additional potential energy term (Hubbard U parameter or +U) is sometimes included to better capture the Coulomb interaction of localised electrons. An LDA or GGA functional alone will typically not describe this interaction correctly, especially for localised $d$ and $f$ electrons\footnote{Multiple occupation of $d$ and $f$ orbitals incurs an energy penalty which is not accurately modelled by the exchange-correlation functional.}. Of particular concern is the calculated value of the band gap from DFT simulations, as this value may deviate by up to 30\% from experimental values. Remedying this shortcoming with an appropriate +U parameter could therefore be valuable in obtaining accurate energies. 

In the literature, one GGA+U study of Fe-doped tetragonal \zirconia\ has shown that the inclusion of a +U term on Zr $d$ orbital electrons between 0 and 3.3 eV changes the electronic properties (in particular the electronic density of states) of the system significantly, and that the best agreement with experimental data occurs when U = 0 eV \cite{Sangalli2013}. Other GGA+U studies on bulk tetragonal \zirconia\ found that a +U term of 4 eV led to calculated lattice parameters which were in good agreement (within 0.05 \r{A}) with experimental values, but that the calculated band gap was still underestimated by 1.28 eV \cite{RuizPuigdollers2016, Chen2015, Puigdollers2015}. One LDA+U study in \zirconia\ used a +U parameter of 1 Ry (13.6 eV), and reproduced the correct order of stability of the monoclinic, tetragonal and cubic phase. This shows how the value of the +U parameter can vary significantly depending on the other approximations being used in the DFT calculations, and therefore an appropriate +U value for this thesis would have to be found independently. A +U study of the zirconium atom, with an electronic configuration of [Kr]$4d^{2}5s^{2}$, was performed to determine the response to and therefore the viability of this additional potential term for the $d$ electrons.

Figure \ref{Figure:plusubandgap} shows the effect on the calculated band gap when introducing a +U term. While the +U term does increase the band gap, the effect is not significant in bringing the prediction in-line with experimental values. Even with +U terms of 10 eV, the calculated band gap falls short of the experimental band gap by at least 1.5 eV. Moreover, with +U terms greater than 4 eV, we begin to see erratic behaviour in the development of both the band gap, and also in the predicted crystal structure. 

For the tetragonal phase, the calculated band gap (4.2 eV) does not agree with that calculated by Puigdollers \emph{et al.} \cite{RuizPuigdollers2016} (4.5 eV) when a +U term of 4 eV is used. The PBE GGA exchange-correlation functional and a plane-wave basis set was used in both studies, however, their study utilised the VASP 5.3 DFT software package while CASTEP 8.0 was used in this thesis. Different software packages may use different minimisation methods which could contribute to differing values, but determining the cause of this anomaly would require a separate study of the two codes at a low level which is beyond the scope of this thesis.

%Cubic is fine, it just keeps expanding with +U as we expect. The tetragonal phase expands in the short a&b directions but contracts in the long c direction (i.e. becomes more 'cubic') up until 6 eV, after which it grows in the same manner as cubic.

%The monoclinic phase is harder to explain. There is a cross-over in the length of the a and b parameter at around 4 eV, and then the beta angle (the ~99 deg between a and c) snaps into 90 deg at 11 eV, and when I look at the output structure at this energy, the coordination number of Zr is 6, down from 7. That's why the lattice parameters don't fall into a=b=c, because it doesn't become cubic.

%As you can see from the band gap plot below, just +U by itself is not enough to reproduce the experimental band gap, even for monoclinic. We're off by about 1.5 eV in each case.

\begin{figure}[ht] % +U band gaps
\begin{center}
\begin{tikzpicture}
	\begin{axis}
		[width=\linewidth*0.7, xlabel={+U on Zr \emph{d} orbitals (eV)}, ylabel={Band gap (eV)}, ymin=3.2, ymax=5, xmin=0, xmax=12, legend style={{draw=}, at={(0.35,0.95)}, anchor=north east, legend columns=1}]
		\addplot[no marks] table [x=plusU, y=bandgap,]{dat/plus_u_mono.dat}; \addlegendentry{Monoclinic};
        \addplot[no marks, dashed] table [x=plusU, y=bandgap, ]{dat/plus_u_tet.dat}; \addlegendentry{Tetragonal};
        \addplot[no marks, densely dotted, black] table [x=plusU, y=bandgap,]{dat/plus_u_cubic.dat}; \addlegendentry{Cubic};
			\end{axis}
		\end{tikzpicture}
		\caption{Calculated band gaps for different +U values in monoclinic, tetragonal and cubic \zirconia .}
		\label{Figure:plusubandgap}
	\end{center}
\end{figure}

\subsubsection{Monoclinic}

In monoclinic \zirconia , the use of a +U term causes the lattice parameters to change disproportionately to each other, as seen in Figure \ref{Figure:plusumono}. All lattice parameters increased with larger +U terms, however, expansion in the $a$ direction proceeded faster than in the $b$ direction, resulting in the $a$ lattice parameter becoming larger at a +U of 4 eV. +U terms larger than 10.5 eV caused the lattice parameters to snap suddenly onto new values. A  investigation of the atomic positions revealed that the monoclinic crystal structure had collapsed into an orthorhombic structure (i.e. the cell experienced a shear strain which resulted in a $\beta$ of 90\textdegree), with the co-ordination number of zirconium ions falling to 6 from 7.

\begin{figure}[ht] % +U mono
\begin{center}
\begin{tikzpicture}
	\begin{axis}
		[width=\linewidth*0.7, xlabel={+U on Zr \emph{d} orbitals (eV)}, ylabel={Lattice parameter (\r{A})}, ymin=4.9, ymax=6.3, xmin=0, xmax=12, legend style={{draw=}, at={(0.18,0.95)}, anchor=north east, legend columns=1}]
		\addplot[no marks] table [x=plusU, y=a,]{dat/plus_u_mono.dat}; \addlegendentry{$a$};
        \addplot[no marks, dashed] table [x=plusU, y=b, ]{dat/plus_u_mono.dat}; \addlegendentry{$b$};
        \addplot[no marks, densely dotted, black] table [x=plusU, y=c,]{dat/plus_u_mono.dat}; \addlegendentry{$c$};
			\end{axis}
		\end{tikzpicture}
		\caption{Individual lattice parameters as a function of +U term in monoclinic \zirconia .}
		\label{Figure:plusumono}
	\end{center}
\end{figure}

\subsubsection{Tetragonal}

In tetragonal \zirconia , increasing the +U term (Figure \ref{Figure:plusutet}) has a strong anisotropic effect on the lattice parameters. Unusually, the $c$ parameter falls (up to an +U energy of 6 eV) while the $a$ parameter increases. Typically it would be expected that both parameters would increase, perhaps at different rates, because the +U term increases the total energy (by increasing the Coulombic contribution) in the system. This increase in energy leads to higher stresses and therefore larger interatomic spacings (cells are permitted to relax in these calculations).

Systems will always tend towards the lowest energy configuration. Therefore the reduction of the $c$ parameter suggests that it is already in a high energy configuration in the $c$ direction (overextended) and can reduce its energy by being compressed in that direction (i.e. becoming more cubic). This is consistent with the interpretation that lower temperature phases of \zirconia\ are distortions of the cubic fluorite phase caused by a small cation radius. 

Above a +U parameter of 6 eV however, the $c$ parameter suddenly begins to increase. Upon further inspection of the resulting cell, it was found that it had transitioned completely to cubic fluorite from tetragonal. This can also be confirmed by observing that the relationship between the parameters becomes $2a^2 = c^2$ (i.e. the $c$ parameter is the same length as the unit cell's [110] diagonal, see Figure \ref{figure:tetvscubic}).

\begin{figure}[ht] % +U tet
\begin{center}
\begin{tikzpicture}
	\begin{axis}
		[width=\linewidth*0.7, xlabel={+U on Zr \emph{d} orbitals (eV)}, ylabel={$a$ parameter (\r{A})}, ymin=3.6, ymax=3.8, xmin=0, xmax=12, legend style={{draw=}, at={(0.18,0.95)}, anchor=north east, legend columns=1}, tick pos=left]
		\addplot[no marks] table [x=plusU, y=a,]{dat/plus_u_tet.dat}; \addlegendentry{$a$};
        %\addplot[no marks, dashed] table [x=plusU, y=b, ]{dat/plus_u_tet.dat}; \addlegendentry{b};
        \addplot[no marks, dashed, black] table [x=plusU, y=c,]{dat/plus_u_tet.dat}; \addlegendentry{$c$};
			\end{axis}
            \begin{axis}[width=\linewidth*0.7,
     xmin = 0, xmax = 12,
     ymin = 5.16, ymax = 5.32,
     hide x axis,
     hide y axis, tick pos=right]
     \addplot[no marks, dashed, black] table [x=plusU, y=c,]{dat/plus_u_tet.dat};
   			\end{axis}
            \pgfplotsset{every axis y label/.append style={rotate=180}}
   \begin{axis}[width=\linewidth*0.7,
         xmin=0, xmax=12,
         ymin=5.16, ymax=5.32,
         hide x axis,
         axis y line*=right,
         ylabel={$c$ parameter (\r{A})}
     ]
   \end{axis}
		\end{tikzpicture}
		\caption{Individual lattice parameters as a function of +U term in tetragonal \zirconia .}
		\label{Figure:plusutet}
	\end{center}
\end{figure}

\subsubsection{Cubic}

The effect of a +U term on a lattice parameter of cubic \zirconia\ is shown in Figure \ref{Figure:plusucubic}. Notably, the symmetry of the cell remains intact even up to a +U term of 12 eV, unlike in the monoclinic and tetragonal phases. The lattice parameter also increases superlinearly as the +U term is increased. This is the typical response that is expected when a +U term is introduced. 

\begin{figure}[ht] % +U cubic
\begin{center}
\begin{tikzpicture}
	\begin{axis}
		[width=\linewidth*0.7, xlabel={+U on Zr \emph{d} orbitals (eV)}, ylabel={Lattice parameter (\r{A})}, ymin=5.1, ymax=5.35, xmin=0, xmax=12, legend style={{draw=}, at={(0.18,0.95)}, anchor=north east, legend columns=1}]
		\addplot[no marks] table [x=plusU, y=a,]{dat/plus_u_cubic.dat}; \addlegendentry{$a$};
        %\addplot[no marks, dashed] table [x=plusU, y=b, ]{dat/plus_u_cubic.dat}; \addlegendentry{b};
        %\addplot[no marks, densely dotted, black] table [x=plusU, y=c,]{dat/plus_u_cubic.dat}; \addlegendentry{c};
			\end{axis}
		\end{tikzpicture}
		\caption{Lattice parameter as a function of +U term in cubic \zirconia .}
		\label{Figure:plusucubic}
	\end{center}
\end{figure}

There was however one instance of unexpected behaviour. Knowing that the tetragonal phase collapses to cubic at +U terms greater than 6 eV, we would expect both to exhibit the same band gap at this +U value. Looking at the band gap results from Figure \ref{Figure:plusubandgap}, we see that the band gaps of tetragonal and cubic \zirconia\ continue to be different above 6 eV, despite the crystal structures being the same in this region. One difference between the cells is that the cubic unit cell has 12 atoms while the tetragonal unit cell has 6 atoms. While this suggests a size effect, the band gap difference did not appear when comparing the band gaps of unit cells and supercells in the absence of a +U term. 

After considering the impact of a +U term in DFT calculations, the decision was made not to include the term. While it would provide a small improvement in the calculated band gap for the cubic phase, the effect on cell symmetry of the other phases would present a confounding variable, especially when placing defects into the structure. It is therefore more useful to run calculations without a +U term to maintain consistency of results between different phases in this thesis. % Good
%\chapter{Defects}

\label{ch:defects}

\section{Fission product empirical potential}
 % Good
\chapter{Structure properties and intrinsic defects} \label{ch:results1}  

\label{ch:defects}

\section{Introduction}  

It is important to fully understand the behaviour of intrinsic defects in \zirconia\ before performing studies with dopant ions. In this chapter, intrinsic defects in monoclinic, tetragonal and cubic \zirconia\ are compared and contrasted, including values for formation energy, defect volumes and defect equilibria. Elastic constants, electronic density of states, band gaps and free energies of the non-defective structures are also reported. % because useful material properties may be exploited to improve performance, e.g. by doping with other ions to stabilise one crystal structure. For example, \zirconia\ doped with enough yttrium cations will stabilise the cubic phase and increase the concentration of oxygen vacancies. This would then affect the behaviour of dopant atoms also present in the lattice.

\subsection{Previous work} 

Previous works studying intrinsic defects in the \zirconia\ system have utilised quantum mechanical methods to determine defect formation energies in the monoclinic phase \cite{zheng2007first,foster2002modelling,foster2001structure} and defect equilibria in the tetragonal phase \cite{youssef2012intrinsic}. The cubic phase is mainly studied as a dopant-stabilised system \cite{orera1990intrinsic,jiang2011first}, with few undoped defect studies in the literature \cite{mackrodt1986theoretical,aarhammar2009energetics}. Building upon previous quantum mechanical studies, a comprehensive account of intrinsic defect energies, defect volumes, and defect equilibria for all three common crystal structures of \zirconia\ is provided, using state-of-the-art, accessible methods.

\section{Methodology}
\subsection{Simulation parameters}

As discussed in Chapter \ref{ch:compmethodology}, DFT calculations were performed using CASTEP 8.0 \cite{Clark2005}. Ultra-soft pseudo-potentials were used throughout, employing a 600 eV cut-off energy. The Perdew, Burke and Ernzerhof (PBE) \cite{Perdew1996} parameterisation of the generalised gradient approximation (GGA) was used to describe the exchange correlation functional. A Monkhorst-Pack sampling scheme \cite{Monkhorst1976} was used for Brillouin zone integration, with a minimum \emph{k}-point separation of 0.09 \r{A}\textsuperscript{-1}. The Pulay method for density mixing \cite{Pulay1980} was used to improve convergence of simulations. 

The electrical energy convergence criterion was set to $1\times10^{-6} $ eV. The maximum force between atoms was limited to $1\times10^{-2}$ eV \r{A}\textsuperscript{-1}. A gradient-descent geometry optimisation task was run on the cell until consecutive iterations differed in energy and atomic displacement by less than $1\times10^{-5}$ eV and $5\times10^{-4}$ \r{A}, respectively. 


\subsection{Helmholtz free energy}

To determine the temperature dependence of the ground states for the pure crystal structures, a harmonic approximation method as described by Burr et al. was used \cite{burr2015crystal,jackson2016resolving}. A constant-volume phonon calculation was performed for each structure, from which the vibrational enthalpy $H_{vib}(T, V)$ and entropy $S_{vib}(T, V)$ contributions to the Helmholtz free energy were calculated up to a temperature of 2500 K. The complete Helmholtz free energy $F(T, V)$ was then obtained by including the internal energy $U(V)$ and configurational entropy $S_{conf}$ of the system:
\begin{equation} \label{helmholtz_equation}
F(T, V) = U(V) + H_{vib}(T, V) - TS_{vib}(T, V) - TS_{conf}
\end{equation}

\subsection{Brouwer diagrams}

Using the method outlined in § \ref{brouwer_method}, Brouwer diagrams of the intrinsic defect equilibria for monoclinic, tetragonal and cubic \zirconia\ were generated. The monoclinic and tetragonal Brouwer diagrams were generated at temperatures of 650 K and 1500 K respectively, corresponding to temperatures at which these phases are thermally stabilised. Defect concentrations are reported in parts/fu (i.e. parts per formula unit \zirconia). The cubic Brouwer diagram was generated at 2000K despite being thermally stabilised at temperatures greater than 2400 K. This was because such high temperatures resulted in very large intrinsic defect concentrations such that defect behaviour could not be meaningfully examined (the system was completely defective). As discussed in § \ref{dis_form_energy_intrinsic} and § \ref{brouwer_discussion_intrinsic}, issues with modelling cubic \zirconia\ in DFT mean that calculated energy values may become unreliable, but are presented in this thesis for the purpose of completeness.

%\section{Cubic phase collapse}
%
%\begin{itemize}
%\item When some oxygen Frenkel defects were introduced to the cubic phase supercell, relaxation under constant volume conditions caused a collapse into a pseudo-tetragonal structure.
%\item This indicated that the cubic phase as modelled in DFT may not be fully stable.
%\item Further investigation indicated that the structure of a supercell of c-\zirconia\ broke down even with constrained symmetry, a result corroborated by Burr {et al}. \cite{burr2017importance}. 
%\end{itemize}

\subsection{Unit cells}

Having selected and optimised the parameters and functionals, unit cells of \zirconia\ in each phase were fully relaxed at constant pressure and the resulting structures were compared in detail to experimental data. Table \ref{lattice_params} shows the calculated lattice parameters and energy differences between the three \zirconia\ phases. 

The first thing to note is that the correct order of \zirconia\ phases is predicted in the total energy calculations, with monoclinic being the lowest energy phase and cubic being the highest. In addition, the energy difference between phases is small ($<$ 0.1 eV/fu). This is a good indication that the choice of exchange correlation functional can reproduce the energy landscape of the system accurately. This is especially important for when defects are introduced because they may promote stabilisation of one phase over another, and an inaccurate model will not capture this behaviour. In all cases the predicted cell volumes are consistently within approximately 2\% of experimental values. 

\begin{table}[ht] % Unit cell parameters
\onehalfspacing
\centering
\caption[Calculated unit cell parameters for the different crystal structures of \zirconia . Experimental data for pure monoclinic, yttria-stabilised tetragonal and magnesia-stabilised cubic phases at 295 K are shown in parentheses. Energy difference between structures is shown with respect to the cubic phase.]{Calculated unit cell parameters for the different crystal structures of \zirconia . Experimental data for monoclinic, tetragonal and cubic phases at 295 K are shown in parentheses \cite{Howard1988}. Energy difference between structures is shown with respect to the cubic phase.}
\label{lattice_params}
\resizebox{\textwidth}{!}{%
\begin{tabular}{ccccccc}
\hline Phase    & a (\AA) & b (\AA) & c (\AA) & $\beta$ ($\degree$) & Volume (\AA\textsuperscript{3}/fu) & $\Delta$E (eV/fu) \\ \hline
m-\zirconia   &    5.18 (5.15)          &    5.24 (5.21)         &    5.37 (5.32)         & 99.63 (99.23)             &       35.96 (35.22)                 &    -0.215              \\
t-\zirconia &    3.62 (3.61)         &              &    5.28  (5.18)        & 90             &   34.60 (33.75)                      &     -0.105             \\
c-\zirconia        &   5.11 (5.09)           &              &              & 90             &     33.36 (32.97)                   &      N/A     \\ \hline      
\end{tabular}}
\end{table}

\subsection{Electronic density of states} 

The electronic density of states for monoclinic, tetragonal and cubic \zirconia\ are generated in a two-step process. First, the non-defective structures are fully relaxed using the geometry optimisation task in CASTEP. This task will also calculate electronic eigenvalues for all k-points and save them to a \texttt{.bands} file. Second, the electronic band data is parsed from the \texttt{.bands} file using the OptaDOS code \cite{Nicholls2012, Morris2014} and the density of states is output to a text file. Further details on using OptaDOS to view the electronic density of states are given in Appendix \ref{castep_scripts}.

The electronic density of states for the three \zirconia\ phase are given in Figure \ref{figure:densityofstates}. In this figure, the valence and conduction bands can clearly be seen at 2-8 eV and 10-15 eV respectively. Most importantly, the energy values of the valence band maximum (VBM) and the conduction band minimum (CBM) for each phase can be obtained from this figure. These values are used to calculate the band gap in the different phases, shown in Figure \ref{table:bandgap} alongside experimental values. The VBM value is also used in the calculation of defect formation energies when electrons are added or removed from a system.

\begin{figure}[ht]
\begin{center}
\begin{tikzpicture}
	\begin{axis}
		[width=\linewidth*0.7, xlabel={Energy (eV)}, ylabel={Electronic density of states}, ymin=0, ymax=12, xmin=0, xmax=16, legend style={{draw=}, at={(0.05,0.95)}, anchor=north west, legend columns=1}]
       \addplot[no marks] table [x=mono_x, y=mono_y,]{dat/eDOS.dat}; \addlegendentry{Monoclinic};
       \addplot[no marks, dashed] table [x=tet_x, y=tet_y,]{dat/eDOS.dat}; \addlegendentry{Tetragonal};
       \addplot[no marks, densely dotted] table [x=cubic_x, y=cubic_y,]{dat/eDOS.dat}; \addlegendentry{Cubic};
			\end{axis}
		\end{tikzpicture}
		\caption{Electronic density of states for the different crystal structures of \zirconia\ showing the band gap predicted by DFT.}
		\label{figure:densityofstates}
	\end{center}
\end{figure}

The electronic density of states show that the VBM and CBM energies increase from monoclinic to tetragonal to cubic \zirconia . This means that at the same Fermi level, the total electronic energy will be smallest in the monoclinic phase and largest in the cubic phase. This corresponds to the correct order of thermal stability that is seen in experiments. Other features that can be seen are the band gaps of the different phases between 7 and 12 eV. These band gaps are significantly underestimated for each phase (see Table \ref{table:bandgap}), as is typical when using a GGA exchange-correlation functional.

\begin{table}[ht] % Band Gap
\onehalfspacing
\centering
\caption[Experimentally determined band gaps alongside values calculated from DFT simulations for each crystal structure of zirconia.]{Experimentally determined band gaps alongside values calculated from DFT simulations for each crystal structure of zirconia. Experimental values taken from \cite{French1994}.}
\begin{tabular}{ccc}
{\bf }                                       & \multicolumn{2}{c}{{\bf Band gap (eV)}}      \\ \hline
\multicolumn{1}{c}{{\bf Crystal Structure}} & \multicolumn{1}{c}{{\bf Expt.}} & {\bf DFT} \\ \hline
\multicolumn{1}{c}{Monoclinic}              & \multicolumn{1}{c}{5.83}        & 3.45      \\
\multicolumn{1}{c}{Tetragonal}              & \multicolumn{1}{c}{5.78}        & 4.00      \\
\multicolumn{1}{c}{Cubic}                   & \multicolumn{1}{c}{6.10}         &   3.55 \\ \hline
\label{table:bandgap}
\end{tabular}
\end{table}


\section{Frenkel and Schottky defects}

\subsection{Incorporation and defect formation energies}

\subsubsection*{Isolated Frenkel defects}

Zr and O Frenkel pair defect formation energies were determined via point defect DFT calculations for the three structures. The formation energies of the isolated Frenkel defect pairs were defined as: % Interstitial iodine defects were simulated in the neutral charge state at different interstitial sites in each phase. The incorporation energy of these defects, assuming a perfect lattice, was calculated using Equation \ref{equation_incorporation}:
\begin{equation}
\label{equation_frenkel}
E_{Frenkel} = E_{DFT}(V^{q}_{X}) + E_{DFT}(X^{-q}_{i}) - 2E_{DFT}(ZrO_2)% - \frac{E_{I_2}}{2}
\end{equation}

where $X$ is either Zr or O, $E_{DFT}(V^{q}_{X})$ is the energy of a supercell of \zirconia\ containing a single vacancy of charge $q$, $E_{DFT}(X^{-q}_{i})$ is the energy of a supercell of \zirconia\ containing a single interstitial with opposing charge $-q$, and $E_{DFT}(ZrO_2)$ is the energy of the non-defective supercell. Charges ranged from the fully charged case (+2 for oxygen vacancies, -4 for zirconium vacancies) to neutral. The interstitial sites, shown in Table \ref{table:interstitials}, were chosen based on standard vacant Wyckoff positions in each crystal structure \cite{theo1996international}.  In the case of oxygen vacancies in monoclinic \zirconia , a defect energy was obtained for both the (III) and (IV) co-ordinated oxygen sites, with the lowest energy value being used in the calculation of the Frenkel defect energy.

\begin{table}[ht] % Wyckoff positions of interstitials
\onehalfspacing
\centering
\caption{Wyckoff positions of interstitial sites used for each crystal structure.}
\label{table:interstitials}
\begin{tabular}{lcc}
\hline
\hspace{0.7 cm} {\bf Crystal Structure} \hspace{0.7 cm}                              & \hspace{0.7 cm} {\bf Interstitial Sites} \hspace{0.7 cm}                                               \\ \hline
\multicolumn{1}{c}{\textbf{Monoclinic}}              & $2a$, $2b$, $2c$, $2d$ \\
\multicolumn{1}{c}{\textbf{Tetragonal}}            & $2b$, $8e$                                   \\
\multicolumn{1}{c}{\textbf{Cubic}}       & $24d$, $4b$                                          \\ \hline
\end{tabular}
\end{table}

The isolated defect formation energies reported in Table \ref{isolated_defects} indicate that fully-charged Schottky defects have the lowest formation energy per atom (most energetically favourable) in all phases, followed by oxygen Frenkel defects and then zirconium Frenkel defects. A trend is seen where the high-temperature phases result in lower formation energies for both Schottky and oxygen Frenkel defects, whereas zirconium Frenkel defects have similar formation energies in all three phases. 

It has been suggested that the relatively small cation size leads to defect structures where oxygen vacancies are favoured over interstitial defects \cite{dwivedi1990computer}. As the zirconium ion is too small to maintain a strong 8-fold bond coordination with its neighbouring oxygen ions, the introduction of oxygen vacancies (which have the added effect of reducing cell volume) will have a stabilising effect.

\begin{table}[ht] % Isolated formation energies
\onehalfspacing
\centering
\caption{Formation energies in eV of isolated \zirconia\ defects.}
\label{isolated_defects}
\begin{tabular}{cccll}
\hline
\multirow{2}{*}{\textbf{Defect}}                      & \multirow{2}{*}{\textbf{Equation}}                                        & \multicolumn{3}{c}{\textbf{Formation Energy (eV)}} \\ \cline{3-5}
	&	& \multicolumn{1}{l}{Monoclinic} & Tetragonal & Cubic \\ \hline
\multirow{5}{*}{\textbf{Zr Frenkel}} & \ch{Zr_{Zr}^{x}} $\rightarrow$ \ch{V_{Zr}^{''''}} + \ch{Zr_{i}^{****}}              & 5.428 & 5.639 & 5.610                             \\
                                     & \ch{Zr_{Zr}^{x}} $\rightarrow$ \ch{V_{Zr}^{'''}} + \ch{Zr_{i}^{***}}               & 8.695 & 8.939 & 8.476                            \\
                                     & \ch{Zr_{Zr}^{x}} $\rightarrow$ \ch{V_{Zr}^{''}} + \ch{Zr_{i}^{**}}                & 12.118 & 12.058 & 11.628                             \\
                                     & \ch{Zr_{Zr}^{x}} $\rightarrow$ \ch{V_{Zr}^{'}} + \ch{Zr_{i}^{*}}                & 16.021 &	15.696 &	13.319                             \\
                                     & \ch{Zr_{Zr}^{x}} $\rightarrow$ \ch{V_{Zr}^{x}} + \ch{Zr_{i}^{x}}                  & 20.563	& 20.094 &	18.170                            \\ \hline
\multirow{3}{*}{\textbf{O Frenkel}}  & \ch{O_{O}^{x}} $\rightarrow$ \ch{V_{O}^{**}} + \ch{O_{i}^{''}}                   & 4.457 &	4.000 & 	3.728                             \\
                                     & \ch{O_{O}^{x}} $\rightarrow$ \ch{V_{O}^{*}} + \ch{O_{i}^{'}}                   & 6.432	& 6.588 &	7.055                             \\
                                     & \ch{O_{O}^{x}} $\rightarrow$ \ch{V_{O}^{x}} + \ch{O_{i}^{x}}                     & 7.518 &	7.452 &	8.477                             \\ \hline
\multirow{3}{*}{\textbf{Schottky}}   & $\varnothing$ $\rightarrow$ \ch{V_{Zr}^{''''}} + 2\ch{V_{O}^{**}} & 5.120 &	3.778	& 1.752                             \\
                                     & $\varnothing$ $\rightarrow$ \ch{V_{Zr}^{''}} + 2\ch{V_{O}^{*}} & 11.353 &	10.832 &	9.624                             \\
                                     & $\varnothing$ $\rightarrow$ \ch{V_{Zr}^{x}} + 2\ch{V_{O}^{x}}   & 18.554 &	18.232 &	17.073  \\ \hline                          
\end{tabular}
\end{table}

\subsubsection*{Bound Frenkel Defects}

Bound Zr and O Frenkel defect formation energies were calculated from DFT energies of supercells where a single ion was moved from its lattice site to an interstitial site. The formation energies of the bound Frenkel defect pairs were defined as:
% Interstitial iodine defects were simulated in the neutral charge state at different interstitial sites in each phase. The incorporation energy of these defects, assuming a perfect lattice, was calculated using Equation \ref{equation_incorporation}:

\begin{equation}
\label{equation_frenkel_bound}
E_{BoundFrenkel} = E_{DFT}(BoundFrenkel) - E_{DFT}(ZrO_2)% - \frac{E_{I_2}}{2}
\end{equation}

where $E_{DFT}(BoundFrenkel)$ is the energy of a supercell of \zirconia\ containing both a vacancy and interstitial defect of the same ion. The two defects were placed as far apart in the supercell as possible (7-8 \r{A}) to avoid recombination. The interstitial defect is assumed to fully compensate the charge of the vacancy defect, resulting in no overall charge on the supercell. The number and type of ions in the defective and non-defective supercell are the same, requiring no further steps to calculate the formation energy. The formation energies calculated for these defects in each crystal structure are presented in Table \ref{table:bound_defects}.

\begin{table}[ht] % Bound formation energies
%\setlength{\tabcolsep}{10pt} % Default value: 6pt
\onehalfspacing
\centering
\caption{Formation energies of bound defects in \zirconia.}
\label{table:bound_defects}
\begin{tabular}{cccc}
\hline
\multirow{2}{*}{\textbf{Defect}} & \multicolumn{3}{c}{\textbf{Formation Energy (eV)}} \\ \cline{2-4} 
 & \textbf{Monoclinic} & \textbf{Tetragonal} & \textbf{Cubic} \\ \hline
\textbf{O Frenkel} & 4.1212 & 4.0290 & 6.4397 \\
\textbf{Zr Frenkel} & 8.4232 & 7.8633 & 6.3274 \\
\textbf{NTV1} & 5.2272 & 3.5813 & 2.6961 \\
\textbf{NTV2} & 5.1405 & 4.2312 & 0.1798 \\
\textbf{NTV3} & 4.6620 & 3.3623 & 2.4089 \\ \hline
\end{tabular}
\end{table}

%Charges ranged from the fully charged case (+2 for oxygen, -4 for zirconium) to neutral. The interstitial sites, shown in Table \ref{table:interstitials}, were chosen based on standard vacant Wyckoff positions in each crystal structure \cite{theo1996international}.  In the case of oxygen vacancies in monoclinic \zirconia , a defect energy was obtained for both the (III) and (IV) co-ordinated oxygen sites. The lowest energies were used in the calculation of the Frenkel defect energy.

\subsubsection*{Isolated Schottky Defects}

Three Schottky energies were calculated for each structure, corresponding to fully charged, partially charged, and uncharged point defect energies. The Schottky formation energy was defined as:

\begin{equation}
\label{equation_schottky}
E_{Schottky} = E_{DFT}(V^{-2q}_{Zr}) + 2E_{DFT}(V^{q}_{O}) -\frac{3(n-1)}{n}E_{DFT}(ZrO_2)% - \frac{E_{I_2}}{2}
\end{equation}

where $n$ denotes the number of atoms in the supercell, $V^{q}_{O}$ denotes an oxygen vacancy with charge $q$, where $q$ varies from 2 to 0. This form maintains both the mass and charge balance of the Schottky defect description for \zirconia :

\begin{equation}
\label{generic_schottky}
Zr^{x}_{Zr} + 2O^{x}_{O} = V^{-2q}_{Zr} + 2V^{q}_{O} + ZrO_{2}
\end{equation}

This implies a rearrangement rather than complete removal of ions from the system. As with the Frenkel defects, the lowest energy vacancy energies were used to calculate Schottky formation energies. While there are multiple configurations of Schottky defects, such nuance cannot be accurately represented through a sum of individual vacancy defect energies. The values presented for Schottky defect formation energies should therefore be considered the lower bound for defect formation. 


\subsubsection*{Bound Schottky Defects}


\begin{figure}[ht] % Tet Zr centre
\centering
\includegraphics[width=8cm]{images/zr_centre_tet.png}
\caption{Zirconium centre  cell showing nearest oxygen atoms in tetragonal \zirconia. Schottky trios indicated by oxygen enumeration with Zr, O and a second oxygen in either the 1\textsuperscript{st}, 2\textsuperscript{nd} or 3\textsuperscript{rd} nearest neighbour with respect to the initial oxygen. Zirconium atoms are shown in green and oxygen atoms in red.}
\label{figure:tetschottky}
\end{figure}

\begin{figure}[ht] % Cubic Zr centre
\centering
\includegraphics[width=8cm]{images/sd_cubic_zro2.png}
\caption{Zirconium centre cell showing nearest oxygen atoms in cubic \zirconia. Schottky trios indicated by oxygen enumeration with Zr, O and a second oxygen in either the 1\textsuperscript{st}, 2\textsuperscript{nd} or 3\textsuperscript{rd} nearest neighbour with respect to the initial oxygen.. Zirconium atoms are shown in green and oxygen atoms in red.}
\label{figure:cubicschottky}
\end{figure}

Bound Schottky defects were modelled in a supercell of \zirconia\ by removing one Zr and two O atoms, in one of several possible nearest neighbour configurations as shown in Figures \ref{figure:monoschottky}, \ref{figure:tetschottky} and \ref{figure:cubicschottky}. Charge neutrality is maintained by the removal of a stoichiometric unit, therefore these defects were defined as neutral tri-vacancies (NTVs). The NTV formation energy was defined as:
\begin{equation}
\label{equation_NTV}
E_{NTV} = E_{DFT}(NTV) - \frac{n-3}{n}E_{DFT}(ZrO_2)% - \frac{E_{I_2}}{2}
\end{equation}

Where $E_{DFT}(NTV)$ is the energy of a supercell containing the NTV defect. As the defective supercell contains three fewer ions than the non-defective cell, the energy of the non-defective cell was adjusted by a proportional factor in our calculation. This form maintains both mass and charge balance of the Schottky defect description for \zirconia\ described in Equation \ref{generic_schottky}.




\section{Defect formation energies} \label{dis_form_energy_intrinsic}

\begin{figure}[ht] % Mono vacancies Fermi level
\begin{center}
\begin{tikzpicture}
	\begin{axis}
		[width=11cm, xlabel={Fermi level $\mu_{e}$ (eV)}, ylabel={Formation energy (eV) per \zirconia\ }, ymin=-10, ymax=18, xmin=0, xmax=6, legend style={{draw=}, at={(0.95,0.95)}, anchor=north east, legend columns=1}]
		\addplot[no marks, blue] table [x=ZRmono1, y=ZRmono2,]{dat/vacancies.dat}; \addlegendentry{Zr};
        \addplot[no marks, red, dashed] table [x=O3mono1, y=O3mono2,]{dat/vacancies.dat}; \addlegendentry{O (III)};
        \addplot[no marks, red] table [x=O4mono1, y=O4mono2,]{dat/vacancies.dat}; \addlegendentry{O (IV)};
			\end{axis}
		\end{tikzpicture}
		\caption{Monoclinic phase formation energies of intrinsic vacancy defects as a function of Fermi level. Gradient indicates defect charge. Oxygen coordination number shown in legend.}
		\label{figure:monovacancies}
	\end{center}
\end{figure}


\begin{figure}[ht] % Tet vacancies Fermi level
\begin{center}
\begin{tikzpicture}
	\begin{axis}
		[width=11cm, xlabel={Fermi level $\mu_{e}$ (eV)}, ylabel={Formation energy (eV) per \zirconia\ }, ymin=-10, ymax=18, xmin=0, xmax=6, legend style={{draw=}, at={(0.95,0.95)}, anchor=north east, legend columns=1}]
		\addplot[no marks, blue] table [x=ZRtet1, y=ZRtet2,]{dat/vacancies.dat}; \addlegendentry{Zr};
        \addplot[no marks, red] table [x=Otet1, y=Otet2,]{dat/vacancies.dat}; \addlegendentry{O};
			\end{axis}
		\end{tikzpicture}
		\caption{Tetragonal phase formation energies of intrinsic vacancy defects as a function of Fermi level. Gradient indicates defect charge.}
		\label{figure:tetvacancies}
	\end{center}
\end{figure}

\begin{figure}[ht]
\begin{center}
\begin{tikzpicture}
	\begin{axis}
		[width=11cm, xlabel={Fermi level $\mu_{e}$ (eV)}, ylabel={Formation energy (eV) per \zirconia\ }, ymin=-10, ymax=18, xmin=0, xmax=6, legend style={{draw=}, at={(0.95,0.95)}, anchor=north east, legend columns=1}]
		\addplot[no marks, blue] table [x=ZRcubic1, y=ZRcubic2,]{dat/vacancies.dat}; \addlegendentry{Zr};
        \addplot[no marks, red] table [x=Ocubic1, y=Ocubic2,]{dat/vacancies.dat}; \addlegendentry{O};
			\end{axis}
		\end{tikzpicture}
		\caption{Cubic phase formation energies of intrinsic vacancy defects as a function of Fermi level. Gradient indicates defect charge.}
		\label{figure:cubicvacancies}
	\end{center}
\end{figure}


\subsection{Defect Volumes}

Tables \ref{defect_volumes_raw} and \ref{defect_volumes_clusters_isolated} show the calculated point defect and cluster defect volumes respectively. The Frenkel and Schottky defect volumes are calculated from the sum of the relevant point defects that would result in an overall neutral charge, with clusters of fully-charged point defects being the expected defect structures in a real material.

\begin{table}[ht!] % Isolated defect volumes
\onehalfspacing
\centering
\caption{Isolated defect volumes in the three \zirconia\ structures.}
\label{defect_volumes_raw}
\begin{tabular}{cccc}
\hline
                      & \multicolumn{3}{c}{\textbf{Defect volume relative to non-defective cell (\r{A}\textsuperscript{3})}}  \\ \cline{2-4} 
\textbf{Defect}       & \textbf{Monoclinic} & \hspace{1cm} \textbf{Tetragonal} & \textbf{Cubic} \\ \hline
\ch{V_{Zr}^{''''}}             & 55.95             & 67.41             & 48.47         \\
\ch{V_{Zr}^{'''}}             & 42.48             &         51.08     &     36.94      \\
\ch{V_{Zr}^{''}}            & 29.90             &  34.28            &     25.93           \\
\ch{V_{Zr}^{'}}             & 17.10             &  18.43            &     15.08           \\
\ch{V_{Zr}^{x}}              & 4.06             &  4.70            &    4.32       \\
\ch{Zr_{i}^{****}}             & -34.62            & -41.94            & -27.34       \\
\ch{Zr_{i}^{***}}             &  -22.76           &	-27.74 		  &	-16.95         \\
\ch{Zr_{i}^{**}}             &  -11.79 	        &	-12.02 		  &	-6.24          \\
\ch{Zr_{i}^{*}}            &  2.68			& -0.02 		  & 	4.69             \\
\ch{Zr_{i}^{x}}              &  15.94		 	& 13.40	 		  & 15.97         \\
\ch{V_{O}^{**}} {[}4coord{]} & -22.52            & -37.53            & -22.76       \\
\ch{V_{O}^{*}} {[}4coord{]} &  -12.41           &    -19.53         &     -12.19           \\
\ch{V_{O}^{x}} {[}4coord{]}  &  -0.69          &  -2.80           &      -1.11          \\
\ch{V_{O}^{**}} {[}3coord{]} & -26.13            &                     &                \\
\ch{V_{O}^{*}} {[}3coord{]} &  -14.42           &                     &                \\
\ch{V_{O}^{x}} {[}3coord{]}  &   -1.71          &                     &                \\
\ch{O_{i}^{''}}              & 27.01             & 40.00              & 28.58        \\
\ch{O_{i}^{'}}              &  15.36            &    24.56         &  16.30              \\
\ch{O_{i}^{x}}               & 2.66             &    11.06          &   8.95        \\ \hline
\end{tabular}
\end{table}

\begin{table}[ht] % Isolated Frenkel volumes
\onehalfspacing
\centering
\caption{Isolated cluster defect volumes in the three \zirconia\ structures.}
\label{defect_volumes_clusters_isolated}
\begin{tabular}{cccc}
\hline
\multirow{2}{*}{\textbf{Defect}}   & \multicolumn{3}{c}{\textbf{Defect volume (\r{A}\textsuperscript{3})}}  \\ \cline{2-4} 
 & \textbf{Monoclinic} & \textbf{Tetragonal} & \textbf{Cubic} \\ \hline
\ch{V_{Zr}^{''''}} + \ch{Zr_{i}^{****}}          & 21.331	 & 25.4702 &	21.1309         \\
\ch{V_{Zr}^{'''}} + \ch{Zr_{i}^{***}}          & 19.7155 &	23.3463 &	19.9954      \\
\ch{V_{Zr}^{''}} + \ch{Zr_{i}^{**}}          & 18.1149 &	22.2525 &	19.68618           \\
\ch{V_{Zr}^{'}} + \ch{Zr_{i}^{*}}          & 19.78339 &	18.4096913 &	19.76396           \\
\ch{V_{Zr}^{x}} + \ch{Zr_{i}^{x}}          & 19.99485 &	18.1061 &	20.29223       \\
\ch{V_{O}^{**}} + \ch{O_{i}^{''}}           & 0.8839 &	2.4704 &	5.8217       \\
\ch{V_{O}^{*}} + \ch{O_{i}^{'}}           &  0.9486 &	5.032 &	4.1146        \\
\ch{V_{O}^{x}} + \ch{O_{i}^{x}}           &  0.9576 &	8.26065 &	7.83687          \\
\ch{V_{Zr}^{''''}} + 2\ch{V_{O}^{**}}       &  3.6979 &	-7.647 &	2.9448             \\
\ch{V_{Zr}^{''}} + 2\ch{V_{O}^{*}}       &  1.0707 &	-4.7866 &	1.5564         \\
\ch{V_{Zr}^{x}} + 2\ch{V_{O}^{x}}        & 0.64517 &	-0.8985 &	2.08973       \\ \hline
\end{tabular}
\end{table}

The oxygen Frenkel defect has the smallest defect volume in the monoclinic phase, followed by the tetragonal phase. This can be explained by the competition between phase density and matrix stiffness. As the monoclinic phase has the highest specific volume (see Table \ref{lattice_params}), we can argue that the monoclinic phase can best absorb the lattice strains imposed by the defect, despite having a lower stiffness than the cubic phase.

The zirconium Frenkel defect is significantly larger than the oxygen Frenkel, mostly due to the large positive strain contribution from the zirconium vacancy. This can explain the larger defect formation energy of zirconium Frenkel defects.

%\subsubsection*{Isolated Defects}
%
%The isolated defect formation energies reported in Table \ref{isolated_defects} indicate that fully-charged Schottky defects have the lowest formation energy per atom (most energetically favourable) in all phases, followed by oxygen Frenkel defects. A trend is seen where the high-temperature phases result in lower formation energies for both Schottky and oxygen Frenkel defects, whereas zirconium Frenkel defects have similar formation energies in all three phases. It has been suggested that the relatively small cation size leads to defect structures where oxygen vacancies are favoured over interstitial defects \cite{dwivedi1990computer}. As the zirconium ion is too small to maintain a strong 8-fold bond coordination with its neighbouring oxygen ions, the introduction of oxygen vacancies (which have the added effect of reducing cell volume) will have a stabilising effect.


\subsubsection*{Bound Defects}
The bound defect formation energies shown in Table \ref{table:bound_defects} show that NTV defects, on a per defect atom basis, are the most energetically favourable defects, followed by oxygen and zirconium Frenkel defects respectively. The NTV3 exhibited the smallest formation energy in all three crystal structures, with a single exception of the NTV2 in the cubic phase where a much smaller formation energy was observed due to collapse\footnote{Upon inspecting the output cell, it was found that all the oxygen atoms shifted positions along the [001] direction, becoming more like the tetragonal \zirconia\ structure. The cell size was constrained so the lattice parameters could not be changed, so this was not a true tetragonal cell.} of the supercell during geometry optimisation.

\section{Elastic constants and defect relaxation volumes}

Table \ref{stiffness_tensor} shows the calculated elastic constants for the monoclinic, tetragonal, and cubic phases of \zirconia . The cubic phase has the highest stiffness, likely due to the short Zr-O bond lengths in the energy-minimised structure (Figure \ref{figure:zrobonddistance}). It is expected that at high temperatures where the cubic phase is stable, the resulting increase in bond length would cause a reduction in stiffness. 

\begin{table}[ht] % Elastic constants
\onehalfspacing
\centering
\caption{Elastic constants for different phases of \zirconia\ from DFT calculations.}
\label{stiffness_tensor}
\begin{tabular}{cccc}
\hline
\multirow{2}{*}{\textbf{Elastic Component}} & \multicolumn{3}{c}{\textbf{Stiffness (GPa)}}               \\ \cline{2-4} 
                                            & \textbf{Monoclinic} & \textbf{Tetragonal} & \textbf{Cubic} \\ \hline
$C_{11}$                                         & 338.86        & 334.30               & 523.38    \\
$C_{12}$                                         & 151.80        & 207.30               & 92.93     \\
$C_{13}$                                         & 89.37         & 48.93               & 92.93     \\
$C_{22}$                                         & 348.37        & 334.20               & 523.39    \\
$C_{23}$                                         & 143.04        & 48.93               & 92.93    \\
$C_{33}$                                         & 262.17        & 250.50               & 523.38   \\
$C_{44}$                                         & 76.35         & 9.38                & 61.98    \\
$C_{55}$                                         & 71.65         & 9.38                & 61.98   \\
$C_{66}$                                         & 114.19        & 152.60               & 61.99     \\ \hline
\end{tabular}
\end{table}


The monoclinic and tetragonal phases have similar stiffness along the principal axes, but vary significantly under shearing conditions. In particular, the tetragonal phase exhibits much smaller $C_{44}$ and $C_{55}$ components. This may be attributed to the strong directional anisotropy of the tetragonal phase due to the larger $c$ parameter. 





\section{Helmholtz energies}

%The calculated Helmholtz energies plotted in Figure \ref{Figure:helmholtz} show the correct order of stability for the three phases of \zirconia\ at low temperatures (monoclinic $\rightarrow$ tetragonal $\rightarrow$ cubic) . However, as temperature is increased, only a transition from monoclinic to tetragonal is seen. The cubic phase Helmholtz energy does fall below the monoclinic curve, but never below the tetragonal curve, thus predicting no tetragonal to cubic phase transition. 

%It must also be noted that the transition temperatures are not predicted accurately. The tetragonal phase is predicted to have a lower energy than monoclinic at approximately 400 K, while experiments indicate that the transition temperature is above 1400 K. This difference is too large to attribute to a kinetic barrier.

The Helmholtz free energy results (Figure \ref{Figure:helmholtz}) show the correct order of crystal structure stability at low temperature. A transition from the monoclinic to tetragonal crystal structure is seen at 390K, but no further transition is seen from tetragonal to cubic. The low monoclinic-tetragonal transition temperature may be due to both the kinetic barrier \cite{bansal1972martensitic,bansal1974martensitic}, and the inability of the constant volume harmonic model to take into account the effects of thermal expansion. The lack of an observed transition to the cubic phase may indicate an inability to accurately simulate the high-temperature phase using DFT techniques. 

\begin{figure}[ht] % Helmholtz
\begin{center}
\begin{tikzpicture}
	\begin{axis}
		[width=\linewidth*0.7, xlabel={Temperature (K)}, ylabel={Helmholtz free energy (eV)}, ymin=-28, ymax=-10, xmin=0, xmax=3000, legend style={{draw=}, at={(0.95,0.95)}, anchor=north east, legend columns=1}]
		\addplot[no marks] table [x=temperature, y=monoclinic,]{dat/helmholtz.dat}; \addlegendentry{Monoclinic};
        \addplot[no marks, dashed] table [x=temperature, y=tetragonal, ]{dat/helmholtz.dat}; \addlegendentry{Tetragonal};
        \addplot[no marks, densely dotted, black] table [x=temperature, y=cubic,]{dat/helmholtz.dat}; \addlegendentry{Cubic};
			\end{axis}
		\end{tikzpicture}
		\caption{Helmholtz free energy as a function of temperature for the monoclinic, tetragonal, and cubic crystal structures of \zirconia .}
		\label{Figure:helmholtz}
	\end{center}
\end{figure}

\section{Defect equilibria} \label{brouwer_discussion_intrinsic}

\subsubsection*{Monoclinic}

The monoclinic Brouwer diagram (Figure \ref{figure:mono_intrinsic_brouwer}) predicts that at 635 K, few types of defects will be present and at very low (\textless 10 ppb \zirconia ) concentrations. This is typical of defect behaviour in a ceramics at temperatures far below their melting points \cite{kingery1997physical,ball2006computer}. Fully-charged zirconium vacancies, charge-compensated by holes, are the major defect type we expect to observe at $p_{O_{2}}$ \textgreater $10^{-15}$ atm. Below this, only electronic defects compensated by electron hole defects are expected. 

%We briefly see increased concentrations of uncharged oxygen interstitial defects at very high levels of $p_{O_{2}}$.
%The intrinsic defect equilibria are shown in Brouwer diagrams in Figures \ref{figure:mono_intrinsic_brouwer}, \ref{figure:tet_intrinsic_brouwer} and \ref{figure:cubic_intrinsic_brouwer}. The monoclinic phase exhibits the smallest overall concentration of intrinsic defects due to the low temperature (650 K) relative to the other tetragonal (1500 K) and cubic (XXX K) phases.

\begin{figure}[ht] % Mono intrinsic Brouwer
\begin{center}
\begin{tikzpicture}
	\begin{axis}
		[width=\linewidth*0.7, xlabel={\ch{log_{10}}($p_{O_{2}}$) (atm)}, ylabel={\ch{log_{10}}([D]) (per f.u.)}, ymin=-18, ymax=0, xmin=-35, xmax=0, legend style={{draw=}, at={(0.40,0.94)}, anchor=north west, legend columns=4, nodes={scale=1, transform shape}}]
        \addplot[no marks, draw=blue!70!black] table [x=pO2, y=electrons,]{dat/intrinsic_mono.dat}; \addlegendentry{\ch{e^{'}}}; \node at (-4.8,-13) {\ch{e^{'}}};
        \addplot[no marks, draw=red!85!black] table [x=pO2, y=holes,]{dat/intrinsic_mono.dat}; \addlegendentry{\ch{h^{\textperiodcentered}}}; \node at (-4.5,-8) {\ch{h^{\textperiodcentered}}};
        \addplot[no marks, draw=black!70!green] table [x=pO2, y=VO{2},]{dat/intrinsic_mono.dat}; \addlegendentry{\ch{V_{O}^{\textperiodcentered\textperiodcentered}}}; \node at (-33,-16.5) {\ch{V_{O}^{\textperiodcentered\textperiodcentered}}};
%         \addplot[no marks, draw=black!55!green] table [x=pO2, y=VO{1},]{dat/intrinsic_mono.dat}; \addlegendentry{\ch{V_{O}^{*}}};
%         \addplot[no marks, draw=black!30!green] table [x=pO2, y=VO{0},]{dat/intrinsic_mono.dat}; \addlegendentry{\ch{V_{O}^{x}}};
        \addplot[no marks, draw=yellow!85!blue] table [x=pO2, y=VM{-4},]{dat/intrinsic_mono.dat}; \addlegendentry{\ch{V_{Zr}^{''''}}}; \node at (-5,-10.5) {\ch{V_{Zr}^{''''}}};
%         \addplot[no marks, draw=yellow!75!blue] table [x=pO2, y=VM{-3},]{dat/intrinsic_mono.dat}; \addlegendentry{\ch{V_{Zr}^{'''}}};
%         \addplot[no marks, draw=yellow!65!blue] table [x=pO2, y=VM{-2},]{dat/intrinsic_mono.dat}; \addlegendentry{\ch{V_{Zr}^{''}}};
%         \addplot[no marks, draw=yellow!55!blue] table [x=pO2, y=VM{-1},]{dat/intrinsic_mono.dat}; \addlegendentry{\ch{V_{Zr}^{'}}};
%         \addplot[no marks, draw=yellow!45!blue] table [x=pO2, y=VM{0},]{dat/intrinsic_mono.dat}; \addlegendentry{\ch{V_{Zr}^{x}}};
%         \addplot[no marks, draw=red!60!yellow] table [x=pO2, y=Oi{-2},]{dat/intrinsic_mono.dat}; \addlegendentry{\ch{O_{i}^{''}}};
%         \addplot[no marks, draw=red!50!yellow] table [x=pO2, y=Oi{-1},]{dat/intrinsic_mono.dat}; \addlegendentry{\ch{O_{i}^{'}}};
%         \addplot[no marks, draw=red!40!yellow] table [x=pO2, y=Oi{0},]{dat/intrinsic_mono.dat}; \addlegendentry{\ch{O_{i}^{x}}};
%         \addplot[no marks, draw=green!80!pink] table [x=pO2, y=Mi{4},]{dat/intrinsic_mono.dat}; \addlegendentry{\ch{Zr_{i}^{****}}};
%         \addplot[no marks, draw=green!70!pink] table [x=pO2, y=Mi{3},]{dat/intrinsic_mono.dat}; \addlegendentry{\ch{Zr_{i}^{***}}};
%         \addplot[no marks, draw=green!60!pink] table [x=pO2, y=Mi{2},]{dat/intrinsic_mono.dat}; \addlegendentry{\ch{Zr_{i}^{\textbf{**}}}};
%         \addplot[no marks, draw=green!50!pink] table [x=pO2, y=Mi{1},]{dat/intrinsic_mono.dat}; \addlegendentry{\ch{Zr_{i}^{*}}};
%         \addplot[no marks, draw=green!40!pink] table [x=pO2, y=Mi{0},]{dat/intrinsic_mono.dat}; \addlegendentry{\ch{Zr_{i}^{x}}};
%         \addplot[no marks] table [x=pO2, y=Stoich,]{dat/intrinsic_mono.dat}; \addlegendentry{Stoich};
%\node at (-33.7,-0.5) {\textbf{a)}};
			\end{axis}            
\end{tikzpicture}
		\caption{Monoclinic phase Brouwer diagram of intrinsic defects at 650 K.}
		\label{figure:mono_intrinsic_brouwer}
	\end{center}
\end{figure}


\subsubsection*{Tetragonal}

Figure \ref{figure:tet_intrinsic_brouwer} shows a much greater concentration of defects across a wide range of $p_{O_{2}}$, mainly owing to an elevated temperature of 1500 K where the tetragonal crystal structure is fully stabilised. At low levels of $p_{O_{2}}$, electronic defects are again the dominant defect, but are now charge-compensated by the formation of fully-charged oxygen vacancies. A clear neutrality condition is seen at a $p_{O_{2}}$ of $10^{-11}$ atm where $[\ch{V_{O}^{**}}] = 2[\ch{V_{Zr}^{''''}}]$, with higher levels of $p_{O_{2}}$ being dominated by fully-charged zirconium vacancies charge-compensated by the formation of electron hole defects.

\begin{figure}[ht] % Tet intrinsic Brouwer
\begin{center}
\begin{tikzpicture}
	\begin{axis}
		[width=\linewidth*0.7, xlabel={\ch{log_{10}}($p_{O_{2}}$) (atm)}, ylabel={\ch{log_{10}}([D]) (per f.u.)}, ymin=-10, ymax=0, xmin=-35, xmax=0, legend style={{draw=}, at={(0.40,0.97)}, anchor=north west, legend columns=4, nodes={scale=1, transform shape}}]
        \addplot[no marks, draw=blue!70!black] table [x=pO2, y=electrons,]{dat/intrinsic_tet.dat}; \addlegendentry{\ch{e^{'}}}; \node at (-26.0,-2) {\ch{e^{'}}};
        \addplot[no marks, draw=red!85!black] table [x=pO2, y=holes,]{dat/intrinsic_tet.dat}; \addlegendentry{\ch{h^{\textperiodcentered}}}; \node at (-7,-3.6) {\ch{h^{\textperiodcentered}}};
        \addplot[no marks, draw=black!70!green] table [x=pO2, y=VO{2},]{dat/intrinsic_tet.dat}; \addlegendentry{\ch{V_{O}^{\textperiodcentered\textperiodcentered}}}; \node at (-28,-3) {\ch{V_{O}^{\textperiodcentered\textperiodcentered}}};
%         \addplot[no marks, draw=black!55!green] table [x=pO2, y=VO{1},]{dat/intrinsic_tet.dat}; \addlegendentry{\ch{V_{O}^{*}}};
%         \addplot[no marks, draw=black!30!green] table [x=pO2, y=VO{0},]{dat/intrinsic_tet.dat}; \addlegendentry{\ch{V_{O}^{x}}};
        \addplot[no marks, draw=yellow!85!blue] table [x=pO2, y=VM{-4},]{dat/intrinsic_tet.dat}; \addlegendentry{\ch{V_{Zr}^{''''}}}; \node at (-3,-4.5) {\ch{V_{Zr}^{''''}}};
%         \addplot[no marks, draw=yellow!75!blue] table [x=pO2, y=VM{-3},]{dat/intrinsic_tet.dat}; \addlegendentry{\ch{V_{Zr}^{'''}}};
%         \addplot[no marks, draw=yellow!65!blue] table [x=pO2, y=VM{-2},]{dat/intrinsic_tet.dat}; \addlegendentry{\ch{V_{Zr}^{''}}};
%         \addplot[no marks, draw=yellow!55!blue] table [x=pO2, y=VM{-1},]{dat/intrinsic_tet.dat}; \addlegendentry{\ch{V_{Zr}^{'}}};
%         \addplot[no marks, draw=yellow!45!blue] table [x=pO2, y=VM{0},]{dat/intrinsic_tet.dat}; \addlegendentry{\ch{V_{Zr}^{x}}};
%         \addplot[no marks, draw=red!60!yellow] table [x=pO2, y=Oi{-2},]{dat/intrinsic_tet.dat}; \addlegendentry{\ch{O_{i}^{''}}};
%         \addplot[no marks, draw=red!50!yellow] table [x=pO2, y=Oi{-1},]{dat/intrinsic_tet.dat}; \addlegendentry{\ch{O_{i}^{'}}};
%         \addplot[no marks, draw=red!40!yellow] table [x=pO2, y=Oi{0},]{dat/intrinsic_tet.dat}; \addlegendentry{\ch{O_{i}^{x}}};
%         \addplot[no marks, draw=green!80!pink] table [x=pO2, y=Mi{4},]{dat/intrinsic_tet.dat}; \addlegendentry{\ch{Zr_{i}^{****}}};
%         \addplot[no marks, draw=green!70!pink] table [x=pO2, y=Mi{3},]{dat/intrinsic_tet.dat}; \addlegendentry{\ch{Zr_{i}^{***}}};
%         \addplot[no marks, draw=green!60!pink] table [x=pO2, y=Mi{2},]{dat/intrinsic_tet.dat}; \addlegendentry{\ch{Zr_{i}^{\textbf{**}}}};
%         \addplot[no marks, draw=green!50!pink] table [x=pO2, y=Mi{1},]{dat/intrinsic_tet.dat}; \addlegendentry{\ch{Zr_{i}^{*}}};
%         \addplot[no marks, draw=green!40!pink] table [x=pO2, y=Mi{0},]{dat/intrinsic_tet.dat}; \addlegendentry{\ch{Zr_{i}^{x}}};
%         \addplot[no marks] table [x=pO2, y=Stoich,]{dat/intrinsic_tet.dat}; \addlegendentry{Stoich};
%\node at (-33.7,-0.5) {\textbf{a)}};
			\end{axis}            
\end{tikzpicture}
		\caption{Tetragonal phase Brouwer diagrams of intrinsic defects at 1500 K.}
		\label{figure:tet_intrinsic_brouwer}
	\end{center}
\end{figure}

\begin{figure}[ht] % cubic intrinsic Brouwer
\begin{center}
\begin{tikzpicture}
	\begin{axis}
		[width=\linewidth*0.7, xlabel={\ch{log_{10}}($p_{O_{2}}$) (atm)}, ylabel={\ch{log_{10}}([D]) (per f.u.)}, ymin=-10, ymax=0, xmin=-35, xmax=0, legend style={{draw=}, at={(0.60,0.97)}, anchor=north west, legend columns=3, nodes={scale=1, transform shape}}]
        \addplot[no marks, draw=blue!70!black] table [x=pO2, y=electrons,]{dat/intrinsic_cubic.dat}; \addlegendentry{\ch{e^{'}}}; \node at (-17.0,-1) {\ch{e^{'}}};
        \addplot[no marks, draw=red!85!black] table [x=pO2, y=holes,]{dat/intrinsic_cubic.dat}; \addlegendentry{\ch{h^{\textperiodcentered}}}; \node at (-8,-7.5) {\ch{h^{\textperiodcentered}}};
        \addplot[no marks, draw=black!70!green] table [x=pO2, y=VO{2},]{dat/intrinsic_cubic.dat}; \addlegendentry{\ch{V_{O}^{\textperiodcentered\textperiodcentered}}}; \node at (-20,-1.7) {\ch{V_{O}^{\textperiodcentered\textperiodcentered}}};
%         \addplot[no marks, draw=black!55!green] table [x=pO2, y=VO{1},]{dat/intrinsic_cubic.dat}; \addlegendentry{\ch{V_{O}^{*}}};
%         \addplot[no marks, draw=black!30!green] table [x=pO2, y=VO{0},]{dat/intrinsic_cubic.dat}; \addlegendentry{\ch{V_{O}^{x}}};
        \addplot[no marks, draw=yellow!85!blue] table [x=pO2, y=VM{-4},]{dat/intrinsic_cubic.dat}; \addlegendentry{\ch{V_{Zr}^{''''}}}; \node at (-20,-6) {\ch{V_{Zr}^{''''}}};
%         \addplot[no marks, draw=yellow!75!blue] table [x=pO2, y=VM{-3},]{dat/intrinsic_cubic.dat}; \addlegendentry{\ch{V_{Zr}^{'''}}};
%         \addplot[no marks, draw=yellow!65!blue] table [x=pO2, y=VM{-2},]{dat/intrinsic_cubic.dat}; \addlegendentry{\ch{V_{Zr}^{''}}};
%         \addplot[no marks, draw=yellow!55!blue] table [x=pO2, y=VM{-1},]{dat/intrinsic_cubic.dat}; \addlegendentry{\ch{V_{Zr}^{'}}};
%         \addplot[no marks, draw=yellow!45!blue] table [x=pO2, y=VM{0},]{dat/intrinsic_cubic.dat}; \addlegendentry{\ch{V_{Zr}^{x}}};
%         \addplot[no marks, draw=red!60!yellow] table [x=pO2, y=Oi{-2},]{dat/intrinsic_cubic.dat}; \addlegendentry{\ch{O_{i}^{''}}};
%         \addplot[no marks, draw=red!50!yellow] table [x=pO2, y=Oi{-1},]{dat/intrinsic_cubic.dat}; \addlegendentry{\ch{O_{i}^{'}}};
%         \addplot[no marks, draw=red!40!yellow] table [x=pO2, y=Oi{0},]{dat/intrinsic_cubic.dat}; \addlegendentry{\ch{O_{i}^{x}}};
%         \addplot[no marks, draw=green!80!pink] table [x=pO2, y=Mi{4},]{dat/intrinsic_cubic.dat}; \addlegendentry{\ch{Zr_{i}^{****}}};
%         \addplot[no marks, draw=green!70!pink] table [x=pO2, y=Mi{3},]{dat/intrinsic_cubic.dat}; \addlegendentry{\ch{Zr_{i}^{***}}};
%         \addplot[no marks, draw=green!60!pink] table [x=pO2, y=Mi{2},]{dat/intrinsic_cubic.dat}; \addlegendentry{\ch{Zr_{i}^{\textbf{**}}}};
%         \addplot[no marks, draw=green!50!pink] table [x=pO2, y=Mi{1},]{dat/intrinsic_cubic.dat}; \addlegendentry{\ch{Zr_{i}^{*}}};
%         \addplot[no marks, draw=green!40!pink] table [x=pO2, y=Mi{0},]{dat/intrinsic_cubic.dat}; \addlegendentry{\ch{Zr_{i}^{x}}};
%         \addplot[no marks, dashed, draw=red!70!black] table [x=pO2, y=Ii{0},]{dat/intrinsic_cubic.dat}; \addlegendentry{\ch{I_{i}^{x}}};
%         \addplot[no marks] table [x=pO2, y=Stoich,]{dat/intrinsic_cubic.dat}; \addlegendentry{Stoich};
%\node at (-33.7,-0.5) {\textbf{a)}};
			\end{axis}            
\end{tikzpicture} 
		\caption{Cubic phase Brouwer diagrams of intrinsic defects at 2000 K.}
		\label{figure:cubic_intrinsic_brouwer}
	\end{center}
\end{figure}

\section{Summary}

The main defects in \zirconia\ are oxygen vacancies and zirconium vacancies at low and high oxygen pressures respectively. 

The cubic phase cannot be modelled accurately due to instabilities outlined by Burr \emph{et al}. \cite{burr2017importance}. For this reason, it was decided that the cubic phase would not be considered when conducting extrinsic dopant simulation studies in \zirconia .
 % Good
\chapter{Iodine defect energies and equilibria in \zirconia}

\emph{The work in this chapter has been published in:} \\ A. Kenich \emph{et al.} J. Nucl. Mater. \textbf{511} (2018) 390-395. Ref \cite{kenichiodine2018}.

\label{ch:results2}

\section{Introduction}

As discussed in Chapter \ref{introduction}, stress-corrosion cracking (SCC) in nuclear fuel pins is an issue related to early integrity of fuel assemblies in light water reactors (LWRs). SCC studies of the internal surface of zirconium-based fuel claddings have been conducted, which indicate that iodine is likely to be one of the main corrosive species involved in promoting crack growth \cite{rosenbaum1966interaction, bcoxpelletclad1990,fregonese2000failure,Sidky1998}. The exact mechanism for iodine SCC has not yet been determined due to difficulties observing the internal cladding surface in-situ, while experimental studies are not yet capable of reproducing the conditions under which such failures occur. The quantum-mechanical simulation approach is therefore particularly useful to model the behaviour of iodine within the oxide layer of the cladding, the layer preceding the zirconium metal. 

Iodine is produced in the fuel pellet directly from fission (see Chapter I for details) and also from the decay of tellurium precursors. As shown in Figure \ref{table:decaydata_chap1}, both iodine and tellurium are relatively common fission products, with combined independent yields from thermal fission of U$_{235}$ above 26\% \cite{kennett1956mass, iodine129fissionyield, imanishi1976independent, iodinefissionyields, iodine132, amiel1975odd, iaeafissionyield}. The majority of thermal fission events occur in the outer rim of the fuel pellet, and a fission product penetration depth of up to 8 $\mu$m in \zirconia\ \cite{degueldre2001behaviour} suggests a large degree of implantation within the oxide and the Zr metal into which the oxide grows, raising the concentration of I well above the equilibrium value. Iodine and many of its relevant compounds (as ZrI$_{4}$, CsI) are volatile and fuel pellets contain many cracks and spaces through which iodine may be rapidly transported to the cladding. When reactor power is increased during start-up, iodine is released in substantial quantities from the UO$_{2}$ pellet \cite{peehs1982experimental}. This is believed to cause crack propagation in the cladding when combined with stresses imposed on the cladding by the fuel pellet, and this contributes to pellet-cladding interaction (PCI), a phenomenon discussed in Chapter \ref{introduction}. Upper limits on power ramping and holding times have therefore been established by fuel suppliers to mitigate potential PCI failures \cite{yagnik2005effect}. While these restrictions have reduced or prevented the incidence of PCI failures, they also impose costs on the operator due to longer ramping periods. This also restricts the ability of the nuclear reactor to load-follow grid demand. Cladding/fuel materials resistant to PCI failure are therefore of great interest in the nuclear power industry, promoting research into solutions such as cladding liners and doped fuel pellets \cite{nonon2005pci,yang2012effect}. 

%\item Power ramping: Increasing power, such as during reactor start-up, can lead to cladding failure.   %power must be ramped up gradually in order to avoid excessive temperature gradients in the fuel pins, but also to manage fission product concentrations. Due to the different half-lives of various fission products, a power ramp will cause a transient increase in the iodine concentration within a fuel pin


Iodine is an oxidising agent, which, under standard conditions, will oxidise Zr metal to produce ZrI$_{4}$. However, oxygen is also present in the internal fuel pin environment, both from the native \zirconia\ layer on the cladding, and the evolution of oxygen from the fuel pellet during burnup. Liberated oxygen will compete with iodine in the oxidation of the Zr metal, but whereas iodine promotes crack growth under stress, oxygen provides a more protective effect, self-limiting its diffusion into the metal \cite{farina2002stress, causey2005review}. Furthermore, oxygen is a more powerful oxidising agent than iodine, reacting together to produce I${_2}$O$_{5}$. For these reasons, the internal oxide layer of the cladding is often considered a barrier to the ingress of iodine into the Zr metal. 

Unlike oxygen and hydrogen, which readily diffuse into Zr metal to occupy interstitial sites, iodine atoms have been predicted in atomistic studies to have very high energy barriers to bulk interstitial diffusion \cite{rossi2015first,legris2005ab,carlot2002energetically}. This is due to the relatively large radius of the iodine atom, which imposes large local strains when penetrating the Zr lattice. This suggests that iodine will instead be transported towards crack tips via grain boundaries. Indeed, intergranular cracking has been observed in PCI failures, but only for a few hundred nm before a more rapid transgranular crack propagation \cite{fregonese2000failure, une1984threshold, wood1983effects, lunde1981stress, vilpponen1981fuel}. Conversely, no atomic scale studies of iodine in \zirconia\ were found in the literature.  

As discussed in § \ref{section:outervsinner}, there is an oxide layer on the internal surface of the cladding consisting of monoclinic and tetragonal oxide grains. The effectiveness of the oxide layer as a barrier to iodine is debated, with one study presuming that the oxide is bypassed entirely by iodine due to fracturing, leaving the Zr metal underneath exposed \cite{rossi2015first}. The outermost part of the oxide, which is porous, exhibits networks of interconnected grain boundary diffusion pathways towards the oxide/metal interface which are certainly wide enough (1-3 nm) to allow iodine transport \cite{ni2010porosity}. The oxygen-saturated Zr at the oxide/metal interface is not, however, taken into account, and it is expected that this will influence the corrosion mechanism due to iodine-oxygen competition: even the much smaller hydrogen atom has its rate of diffusion into the metal reduced by the presence of oxygen, as shown in both computational \cite{glazoff2014oxidation} and experimental hydrogen pick-up studies \cite{couet2014hydrogen}. This means that some barrier to iodine ingress must already exist near the oxide/metal interface. The varying levels of oxygen across the oxide layer itself also have an effect on defect behaviour, and will therefore influence the initiation mechanisms behind PCI failures. Thus here, we predict iodine incorporation energies and defect equilibria in \zirconia\ as a function of oxygen pressure through Brouwer diagrams, in order to predict the resulting iodine defect response.


%\subsection{Pellet-cladding interaction}
%
%Pellet-cladding interaction refers to the interaction between the fuel and the cladding at higher burnups where the gas gap has been closed by the swelling of the fuel pellets. PCI has both a mechanical and a chemical component, sometimes referred to specifically as pellet cladding mechanical interaction (PCMI) and pellet cladding chemical interaction (PCCI) respectively.
%

\section{Methodology}
\subsection{Computational details}

As discussed in Chapter \ref{ch:compmethodology}, calculations were performed using CASTEP 8.0 \cite{Clark2005}. Ultra-soft pseudo-potentials with a cut-off energy of 600 eV were employed. The Perdew, Burke and Ernzerhof (PBE) \cite{Perdew1996} parameterisation of the generalised gradient approximation (GGA) was used to describe the exchange correlation functional. A Monkhorst-Pack sampling scheme \cite{Monkhorst1976} was used for Brillouin zone integration, with a minimum \emph{k}-point separation of 0.09 \r{A}\textsuperscript{-1}. The Pulay method for density mixing \cite{Pulay1980} was used to improve simulation convergence. 

The electronic energy convergence criterion was set to $1\times10^{-6}$ eV and the maximum force between atoms limited to $1\times10^{-2}$ eV \r{A}\textsuperscript{-1}, which are values demonstrated as appropriate in § \ref{convergence_criteria}. A gradient-descent geometry optimisation task was run on the cell until consecutive iterations differed in energy and atomic displacement by less than $1\times10^{-5}$ eV and $5\times10^{-4}$ \r{A} respectively, again demonstrated in § \ref{convergence_criteria}. 


\subsection{Defect Equilibrium Response to Oxygen Partial Pressure}

Brouwer diagrams were generated using the method previously outlined in § \ref{brouwer_method}. Defect concentrations for the monoclinic and tetragonal phases were calculated at 650 K and 1500 K respectively to reflect the temperatures at which these structures are stable. Brouwer diagrams at extrinsic defect concentrations of $10^{-5}$ and $10^{-3}$ parts/fu (i.e. parts per \zirconia\ formula unit) were generated to examine low and high dopant concentrations, respectively. These two concentrations were examined because the amount of fission products present at a particular point in a fuel pellet depends on macroscopic parameters, including its position in the core and the time since the last shutdown, but also microscopic parameters such as the radial position in the pellet. These two concentrations were selected because $10^{-3}$ parts/fu is high enough to model an aggregation of iodine (such as at a crack tip), and $10^{-5}$ parts/fu was found to be the concentration below which iodine did not have a significant effect on defect equilibria. 


%Brouwer diagrams, also known as Kr{\"o}ger-Vink diagrams, were produced using a method outlined by Murphy et al. \cite{Murphy2014} to determine defect concentrations as a function of oxygen partial pressure. We start from the statement that the chemical potential of \zirconia\ is equivalent to the sum of the chemical potentials $\mu$ of its constituent species, Zr and O:
%
%\begin{equation}
%{\mu}_{ZrO_2(s)} = {\mu}_{Zr}(p_{O_2}, T) + {\mu}_{O_2}(p_{O_2}, T)
%\label{mewZrO2results2}
%\end{equation}
%
%where $T$ denotes temperature and $p_{O_2}$ denotes oxygen partial pressure. The chemical potential of \zirconia\ in the solid state is assumed to have negligible dependence on $T$ and $p_{O_2}$ relative to ${\mu}_{Zr}$ and ${\mu}_{O_2}$. Energies can be obtained for bulk \zirconia\ and Zr, but the ground state of oxygen is not correctly reproduced in DFT \cite{Batyrev2000,Lozovoi2001}. Instead, we use the approach of Finnis et al. \cite{Finnis2005} to infer the oxygen chemical potential from standard state values. We can use the experimental Gibbs free energy to produce an equation where $\mu_{O_2}$ is the only unknown:
%
%\begin{equation}
%\Delta{G^{\plimsoll}_{f, ZrO_2}} = \mu_{ZrO_2(s)} - (\mu_{Zr(s)} + \mu^{\plimsoll}_{O_2})
%\end{equation}
%
%where $\Delta{G^{\plimsoll}_{f, ZrO_2}}$ is the experimental Gibbs energy at standard temperature and pressure and $\mu^{\plimsoll}_{O_2}$ is the oxygen chemical potential under the same conditions. The values of $\mu_{ZrO_2(s)}$ and $\mu_{Zr(s)}$ are calculated from the DFT energies. Once $\mu^{\plimsoll}_{O_2}$ is calculated, we can generalise the chemical potential of oxygen for any value of $T$ and $p_{O_2}$ by appending an ideal gas relationship $\Delta{\mu(T)}$ and a Boltzmann distribution:
%
%\begin{equation}
%\mu_{O_2}(p_{O_2},T) = \mu^{\plimsoll}_{O_2} + \Delta{\mu(T)} + \frac{1}{2}{k_B}log(\frac{p_{O_2}}{p^{\plimsoll}_{O_2}})
%\end{equation}
%
%Using our generalised formula for $\mu_{O_2}$, we fix the temperature within the range of thermal phase-stabilisation (1500 K for tetragonal \zirconia) and calculate $\mu_{O_2}$ for many different values of $p_{O_2}$ between $10^{-35}$ and 10$^{0}$ atm, corresponding to oxygen deficient and oxygen rich environments, respectively ($p_{O_2}$ in air is approximately 0.2 atm). While the tetragonal phase will be stress-stabilised in practice, thermal-stabilisation in such models has been shown to qualitatively approximate the effect of stress-stabilisation, while allowing a wider range of dopant behaviours to be predicted \cite{Bell2016}. 

\section{Results}

\subsection{Incorporation energies}

\subsubsection*{Interstitial Sites}
Neutral iodine incorporation energies at interstitial sites for each phase are reported in Table \ref{i_incorp_interstitial}. The $2a$ and $2c$ sites in monoclinic \zirconia\ provide the least unfavourable iodine incorporation energy, followed by the $2b$ and $8e$ sites in tetragonal \zirconia , although in all cases energies are positive and large, indicating a large energy penalty against interstitial incorporation. The difference in incorporation energies between monoclinic and tetragonal \zirconia\ is approximately 1 eV, whereas the difference between tetragonal and cubic is 3.5 eV, indicating especially unfavourable conditions in cubic \zirconia . These differences are likely due to the larger interstitial sites in the lower-temperature phases, as monoclinic \zirconia\ exhibits the least and cubic \zirconia\ the most dense cell, (see Chapter \ref{ch:crystallography}). 
%It is expected that these densities will be representative because the decreased density due to thermal expansion at 650 K will be countered by attractive dispersion forces. 

\begin{table}[ht]
\onehalfspacing
\centering
\caption{Incorporation energies of iodine interstitials in non-defective supercells.}
\label{i_incorp_interstitial}
\begin{tabular}{ccccc}
\hline
\multirow{2}{*}{\textbf{Structure}} & \multirow{2}{*}{\textbf{Site}} & \multicolumn{3}{c}{\textbf{Incorporation Energy (eV)}} \\ \cline{3-5} 
 &  &  \textbf{\ch{I^{x}_{i}}} & \textbf{\ch{I^{'}_{i}}} & \textbf{\ch{I^{*}_{i}}} \\ \hline % \hspace{0.7 cm}
\multirow{4}{*}{\textbf{Monoclinic}} & 2a & 8.55 & 12.10 & 6.55 \\
 & 2b & 10.81 & 16.40 & 5.63 \\
 & 2c & 8.79 & 12.20 & 4.62 \\
 & 2d & 10.94 & 13.66 & 6.92 \\ \hline
\multirow{2}{*}{\textbf{Tetragonal}} & 2b & 9.49 & 13.96 & 5.99 \\
 & 8e & 9.53 & 12.73 & 5.10 \\ \hline
\multirow{2}{*}{\textbf{Cubic}} & 24d & 13.02 & 18.24 & 7.62 \\
 & 4b & 13.08 & 16.46 & 9.82 \\ \hline
\end{tabular}%
\end{table}

While the incorporation energies of iodine in the interstitial sites of \zirconia\ are large, for a fixed iodine concentration, they become relevant as the intrinsic defect populations become small, such as at low temperatures relative to the melting point. This is because interstitial sites are always available, whereas at low intrinsic defect concentrations, substitutional sites become saturated and accommodation at a lattice site first requires the creation of a vacancy defect, which has a formation energy penalty associated with it. 

When Brouwer diagrams are generated, iodine will also be considered as a charged species at an interstitial site (see § \ref{results2_brouwer}). This includes I$^{+}$ and I$^{-}$, where I$^{+}$ is a smaller ion that is more easily accommodated at an interstitial site. Energy values for I$^{+}$ and I$^{-}$ are also reported in Table \ref{i_incorp_interstitial}, however, values for different charge states cannot be compared because an electron has been added or removed from the I atom to form the specific charge state. An interstitial site will always be uncharged (resultant charge is distributed onto nearby ions instead), meaning that there is no pre-existing charged interstitial site to incorporate an atom onto. It is therefore only appropriate to consider the \emph{formation} energy (i.e. including the addition or removal of an electron) of a charged interstitial defect. 

\subsubsection*{Oxygen Sites}

Table \ref{i_incorp_oxygen} reports incorporation energies of iodine at various oxygen sites. In each phase, the lowest incorporation energy was that for accommodation at a vacant oxygen site such that iodine is in the 1- oxidation state, resulting in the overall defect \ch{I_{O}^{*}}. This anionic behaviour is expected from a halogen atom in a highly reducing site as it promotes the filling of the \emph{p} shell. I is most readily accommodated in the monoclinic phase for all I charge states.

\begin{table}[ht] % Iodine O site incorporation
\onehalfspacing
\centering
\caption{Incorporation energies of iodine in oxygen sites of the monoclinic, tetragonal, and cubic \zirconia\ phases.}
\label{i_incorp_oxygen}
\begin{tabular}{cccc} % \ch{V_{O}^{x}}
\hline
\multirow{2}{*}{\textbf{Structure}} & \multicolumn{3}{c}{\textbf{Incorporation energy (eV)}} \\ \cline{2-4} 
                                    & \hspace{0.7 cm} \textbf{\ch{I_{O}^{**}}} \hspace{0.7 cm} & \textbf{\ch{I_{O}^{*}}} & \textbf{\ch{I_{O}^{x}}} \\ \hline
\textbf{Monoclinic (3 co-ord)}      & 4.54             & 2.90             & 3.67             \\
\textbf{Monoclinic (4 co-ord)}      & 5.63             & 3.77             & 4.87             \\
\textbf{Tetragonal}                 & 6.19             & 4.02             & 4.44             \\
\textbf{Cubic}                      & 8.37             & 5.74             & 6.66      \\     \hline  
\end{tabular}
\end{table}


 % and these atoms have large electron affinities since they require only one electron to achieve the relatively stable noble gas electron configuration. 

\subsubsection*{Zirconium Sites}

Incorporation energies of iodine on zirconium sites are reported in Table \ref{i_incorp_zirconium}. The  incorporation energy decreases as the charge of the defect decreases from -4 to 0 (i.e. nominally from I$^{0}$ to I$^{4+}$). This is due to the decrease in the size of the iodine species with increasing positive charge, fitting better into the small vacant Zr$^{4+}$ cation site. This alone does not guarantee the emergence of uncharged iodine defects (I$^{4+}$) on zirconium sites when all energy terms are considered. In particular, there is also an energy penalty incurred in the change in charge of iodine. A Mulliken population analysis revealed a charge localised on the iodine of +2.31 at the \ch{I_{Zr}^{x}} defect, and a +0.86 charge on the \ch{I_{Zr}^{''''}} defect, with the remaining charge accommodated by other ions in the lattice. Again, I is most readily accommodated in the monoclinic phase for all charge states.

\begin{table}[ht] % Iodine Zr site incorporation
\onehalfspacing
\centering
\caption{Incorporation energies of iodine in zirconium sites of \zirconia.}
\label{i_incorp_zirconium}
\begin{tabular}{ccllll}
\hline
\hspace{0.7 cm} \multirow{2}{*}{\textbf{Structure}} \hspace{0.7 cm} & \multicolumn{5}{c}{\hspace{0.7 cm} \textbf{Incorporation energy (eV)}} \hspace{0.7 cm}                                                          \\ \cline{2-6} 
                                    & \multicolumn{1}{l}  {\textbf{\hspace{0.45 cm} \ch{I_{Zr}^{''''}}}} \hspace{0.45 cm} & \textbf{\ch{I_{Zr}^{'''}}} \hspace{0.45 cm} & \textbf{\ch{I_{Zr}^{''}}} \hspace{0.45 cm} & \textbf{\ch{I_{Zr}^{'}}} \hspace{0.45 cm} & \textbf{\ch{I_{Zr}^{x}}} \\ \hline
\textbf{Monoclinic}                 & 6.78                             &       3.65            &        0.89          &         -2.84         &     -5.08             \\
\textbf{Tetragonal}                 & 7.58                            &         3.64         &        1.69          &      -2.13            &     -4.57             \\
\textbf{Cubic}                      & 9.70                            &         6.81         &        3.01          &        0.38          &      -3.14     \\      \hline
\end{tabular}
\end{table}

%\subsection{Temperature dependence}

%To examine the temperature dependence of the defect equilibria, the concentration of dopant iodine was held constant while temperature was changed.


%\subsection{Dopant concentration dependence}
%
%To examine the dependence of the defect equilibria on iodine dopant concentration, the temperature was held constant while iodine concentration was changed.

%\begin{figure}[ht] % Tet intrinsic no space charge
%\begin{center}
%\begin{tikzpicture}
%	\begin{groupplot}[group style={group size=1 by 2}, width=13cm, height=10.2cm]
%	\nextgroupplot[
%		 ylabel={\ch{log_{10}}([D]) (per f.u.)}, ymin=-10, ymax=0, xmin=-35, xmax=0, legend style={{draw=}, at={(0.40,0.97)}, anchor=north west, legend columns=2, nodes={scale=1, transform shape}}]
%
%	\nextgroupplot[
%		 xlabel={\ch{log_{10}}($p_{O_{2}}$) (atm)}, ylabel={\ch{log_{10}}([D]) (per f.u.)}, ymin=-10, ymax=0, xmin=-35, xmax=0, legend style={{draw=}, at={(0.40,0.97)}, anchor=north west, legend columns=4, nodes={scale=1, transform shape}}]
%      
%	\end{groupplot}           
%\end{tikzpicture}
%		\caption{Tetragonal phase Brouwer diagrams of intrinsic point defects at a temperature of 1500 K \textbf{a)} without a space charge and \textbf{b)} with a space charge of $10^{-1}$ e$^{-1}$ per f.u.}
%		\label{figure:spacechargeexample}
%	\end{center}
%\end{figure}

\subsection{Dopant interstitial defects}

The formation energies of iodine interstitial defects are useful not only to determine whether interstitial defects will form, but also which interstitial sites in particular they will occupy. As shown in Table \ref{table:interstitials}, ZrO$_{2}$ has four interstitial sites in the monoclinic phase, and two interstitial sites in the tetragonal and cubic phase (based on crystallographic data of the three phases). For each phase, the formation energies of each iodine interstitial defect as a function of Fermi level are calculated and provided below.

\subsubsection{Monoclinic interstitial defects}

Figure \ref{figure:monointer} shows how the formation energy of an iodine interstitial defect in monoclinic ZrO$_{2}$ varies based on site and Fermi level. In this phase, the 2$b$ Wyckoff position is the site of lowest formation energy across the entire range of Fermi levels that span the band gap, with a maximum of 8.1 eV at a Fermi level of 2.8 eV.  The next most favourable site is at 2$a$, with formation energies at least 0.4 eV greater at similar Fermi levels. Iodine at the 2$c$ and 2$d$ sites exhibits formation energies between 1 and 4 eV larger than at the 2$b$ site, making these sites significantly unfavourable. This shows that there are indeed four unique sites which iodine atoms can occupy in the monoclinic phase (as evidenced by the different formation energy evolutions at each site), and that of these sites, the 2$c$ is the most energetically favourable for iodine.

\begin{figure}[ht] % Mono iodine interstitials E vs Fermi
\begin{center}
\begin{tikzpicture}
	\begin{axis}
		[width=11.5cm, xlabel={Fermi level $\mu_{e}$ (eV)}, ylabel={Formation energy (eV) per \zirconia\ }, ymin=4, ymax=11, xmin=0, xmax=6, legend style={{draw=}, at={(0.5,0.05)}, anchor=south, legend columns=2}]
		\addplot[no marks, red] table [x=2a1, y=2a2,]{dat/monointer.dat}; \addlegendentry{$2a$};
        \addplot[no marks, red, dashed] table [x=2b1, y=2b2, ]{dat/monointer.dat}; \addlegendentry{$2b$};
        \addplot[no marks, blue] table [x=2c1, y=2c2,]{dat/monointer.dat}; \addlegendentry{$2c$};
        \addplot[no marks, blue, dashed] table [x=2d1, y=2d2,]{dat/monointer.dat}; \addlegendentry{$2d$};
			\end{axis}
		\end{tikzpicture}
		\caption{Iodine interstitial formation energies in monoclinic \zirconia\ as a function of Fermi level. Gradient indicates defect charge.}
		\label{figure:monointer}
	\end{center}
\end{figure}

It should also be noted that iodine will occupy these interstitial sites as either \ch{I^{*}_{i}} or \ch{I^{'}_{i}} (the slope of the curve indicates defect charge). \ch{I^{x}_{i}} appears for a small range of Fermi levels in the 2$c$ site, but the formation energy is so large compared to the other sites that this will not be present in the crystal. The preference for a charged state of +1 or -1 may be due to a combination of electron availability and ionic radius. \ch{I^{*}_{i}}, being positively charged, has a smaller ionic radius than \ch{I^{x}_{i}}. The smaller size of this defect comes with a smaller energy penalty when occupying an interstitial site, and at low Fermi levels the iodine is more susceptible to oxidation. At high Fermi levels, \ch{I^{'}_{i}} forms because electrons are more readily available and iodine has a high electron affinity, which compensates energetically for the increased ionic radius.

\subsubsection{Tetragonal interstitial defects}

\begin{figure}[ht] % Tet iodine interstitials E vs Fermi
\begin{center}
\begin{tikzpicture}
	\begin{axis}
		[width=11cm, xlabel={Fermi level $\mu_{e}$ (eV)}, ylabel={Formation energy (eV) per \zirconia\ }, ymin=6, ymax=11, xmin=0, xmax=6, legend style={{draw=}, at={(0.5,0.05)}, anchor=south, legend columns=1}]
		\addplot[no marks, red] table [x=2atet1, y=2atet2,]{dat/tetcubicinter.dat}; \addlegendentry{$2a$};
        \addplot[no marks, blue] table [x=8etet1, y=8etet2, ]{dat/tetcubicinter.dat}; \addlegendentry{$8e$};
			\end{axis}
		\end{tikzpicture}
		\caption{Iodine interstitial formation energies in tetragonal \zirconia\ as a function of Fermi level. Gradient indicates defect charge.}
		\label{figure:tetinter}
	\end{center}
\end{figure}

\subsubsection{Cubic interstitial defects}

\begin{figure}[ht] % Cubic iodine interstitials E vs Fermi
\begin{center}
\begin{tikzpicture}
	\begin{axis}
		[width=11cm, xlabel={Fermi level $\mu_{e}$ (eV)}, ylabel={Formation energy (eV) per \zirconia\ }, ymin=8, ymax=13, xmin=0, xmax=6, legend style={{draw=}, at={(0.5,0.05)}, anchor=south, legend columns=1}]
		\addplot[no marks, red] table [x=24cubic1, y=24cubic2,]{dat/tetcubicinter.dat}; \addlegendentry{$24d$};
        \addplot[no marks, blue] table [x=4bmcubic1, y=4bmcubic2, ]{dat/tetcubicinter.dat}; \addlegendentry{$4b$};
			\end{axis}
		\end{tikzpicture}
		\caption{Iodine interstitial formation energies in cubic \zirconia\ as a function of Fermi level. Gradient indicates defect charge.}
		\label{figure:cubicinter}
	\end{center}
\end{figure}

\subsection{Brouwer Diagrams}  \label{results2_brouwer}

\subsubsection*{Monoclinic Phase}

Brouwer diagrams associated with the monoclinic phase, at 650 K, at which temperature this \zirconia\ phase is stable, are shown in Figure \ref{figure:tikzbrouwerconcmono}. At 650 K, this phase exhibits a relatively low concentration of intrinsic defects; concentrations of \ch{V_{O}^{**}} and \ch{V_{Zr}^{''''}} remained below $10^{-10}$ parts/fu across the majority of oxygen pressures at both iodine concentrations and do not appear in the diagrams. At lower iodine concentrations, the intrinsic electronic defects, \ch{e^{'}} and \ch{h^{*}}, were more significant, with \ch{h^{*}} defects being a major fraction of the total defect population near stoichiometry (i.e. at an oxygen pressure of approximately $10^{-7.5}$ atm). 

\begin{figure}[ht!] % Mono conc sweep
\begin{center}
\begin{tikzpicture} % 10e-3 iodine conc in mono
	\begin{groupplot}[group style={group size=1 by 2}, width=13cm, height=10.2cm]
	\nextgroupplot[
		 ylabel={\ch{log_{10}}([D]) (per f.u.)}, ymin=-10, ymax=0, xmin=-35, xmax=0, legend style={{draw=}, at={(0.40,0.97)}, anchor=north west, legend columns=2, nodes={scale=1, transform shape}}]
        \addplot[no marks, draw=blue!70!black] table [x=pO2, y=electrons,]{dat/1e5iconcmono650.dat};  \node at (-28.1,-7.5) {\ch{e^{'}}};  \addlegendentry{\ch{e^{'}}};
        \addplot[no marks, draw=red!85!black] table [x=pO2, y=holes,]{dat/1e5iconcmono650.dat}; \addlegendentry{\ch{h^{\textperiodcentered}}}; % \node at (-1,-4.5) {\ch{h^{\textperiodcentered}}};
%         \addplot[no marks, draw=black!70!green] table [x=pO2, y=VO{2},]{dat/1e5iconcmono650.dat}; \addlegendentry{\ch{V_{O}^{\textperiodcentered\textperiodcentered}}};
%         \addplot[no marks, draw=black!55!green] table [x=pO2, y=VO{1},]{dat/1e5iconcmono650.dat}; \addlegendentry{\ch{V_{O}^{\textperiodcentered}}};
%         \addplot[no marks, draw=black!30!green] table [x=pO2, y=VO{0},]{dat/1e5iconcmono650.dat}; \addlegendentry{\ch{V_{O}^{x}}};
%         \addplot[no marks, draw=yellow!85!blue] table [x=pO2, y=VM{-4},]{dat/1e5iconcmono650.dat}; \addlegendentry{\ch{V_{Zr}^{''''}}};
%         \addplot[no marks, draw=yellow!75!blue] table [x=pO2, y=VM{-3},]{dat/1e5iconcmono650.dat}; \addlegendentry{\ch{V_{Zr}^{'''}}};
%         \addplot[no marks, draw=yellow!65!blue] table [x=pO2, y=VM{-2},]{dat/1e5iconcmono650.dat}; \addlegendentry{\ch{V_{Zr}^{''}}};
%         \addplot[no marks, draw=yellow!55!blue] table [x=pO2, y=VM{-1},]{dat/1e5iconcmono650.dat}; \addlegendentry{\ch{V_{Zr}^{'}}};
%         \addplot[no marks, draw=yellow!45!blue] table [x=pO2, y=VM{0},]{dat/1e5iconcmono650.dat}; \addlegendentry{\ch{V_{Zr}^{x}}};
%         \addplot[no marks, draw=red!60!yellow] table [x=pO2, y=Oi{-2},]{dat/1e5iconcmono650.dat}; \addlegendentry{\ch{O_{i}^{''}}};
%         \addplot[no marks, draw=red!50!yellow] table [x=pO2, y=Oi{-1},]{dat/1e5iconcmono650.dat}; \addlegendentry{\ch{O_{i}^{'}}};
%         \addplot[no marks, draw=red!40!yellow] table [x=pO2, y=Oi{0},]{dat/1e5iconcmono650.dat}; \addlegendentry{\ch{O_{i}^{x}}};
%         \addplot[no marks, draw=green!80!pink] table [x=pO2, y=Mi{4},]{dat/1e5iconcmono650.dat}; \addlegendentry{\ch{Zr_{i}^{\textperiodcentered\textperiodcentered\textperiodcentered\textperiodcentered}}};
%         \addplot[no marks, draw=green!70!pink] table [x=pO2, y=Mi{3},]{dat/1e5iconcmono650.dat}; \addlegendentry{\ch{Zr_{i}^{\textperiodcentered\textperiodcentered\textperiodcentered}}};
%         \addplot[no marks, draw=green!60!pink] table [x=pO2, y=Mi{2},]{dat/1e5iconcmono650.dat}; \addlegendentry{\ch{Zr_{i}^{\textbf{\textperiodcentered\textperiodcentered}}}};
%         \addplot[no marks, draw=green!50!pink] table [x=pO2, y=Mi{1},]{dat/1e5iconcmono650.dat}; \addlegendentry{\ch{Zr_{i}^{\textperiodcentered}}};
%         \addplot[no marks, draw=green!40!pink] table [x=pO2, y=Mi{0},]{dat/1e5iconcmono650.dat}; \addlegendentry{\ch{Zr_{i}^{x}}};
%         \addplot[no marks, dashed, draw=red!70!black] table [x=pO2, y=Ii{0},]{dat/1e5iconcmono650.dat}; \addlegendentry{\ch{I_{i}^{x}}};
%         \addplot[no marks, dashed, draw=red!50!black] table [x=pO2, y=Ii{-1},]{dat/1e5iconcmono650.dat}; \addlegendentry{\ch{I_{i}^{'}}};
        \addplot[no marks, dashed, draw=purple!60!white] table [x=pO2, y=Ii{1},]{dat/1e5iconcmono650.dat}; \addlegendentry{\ch{I_{i}^{\textperiodcentered}}}; 
        \addplot[no marks, dashed, draw=blue!50!white] table [x=pO2, y=IsubO{1},]{dat/1e5iconcmono650.dat}; \addlegendentry{\ch{I_{O}^{\textperiodcentered}}}; \node at (-22,-4.5) {\ch{I_{O}^{\textperiodcentered}}};
        \addplot[no marks, dashed, draw=green!60!black] table [x=pO2, y=IsubO{2},]{dat/1e5iconcmono650.dat}; \addlegendentry{\ch{I_{O}^{\textperiodcentered\textperiodcentered}}}; 
        \addplot[no marks, dashed, draw=black] table [x=pO2, y=IsubO{3},]{dat/1e5iconcmono650.dat}; \addlegendentry{\ch{I_{O}^{\textperiodcentered\textperiodcentered\textperiodcentered}}};
        \addplot[no marks, dashed, draw=orange!80!black] table [x=pO2, y=IsubZr{-3},]{dat/1e5iconcmono650.dat};  \node at (-22,-6.2) {\ch{I_{Zr}^{'''}}}; \addlegendentry{\ch{I_{Zr}^{'''}}};
%         \addplot[no marks, dashed, draw=green!50!black] table [x=pO2, y=IsubZr{-4},]{dat/1e5iconcmono650.dat}; \addlegendentry{\ch{I_{Zr}^{''''}}};
%         \addplot[no marks, dashed, draw=green!30!black] table [x=pO2, y=IsubZr{-5},]{dat/1e5iconcmono650.dat}; \addlegendentry{\ch{I_{Zr}^{'''''}}};
%         \addplot[no marks] table [x=pO2, y=Stoich,]{dat/1e5iconcmono650.dat}; \addlegendentry{Stoich};
\node at (-33.7,-0.5) {\textbf{a)}};
			\nextgroupplot[
		 xlabel={\ch{log_{10}}($p_{O_{2}}$) (atm)}, ylabel={\ch{log_{10}}([D]) (per f.u.)}, ymin=-10, ymax=0, xmin=-35, xmax=0, legend style={{draw=}, at={(0.40,0.97)}, anchor=north west, legend columns=4, nodes={scale=1, transform shape}}]
        \addplot[no marks, draw=blue!70!black] table [x=pO2, y=electrons,]{dat/1e3iconcmono650.dat}; \node at (-29,-7) {\ch{e^{'}}};
        \addplot[no marks, draw=red!85!black] table [x=pO2, y=holes,]{dat/1e3iconcmono650.dat}; 
%         \addplot[no marks, draw=black!70!green] table [x=pO2, y=VO{2},]{dat/1e3iconcmono650.dat}; 
%         \addplot[no marks, draw=black!55!green] table [x=pO2, y=VO{1},]{dat/1e3iconcmono650.dat}; 
%         \addplot[no marks, draw=black!30!green] table [x=pO2, y=VO{0},]{dat/1e3iconcmono650.dat}; 
%         \addplot[no marks, draw=yellow!85!blue] table [x=pO2, y=VM{-4},]{dat/1e3iconcmono650.dat}; 
%         \addplot[no marks, draw=yellow!75!blue] table [x=pO2, y=VM{-3},]{dat/1e3iconcmono650.dat}; 
%         \addplot[no marks, draw=yellow!65!blue] table [x=pO2, y=VM{-2},]{dat/1e3iconcmono650.dat}; 
%         \addplot[no marks, draw=yellow!55!blue] table [x=pO2, y=VM{-1},]{dat/1e3iconcmono650.dat}; 
%         \addplot[no marks, draw=yellow!45!blue] table [x=pO2, y=VM{0},]{dat/1e3iconcmono650.dat}; 
%         \addplot[no marks, draw=red!60!yellow] table [x=pO2, y=Oi{-2},]{dat/1e3iconcmono650.dat}; 
%         \addplot[no marks, draw=red!50!yellow] table [x=pO2, y=Oi{-1},]{dat/1e3iconcmono650.dat}; 
%         \addplot[no marks, draw=red!40!yellow] table [x=pO2, y=Oi{0},]{dat/1e3iconcmono650.dat}; 
%         \addplot[no marks, draw=green!80!pink] table [x=pO2, y=Mi{4},]{dat/1e3iconcmono650.dat}; 
%         \addplot[no marks, draw=green!70!pink] table [x=pO2, y=Mi{3},]{dat/1e3iconcmono650.dat}; 
%         \addplot[no marks, draw=green!60!pink] table [x=pO2, y=Mi{2},]{dat/1e3iconcmono650.dat}; 
%         \addplot[no marks, draw=green!50!pink] table [x=pO2, y=Mi{1},]{dat/1e3iconcmono650.dat}; 
%         \addplot[no marks, draw=green!40!pink] table [x=pO2, y=Mi{0},]{dat/1e3iconcmono650.dat}; 
%         \addplot[no marks, dashed, draw=red!70!black] table [x=pO2, y=Ii{0},]{dat/1e3iconcmono650.dat}; 
%         \addplot[no marks, dashed, draw=red!50!black] table [x=pO2, y=Ii{-1},]{dat/1e3iconcmono650.dat}; 
        \addplot[no marks, dashed, draw=purple!60!white] table [x=pO2, y=Ii{1},]{dat/1e3iconcmono650.dat}; 
        \addplot[no marks, dashed, draw=blue!50!white] table [x=pO2, y=IsubO{1},]{dat/1e3iconcmono650.dat}; \node at (-22,-2.5) {\ch{I_{O}^{\textperiodcentered}}};
        \addplot[no marks, dashed, draw=green!60!black] table [x=pO2, y=IsubO{2},]{dat/1e3iconcmono650.dat}; 
        \addplot[no marks, dashed, draw=black] table [x=pO2, y=IsubO{3},]{dat/1e3iconcmono650.dat}; 
        \addplot[no marks, dashed, draw=orange!80!black] table [x=pO2, y=IsubZr{-3},]{dat/1e3iconcmono650.dat}; \node at (-22,-4.4) {\ch{I_{Zr}^{'''}}}; 
%         \addplot[no marks, dashed, draw=green!50!black] table [x=pO2, y=IsubZr{-4},]{dat/1e3iconcmono650.dat}; 
%         \addplot[no marks, dashed, draw=green!30!black] table [x=pO2, y=IsubZr{-5},]{dat/1e3iconcmono650.dat}; 
%         \addplot[no marks] table [x=pO2, y=Stoich,]{dat/1e3iconcmono650.dat}; 
\node at (-33.7,-0.5) {\textbf{b)}};
			\end{groupplot}          
\end{tikzpicture}
		\caption{Monoclinic phase Brouwer diagrams of point defects at iodine concentrations of a) $10^{-5}$ and b) $10^{-3}$, at a temperature of 650 K.}
		\label{figure:tikzbrouwerconcmono}
	\end{center}
\end{figure} % 10e-3 iodine conc in mono

Between oxygen pressures of $10^{-35}$ and $10^{-10}$ atm, the dominant defects were \ch{I_{O}^{*}} charge-compensated by \ch{I_{Zr}^{'''}}. Above an oxygen pressure of $10^{-10}$ atm, a combination of \ch{I_{i}^{*}}, \ch{I_{Zr}^{'''}} and \ch{I_{O}^{***}} defects were dominant. This demonstrates that iodine will adopt a +1 oxidation state in order to facilitate iodine incorporation into the lattice. The effective ionic radius of I$^{-}$ is 2.20 \r{A} in VI-fold coordination, compared to 1.38 \r{A} for O$^{2-}$ in IV-fold coordination, as is the case in \zirconia\ \cite{Shannon1976}. Iodine with a higher positive charge state will have a smaller ionic radius, and thus impose less strain on the lattice (and therefore a smaller energy penalty) in each defect configuration, including substitution on a Zr site. At the highest oxygen pressures, the Brouwer diagrams show that oxidation of iodine, substituted at an oxygen site, to the +1 oxidation state (i.e. \ch{I_{O}^{***}}) becomes a necessary charge compensating defect. This is because the energy penalty to form hole defects in this broad band insulator is too great, as is the formation of other positive charge defects such as \ch{Zr_{i}^{****}}. This may translate to iodine out-competing oxygen for oxygen sites in monoclinic \zirconia , with higher oxygen pressures providing very little in terms of a barrier effect. 

% \begin{itemize}
% %\item Figure 1 shows monoclinic Brouwer diagrams generated at different assumed iodine concentrations.
% %\item The monoclinic Brouwer diagrams were generated at a temperature of 650 K. This is because the structure is stable at this temperature, and it is representative of the temperature which the \zirconia\ layer on the internal surface of the cladding would experience.
% %\item At an iodine concentration of 1e-5 and low oxygen pressures, the dominant defects were iodine -1 substitutional defects on the oxygen site, charged compensated by iodine +1 substitutional defects on the zirconium site. At higher oxygen pressures, the zirconium substitutional defects remained but were now charge compensated by iodine +1 substitutional defects on the oxygen site. At very high oxygen pressures, hole defects were preferred to oxygen substitutional defects.

% \item At an iodine concentration of 1e-3, intrinsic defects were found to be negligible compared to the extrinsic iodine defects. The same pattern was seen as with an iodine concentration of 1e-5, except hole defects were no longer significant at high oxygen pressures. Very low concentrations of iodine +1 interstitial defects began to appear at oxygen pressures greater than 1e-20.
% \end{itemize}
% The monoclinic Brouwer diagram (Figure \ref{figure:monoBrouwer}) predicts that at 635 K, few types of defects will be present and at very low (\textless 10 ppb \zirconia ) concentrations. This is typical of defect behaviour in a ceramics at temperatures far below their melting points \cite{kingery1997physical,ball2006computer}. Fully-charged zirconium vacancies, charge-compensated by holes, are the major defect type we expect to observe at $p_{O_{2}}$ \textgreater $10^{-15}$. Below this, only electronic defects compensated by electron hole defects are expected. We briefly see increased concentrations of uncharged oxygen interstitial defects at very high levels of $p_{O_{2}}$.

\subsubsection*{Tetragonal Phase}

Brouwer diagrams for the tetragonal phase are shown in Figure \ref{figure:tikzbrouwerconctet}. As these diagrams were generated at a temperature of 1500 K (at which the tetragonal phase becomes stable), intrinsic defect concentrations were significantly higher than in the monoclinic diagrams for all oxygen pressures (though trends remained the same). Intrinsic defects \ch{e^{'}}, \ch{h^{*}}, \ch{V_{O}^{**}} and \ch{V_{Zr}^{''''}} were dominant across most oxygen pressures at an iodine concentration of $10^{-5}$ parts/fu. Only around stoichiometry do extrinsic defect concentrations approach intrinsic values (which as mentioned earlier is why this concentration of iodine was chosen). Across all oxygen pressures, \ch{I_{O}^{*}} and \ch{I_{Zr}^{'''}} are the major iodine defects. Between $10^{-15}$ and $10^{-5}$ atm, Figure \ref{figure:tikzbrouwerconctet} illustrates that the major iodine defect swaps from being \ch{I_{O}^{*}} to \ch{I_{Zr}^{'''}}. 

\begin{figure}[ht!] % 10e-5 iodine conc in tet
\begin{center}
\begin{tikzpicture}
	\begin{groupplot}[group style={group size=1 by 2}, width=13cm, height=10.2cm]
	\nextgroupplot[
		 ylabel={\ch{log_{10}}([D]) (per f.u.)}, ymin=-10, ymax=0, xmin=-35, xmax=0, legend style={{draw=}, at={(0.40,0.97)}, anchor=north west, legend columns=2, nodes={scale=1, transform shape}}]
        \addplot[no marks, draw=blue!70!black] table [x=pO2, y=electrons,]{dat/1e5iconctet1500.dat}; \addlegendentry{\ch{e^{'}}}; \node at (-26.0,-1.9) {\ch{e^{'}}};
        \addplot[no marks, draw=red!85!black] table [x=pO2, y=holes,]{dat/1e5iconctet1500.dat}; \addlegendentry{\ch{h^{\textperiodcentered}}}; \node at (-7,-3.6) {\ch{h^{\textperiodcentered}}};
        \addplot[no marks, draw=black!70!green] table [x=pO2, y=VO{2},]{dat/1e5iconctet1500.dat}; \addlegendentry{\ch{V_{O}^{\textperiodcentered\textperiodcentered}}}; \node at (-26.7,-3.3) {\ch{V_{O}^{\textperiodcentered\textperiodcentered}}};
%         \addplot[no marks, draw=black!55!green] table [x=pO2, y=VO{1},]{dat/1e5iconctet1500.dat}; \addlegendentry{\ch{V_{O}^{\textperiodcentered}}};
%         \addplot[no marks, draw=black!30!green] table [x=pO2, y=VO{0},]{dat/1e5iconctet1500.dat}; \addlegendentry{\ch{V_{O}^{x}}};
        \addplot[no marks, draw=yellow!85!blue] table [x=pO2, y=VM{-4},]{dat/1e5iconctet1500.dat}; \addlegendentry{\ch{V_{Zr}^{''''}}};
%         \addplot[no marks, draw=yellow!75!blue] table [x=pO2, y=VM{-3},]{dat/1e5iconctet1500.dat}; \addlegendentry{\ch{V_{Zr}^{'''}}};
%         \addplot[no marks, draw=yellow!65!blue] table [x=pO2, y=VM{-2},]{dat/1e5iconctet1500.dat}; \addlegendentry{\ch{V_{Zr}^{''}}};
%         \addplot[no marks, draw=yellow!55!blue] table [x=pO2, y=VM{-1},]{dat/1e5iconctet1500.dat}; \addlegendentry{\ch{V_{Zr}^{'}}};
%         \addplot[no marks, draw=yellow!45!blue] table [x=pO2, y=VM{0},]{dat/1e5iconctet1500.dat}; \addlegendentry{\ch{V_{Zr}^{x}}};
%         \addplot[no marks, draw=red!60!yellow] table [x=pO2, y=Oi{-2},]{dat/1e5iconctet1500.dat}; \addlegendentry{\ch{O_{i}^{''}}};
%         \addplot[no marks, draw=red!50!yellow] table [x=pO2, y=Oi{-1},]{dat/1e5iconctet1500.dat}; \addlegendentry{\ch{O_{i}^{'}}};
%         \addplot[no marks, draw=red!40!yellow] table [x=pO2, y=Oi{0},]{dat/1e5iconctet1500.dat}; \addlegendentry{\ch{O_{i}^{x}}};
%         \addplot[no marks, draw=green!80!pink] table [x=pO2, y=Mi{4},]{dat/1e5iconctet1500.dat}; \addlegendentry{\ch{Zr_{i}^{\textperiodcentered\textperiodcentered\textperiodcentered\textperiodcentered}}};
%         \addplot[no marks, draw=green!70!pink] table [x=pO2, y=Mi{3},]{dat/1e5iconctet1500.dat}; \addlegendentry{\ch{Zr_{i}^{\textperiodcentered\textperiodcentered\textperiodcentered}}};
%         \addplot[no marks, draw=green!60!pink] table [x=pO2, y=Mi{2},]{dat/1e5iconctet1500.dat}; \addlegendentry{\ch{Zr_{i}^{\textbf{\textperiodcentered\textperiodcentered}}}};
%         \addplot[no marks, draw=green!50!pink] table [x=pO2, y=Mi{1},]{dat/1e5iconctet1500.dat}; \addlegendentry{\ch{Zr_{i}^{\textperiodcentered}}};
%         \addplot[no marks, draw=green!40!pink] table [x=pO2, y=Mi{0},]{dat/1e5iconctet1500.dat}; \addlegendentry{\ch{Zr_{i}^{x}}};
%         \addplot[no marks, dashed, draw=red!70!black] table [x=pO2, y=Ii{0},]{dat/1e5iconctet1500.dat}; \addlegendentry{\ch{I_{i}^{x}}};
%         \addplot[no marks, dashed, draw=red!50!black] table [x=pO2, y=Ii{-1},]{dat/1e5iconctet1500.dat}; \addlegendentry{\ch{I_{i}^{'}}};
        \addplot[no marks, dashed, draw=purple!60!white] table [x=pO2, y=Ii{1},]{dat/1e5iconctet1500.dat}; \addlegendentry{\ch{I_{i}^{\textperiodcentered}}};
        \addplot[no marks, dashed, draw=blue!50!white] table [x=pO2, y=IsubO{1},]{dat/1e5iconctet1500.dat}; \addlegendentry{\ch{I_{O}^{\textperiodcentered}}};
        \addplot[no marks, dashed, draw=green!60!black] table [x=pO2, y=IsubO{2},]{dat/1e5iconctet1500.dat}; \addlegendentry{\ch{I_{O}^{\textperiodcentered\textperiodcentered}}};
        \addplot[no marks, dashed, draw=black] table [x=pO2, y=IsubO{3},]{dat/1e5iconctet1500.dat}; \addlegendentry{\ch{I_{O}^{\textperiodcentered\textperiodcentered\textperiodcentered}}};
        \addplot[no marks, dashed, draw=orange!80!black] table [x=pO2, y=IsubZr{-3},]{dat/1e5iconctet1500.dat}; \addlegendentry{\ch{I_{Zr}^{'''}}};
%         \addplot[no marks, dashed, draw=pink] table [x=pO2, y=IsubZr{-4},]{dat/1e5iconctet1500.dat}; \addlegendentry{\ch{I_{Zr}^{''''}}};
%         \addplot[no marks, dashed, draw=purple] table [x=pO2, y=IsubZr{-5},]{dat/1e5iconctet1500.dat}; \addlegendentry{\ch{I_{Zr}^{'''''}}};
%         \addplot[no marks] table [x=pO2, y=Stoich,]{dat/1e5iconctet1500.dat}; \addlegendentry{Stoich};
\node at (-33.7,-0.5) {\textbf{a)}};
			\nextgroupplot[
		 xlabel={\ch{log_{10}}($p_{O_{2}}$) (atm)}, ylabel={\ch{log_{10}}([D]) (per f.u.)}, ymin=-10, ymax=0, xmin=-35, xmax=0, legend style={{draw=}, at={(0.40,0.97)}, anchor=north west, legend columns=4, nodes={scale=1, transform shape}}]
        \addplot[no marks, draw=blue!70!black] table [x=pO2, y=electrons,]{dat/1e3iconctet1500.dat}; \node at (-27,-1.7) {\ch{e^{'}}};
        \addplot[no marks, draw=red!85!black] table [x=pO2, y=holes,]{dat/1e3iconctet1500.dat}; \node at (-2.5,-2.1) {\ch{h^{\textperiodcentered}}};
        \addplot[no marks, draw=black!70!green] table [x=pO2, y=VO{2},]{dat/1e3iconctet1500.dat}; 
%         \addplot[no marks, draw=black!55!green] table [x=pO2, y=VO{1},]{dat/1e3iconctet1500.dat}; 
%         \addplot[no marks, draw=black!30!green] table [x=pO2, y=VO{0},]{dat/1e3iconctet1500.dat}; 
        \addplot[no marks, draw=yellow!85!blue] table [x=pO2, y=VM{-4},]{dat/1e3iconctet1500.dat}; 
%         \addplot[no marks, draw=yellow!75!blue] table [x=pO2, y=VM{-3},]{dat/1e3iconctet1500.dat}; 
%         \addplot[no marks, draw=yellow!65!blue] table [x=pO2, y=VM{-2},]{dat/1e3iconctet1500.dat}; 
%         \addplot[no marks, draw=yellow!55!blue] table [x=pO2, y=VM{-1},]{dat/1e3iconctet1500.dat}; 
%         \addplot[no marks, draw=yellow!45!blue] table [x=pO2, y=VM{0},]{dat/1e3iconctet1500.dat}; 
%         \addplot[no marks, draw=red!60!yellow] table [x=pO2, y=Oi{-2},]{dat/1e3iconctet1500.dat}; 
%         \addplot[no marks, draw=red!50!yellow] table [x=pO2, y=Oi{-1},]{dat/1e3iconctet1500.dat}; 
%         \addplot[no marks, draw=red!40!yellow] table [x=pO2, y=Oi{0},]{dat/1e3iconctet1500.dat}; 
%         \addplot[no marks, draw=green!80!pink] table [x=pO2, y=Mi{4},]{dat/1e3iconctet1500.dat}; 
%         \addplot[no marks, draw=green!70!pink] table [x=pO2, y=Mi{3},]{dat/1e3iconctet1500.dat}; 
%         \addplot[no marks, draw=green!60!pink] table [x=pO2, y=Mi{2},]{dat/1e3iconctet1500.dat}; 
%         \addplot[no marks, draw=green!50!pink] table [x=pO2, y=Mi{1},]{dat/1e3iconctet1500.dat}; 
%         \addplot[no marks, draw=green!40!pink] table [x=pO2, y=Mi{0},]{dat/1e3iconctet1500.dat}; 
%         \addplot[no marks, dashed, draw=red!70!black] table [x=pO2, y=Ii{0},]{dat/1e3iconctet1500.dat}; 
%         \addplot[no marks, dashed, draw=red!50!black] table [x=pO2, y=Ii{-1},]{dat/1e3iconctet1500.dat}; 
        \addplot[no marks, dashed, draw=purple!60!white] table [x=pO2, y=Ii{1},]{dat/1e3iconctet1500.dat}; 
        \addplot[no marks, dashed, draw=blue!50!white] table [x=pO2, y=IsubO{1},]{dat/1e3iconctet1500.dat}; \node at (-11,-2.6) {\ch{I_{O}^{\textperiodcentered}}};
        \addplot[no marks, dashed, draw=green!60!black] table [x=pO2, y=IsubO{2},]{dat/1e3iconctet1500.dat}; 
        \addplot[no marks, dashed, draw=black] table [x=pO2, y=IsubO{3},]{dat/1e3iconctet1500.dat}; 
        \addplot[no marks, dashed, draw=orange!80!black] table [x=pO2, y=IsubZr{-3},]{dat/1e3iconctet1500.dat}; 
%         \addplot[no marks, dashed, draw=pink] table [x=pO2, y=IsubZr{-4},]{dat/1e3iconctet1500.dat}; 
%         \addplot[no marks, dashed, draw=purple] table [x=pO2, y=IsubZr{-5},]{dat/1e3iconctet1500.dat}; 
%         \addplot[no marks] table [x=pO2, y=Stoich,]{dat/1e3iconctet1500.dat}; 
\node at (-33.7,-0.5) {\textbf{b)}};
			\end{groupplot}        
\end{tikzpicture} % 10e-3 iodine conc in tet
		\caption{Tetragonal phase Brouwer diagrams of point defects at iodine concentrations of a) $10^{-5}$ and b) $10^{-3}$, at a temperature of 1500 K.}
		\label{figure:tikzbrouwerconctet}
	\end{center}
\end{figure}

When the iodine concentration was increased to $10^{-3}$ parts/fu, a significant change in defect equilibria was predicted. The oxygen pressure at stoichiometry increased from $10^{-10}$ to $10^{-6.5}$ atm (for monoclinic \zirconia, it remained at $10^{-7.5}$ atm regardless of iodine concentration). Nevertheless, \ch{I_{O}^{*}} and \ch{I_{Zr}^{'''}} remain the dominant defect pair between oxygen pressures of $10^{-15}$ and $10^{-5}$ atm (as they are at the lower iodine concentration). However, \ch{I_{O}^{*}} and \ch{I_{Zr}^{'''}} became higher concentration defects than both intrinsic \ch{V_{O}^{**}} and \ch{V_{Zr}^{''''}} defects. We also observe that Zr vacancies no longer serve as the main negative charge-compensation defect near stoichiometry, leaving \ch{I_{Zr}^{'''}} as the most energetically favourable negatively-charged defect.  

Unlike in the Brouwer diagrams for the monoclinic phase, for the tetragonal phase, the concentration of iodine substitutional defects on oxygen sites decreases more steeply at high oxygen pressures, peaking near stoichiometry. \ch{I_{O}^{***}} in particular, which was the dominant defect at high oxygen pressures in monoclinic \zirconia , becomes insignificant under the same conditions in the tetragonal phase, with iodine confined to Zr sites. This behaviour is indicative of a `barrier' effect against iodine at high oxygen partial pressures, with oxygen out-competing iodine for oxygen sites. Given that the inner oxide is likely to have a higher tetragonal phase fraction than the external oxide, due to the incorporation of fission products, this result could help to explain why there appears to be an oxygen effect on PCI-related SCC of zirconium alloys \cite{hofmann1984stress}. 

Another effect considered was the space charge of the system. Electrons have a higher rate of diffusion than oxygen vacancies in \zirconia , leading to a build-up of oxygen vacancies near the metal-oxide interface as corrosion progresses \cite{bojinov2010influence}. This results in an overall positive charge (since the dominant oxygen vacancy is \ch{V_{O}^{**}}) referred to as a space charge. When included in our Brouwer diagrams, this space charge had a negligible effect on the concentration or charge state of iodine up to a charge of $10^{-1}$ holes per f.u. \zirconia . This corresponds to a high concentration of oxygen vacancies relative to the equilibrium concentration, predicting that a significant deviation from equilibrium is not expected near the metal oxide interface as a result of a positive space charge.

\section{Conclusions}

Iodine exhibits lower incorporation energies when occupying defects in monoclinic \zirconia\ than in the tetragonal phase. However, as monoclinic is the low-temperature phase, intrinsic defect concentrations will also be low, thereby requiring additional energy input to produce vacancies when the concentration of iodine is much larger than that of the intrinsic defects. This leads to relatively large concentrations of iodine interstitial defects predicted in the monoclinic Brouwer diagrams, as interstitial sites are always available in the lattice. 

Defects involving iodine in the +1 oxidation state are present in significant concentrations, especially in monoclinic \zirconia , indicating that filling of the $p$ electronic sub-shell is not always energetically favourable compared to forming the smaller iodine ionic radius developed through oxidation. 

The competition between iodine and oxygen for anion sites in \zirconia\ is phase and oxygen pressure dependent. At high oxygen pressures in monoclinic \zirconia , iodine in the +1 oxidation state is predicted to occupy oxygen sites and remains the dominant defect. In tetragonal \zirconia\ at high oxygen pressures, however, the concentration of iodine defects on anion sites decreases steeply, indicating a preference for iodine accommodated at zirconium cation sites. This is indicative of a barrier effect in the tetragonal phase with oxygen out-competing iodine for anion sites.
 % Good
\chapter{Radioparagenesis of fission products in tetragonal \zirconia}

\label{ch:results3}

\section{Introduction}
\subsection{Radioparagenesis}

The nuclei of fission products immediately after a fission event are typically neutron-rich and unstable. In the case of iodine, the stable isotope is I-127, yet isotopes up to I-143 are produced during fission. This is the true for the fission of all large nuclei, including U-233 (thorium cycle), U-235 (conventional) and Pu-239 (breeder/MOX)

Stress-corrosion cracking (SCC) in nuclear fuel pins is an issue related to early loss of structural integrity of fuel assemblies in light water reactors (LWRs). In particular, the phenomenon of pellet-cladding interaction (PCI) in combination with SCC can lead to failures where the cladding is breached, exposing fuel to the coolant \cite{bcoxpelletclad1990}.     

This study follows previous work on defect equilibria in \zirconia\ to determine the oxide layer's effectiveness as a barrier to iodine \cite{kenichiodine2018}. It was found that the tetragonal phase of \zirconia\ is a greater barrier to iodine ingress than monoclinic \zirconia\ as the partial pressure of oxygen is increased. It is also known that tetragonal \zirconia\ will always be present on the inner surface of the cladding in significant quantities because it is self-stabilised by the stresses imposed as the oxide grows into the zirconium metal, in addition to compressive residual stresses induced by radiation damage. The iodine defect study, however, only informs us about one part of the SCC process. For a more holistic understanding, the life cycle of the iodine must be taken into account as well.     

%SCC studies of the internal surface of zirconium-based fuel claddings have been conducted, which indicate that iodine is likely to be one of the main corrosive species involved in promoting crack growth \cite{rosenbaum1966interaction, Cox1990Pellet-cladReview,Fregonese1998AmountIodine,Sidky1998IodineReview}. The exact mechanism for iodine SCC has not yet been determined due to difficulties observing the internal cladding surface in-situ, while experimental studies are not yet capable of reproducing the conditions under which such failures occur. This study focuses on the oxide on the internal surface of the fuel cladding, following from a previous study on iodine in the oxide layer. \\

Nuclear fuel claddings have unique materials challenges associated with them owing to the highly active environment and creation of unstable isotopes. Corrosive species in the pin such as iodine can be produced directly as a result of fission of uranium fuel. While it is known that iodine plays a role in SCC, one must also consider that these iodine nuclei are unstable. Fission of uranium will produce iodine precursors, mainly unstable isotopes of tellurium. Both iodine and tellurium are relatively common fission products, with combined independent yields from thermal fission of U$_{235}$ above 5\% \cite{kennett1956mass, iodine129fissionyield, imanishi1976independent, iodinefissionyields, iodine132, amiel1975odd}. 

Nuclei produced during fission are typically neutron-rich, resulting in decay modes such as $\beta-$ or neutron emission. In the case of tellurium, the vast majority of unstable isotopes will decay into iodine, which then decays into xenon with varying half-lives depending on the isotope. The decay chain continues with xenon nuclei decaying into caesium, many isotopes of which have half lives measured in years. At this point, fuel is typically retired long before a significant quantity of caesium decays into barium. For this reason we only consider the elements tellurium through caesium in this study. It should also be noted that the majority of thermal fission events occur in the outer rim of the fuel pellet, and a fission product penetration depth of up to 8 $\mu$m in \zirconia\ \cite{degueldre2001behaviour} suggests a large degree of fission product implantation within the oxide. With each nuclear decay comes a change in the chemical and therefore physical behaviour of the atom with its immediate environment. For example, an iodine dopant in \zirconia\ may decay into xenon which will then have a significantly different thermodynamic equilibrium site from the one it inherited.   

Determining the effect of each of these elements in the oxide layer may provide information about the initiation of SCC in fuel cladding. We have therefore adopted a quantum-mechanical calculation approach to model the behaviour of the decay chain elements tellurium through caesium within tetragonal phase zirconia. 

\begin{itemize}
\item We propose that crack initiation on the internal surface of the cladding may be in part due to radioparagenesis of fission products
\item One mechanism is neutron-rich iodine making its way through the monoclinic \zirconia\ before being stopped by the highly passivating tetragonal \zirconia\ closer to the metal interface.
\item The iodine nucleus then decays by beta- particle emission, converting from an iodine to a xenon nucleus.
\item This xenon ion quickly fills its valence shell to the noble gas configuration.
\item The uncharged xenon atom then imposes a large strain on the surrounding \zirconia\ due to the volume mismatch.
\item This strain weakens the monoclinic \zirconia , and promotes crack initiation (new surface relieves the strain imposed by the xenon).
\item The tetragonal \zirconia , now less constrained by the monoclinic layer, expands and becomes less inhibiting to iodine and oxygen ingress.
\item If the iodine partial pressure is high enough relative to the oxygen pressure, the \zirconia\ layer will fail to impede iodine corrosive attack on the zirconium metal.
\end{itemize}

Xenon in a reactor will also eventually decay by beta- emission into caesium, a much more chemically reactive element.

\subsection{Site preference of fission products}

\begin{itemize}

\item \textbf{Tellurium} is a group 6 element like oxygen, but it displays some metallic behaviour.
\item Because of its electronic structure, it may be expected to display preference for the oxygen site in \zirconia .
\item It's metallic properties and low electronegativity, however, suggest that it may be able to fill a cation site instead, but this would require the creation of oxygen vacancies since it has a lower valence than zirconium.
\item \textbf{Iodine} was shown in Chapter 4 to adopt either oxygen and zirconium sites under the right conditions
\item \textbf{Xenon} is a noble gas, but is still able to form compounds with very strong oxidising agents (e.g. XeF4). It's large size (comparison here) may make it unfavourable in both cation and anion sites, thus imposing a large lattice strain.
\item \textbf{Caesium} is a group 1 metal. Its second ionisation energy is very large (removing an electron from a full $p$ sub-shell), likely making it very unfavourable on a zirconium site, only made worse by its size.
\end{itemize}

\subsection{Fission product penetration}

\begin{itemize}
\item Fission products can penetrate up to 10 microns into the cladding, with most deposition occurring at 5 microns (REF)
\item This means we can expect some existing fission products in the cladding before crack-assisted diffusion becomes relevant
\item Therefore some defects will already exist, and the Brouwer diagrams lets us predict what the thermodynamically stable (most likely) ones will be.
\end{itemize}

\section{Methodology}
\subsection{Simulation parameters}

\begin{itemize}
\item energy per atom convergence
\item displacement per atom convergence
\item plane-wave cutoff
\item k-point spacing
\item PBE GGA exchange correlation functional
\end{itemize}

\subsection{Brouwer diagram generation}

\begin{itemize}
\item Defect concentration against oxygen partial pressure 
\item Find Fermi level that leads to charge neutrality
\end{itemize}

\subsection{Defect Volumes}

\begin{itemize}
\item compare constant pressure relaxation of defective to perfect supercell
\end{itemize}

\section{Defect equilibria}
\subsection{Tellurium}

\begin{landscape}
\begin{figure}[htp] % Tellurium
\begin{center}
\begin{tikzpicture}
	\begin{axis}
		[width=11.22cm, xlabel={\ch{log_{10}}($p_{O_{2}}$) (atm)}, ylabel={\ch{log_{10}}([D]) (per f.u.)}, ymin=-10, ymax=0, xmin=-35, xmax=0, legend style={{draw=}, at={(0.30,1.47)}, anchor=north west, legend columns=3, nodes={scale=0.75, transform shape}}]
        \addplot[no marks, draw=blue!70!black] table [x=pO2, y=electrons,]{dat/te_tet_10-5.dat}; \addlegendentry{\ch{e^{'}}}; %\node at (-26.0,-1.9) {\ch{e^{'}}};
        \addplot[no marks, draw=red!85!black] table [x=pO2, y=holes,]{dat/te_tet_10-5.dat}; \addlegendentry{\ch{h^{\textperiodcentered}}}; %\node at (-7,-3.6) {\ch{h^{\textperiodcentered}}};
        \addplot[no marks, draw=black!70!green] table [x=pO2, y=VO{2},]{dat/te_tet_10-5.dat}; \addlegendentry{\ch{V_{O}^{\textperiodcentered\textperiodcentered}}}; %\node at (-26.7,-3.3) {\ch{V_{O}^{\textperiodcentered\textperiodcentered}}};
         \addplot[no marks, draw=black!55!green] table [x=pO2, y=VO{1},]{dat/te_tet_10-5.dat}; \addlegendentry{\ch{V_{O}^{\textperiodcentered}}};
         \addplot[no marks, draw=black!30!green] table [x=pO2, y=VO{0},]{dat/te_tet_10-5.dat}; \addlegendentry{\ch{V_{O}^{x}}};
        \addplot[no marks, draw=yellow!85!blue] table [x=pO2, y=VM{-4},]{dat/te_tet_10-5.dat}; \addlegendentry{\ch{V_{Zr}^{''''}}};
         \addplot[no marks, draw=yellow!75!blue] table [x=pO2, y=VM{-3},]{dat/te_tet_10-5.dat}; \addlegendentry{\ch{V_{Zr}^{'''}}};
         \addplot[no marks, draw=yellow!65!blue] table [x=pO2, y=VM{-2},]{dat/te_tet_10-5.dat}; \addlegendentry{\ch{V_{Zr}^{''}}};
         \addplot[no marks, draw=yellow!55!blue] table [x=pO2, y=VM{-1},]{dat/te_tet_10-5.dat}; \addlegendentry{\ch{V_{Zr}^{'}}};
         \addplot[no marks, draw=yellow!45!blue] table [x=pO2, y=VM{0},]{dat/te_tet_10-5.dat}; \addlegendentry{\ch{V_{Zr}^{x}}};
         \addplot[no marks, draw=red!60!yellow] table [x=pO2, y=Oi{-2},]{dat/te_tet_10-5.dat}; \addlegendentry{\ch{O_{i}^{''}}};
         \addplot[no marks, draw=red!50!yellow] table [x=pO2, y=Oi{-1},]{dat/te_tet_10-5.dat}; \addlegendentry{\ch{O_{i}^{'}}};
         \addplot[no marks, draw=red!40!yellow] table [x=pO2, y=Oi{0},]{dat/te_tet_10-5.dat}; \addlegendentry{\ch{O_{i}^{x}}};
         \addplot[no marks, draw=green!80!pink] table [x=pO2, y=Mi{4},]{dat/te_tet_10-5.dat}; \addlegendentry{\ch{Zr_{i}^{\textperiodcentered\textperiodcentered\textperiodcentered\textperiodcentered}}};
         \addplot[no marks, draw=green!70!pink] table [x=pO2, y=Mi{3},]{dat/te_tet_10-5.dat}; \addlegendentry{\ch{Zr_{i}^{\textperiodcentered\textperiodcentered\textperiodcentered}}};
         \addplot[no marks, draw=green!60!pink] table [x=pO2, y=Mi{2},]{dat/te_tet_10-5.dat}; \addlegendentry{\ch{Zr_{i}^{\textbf{\textperiodcentered\textperiodcentered}}}};
        \addplot[no marks, draw=green!50!pink] table [x=pO2, y=Mi{1},]{dat/te_tet_10-5.dat}; \addlegendentry{\ch{Zr_{i}^{\textperiodcentered}}};
         \addplot[no marks, draw=green!40!pink] table [x=pO2, y=Mi{0},]{dat/te_tet_10-5.dat}; \addlegendentry{\ch{Zr_{i}^{x}}};
         \addplot[no marks, dashed, draw=red!70!black] table [x=pO2, y=Tei{0},]{dat/te_tet_10-5.dat}; \addlegendentry{\ch{Te_{i}^{x}}};
         \addplot[no marks, dashed, draw=red!50!black] table [x=pO2, y=Tei{-1},]{dat/te_tet_10-5.dat}; \addlegendentry{\ch{Te_{i}^{'}}};
        \addplot[no marks, dashed, draw=purple] table [x=pO2, y=Tei{1},]{dat/te_tet_10-5.dat}; \addlegendentry{\ch{Te_{i}^{\textperiodcentered}}};
        \addplot[no marks, dashed, draw=blue!50!white] table [x=pO2, y=TesubO{1},]{dat/te_tet_10-5.dat}; \addlegendentry{\ch{Te_{O}^{\textperiodcentered}}};
        \addplot[no marks, dashed, draw=orange] table [x=pO2, y=TesubO{2},]{dat/te_tet_10-5.dat}; \addlegendentry{\ch{Te_{O}^{\textperiodcentered\textperiodcentered}}};
        \addplot[no marks, dashed, draw=black] table [x=pO2, y=TesubO{3},]{dat/te_tet_10-5.dat}; \addlegendentry{\ch{Te_{O}^{\textperiodcentered\textperiodcentered\textperiodcentered}}};
        \addplot[no marks, dashed, draw=green] table [x=pO2, y=TesubZr{-3},]{dat/te_tet_10-5.dat}; \addlegendentry{\ch{Te_{Zr}^{'''}}};
         \addplot[no marks, dashed, draw=blue] table [x=pO2, y=TesubZr{-4},]{dat/te_tet_10-5.dat}; \addlegendentry{\ch{Te_{Zr}^{''''}}};
         \addplot[no marks, dashed, draw=red] table [x=pO2, y=TesubZr{-5},]{dat/te_tet_10-5.dat}; \addlegendentry{\ch{Te_{Zr}^{'''''}}};
%         \addplot[no marks] table [x=pO2, y=Stoich,]{Te_tet.dat}; \addlegendentry{Stoich};
%\node at (-33.7,-0.5) {\textbf{a)}};
			\end{axis}            
\end{tikzpicture}
\begin{tikzpicture} % TELLURIUM 2
	\begin{axis} % change width to 8.22cm for portrait
		[width=11.22cm, xlabel={\ch{log_{10}}($p_{O_{2}}$) (atm)}, yticklabels={}, ymin=-10, ymax=0, xmin=-35, xmax=0]
        \addplot[no marks, draw=blue!70!black] table [x=pO2, y=electrons,]{dat/te_tet_10-3.dat}; %\node at (-27,-1.7) {\ch{e^{'}}};
        \addplot[no marks, draw=red!85!black] table [x=pO2, y=holes,]{dat/te_tet_10-3.dat}; %\node at (-2.5,-2.1) {\ch{h^{\textperiodcentered}}};
        \addplot[no marks, draw=black!70!green] table [x=pO2, y=VO{2},]{dat/te_tet_10-3.dat}; 
         \addplot[no marks, draw=black!55!green] table [x=pO2, y=VO{1},]{dat/te_tet_10-3.dat}; 
         \addplot[no marks, draw=black!30!green] table [x=pO2, y=VO{0},]{dat/te_tet_10-3.dat}; 
        \addplot[no marks, draw=yellow!85!blue] table [x=pO2, y=VM{-4},]{dat/te_tet_10-3.dat}; 
%         \addplot[no marks, draw=yellow!75!blue] table [x=pO2, y=VM{-3},]{dat/te_tet_10-3.dat}; 
%         \addplot[no marks, draw=yellow!65!blue] table [x=pO2, y=VM{-2},]{dat/te_tet_10-3.dat}; 
%         \addplot[no marks, draw=yellow!55!blue] table [x=pO2, y=VM{-1},]{dat/te_tet_10-3.dat}; 
%         \addplot[no marks, draw=yellow!45!blue] table [x=pO2, y=VM{0},]{dat/te_tet_10-3.dat}; 
%         \addplot[no marks, draw=red!60!yellow] table [x=pO2, y=Oi{-2},]{dat/te_tet_10-3.dat}; 
%         \addplot[no marks, draw=red!50!yellow] table [x=pO2, y=Oi{-1},]{dat/te_tet_10-3.dat}; 
%         \addplot[no marks, draw=red!40!yellow] table [x=pO2, y=Oi{0},]{dat/te_tet_10-3.dat}; 
%         \addplot[no marks, draw=green!80!pink] table [x=pO2, y=Mi{4},]{dat/te_tet_10-3.dat}; 
%         \addplot[no marks, draw=green!70!pink] table [x=pO2, y=Mi{3},]{dat/te_tet_10-3.dat}; 
%         \addplot[no marks, draw=green!60!pink] table [x=pO2, y=Mi{2},]{dat/te_tet_10-3.dat}; 
%         \addplot[no marks, draw=green!50!pink] table [x=pO2, y=Mi{1},]{dat/te_tet_10-3.dat}; 
%         \addplot[no marks, draw=green!40!pink] table [x=pO2, y=Mi{0},]{dat/te_tet_10-3.dat}; 
        \addplot[no marks, dashed, draw=red!70!black] table [x=pO2, y=Tei{0},]{dat/te_tet_10-3.dat}; 
        \addplot[no marks, dashed, draw=red!50!black] table [x=pO2, y=Tei{-1},]{dat/te_tet_10-3.dat}; 
        \addplot[no marks, dashed, draw=purple] table [x=pO2, y=Tei{1},]{dat/te_tet_10-3.dat}; 
        \addplot[no marks, dashed, draw=blue!50!white] table [x=pO2, y=TesubO{1},]{dat/te_tet_10-3.dat}; %\node at (-11,-2.6) {\ch{I_{O}^{\textperiodcentered}}};
        \addplot[no marks, dashed, draw=orange] table [x=pO2, y=TesubO{2},]{dat/te_tet_10-3.dat}; 
        \addplot[no marks, dashed, draw=black] table [x=pO2, y=TesubO{3},]{dat/te_tet_10-3.dat}; 
        \addplot[no marks, dashed, draw=green] table [x=pO2, y=TesubZr{-3},]{dat/te_tet_10-3.dat}; 
        \addplot[no marks, dashed, draw=blue] table [x=pO2, y=TesubZr{-4},]{dat/te_tet_10-3.dat}; 
        \addplot[no marks, dashed, draw=red] table [x=pO2, y=TesubZr{-5},]{dat/te_tet_10-3.dat}; 
%        \addplot[no marks] table [x=pO2, y=Stoich,]{dat/te_tet_10-3.dat}; 
%\node at (-33.7,-0.5) {\textbf{b)}};
			\end{axis}            
\end{tikzpicture}
		\caption{Tetragonal phase Brouwer diagrams of point defects at Tellurium concentrations of a) $10^{-5}$ and b) $10^{-3}$, at a temperature of 1500 K. Space charge = 0}
		\label{figure:telluriumbrouwer-5-3}
	\end{center}
\end{figure}
\end{landscape}

\subsection{Iodine}

\begin{itemize}
\item Should we just reference the Brouwer diagram for iodine again? 
\end{itemize}

\subsection{Xenon}

\begin{itemize}
\item Xenon point defects showed a change in behaviour at high and low oxygen pressures
\end{itemize}

\begin{landscape}
\begin{figure}[htp] % XENON
\begin{center}
\begin{tikzpicture}
	\begin{axis}
		[width=11.22cm, xlabel={\ch{log_{10}}($p_{O_{2}}$) (atm)}, ylabel={\ch{log_{10}}([D]) (per f.u.)}, ymin=-10, ymax=0, xmin=-35, xmax=0, legend style={{draw=}, at={(0.30,1.47)}, anchor=north west, legend columns=3, nodes={scale=0.75, transform shape}}]
        \addplot[no marks, draw=blue!70!black] table [x=pO2, y=electrons,]{dat/xe_tet_10-5.dat}; \addlegendentry{\ch{e^{'}}}; %\node at (-26.0,-1.9) {\ch{e^{'}}};
        \addplot[no marks, draw=red!85!black] table [x=pO2, y=holes,]{dat/xe_tet_10-5.dat}; \addlegendentry{\ch{h^{\textperiodcentered}}}; %\node at (-7,-3.6) {\ch{h^{\textperiodcentered}}};
        \addplot[no marks, draw=black!70!green] table [x=pO2, y=VO{2},]{dat/xe_tet_10-5.dat}; \addlegendentry{\ch{V_{O}^{\textperiodcentered\textperiodcentered}}}; %\node at (-26.7,-3.3) {\ch{V_{O}^{\textperiodcentered\textperiodcentered}}};
         \addplot[no marks, draw=black!55!green] table [x=pO2, y=VO{1},]{dat/xe_tet_10-5.dat}; \addlegendentry{\ch{V_{O}^{\textperiodcentered}}};
         \addplot[no marks, draw=black!30!green] table [x=pO2, y=VO{0},]{dat/xe_tet_10-5.dat}; \addlegendentry{\ch{V_{O}^{x}}};
        \addplot[no marks, draw=yellow!85!blue] table [x=pO2, y=VM{-4},]{dat/xe_tet_10-5.dat}; \addlegendentry{\ch{V_{Zr}^{''''}}};
         \addplot[no marks, draw=yellow!75!blue] table [x=pO2, y=VM{-3},]{dat/xe_tet_10-5.dat}; \addlegendentry{\ch{V_{Zr}^{'''}}};
         \addplot[no marks, draw=yellow!65!blue] table [x=pO2, y=VM{-2},]{dat/xe_tet_10-5.dat}; \addlegendentry{\ch{V_{Zr}^{''}}};
         \addplot[no marks, draw=yellow!55!blue] table [x=pO2, y=VM{-1},]{dat/xe_tet_10-5.dat}; \addlegendentry{\ch{V_{Zr}^{'}}};
         \addplot[no marks, draw=yellow!45!blue] table [x=pO2, y=VM{0},]{dat/xe_tet_10-5.dat}; \addlegendentry{\ch{V_{Zr}^{x}}};
         \addplot[no marks, draw=red!60!yellow] table [x=pO2, y=Oi{-2},]{dat/xe_tet_10-5.dat}; \addlegendentry{\ch{O_{i}^{''}}};
         \addplot[no marks, draw=red!50!yellow] table [x=pO2, y=Oi{-1},]{dat/xe_tet_10-5.dat}; \addlegendentry{\ch{O_{i}^{'}}};
         \addplot[no marks, draw=red!40!yellow] table [x=pO2, y=Oi{0},]{dat/xe_tet_10-5.dat}; \addlegendentry{\ch{O_{i}^{x}}};
         \addplot[no marks, draw=green!80!pink] table [x=pO2, y=Mi{4},]{dat/xe_tet_10-5.dat}; \addlegendentry{\ch{Zr_{i}^{\textperiodcentered\textperiodcentered\textperiodcentered\textperiodcentered}}};
         \addplot[no marks, draw=green!70!pink] table [x=pO2, y=Mi{3},]{dat/xe_tet_10-5.dat}; \addlegendentry{\ch{Zr_{i}^{\textperiodcentered\textperiodcentered\textperiodcentered}}};
         \addplot[no marks, draw=green!60!pink] table [x=pO2, y=Mi{2},]{dat/xe_tet_10-5.dat}; \addlegendentry{\ch{Zr_{i}^{\textbf{\textperiodcentered\textperiodcentered}}}};
        \addplot[no marks, draw=green!50!pink] table [x=pO2, y=Mi{1},]{dat/xe_tet_10-5.dat}; \addlegendentry{\ch{Zr_{i}^{\textperiodcentered}}};
         \addplot[no marks, draw=green!40!pink] table [x=pO2, y=Mi{0},]{dat/xe_tet_10-5.dat}; \addlegendentry{\ch{Zr_{i}^{x}}};
         \addplot[no marks, dashed, draw=red!70!black] table [x=pO2, y=Xei{0},]{dat/xe_tet_10-5.dat}; \addlegendentry{\ch{Xe_{i}^{x}}};
         \addplot[no marks, dashed, draw=red!50!black] table [x=pO2, y=Xei{-1},]{dat/xe_tet_10-5.dat}; \addlegendentry{\ch{Xe_{i}^{'}}};
        \addplot[no marks, dashed, draw=purple] table [x=pO2, y=Xei{1},]{dat/xe_tet_10-5.dat}; \addlegendentry{\ch{Xe_{i}^{\textperiodcentered}}};
        \addplot[no marks, dashed, draw=blue!50!white] table [x=pO2, y=XesubO{1},]{dat/xe_tet_10-5.dat}; \addlegendentry{\ch{Xe_{O}^{\textperiodcentered}}};
        \addplot[no marks, dashed, draw=orange] table [x=pO2, y=XesubO{2},]{dat/xe_tet_10-5.dat}; \addlegendentry{\ch{Xe_{O}^{\textperiodcentered\textperiodcentered}}};
        \addplot[no marks, dashed, draw=black] table [x=pO2, y=XesubO{3},]{dat/xe_tet_10-5.dat}; \addlegendentry{\ch{Xe_{O}^{\textperiodcentered\textperiodcentered\textperiodcentered}}};
        \addplot[no marks, dashed, draw=green] table [x=pO2, y=XesubZr{-3},]{dat/xe_tet_10-5.dat}; \addlegendentry{\ch{Xe_{Zr}^{'''}}};
         \addplot[no marks, dashed, draw=blue] table [x=pO2, y=XesubZr{-4},]{dat/xe_tet_10-5.dat}; \addlegendentry{\ch{Xe_{Zr}^{''''}}};
         \addplot[no marks, dashed, draw=red] table [x=pO2, y=XesubZr{-5},]{dat/xe_tet_10-5.dat}; \addlegendentry{\ch{Xe_{Zr}^{'''''}}};
%         \addplot[no marks] table [x=pO2, y=Stoich,]{xe_tet.dat}; \addlegendentry{Stoich};
%\node at (-33.7,-0.5) {\textbf{a)}};
			\end{axis}            
\end{tikzpicture}
\begin{tikzpicture} % XENON 2
	\begin{axis} % change width to 8.22cm for portrait
		[width=11.22cm, xlabel={\ch{log_{10}}($p_{O_{2}}$) (atm)}, yticklabels={}, ymin=-10, ymax=0, xmin=-35, xmax=0]
        \addplot[no marks, draw=blue!70!black] table [x=pO2, y=electrons,]{dat/xe_tet_10-3.dat}; %\node at (-27,-1.7) {\ch{e^{'}}};
        \addplot[no marks, draw=red!85!black] table [x=pO2, y=holes,]{dat/xe_tet_10-3.dat}; %\node at (-2.5,-2.1) {\ch{h^{\textperiodcentered}}};
        \addplot[no marks, draw=black!70!green] table [x=pO2, y=VO{2},]{dat/xe_tet_10-3.dat}; 
         \addplot[no marks, draw=black!55!green] table [x=pO2, y=VO{1},]{dat/xe_tet_10-3.dat}; 
         \addplot[no marks, draw=black!30!green] table [x=pO2, y=VO{0},]{dat/xe_tet_10-3.dat}; 
        \addplot[no marks, draw=yellow!85!blue] table [x=pO2, y=VM{-4},]{dat/xe_tet_10-3.dat}; 
%         \addplot[no marks, draw=yellow!75!blue] table [x=pO2, y=VM{-3},]{dat/xe_tet_10-3.dat}; 
%         \addplot[no marks, draw=yellow!65!blue] table [x=pO2, y=VM{-2},]{dat/xe_tet_10-3.dat}; 
%         \addplot[no marks, draw=yellow!55!blue] table [x=pO2, y=VM{-1},]{dat/xe_tet_10-3.dat}; 
%         \addplot[no marks, draw=yellow!45!blue] table [x=pO2, y=VM{0},]{dat/xe_tet_10-3.dat}; 
%         \addplot[no marks, draw=red!60!yellow] table [x=pO2, y=Oi{-2},]{dat/xe_tet_10-3.dat}; 
%         \addplot[no marks, draw=red!50!yellow] table [x=pO2, y=Oi{-1},]{dat/xe_tet_10-3.dat}; 
%         \addplot[no marks, draw=red!40!yellow] table [x=pO2, y=Oi{0},]{dat/xe_tet_10-3.dat}; 
%         \addplot[no marks, draw=green!80!pink] table [x=pO2, y=Mi{4},]{dat/xe_tet_10-3.dat}; 
%         \addplot[no marks, draw=green!70!pink] table [x=pO2, y=Mi{3},]{dat/xe_tet_10-3.dat}; 
%         \addplot[no marks, draw=green!60!pink] table [x=pO2, y=Mi{2},]{dat/xe_tet_10-3.dat}; 
%         \addplot[no marks, draw=green!50!pink] table [x=pO2, y=Mi{1},]{dat/xe_tet_10-3.dat}; 
%         \addplot[no marks, draw=green!40!pink] table [x=pO2, y=Mi{0},]{dat/xe_tet_10-3.dat}; 
        \addplot[no marks, dashed, draw=red!70!black] table [x=pO2, y=Xei{0},]{dat/xe_tet_10-3.dat}; 
        \addplot[no marks, dashed, draw=red!50!black] table [x=pO2, y=Xei{-1},]{dat/xe_tet_10-3.dat}; 
        \addplot[no marks, dashed, draw=purple] table [x=pO2, y=Xei{1},]{dat/xe_tet_10-3.dat}; 
        \addplot[no marks, dashed, draw=blue!50!white] table [x=pO2, y=XesubO{1},]{dat/xe_tet_10-3.dat}; %\node at (-11,-2.6) {\ch{I_{O}^{\textperiodcentered}}};
        \addplot[no marks, dashed, draw=orange] table [x=pO2, y=XesubO{2},]{dat/xe_tet_10-3.dat}; 
        \addplot[no marks, dashed, draw=black] table [x=pO2, y=XesubO{3},]{dat/xe_tet_10-3.dat}; 
        \addplot[no marks, dashed, draw=green] table [x=pO2, y=XesubZr{-3},]{dat/xe_tet_10-3.dat}; 
        \addplot[no marks, dashed, draw=blue] table [x=pO2, y=XesubZr{-4},]{dat/xe_tet_10-3.dat}; 
        \addplot[no marks, dashed, draw=red] table [x=pO2, y=XesubZr{-5},]{dat/xe_tet_10-3.dat}; 
%        \addplot[no marks] table [x=pO2, y=Stoich,]{dat/xe_tet_10-3.dat}; 
%\node at (-33.7,-0.5) {\textbf{b)}};
			\end{axis}            
\end{tikzpicture}
		\caption{Tetragonal phase Brouwer diagrams of point defects at Xenon concentrations of a) $10^{-5}$ and b) $10^{-3}$, at a temperature of 1500 K. Space charge = 0}
		%\label{figure:tikzbrouwerconctet}
	\end{center}
\end{figure}
\end{landscape}

\subsection{Caesium}

\begin{itemize}
\item Cs point defects didn't show much change in behaviour
\item Defects behaviour strongly follows single ionisation preference (as expected).
\end{itemize}

\begin{landscape}
\begin{figure}[htp] % CAESIUM
\begin{center}
\begin{tikzpicture}
	\begin{axis}
		[width=11.22cm, xlabel={\ch{log_{10}}($p_{O_{2}}$) (atm)}, ylabel={\ch{log_{10}}([D]) (per f.u.)}, ymin=-10, ymax=0, xmin=-35, xmax=0, legend style={{draw=}, at={(0.30,1.47)}, anchor=north west, legend columns=3, nodes={scale=0.75, transform shape}}]
        \addplot[no marks, draw=blue!70!black] table [x=pO2, y=electrons,]{dat/cs_tet_10-5.dat}; \addlegendentry{\ch{e^{'}}}; %\node at (-26.0,-1.9) {\ch{e^{'}}};
        \addplot[no marks, draw=red!85!black] table [x=pO2, y=holes,]{dat/cs_tet_10-5.dat}; \addlegendentry{\ch{h^{\textperiodcentered}}}; %\node at (-7,-3.6) {\ch{h^{\textperiodcentered}}};
        \addplot[no marks, draw=black!70!green] table [x=pO2, y=VO{2},]{dat/cs_tet_10-5.dat}; \addlegendentry{\ch{V_{O}^{\textperiodcentered\textperiodcentered}}}; %\node at (-26.7,-3.3) {\ch{V_{O}^{\textperiodcentered\textperiodcentered}}};
         \addplot[no marks, draw=black!55!green] table [x=pO2, y=VO{1},]{dat/cs_tet_10-5.dat}; \addlegendentry{\ch{V_{O}^{\textperiodcentered}}};
         \addplot[no marks, draw=black!30!green] table [x=pO2, y=VO{0},]{dat/cs_tet_10-5.dat}; \addlegendentry{\ch{V_{O}^{x}}};
        \addplot[no marks, draw=yellow!85!blue] table [x=pO2, y=VM{-4},]{dat/cs_tet_10-5.dat}; \addlegendentry{\ch{V_{Zr}^{''''}}};
         \addplot[no marks, draw=yellow!75!blue] table [x=pO2, y=VM{-3},]{dat/cs_tet_10-5.dat}; \addlegendentry{\ch{V_{Zr}^{'''}}};
         \addplot[no marks, draw=yellow!65!blue] table [x=pO2, y=VM{-2},]{dat/cs_tet_10-5.dat}; \addlegendentry{\ch{V_{Zr}^{''}}};
         \addplot[no marks, draw=yellow!55!blue] table [x=pO2, y=VM{-1},]{dat/cs_tet_10-5.dat}; \addlegendentry{\ch{V_{Zr}^{'}}};
         \addplot[no marks, draw=yellow!45!blue] table [x=pO2, y=VM{0},]{dat/cs_tet_10-5.dat}; \addlegendentry{\ch{V_{Zr}^{x}}};
         \addplot[no marks, draw=red!60!yellow] table [x=pO2, y=Oi{-2},]{dat/cs_tet_10-5.dat}; \addlegendentry{\ch{O_{i}^{''}}};
         \addplot[no marks, draw=red!50!yellow] table [x=pO2, y=Oi{-1},]{dat/cs_tet_10-5.dat}; \addlegendentry{\ch{O_{i}^{'}}};
         \addplot[no marks, draw=red!40!yellow] table [x=pO2, y=Oi{0},]{dat/cs_tet_10-5.dat}; \addlegendentry{\ch{O_{i}^{x}}};
         \addplot[no marks, draw=green!80!pink] table [x=pO2, y=Mi{4},]{dat/cs_tet_10-5.dat}; \addlegendentry{\ch{Zr_{i}^{\textperiodcentered\textperiodcentered\textperiodcentered\textperiodcentered}}};
         \addplot[no marks, draw=green!70!pink] table [x=pO2, y=Mi{3},]{dat/cs_tet_10-5.dat}; \addlegendentry{\ch{Zr_{i}^{\textperiodcentered\textperiodcentered\textperiodcentered}}};
         \addplot[no marks, draw=green!60!pink] table [x=pO2, y=Mi{2},]{dat/cs_tet_10-5.dat}; \addlegendentry{\ch{Zr_{i}^{\textbf{\textperiodcentered\textperiodcentered}}}};
        \addplot[no marks, draw=green!50!pink] table [x=pO2, y=Mi{1},]{dat/cs_tet_10-5.dat}; \addlegendentry{\ch{Zr_{i}^{\textperiodcentered}}};
         \addplot[no marks, draw=green!40!pink] table [x=pO2, y=Mi{0},]{dat/cs_tet_10-5.dat}; \addlegendentry{\ch{Zr_{i}^{x}}};
         \addplot[no marks, dashed, draw=red!70!black] table [x=pO2, y=Csi{0},]{dat/cs_tet_10-5.dat}; \addlegendentry{\ch{Cs_{i}^{x}}};
         \addplot[no marks, dashed, draw=red!50!black] table [x=pO2, y=Csi{-1},]{dat/cs_tet_10-5.dat}; \addlegendentry{\ch{Cs_{i}^{'}}};
        \addplot[no marks, dashed, draw=purple] table [x=pO2, y=Csi{1},]{dat/cs_tet_10-5.dat}; \addlegendentry{\ch{Cs_{i}^{\textperiodcentered}}};
        \addplot[no marks, dashed, draw=blue!50!white] table [x=pO2, y=CssubO{1},]{dat/cs_tet_10-5.dat}; \addlegendentry{\ch{Cs_{O}^{\textperiodcentered}}};
        \addplot[no marks, dashed, draw=green!60!black] table [x=pO2, y=CssubO{2},]{dat/cs_tet_10-5.dat}; \addlegendentry{\ch{Cs_{O}^{\textperiodcentered\textperiodcentered}}};
        \addplot[no marks, dashed, draw=black] table [x=pO2, y=CssubO{3},]{dat/cs_tet_10-5.dat}; \addlegendentry{\ch{Cs_{O}^{\textperiodcentered\textperiodcentered\textperiodcentered}}};
        \addplot[no marks, dashed, draw=orange!80!black] table [x=pO2, y=CssubZr{-3},]{dat/cs_tet_10-5.dat}; \addlegendentry{\ch{Cs_{Zr}^{'''}}};
         \addplot[no marks, dashed, draw=pink] table [x=pO2, y=CssubZr{-4},]{dat/cs_tet_10-5.dat}; \addlegendentry{\ch{Cs_{Zr}^{''''}}};
         \addplot[no marks, dashed, draw=purple] table [x=pO2, y=CssubZr{-5},]{dat/cs_tet_10-5.dat}; \addlegendentry{\ch{Cs_{Zr}^{'''''}}};
%         \addplot[no marks] table [x=pO2, y=Stoich,]{cs_tet.dat}; \addlegendentry{Stoich};
%\node at (-33.7,-0.5) {\textbf{a)}};
			\end{axis}            
\end{tikzpicture}
\begin{tikzpicture} % CAESIUM 2
	\begin{axis}
		[width=11.22cm, xlabel={\ch{log_{10}}($p_{O_{2}}$) (atm)}, yticklabels={}, ymin=-10, ymax=0, xmin=-35, xmax=0]
        \addplot[no marks, draw=blue!70!black] table [x=pO2, y=electrons,]{dat/cs_tet_10-3.dat}; %\node at (-27,-1.7) {\ch{e^{'}}};
        \addplot[no marks, draw=red!85!black] table [x=pO2, y=holes,]{dat/cs_tet_10-3.dat}; %\node at (-2.5,-2.1) {\ch{h^{\textperiodcentered}}};
        \addplot[no marks, draw=black!70!green] table [x=pO2, y=VO{2},]{dat/cs_tet_10-3.dat}; 
         \addplot[no marks, draw=black!55!green] table [x=pO2, y=VO{1},]{dat/cs_tet_10-3.dat}; 
         \addplot[no marks, draw=black!30!green] table [x=pO2, y=VO{0},]{dat/cs_tet_10-3.dat}; 
        \addplot[no marks, draw=yellow!85!blue] table [x=pO2, y=VM{-4},]{dat/cs_tet_10-3.dat}; 
%         \addplot[no marks, draw=yellow!75!blue] table [x=pO2, y=VM{-3},]{dat/cs_tet_10-3.dat}; 
%         \addplot[no marks, draw=yellow!65!blue] table [x=pO2, y=VM{-2},]{dat/cs_tet_10-3.dat}; 
%         \addplot[no marks, draw=yellow!55!blue] table [x=pO2, y=VM{-1},]{dat/cs_tet_10-3.dat}; 
%         \addplot[no marks, draw=yellow!45!blue] table [x=pO2, y=VM{0},]{dat/cs_tet_10-3.dat}; 
%         \addplot[no marks, draw=red!60!yellow] table [x=pO2, y=Oi{-2},]{dat/cs_tet_10-3.dat}; 
%         \addplot[no marks, draw=red!50!yellow] table [x=pO2, y=Oi{-1},]{dat/cs_tet_10-3.dat}; 
%         \addplot[no marks, draw=red!40!yellow] table [x=pO2, y=Oi{0},]{dat/cs_tet_10-3.dat}; 
%         \addplot[no marks, draw=green!80!pink] table [x=pO2, y=Mi{4},]{dat/cs_tet_10-3.dat}; 
%         \addplot[no marks, draw=green!70!pink] table [x=pO2, y=Mi{3},]{dat/cs_tet_10-3.dat}; 
%         \addplot[no marks, draw=green!60!pink] table [x=pO2, y=Mi{2},]{dat/cs_tet_10-3.dat}; 
%         \addplot[no marks, draw=green!50!pink] table [x=pO2, y=Mi{1},]{dat/cs_tet_10-3.dat}; 
%         \addplot[no marks, draw=green!40!pink] table [x=pO2, y=Mi{0},]{dat/cs_tet_10-3.dat}; 
        \addplot[no marks, dashed, draw=red!70!black] table [x=pO2, y=Csi{0},]{dat/cs_tet_10-3.dat}; 
        \addplot[no marks, dashed, draw=red!50!black] table [x=pO2, y=Csi{-1},]{dat/cs_tet_10-3.dat}; 
        \addplot[no marks, dashed, draw=purple] table [x=pO2, y=Csi{1},]{dat/cs_tet_10-3.dat}; 
        \addplot[no marks, dashed, draw=blue!50!white] table [x=pO2, y=CssubO{1},]{dat/cs_tet_10-3.dat}; %\node at (-11,-2.6) {\ch{I_{O}^{\textperiodcentered}}};
        \addplot[no marks, dashed, draw=green!60!black] table [x=pO2, y=CssubO{2},]{dat/cs_tet_10-3.dat}; 
        \addplot[no marks, dashed, draw=black] table [x=pO2, y=CssubO{3},]{dat/cs_tet_10-3.dat}; 
        \addplot[no marks, dashed, draw=orange!80!black] table [x=pO2, y=CssubZr{-3},]{dat/cs_tet_10-3.dat}; 
        \addplot[no marks, dashed, draw=pink] table [x=pO2, y=CssubZr{-4},]{dat/cs_tet_10-3.dat}; 
        \addplot[no marks, dashed, draw=purple] table [x=pO2, y=CssubZr{-5},]{dat/cs_tet_10-3.dat}; 
%        \addplot[no marks] table [x=pO2, y=Stoich,]{dat/cs_tet_10-3.dat}; 
%\node at (-33.7,-0.5) {\textbf{b)}};
			\end{axis}            
\end{tikzpicture}
		\caption{Tetragonal phase Brouwer diagrams of point defects at caesium concentrations of a) $10^{-5}$ and b) $10^{-3}$, at a temperature of 1500 K. Space charge = 0}
		%\label{figure:tikzbrouwerconctet}
	\end{center}
\end{figure}
\end{landscape}

\section{Summary}
 % Good
\chapter{Summary and further work}

\label{ch:summary}

\section{Structure properties and intrinsic defects in \zirconia}

DFT calculations of non-defective \zirconia\ predicted the correct order of phase stability for the monoclinic, tetragonal and cubic phases. Calculated lattice parameters of each phase also agreed to within 2\% of experimental values. Calculated band gaps of each phase were underestimated by approximately 2 eV, which is typical when modelling using GGA-DFT. A Hubbard +U study revealed that the calculated band gaps could be increased by up to 1 eV, but that the symmetry of the supercells would be distorted in the process.

Helmholtz free energy calculations at temperatures ranging from 0 to 2500 K showed that the lowest energy phase changes from monoclinic to tetragonal, but not from tetragonal to cubic. Furthermore, the predicted transition temperature of the monoclinic-tetragonal phase change is underestimated by almost 1100 K. This is attributed to lack of thermal expansion in the constant-volume harmonic model but could also be a consequence of the kinetic barrier of the phase transformation.

Elastic constants were calculated and reported for each phase. The cubic phase was predicted to have the highest stiffness, followed by monoclinic and then tetragonal. The elastic constants were used to calculate the bulk modulus of each phase, and it was shown that the monoclinic and tetragonal bulk moduli agreed with experimental values to within 5\%. The calculated bulk modulus in the cubic phase could not be compared to an experimental value in pure \zirconia , but was 15\% larger than that of YSZ.

Formation energy calculations showed that Zr vacancies in all three phases are fully-charged (\ch{V_{Zr}^{''''}}) over most Fermi levels, while O vacancies change from \ch{V_{O}^{**}} at low Fermi levels to \ch{V_{O}^{x}} when the Fermi level is larger than 3 eV. The defect cluster with the lowest formation energy per defect was the fully-charged Schottky, followed by O Frenkel defects and then Zr Frenkel defects. This can be partly attributed to defect sizes, with Schottky defects predicted to have the smallest defect volumes in all phases, again followed by O Frenkel defects and then Zr Frenkel defects.

Brouwer diagrams were constructed for each phase, and showed that the defect equilibria are dominated by \ch{V_{Zr}^{''''}}, \ch{V_{O}^{**}}, \ch{h^{*}} and \ch{e^{'}} defects. Interstitial defects appear only at very low concentrations relative to these defects. This suggests that Schottky defects will be the main cluster defect near stoichiometry. At high $p_{O_{2}}$, \ch{V_{Zr}^{''''}} is charge-compensated by \ch{h^{*}} defects, while at low $p_{O_{2}}$, \ch{V_{O}^{**}} is charge-compensated by \ch{e^{'}} defects.

Cubic \zirconia\ supercells sometimes exhibited a loss of symmetry when performing defect energy minimisations. This behaviour and anomalous results from other calculations led to the conclusion that the cubic phase might not be modelled accurately with present DFT methods. Work by Burr \emph{et al.} \cite{burr2017importance} supports this result. In addition, the high temperatures required to stabilise the cubic phase limit the ability to examine defect equilibria due to the high intrinsic defect concentrations. Cubic phase Brouwer diagrams were therefore not constructed in subsequent extrinsic defect studies.

\section{Iodine doping and oxygen competition}

Iodine point defects were studied in the three phases to investigate defect energies and these were used to construct Brouwer diagrams in the monoclinic and tetragonal phases. It was found that in monoclinic \zirconia , iodine has lower incorporation energies onto both Zr and O sites than in the tetragonal phase. In all phases, \ch{I_{O}} defects exhibit the smallest incorporation energies, followed by \ch{I_{Zr}} and then \ch{I_{i}}. 

Four unique interstitial sites for iodine were found in the monoclinic phase, and two in the tetragonal and cubic phases. It was found that the lowest energy interstitial sites changed in the tetragonal and cubic phases depending on the Fermi level, whereas only one site was favourable in the monoclinic phase. Iodine interstitial defects will only exist as \ch{I_{i}^{*}} at low Fermi levels or \ch{I_{i}^{'}} at high Fermi levels, whereas \ch{I_{i}^{x}} defects will have a higher formation energy regardless of the Fermi level. At the O site, iodine will form \ch{I_{O}^{*}} and to a lesser extent, \ch{I_{O}^{***}} defects. At the Zr site, iodine will form \ch{I_{Zr}^{'''}} defects. 

Brouwer diagrams showed that there is competition between iodine and oxygen for anion sites in \zirconia , and that this competition is dependent on both phase and oxygen pressure. It was found that higher oxygen pressures will reduce the equilibrium concentration of \ch{I_{O}} and \ch{I_{i}} defects in both phases, but with a more significant effect in the tetragonal phase. A combination of high oxygen pressure and tetragonal phase led to oxygen out-competing iodine for anion sites. This will reduce diffusion rates of iodine in the oxide layer.

% iodine interstitial defects are predicted to be mostly \ch{I_{i}^{*}}, and will be observed in higher concentrations in monoclinic \zirconia . This is likely due to the larger size of the interstitial sites and therefore better accommodation of the large iodine atom.

\section{Decay chain elements in \zirconia}

I nuclei produced during nuclear fission are unstable and will therefore be part of a decay chain. Decay of Te precursors are the major source of I. Furthermore, Xe and Cs are predicted to be present in significant concentrations based on decay rates and fission yields. Decay rates of Te and I isotopes were also found to be commensurate with the time taken for PCI failures to occur following a power ramp, hinting that these phenomena may be related. From the iodine study, tetragonal phase \zirconia\ was predicted to provide a greater barrier effect against iodine. Thus, defect equilibria and defect volumes of decay chain elements Te to Cs were investigated in tetragonal \zirconia . 

Brouwer diagrams predicted low concentrations of interstitial defects relative to O and Zr substitutional defects. At the O site, the defect evolution is predicted to be \ch{Te_{O}^{**}} $\rightarrow$ \ch{I_{O}^{*}} $\rightarrow$ \ch{Xe_{O}^{**}} $\rightarrow$ \ch{Cs_{O}^{**}} while at the Zr site, the Brouwer diagrams predict \ch{Te_{Zr}^{'''}} $\rightarrow$ \ch{I_{Zr}^{'''}} $\rightarrow$ \ch{Xe_{Zr}^{''''}} $\rightarrow$ \ch{Cs_{Zr}^{'''}}.

As Te decays towards Xe, the defects produced have progressively larger volumes. All substitutional defects have positive volumes ranging from +21 to +46 \r{A}$^{3}$ relative to the non-defective crystal, with the largest major defect being \ch{I_{O}^{*}} (46.94 \r{A}$^{3}$). All O substitutional defects were found to be larger than those on the Zr site. The stresses generated by defects on a constrained supercell were also calculated and shown to follow the same trends as the defect volumes.

Stresses imposed by implanted fission products and their decay chain elements will promote the formation of new surfaces. During a power ramp, the rate of fission product implantation within the oxide will increase. The largest stresses will occur in the hours following the ramp when I concentrations reach a maximum level. Stresses will begin to fall as more I nuclei decay into Xe and Cs, which have smaller defect volumes. This mechanism is proposed as a contributing factor to PCI failures, as larger ramps are more likely to lead to failure and the rate of new crack formation may exceed the rate at which a passivating oxide is developed.

\section{Further work}

\subsection{Temperature effects in intrinsic defect simulations}

ZrO$_{2}$ crystal structures are sensitive to changes in interatomic spacing, sometimes causing a loss of supercell symmetry when defects are modelled. There is a need to accurately represent anisotropic effects such as thermal expansion in the different phases over a broad temperature range. This can be achieved by developing potentials and conducting classical or quantum MD simulations to test their performance through comparison to experimental data. Quantum MD simulations, however, are very computationally expensive. A good starting point is to perform phonon calculations of the different ZrO$_{2}$ phases in conjunction with the quasi-harmonic method. This is an extension to the method outlined in § \ref{helmholtz_method} which can be used to find the ground state lattice positions (and therefore interatomic spacings) at different temperatures. These values are used to determine thermal expansion coefficients of ZrO$_{2}$ along each lattice vector. Potentials can then be developed with these thermal expansion coefficients, which will be able to reproduce the anisotropic effects in different phases.

\subsection{Fission product empirical potential}

In order to study the interaction of fission products with larger features in the cladding microstructure, such as dislocations and grain boundaries, it is necessary to develop empirical potentials for use in molecular dynamics simulations. Grain boundary transport is of particular interest, but this would require something on the order of 10$^{4}$ atoms to represent a system large enough to develop a reasonable degree of accuracy. This cannot currently be achieved using DFT due to the significant amount of computing resources required to run such a simulation. 

The development of an iodine and xenon potential with \zirconia\ should be prioritised in order to run simulations to determine the migration of iodine within \zirconia, followed by the behaviour of xenon at sites occupied by iodine.

\subsection{Zr/ZrO/\zirconia\ interface study}

The inner oxide is not a homogeneous structure, as described in Chapter \ref{ch:crystallography}. Figure \ref{fig:zro_interface} shows the existence of a ZrO phase up to 200 nm thick at the interface between \zirconia\ and Zr metal. The presence of ZrO and even oxygen-saturated Zr metal will have an effect on the thermodynamic equilibria of different fission products. An interface study can be conducted using DFT, to determine stresses at the interfaces of Zr and ZrO, and ZrO and \zirconia . Studying the aggregate effect of these interfaces on fission product behaviour may require larger molecular dynamics simulations. 

The crystal structure of the ZrO phase has been studied using both simulation and high-resolution electron microscopy, with two likely crystal structures being proposed \cite{Nicholls2015}. Further atomistic studies must be conducted to determine the stability of each crystal structure of ZrO when constrained by \zirconia\ and oxygen-saturated Zr metal interfaces (i.e. can we determine if stress stabilises the ZrO layer?).

\begin{figure}[ht] % ZrO interface
    \centering
    \includegraphics[height=9cm]{images/zro_interface.png}
    \caption[STEM image of a Zr-1.0\%Nb sample oxidised in simulated PWR water at 360 C for 120 days.]{STEM image of a Zr-1.0\%Nb sample oxidised in simulated PWR water at 360 C for 120 days. Taken from \cite{inproceedings}.}
    \label{fig:zro_interface}
\end{figure}

\subsubsection{Bromine}

Bromine is another halogen atom produced from fission, although in smaller quantities than iodine. Since bromine has similar chemical behaviour to iodine, though it is somewhat smaller (I$^{-}$ radius = 2.20 \r{A}, Br$^{-}$ radius = 1.96 \r{A} \cite{Shannon1976}), it would be of interest to conduct a defect study with bromine in \zirconia\ similar to that in Chapter \ref{ch:results2}. Iodine produced defects which were in either the +1 or -1 oxidation states (e.g. \ch{I_{Zr}^{'''}} and \ch{I_{O}^{*}}). This would help determine if Br defects also exhibit this behaviour, or if the 0 oxidation state demonstrates greater stability at certain sites. Furthermore, the pathway to Br isotopes will be different to I in terms of decay rates and this must also be considered.

\subsection{Defect volumes}

Defect volumes are useful quantities for comparing different defects and explaining their behaviour, however, the calculation of defect volumes for charged defects in DFT models is a contentious topic. While defect volumes of charged point defects are often reported in the literature, reported values are often unphysically large \cite{Bruneval2015}. For this reason, the charged defect volumes reported in this thesis used the volume of an equivalently charged non-defective supercell as the reference volume. This method has been shown to yield more reasonable defect volumes for charged defects \cite{goyal2017conundrum}, but this is still only a rudimentary correction. 

One method for obtaining more accurate defect volumes is to use larger supercells (to reduce finite size effects and self-interaction errors across the periodic boundary). Another possibility is using hybrid-AE functionals which reproduce a more accurate electronic band structure, thereby reducing energy errors and improving the accuracy of interatomic forces when adding or removing electrons in a defective system with non-zero charge \cite{Brothers2008}.
% \cite{He2006} Jellium

\subsection{Experimental work}

\subsubsection{Tetragonal phase stabilisation}

A key finding in this thesis is that tetragonal phase \zirconia\ (as opposed to monoclinic) provides the barrier effect against corrosive fission products. It would therefore be useful to conduct experiments on claddings designed to maximise the content of tetragonal phase \zirconia\ in the internal oxide layer. While this can be achieved with a tetragonal stabilising dopant (e.g. scandium or yttrium), this will increase the concentration of oxygen vacancies which may lead to reduced corrosion performance. 

Another method to increase the proportion of tetragonal phase in the oxide layer is to reduce the Zr grain size. This is an attractive option because the chemical composition of the cladding will be unchanged, and the increased strength does not come at the cost of ductility. The very small grain sizes required (nanoscale) will, however, negatively impact the creep resistance of the cladding, so if this is not a realistic modification for the entire clad, it may be better to refine only the Zr liner grain size. 

\subsubsection{Oxygen environment control}

A key result in Chapter \ref{ch:results2} was that at a high enough $p_{O_{2}}$ above stoichiometry, the concentration of \ch{I_{O}} and \ch{I_{i}} defects began to fall significantly, indicating competition between oxygen and iodine for anion sites in tetragonal \zirconia . Increasing the $p_{O_{2}}$ in fuel pins may therefore impede the bulk diffusion of iodine through the oxide layer, slowing the corrosion process. In-pile irradiation tests (e.g. power ramps, high burn-up behaviour) to examine the effect of increasing the oxygen content in fuel pins would provide useful data on PCI performance. Presently, fuel pins are filled with helium because it is an inert gas, it does not affect the neutronics within the core and has a high thermal conductivity (compared to other gases). One option for varying oxygen content in a fuel pin is to use a fill gas which is a mixture of helium and oxygen. Although oxygen (most of which is \ch{O^{16}_{8}}) is not inert and has a lower thermal conductivity than helium, it is highly resistant to neutron activation due to a combination of very low thermal neutron absorption cross-section and three stable isotopes (i.e. three consecutive neutron captures are required to produce an unstable \ch{O^{19}_{8}} nucleus). Adding oxygen to the fill gas is also a much simpler (and therefore cheaper) method of increasing oxygen content compared to changing the stoichiometry of the fuel.

 
%\chapter{Future work}

\label{ch:future}

\section{Iodine empirical potential}
\section{Grain boundary transport}
\section{Zr/ZrO/ZrO2 interface study}
 % Moved to summary.tex

\setstretch{1.6}
\addcontentsline{toc}{chapter}{References}
\label{References}
\renewcommand\bibname{References}
\bibliographystyle{unsrt}
\bibliography{Mendeley.bib}

\setstretch{2.1}
\appendix
% Appendices come here
\addcontentsline{toc}{chapter}{Appendix}
\label{Appendix}

\chapter{ParaSweep}

ParaSweep is a generalised sensitivity analysis visualisation tool which was developed during this project. Initially, it was built to help visualise the effects of changing single parameters in Brouwer diagrams, such as temperature or concentration of defects. Generalisation of the sweeping parameters was a natural extension of this tool, allowing sensitivity analyses to be performed in conjunction with any program that has variable inputs. The program has since been open-sourced and is available at \href{https://github.com/v1thesource/ParaSweep}{https://github.com/v1thesource/ParaSweep} along with supporting documentation.

\chapter{CASTEP and HPC Scripts}
\label{castep_scripts}

Throughout the course of this project, many useful scripts were created to help with preparing CASTEP jobs and analysing their outputs. These scripts have been made available online and for free at \href{https://github.com/v1thesource/CASTEP}{https://github.com/v1thesource/CASTEP}. The purpose of open-sourcing these scripts is to simplify the experience for new users of CASTEP and help them save a considerable amount of time.

\chapter{Steady state calculation of iodine inventory}

Consider a reactor such as the AP1000 with thermal efficiency $\eta_{t}$ = 1/3 and electrical power output $P_{e}$ = 1000 MWe. Thermal power output ($P_{t}$) is therefore:
\begin{equation}
P_{t} = \frac{P_{e}}{\eta_{t}} = \frac{1000}{1/3} = 3000 \ \ch{MWt}
\end{equation}
Assuming all thermal energy comes from fission events with each fission of U$^{235}$ releasing 170 MeV, the mean rate of fission ($\dot{f}$) is given by:
\begin{equation}
\dot{f} = \frac{P_{t}}{E_{fission}} = \frac{3 \times 10^{9}}{(170 \times 10^{6})(1.60 \times 10^{-19})} = 1.10 \times 10^{20} \ \ch{s^{-1}}
\end{equation}
Each fission of U$^{235}$ is an independent event, therefore the production rate of a particular isotope from fission ($\dot{f}_{X}$) is related to its independent fission yield ($y_{X}$) as follows:
\begin{equation}
\dot{f}_{X} = \dot{f}y_{X}
\end{equation}
The total inventory of iodine at steady state operation is dependent on four factors:
\begin{itemize}
\item Production directly from fission
\item Production from decay of tellurium precursors
\item Loss through radioactive decay
\item Loss (of a particular iodine isotope) through neutron capture
\end{itemize}
The rate of change of tellurium and iodine populations (\ch{N_{Te}} and \ch{N_{I}}) can therefore be expressed by:
\begin{gather}
\frac{d\ch{N_{Te}}}{dt} = \dot{f}_{Te} - \lambda_{Te}\ch{N_{Te}} \\
\frac{d\ch{N_{I}}}{dt} = \dot{f}_{I} + \lambda_{Te}\ch{N_{Te}} - \lambda_{I}\ch{N_{I}} - \phi\sigma_{a_{I}}\ch{N_{I}}
\end{gather}
where $\lambda$ is the decay constant, $\phi$ is the thermal neutron flux and $\sigma_{a_{I}}$ is the microscopic thermal neutron absorption cross-section of iodine.

\end{document}