\chapter{Crystallography and Point Defects}

\label{ch:crystallography}

\section{\zirconia\ phases and stabilisation}

\zirconia\ is unusual in exhibiting three commonly reported polytypes in its binary phase diagram (Figure \ref{figure:binary_phase_diagram}). Each will now be described and contrasted.

%Zirconia-based ceramics are used in a variety of technological and commercial applications, ranging from thermal barriercoatings on aircraft turbines to affordable diamond substitutes[167,168]. Recently, zirconia has received attention as an alternative dielectrics to silicon dioxide for memory and logic devices [169]. The versatility of zirconia originates entirely fromatomic or point defects in the crystal created by adding aliovalent oxides, e.g., MgO, La2O3, and Y2O3. The substitutionalcations or dopants that are charge balanced by the formation ofoxygen vacancies, and their mutual interactions, dramaticallyaffect the structural, thermal, mechanical, and electrical properties of modified or stabilized zirconia (SZ). These defects aresolely responsible for the high ionic conductivity of doped zirconia that underlies its use in oxygen sensors [170], and hightemperature fuel cells [171]. Zirconia is also used as a support material in heterogeneous catalysis [172].


\subsection{Monoclinic}

Below 1205 K at atmospheric pressure, \zirconia\ adopts a monoclinic Baddeleyite (P$2_{1}/c$) crystal structure. This is the ground state structure of \zirconia . In this phase, Zr ions have a sevenfold O coordination, down from eight in the higher temperature phases due to a single broken bond. A unit cell of monoclinic \zirconia\ is illustrated in Figure \ref{figure:coordination}. The dashed line (approximately 3.7\r{A} in length) shows the Zr-O bond which is broken when transitioning to monoclinic from the tetragonal phase.

\begin{figure}[htp] % Mono coordination figure
\centering
\includegraphics[height=7.5cm]{images/coordination.png}
\caption[A monoclinic zirconia unit cell indicating the two different oxygen bond coordinations. Small spheres represent oxygen ions while large spheres represent zirconium ions.]{A monoclinic zirconia unit cell indicating the two different oxygen bond coordinations. Small spheres represent oxygen ions while large spheres represent zirconium ions. Taken from \cite{Xia2010}.
\label{figure:coordination}}
\end{figure}

Monoclinic phase \zirconia\ also has two distinct oxygen ions in its primitive cell. To maintain the stoichiometry of 1:2 Zr to O, half of the O ions exhibit threefold coordination with Zr (in a planar configuration), while the other half have a fourfold Zr coordination (tetrahedral configuration). Figure \ref{figure:monoschottky} shows the positions of these O ions around a Zr ion centre. The distortion of the oxygen rock salt sub-lattice can be seen, especially with the three coordinated oxygens. This is due to the Zr ion being too small to hold 8 O ions in an octahedral configuration, as in the fluorite crystal structure. As temperature is increased, so too is the mean radius of the ions. At the tetragonal temperature range, bonding between all 8 nearest neighbour O ions and Zr becomes stable and the eightfold coordination can be restored. 

The monoclinic-tetragonal phase transition occurs by a diffusionless martensitic transformation \cite{Subbarao1974}. This is a fast transformation and has a large volume change associated with it. Tetragonal phase \zirconia\ (density 6.10 g/cm$^{3}$) is around 4.6\% more dense than monoclinic \zirconia\ (density 5.83 g/cm$^{3}$) \cite{McCullough2002}, though the volume increase when cooling from tetragonal has been reported to be as high as 9\% \cite{Gupta1977}. The phase transition exhibits a hysteresis loop approximately 200 K wide when undergoing thermal cycling, as shown in Figure \ref{figure:hysteresis_monotet}.

\begin{figure}[htp] % Hysteresis
\centering
\includegraphics[width=12cm]{images/hysteresis_monotet.png}
\caption[Monoclinic-tetragonal phase transition in \zirconia\ as a function of temperature.]{Monoclinic-tetragonal phase transition in \zirconia\ as a function of temperature. Taken from \cite{WOLTEN1963}.}
\label{figure:hysteresis_monotet}
\end{figure}

\begin{figure}[htp] % Mono Zr centre
\centering
\includegraphics[height=8.5cm]{images/zr_centre_mono.png}
\caption{Zirconium centre in monoclinic \zirconia\ showing nearest oxygen atoms and their respective bond co-ordinations. Zirconium atoms are shown in green and oxygen atoms in red.}
\label{figure:monoschottky}
\end{figure}


\begin{landscape}
\begin{center}
\begin{table}[htp]
\onehalfspacing
\centering
\caption[\zirconia\ phases and their details.]{\zirconia\ phases and their details. Adapted from \cite{kisi1998crystal}.}
\label{my-label}
%\resizebox{\textwidth}{!}{%
\begin{tabular}{ccccccccccc}
\hline
\multirow{3}{*}{Phase} & \multirow{3}{*}{\begin{tabular}[c]{@{}c@{}}Stability\\ range\\ (K, GPa)\end{tabular}} & \multicolumn{4}{c}{\multirow{2}{*}{\begin{tabular}[c]{@{}c@{}}Cell parameters (\r{A})\\ (Extrapolated to room temperature)\end{tabular}}} & \multicolumn{4}{c}{\multirow{2}{*}{Atom positions}} & \multirow{3}{*}{\begin{tabular}[c]{@{}c@{}}Space\\ group\end{tabular}} \\
 &  & \multicolumn{4}{c}{} & \multicolumn{4}{c}{} &  \\ \cline{3-10}
 &  & $a$ & $b$ & $c$ & $\beta$ & Atom & $x$ & $y$ & $z$ &  \\ \hline
\multirow{2}{*}{Cubic \cite{terblanche1989thermal}} & \multirow{2}{*}{2377-2710 K} & \multirow{2}{*}{5.117} & \multirow{2}{*}{5.117} & \multirow{2}{*}{5.117} & \multirow{2}{*}{90} & Zr & 0 & 0 & 0 & \multirow{2}{*}{F$m\overline{3}m$} \\
 &  &  &  &  &  & O & 0.25 & 0.25 & 0.25 &  \\ \hline
\multirow{2}{*}{Tetragonal \cite{LANG1964}} & \multirow{2}{*}{1205-2377 K} & \multirow{2}{*}{5.074} & \multirow{2}{*}{5.074} & \multirow{2}{*}{5.188} & \multirow{2}{*}{90} & Zr & 0 & 0 & 0 & \multirow{2}{*}{P$4_{2}/nmc$} \\
 &  &  &  &  &  & O & 0.25 & 0.25 & 0.2044 &  \\ \hline
\multirow{3}{*}{Monoclinic \cite{Howard1988}} & \multirow{3}{*}{0-1205 K} & \multirow{3}{*}{5.1507} & \multirow{3}{*}{5.2028} & \multirow{3}{*}{5.3156} & \multirow{3}{*}{99.194} & Zr & 0.2754 & 0.0395 & 0.2083 & \multirow{3}{*}{P$2_{1}/c$} \\
 &  &  &  &  &  & O1 & 0.0700 & 0.3317 & 0.3477 &  \\
 &  &  &  &  &  & O2 & 0.4416 & 0.7569 & 0.4792 &  \\ \hline
\multirow{3}{*}{Ortho I \cite{ohtaka1990structural}} & \multirow{3}{*}{3.5-15 GPa} & \multirow{3}{*}{5.0431} & \multirow{3}{*}{5.2615} & \multirow{3}{*}{5.0910} & \multirow{3}{*}{90} & Zr & 0.2686 & 0.0332 & 0.2558 & \multirow{3}{*}{P$bca$} \\
 &  &  &  &  &  & O1 & 0.0822 & 0.3713 & 0.1310 &  \\
 &  &  &  &  &  & O2 & 0.5442 & 0.2447 & 0.0052 &  \\ \hline
\multirow{3}{*}{\begin{tabular}[c]{@{}c@{}}Ortho II \cite{Haines1995} \\ (cotunnite)\end{tabular}} & \multirow{3}{*}{\textgreater 15 GPa} & \multirow{3}{*}{5.593} & \multirow{3}{*}{6.484} & \multirow{3}{*}{3.333} & \multirow{3}{*}{90} & Zr & 0.256 & 0.110 & 0.25 & \multirow{3}{*}{P$nam$} \\
 &  &  &  &  &  & O1 & 0.356 & 0.422 & 0.25 &  \\
 &  &  &  &  &  & O2 & 0.022 & 0.331 & 0.75 &  \\ \hline
\multirow{3}{*}{Ortho (PSZ) \cite{kisi1989crystal}} & \multirow{3}{*}{0-500 K} & \multirow{3}{*}{5.068} & \multirow{3}{*}{5.260} & \multirow{3}{*}{5.077} & \multirow{3}{*}{90} & Zr & 0.267 & 0.030 & 0.250 & \multirow{3}{*}{P$bc2_{1}$} \\
 &  &  &  &  &  & O1 & 0.068 & 0.361 & 0.106 &  \\
 &  &  &  &  &  & O2 & 0.537 & 0.229 & 0 &  \\ \hline
\end{tabular}%
\end{table}
\end{center}
\end{landscape}


%\begin{table}[htp]
%\centering
%\onehalfspacing
%\caption{\zirconia\ crystal structures and their stable temperatures at 1 atm \cite{Howard1988}.}
%\label{table:phases}
%\begin{tabular}{ccc}
%\hline
%{Crystal Structure} & {Space Group}    & {Temperature Range (K)} \\ \hline
%\multicolumn{1}{c}{Monoclinic} & \multicolumn{1}{c}{$P2_1/c$} & \multicolumn{1}{c}{$T$ \textless\ 1440}     \\
%\multicolumn{1}{c}{Tetragonal} & \multicolumn{1}{c}{$P4_2/nmc$} & \multicolumn{1}{c}{1440 \textless\ $T$ \textless\ 2640}        \\
%\multicolumn{1}{c}{Cubic} & \multicolumn{1}{c}{$Fm\overline{3}m$}     & \multicolumn{1}{c}{2640 \textless\ $T$ \textless\ 2950}      \\ \hline
%\end{tabular}
%\end{table}

\subsection{Tetragonal}

\subsection{Cubic}

\subsubsection{Cubic phase stability}

\begin{itemize}
\item The existence of a cubic phase in undoped \zirconia\ is debated. The high temperatures required for stabilisation of the cubic phase and the similarity to the tetragonal phase make it difficult to discern as a third phase.
\item Also due to the high temperature, the stability of the cubic phase in DFT calculations may be inaccurate. This is discussed further in Chapter 3.
\end{itemize}

\begin{figure}[htp]
\centering
\includegraphics[width=14cm]{images/tet_vs_cubic.png}
\caption{\textbf{A}) Tetragonal \zirconia\ viewed along the [110] direction. \textbf{B}) Cubic \zirconia\ viewed along the [100] direction. Zirconium atoms are shown in green and oxygen atoms in red.}
\label{figure:tetvscubic}
\end{figure}

\subsection{Other phases}

\subsubsection{Orthorhombic}

Two orthorhombic phases have also been observed at high pressures in pure \zirconia\ \cite{howard1991crystal}. These structures are referred to as OI and OII, the latter of which is isostructural with cotunnite (PbCl$_{2}$). A third orthorhombic phase ($bc2_{1}$) has also been reported in partially stabilised zirconia, but has not been found in pure zirconia \cite{kisi1998crystal}.


\begin{figure}[htp]
  \centering
      \includegraphics[width=\linewidth]{images/orthorhombic_I.png}
  \caption[Illustrations of the orthorhombic OI (P$bca$) crystal structure of \zirconia\ showing \textbf{a)} the unit cell as seen from the $a$ direction and \textbf{b)} the zirconium ion centre with 7 coordinated oxygen ions. Zirconium and oxygen ions are coloured green and red respectively.]{Illustrations of the orthorhombic OI (P$bca$) crystal structure of \zirconia\ showing \textbf{a)} the unit cell as seen from the $a$ direction and \textbf{b)} the zirconium ion centre with 7 coordinated oxygen ions. Zirconium and oxygen ions are coloured green and red respectively. Adapted from \cite{kisi1989crystal}.}
  \label{fig:orthorhombic_I}
\end{figure}


\begin{figure}[htp]
  \centering
      \includegraphics[height=9cm]{images/orthorhombic_II.png}
  \caption[Illustration of the OII cotunnite (P$nam$) crystal structure of \zirconia . Zirconium and oxygen ions are shaded dark and light respectively.]{Illustration of the OII cotunnite (P$nam$) crystal structure of \zirconia . Zirconium and oxygen ions are shaded dark and light respectively. Adapted from \cite{Haines1997}.}
  \label{fig:cotunnite_structure}
\end{figure}

\subsubsection*{Volume expansion}

The phase transitions in \zirconia\ are accompanied by a change in volume, where the monoclinic phase is the least dense and the cubic phase is the most dense (see Figure \ref{figure:zrobonddistance}). This is especially significant in the case of the martensitic t-\zirconia\ to m-\zirconia\ transition, where the volume increases by around 9\% \cite{Gupta1977}. This has substantial implications for the creation and opening of cracks as \zirconia\ is a ceramic material with low toughness. This is especially relevant in a reactor scenario where temperature cycling (shutdown/startup or load-following behaviour) may lead to fatigue if the phase transition threshold is passed.

Another consequence of this large volume expansion is that a significant hysteresis effect is observed in the monoclinic/tetragonal phase transition, as shown in Figure \ref{fig:phasediagram}. 
%as the resulting coherency strain is likely to result in reduced mobility of fission products that have been embedded in the bulk crystal. 

\begin{figure}[htp]
  \centering
      \includegraphics[width=13cm]{images/zirconiaphasediagram.png}
  \caption[Pressure-temperature phase diagram for \zirconia . Dash-dotted lines represent more recent data. Diamonds mark transition points during an increase in pressure/temperature, while open circles are used for a decrease in pressure/temperature. Solid circles represent transition points for a fresh, single crystal sample.]{Pressure-temperature phase diagram for \zirconia . Dash-dotted lines represent more recent data. Diamonds mark transition points during an increase in pressure/temperature, while open circles are used for a decrease in pressure/temperature. Solid circles represent transition points for a fresh, single crystal sample. Taken from \cite{gando2011partial}. \label{fig:phasediagram}}
\end{figure}

\begin{figure}
\begin{center}
\begin{tikzpicture}
	\begin{axis}
		[width=\linewidth*0.7, xlabel={Nearest neighbour Zr-O bond distance (\r{A})}, ylabel={Relative occurrence}, ymin=0, ymax=140, xmin=2.0, xmax=2.50, legend style={{draw=}, at={(0.95,0.95)}, anchor=north east, legend columns=1}]
		\addplot[no marks] table [x=zr_o_dist, y=monoclinic,]{dat/zr_o_bond_distances.dat}; \addlegendentry{Monoclinic};
        \addplot[no marks, dashed] table [x=zr_o_dist, y=tetragonal, ]{dat/zr_o_bond_distances.dat}; \addlegendentry{Tetragonal};
        \addplot[no marks, densely dotted] table [x=zr_o_dist, y=cubic,]{dat/zr_o_bond_distances.dat}; \addlegendentry{Cubic};
			\end{axis}
		\end{tikzpicture}
		\caption{Density plot of the nearest neighbour Zr-O bond distances in \zirconia\ for each crystal structure. Specific volumes from DFT simulations are 11.99 \r{A}$^{3}$ion$^{-1}$, 11.51 \r{A}$^{3}$ion$^{-1}$, and 11.13 \r{A}$^{3}$ion$^{-1}$ for monoclinic, tetragonal, and cubic phases respectively.}
		\label{figure:zrobonddistance}
	\end{center}
\end{figure}


\subsection{Pressure stabilisation (isochoric + autostabilisation)}

The tetragonal and cubic phases of \zirconia\ are stabilised at high pressure. Since the oxide has a larger volume than the underlying metal (pilling-bedworth ratio of 1.5X), the growth of the oxide will itself impose stresses which may stabilise the tetragonal phase.

\subsection{Dopant stabilisation (lower valence cations)}

Particular kinds of dopants will also stabilise the tetragonal and cubic phases of \zirconia. The most technologically significant of which is yttrium, which at concentrations of 15\% (atomic), fully stabilises the cubic phase. Zirconia stabilised this way is known as yttria-stabilised zirconia (YSZ). This works by trivalent yttrium promoting the inclusion of charge compensating oxygen vacancy defects (see Equation \ref{equation:YSZ}). This works in a similar way with several other cation dopants such as trivalent scandium from Sc$_{2}$O$_{3}$, or divalent magnesium from MgO (Equation \ref{equation:mg_stabilised_zro2}).

\begin{equation}
Y_{2}O_{3} = 2\ch{Y_{Zr}^{'}} + \ch{V_{O}^{**}} + 3\ch{O_{O}^{x}} 
\label{equation:YSZ}
\end{equation}

\begin{equation}
MgO= \ch{Mg_{Zr}^{''}} + \ch{V_{O}^{**}} + \ch{O_{O}^{x}} 
\label{equation:mg_stabilised_zro2}
\end{equation}

While \zirconia\ would ideally adopt the regular cubic fluorite structure, we only observe cubic stabilisation at elevated temperatures where the ionic radii of the zirconium ion increases (and possibly a stabilising phonon mode contribution). Yttrium is known to stabilise the cubic phase in \zirconia\ at standard conditions. This stabilising effect is observed because the yttrium ion is of the appropriate size to maintain the oxygen ions and vacancies in the VIII coordination at low temperatures, as can be seen from ionic radii values in Table \ref{figure:ionicradii}. 

%This is accomplished by substituting a Zr\textsuperscript{4+} ion with Y\textsuperscript{3+} and the binding of two neighbouring oxygen vacancies, effectively producing a sub-unit of yttria (Y\textsubscript{2}O\textsubscript{3}) in the \zirconia\ lattice. 

\begin{table}[htp]
\onehalfspacing
\centering
\caption[Ionic radii of Zr\textsuperscript{4+} and Y\textsuperscript{3+} in various coordination environments.]{Ionic radii of Zr\textsuperscript{4+} and Y\textsuperscript{3+} in various coordination environments. Values taken from \cite{Shannon1976}.}
\label{figure:ionicradii}
\begin{tabular}{ccc}
\hline
Ion & Coordination & Ionic Radius (\r{A}) \\ \hline
\multicolumn{1}{c}{\multirow{6}{*}{Zr\textsuperscript{4+}}} & \multicolumn{1}{c}{IV} & 0.59 \\
\multicolumn{1}{c}{} & \multicolumn{1}{c}{V} & 0.66 \\
\multicolumn{1}{c}{} & \multicolumn{1}{c}{VI} & 0.72 \\
\multicolumn{1}{c}{} & \multicolumn{1}{c}{VII} & 0.78 \\
\multicolumn{1}{c}{} & \multicolumn{1}{c}{VIII} & 0.84 \\
\multicolumn{1}{c}{} & \multicolumn{1}{c}{IX} & 0.89 \\ \hline
\multicolumn{1}{c}{\multirow{4}{*}{Y\textsuperscript{3+}}} & \multicolumn{1}{c}{VI} & 0.90 \\
\multicolumn{1}{c}{} & \multicolumn{1}{c}{VII} & 0.96 \\
\multicolumn{1}{c}{} & \multicolumn{1}{c}{VIII} & 1.019 \\
\multicolumn{1}{c}{} & \multicolumn{1}{c}{IX} & 1.075 \\ \hline
\end{tabular}
\end{table}

%\section{Point Defects}

\subsection{Kr\"{o}ger-Vink notation}

Kr\"{o}ger-Vink notation \cite{kroger1956relations} is used throughout this thesis to describe defects. It is widely used in physical chemistry and is a useful shorthand for describing chemical reactions where conservation of mass, charge and lattice sites is required. The notation syntax is of the form \ch{x^{y}_{z}}, where x is the substituted atom or missing atom (i.e. a vacancy V), y is the charge of the defect (relative to the lattice species that originally occupied the site) and z is the site the defect occupies. Positive and negative charges are indicated with dots (\ch{^{*}}) and dashes (\ch{^{'}}) respectively, otherwise a cross (\ch{^{x}}) is used to denote a neutral defect. The site may be either a lattice site (such as Zr or O in \zirconia ) or an interstitial site ($i$). Table \ref{table:krogervink} shows examples of several different types of defects and their respective Kr\"{o}ger-Vink notation.

\begin{table}[htp] % Kroger-Vink notation table
\onehalfspacing
\centering
\caption{Examples of Kr\"{o}ger-Vink notation for several defects in \zirconia .}
\label{table:krogervink}
\begin{tabular}{cc}
\hline
Defect & Kr\"{o}ger-Vink Notation \\ \hline
Anion vacancy & \ch{V_{O}^{**}} \\
Cation vacancy & \ch{V_{Zr}^{''''}} \\
Anion interstitial & \ch{O_{i}^{''}} \\
Cation interstitial & \ch{Zr_{i}^{****}} \\
Iodine (I$^{-}$ anion) on oxygen site & \ch{I_{O}^{*}} \\
Iodine (I$^{+}$ cation) on zirconium site & \ch{I_{Zr}^{'''}} \\ \hline
\end{tabular}
\end{table}

